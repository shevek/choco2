%\part{feaspairac}
\label{feaspairac}
\hypertarget{feaspairac}{}

\section{feasPairAC (constraint)}\label{feaspairac:feaspairacconstraint}\hypertarget{feaspairac:feaspairacconstraint}{}

\begin{notedef}
  \texttt{feasPairAC}$(x,y,feasTuples)$ states an extensional binary constraint on $(x,y)$ defined by the table $feasTuples$ of compatible pairs of values, and then enforces arc consistency. Two APIs are available to define the compatible pairs:
\begin{itemize}
	\item if $feasTuples$ is encoded as a list of pairs \texttt{List<int[2]>}, then:
      $$\exists \text{ tuple } i\ |\quad (x,y)=feasTuples[i]$$
	\item if $feasTuples$ is encoded as a boolean matrix \texttt{boolean[][]}, let $\underline{x}$ and  $\underline{y}$ be the initial minimum values of $x$ and $y$, then:
      $$\exists (u,v)\ |\quad (x,y)=(u+\underline{x},v+\underline{y})\ \land\ feasTuples[u][v]$$
\end{itemize}
\end{notedef}

The two APIs are duplicated to allow definition of options. 
\begin{itemize}
	\item \textbf{API} :
	\begin{itemize}
		\item \mylst{feasPairAC(IntegerVariable x, IntegerVariable y, List<int[]> feasTuples)}
		\item \mylst{feasPairAC(String options, IntegerVariable x, IntegerVariable y, List<int[]> feasTuples)}
		\item \mylst{feasPairAC(IntegerVariable x, IntegerVariable y, boolean[][] feasTuples)}
		\item \mylst{feasPairAC(String options, IntegerVariable x, IntegerVariable y, boolean[][] feasTuples)}
	\end{itemize}
	\item \textbf{return type} : \texttt{Constraint}
	\item \textbf{options} :
	\begin{itemize}
		\item \emph{no option}: use AC3 (default arc consistency)
		\item \texttt{cp:ac3}: to get AC3 algorithm (searching from scratch for supports on all values)
		\item \texttt{cp:ac2001}: to get AC2001 algorithm (maintaining the current support of each value)
		\item \texttt{cp:ac32}: to get AC3rm algorithm (maintaining the current support of each value in a non backtrackable way)
		\item \texttt{cp:ac322}: to get AC3 with the used of \texttt{BitSet} to know if a support still exists
	\end{itemize}
	\item \textbf{favorite domain} : \emph{to complete}
	\item \textbf{references} :\\
      global constraint catalog: \href{http://www.emn.fr/x-info/sdemasse/gccat/Celem.html}{elem}
\end{itemize}



\textbf{Example}:
\begin{lstlisting}
	Model m = new CPModel();
	Solver s = new CPSolver();
	
	ArrayList couples2 = new ArrayList();
	couples2.add(new int[]{1, 2});
	couples2.add(new int[]{1, 3});
	couples2.add(new int[]{2, 1});
	couples2.add(new int[]{3, 1});
	couples2.add(new int[]{4, 1});
	
	IntegerVariable v1 = makeIntVar("v1", 1, 4);
	IntegerVariable v2 = makeIntVar("v2", 1, 4);
	m.addConstraint(feasPairAC("cp:ac32",v1, v2, couples2));
	s.read(m);
	s.solveAll();
\end{lstlisting} 
