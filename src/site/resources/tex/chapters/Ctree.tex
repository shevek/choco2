%\part{tree}
\label{tree}
\hypertarget{tree}{}

\section{tree (constraint)}\label{tree:treeconstraint}\hypertarget{tree:treeconstraint}{}

Let $G=(V,A)$ be a digraph on $V=\{1,\ldots,n\}$. $G$ can be modeled by a sequence of domain variables $x=(x_1,\dots,x_n)\in V^n$ -- the \emph{successors} variables -- whose respective domains are given by $D_i=\{j\in V\ |\ (i,j)\in A\}$. Conversely, when instantiated, $x$ defines a subgraph $G_x=(V,A_x)$ of $G$ with $A_x=\{(i,x_i)\ |\ i\in V\}\subseteq A$. Such a subgraph has one particularity: any connected component of $G_x$ contains either no loop -- and then it contains a cycle -- or exactly one loop $x_i=i$ and then it is a \emph{tree} of root $i$ (literally, it is an anti-arborescence as there exists a path from each node to $i$ and $i$ has a loop).

\begin{notedef}
  \texttt{tree}$(x,restrictions)$ is a vertex-disjoint graph partitioning constraint. It states that $G_x$ is a forest (its connected components are trees) that satisfies some conditions specified by $restrictions$.
\texttt{tree} deals with several kinds of graph restrictions on:
\begin{itemize}
	\item the number of trees
	\item the number of proper trees (a tree is proper if it contains more than 2 nodes)
    \item the weigth of the partition: the sum of the weights of the edges
	\item incomparability: some nodes in pairs have to belong to distinct trees
	\item precedence: some nodes in pairs have to belong to the same tree in a given order
	\item conditional precedence: some nodes in pairs have to respect a given order if they belong to the same tree
	\item the in-degree of the nodes
	\item the time windows on nodes (given travelling times on arcs)
\end{itemize}
\end{notedef}

Many applications require to partition a graph such that each component contains exactly one \emph{resource} node and several \emph{task} nodes. A typical example is a routing problem where vehicle routes are paths (a path is a special case of tree) starting from a depot and delivering goods to several clients. Another example is a local network where each computer has to be connected to one shared printer. Last, one can cite the problem of reconstructing plylogeny trees.
The constraint \texttt{tree} can handle these kinds of problems with many additional constraints on the structure of the partition.

\begin{itemize}
	\item \textbf{API} : \mylst{tree(TreeParametersObject param)}
	\item \textbf{return type} : \texttt{Constraint}
	\item \textbf{options} :\emph{n/a}
	\item \textbf{favorite domain} : \emph{to complete}
	\item \textbf{references} :
      \begin{itemize}
      \item \cite{beldiceanuCONSTRAINTS08}: \emph{Combining tree partitioning, precedence, and incomparability constraints}
      \item global constraint catalog: \href{http://www.emn.fr/x-info/sdemasse/gccat/Cproper_forest.html}{proper\_forest} (variant)
      \end{itemize}

\end{itemize}

The tree constraint API requires a particular Model object, named \textbf{\tt TreeParametersObject}.
It can be created with the following parameters:

\begin{tabular}{p{3cm}p{3cm}p{7cm}}
parameter &type &description\\
\hline
$n$ &\texttt{int} &number of nodes in the initial graph $G$\\
$nTree$ &\texttt{IntegerVariable} &number of trees in the resulting forest $G_x$\\
$nProper$ &\texttt{IntegerVariable} &number of proper trees in $G_x$\\
$objective$ &\texttt{IntegerVariable} &(bounded) total \todo{cost} of $G_x$\\
%$objective$ &\texttt{IntegerVariable} &(bounded) total weight of $G_x$\\
$graphs$ &\texttt{List<BitSet[]>} &
\begin{minipage}[t]{7cm}
graphs encoded as successor lists,\\
  \texttt{graphs[0]} the initial graph $G$,\\
  \texttt{graphs[1]} a precedence graph,\\
  \texttt{graphs[2]} a conditional precedence graph,\\
  \texttt{graphs[3]} an incomparability graph
\end{minipage}\\
$matrix$ &\texttt{List<int[][]>} &\texttt{matrix[0]} the indegree of each node, and \texttt{matrix[1]} the starting time from each node\\
$travel$ &\texttt{int[][]} &the travel time of each arc
\end{tabular}

\textbf{Example}:
\lstinputlisting{java/ctree_import.j2t}
\lstinputlisting{java/ctree.j2t}
