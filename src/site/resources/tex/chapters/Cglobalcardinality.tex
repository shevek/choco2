%\part{globalcardinality}
\label{globalcardinality}
\hypertarget{globalcardinality}{}

\section{globalCardinality (constraint)}\label{globalcardinality:globalcardinalityconstraint}\hypertarget{globalcardinality:globalcardinalityconstraint}{}
\begin{notedef}
  \texttt{globalCardinality}$(x,low, up)$ states bounds on the occurrence numbers of any value $v$ in $x$ (here, offset $min$ is the minimum value over all variables in $x$) :
$$low[v-min]\le|\{i\ |\ x_i=v\}|\le up[v-min],\quad\forall \text{ value } v$$   
\end{notedef}

Mulitple APIs exist:
\begin{itemize}
	\item \emph{bounds on cardinalities:} Given an array of variables $x$, $min$ the minimal value over all variables, and $max$ the maximal value over all variables, the constraint ensures that the number of occurrences of value $v$ among the variables is between $low[v-min]$ and $up[v-min]$. Note that the length of $low$ and $up$ should be $max - min + 1$. Use the propagator of \cite{QuimperCP03}.
	\item \emph{default offset $min=1$:} Given an array of variables $x$, the constraint ensures that the number of occurrences of the value 1 in all the variables $x$ is between $low[0]$ and $up[0]$, and generally the number of occurrences of the value $v$ in $x$ is between $low[v-1]$ and $up[v-1]$.
	\item \emph{variable cardinalities:} Given an array of variables $x$, an array of variables $card$ to represent the cardinalities, the constraint ensures that the number of occurrences of the value $v$ among the variables is equal to $card[v]$. This constraint:
      \begin{itemize}
      \item enforces Bound Consistency over $x$ regarding the lower and upper bounds of $card$, 
      \item maintains the upper bounds of $card$ by counting the number of variables in which each value can occur, 
      \item maintains the lower bounds of $card$ by counting the number of variables instantiated to each value, 
      \item enforces $card[0] + \cdots + card[m] = n$, where \emph{n} is the number of variables and \emph{m} the number of values.
      \end{itemize}

\end{itemize}

The APIs are duplicated to define options. 

\begin{itemize}
	\item \textbf{API} :
      \begin{itemize}
	\item \mylst{globalCardinality(IntegerVariable[] x, int min, int max, int[] low, int[] up)}
	\item \mylst{globalCardinality(String options, IntegerVariable[] x, int min, int max, int[] low, int[] up)}
	\item \mylst{globalCardinality(IntegerVariable[] x, int[] low, int[] up)}
	\item \mylst{globalCardinality(String options, IntegerVariable[] x, int[] low, int[] up)}
	\item \mylst{globalCardinality(IntegerVariable[] x, int min, int max, IntegerVariable[] card)}
      \end{itemize}
	\item \textbf{return type} : \texttt{Constraint}
	\item \textbf{options}:
	\begin{itemize}
		\item \emph{no option} :
		\item \texttt{cp:ac} : for \cite{ReginAAAI96} implementation of arc consistency
		\item \texttt{cp:bc} : for  \cite{QuimperCP03} implementation of bound consistency
	\end{itemize}
	\item \textbf{favorite domain} : \emph{to complete}
	\item \textbf{references} :
      \begin{itemize}
      \item \cite{ReginAAAI96}: \emph{Generalized arc consistency for global cardinality constraint},
      \item \cite{QuimperCP03}: \emph{An efficient bounds consistency algorithm for the global cardinality constraint}
      \item global constraint catalog: \href{http://www.emn.fr/x-info/sdemasse/gccat/Cglobal_cardinality.html}{global\_cardinality}
      \end{itemize}
\end{itemize}

\textbf{Example}:
\begin{lstlisting}
	int n = 5;
	CPModel pb = new CPModel();
	CPSolver s = new CPSolver();
	
	IntegerVariable[] vars = new IntegerVariable[n];
	for (int i = 0; i < n; i++) {
	   vars[i] = makeIntVar("var " + i, 1, n);
	}
	
	int[] LB2 = {0, 1, 1, 0, 3};
	int[] UB2 = {0, 1, 1, 0, 3};
	
	m.addConstraint("cp:bc", globalCardinality(vars, 1, n, LB2, UB2));
	s.read(pb);
	s.solve();
\end{lstlisting}
