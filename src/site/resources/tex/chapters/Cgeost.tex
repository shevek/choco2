%\part{geost}
\label{geost}
\hypertarget{geost}{}

\section{geost (constraint)}\label{geost:geostconstraint}\hypertarget{geost:geostconstraint}{}
\begin{notedef}
\texttt{geost} is a global constraint that generically handles a variety of geometrical placement problems. 
It handles geometrical constraints (non-overlapping, distance, etc.) between polymorphic objects (ex: polymorphism can be used for representing rotation) in any dimension.
%The \texttt{geost}$(K, O, S, C)$ constraint is given set of parameters which will define the environment of \texttt{geost}. The parameters are as follows:
The parameters of \texttt{geost}$(dim, objects, shiftedBoxes, eCtrs)$ are respectively:
the space dimension, the list of geometrical objects, the set of boxes that compose the shapes of the objects, the set of geometrical constraints.
\end{notedef}

\begin{itemize}
	\item \textbf{API} :\\
\mylst{geost(int dim, Vector<GeostObject> objects, Vector<ShiftedBox> shiftedBoxes, Vector<ExternalConstraint> eCtrs)}\\
\mylst{geost(int dim, Vector<GeostObject> objects, Vector<ShiftedBox> shiftedBoxes, Vector<ExternalConstraint> eCtrs, Vector<int[]> ctrlVs)}
	\item \textbf{return type} : \texttt{Constraint}
	\item \textbf{options} :\emph{n/a}
	\item \textbf{favorite domain} : \emph{to complete}
	\item \textbf{references} :\\
      global constraint catalog: \href{http://www.emn.fr/x-info/sdemasse/gccat/Cgeost.html}{geost}
\end{itemize}

The geost constraint requires the creation of different objects:

\centerline{\begin{tabular}{p{3cm}p{5cm}p{6cm}}
parameter &type &description \\
\hline
\emph{objects} &\texttt{Vector<GeostObject>} &geometrical objects\\
\emph{shiftedBoxes} &\texttt{Vector<ShiftedBox>} &boxes that compose the object shapes\\
\emph{eCtrs} &\texttt{Vector<ExternalConstraint>} &geometrical constraints\\
\emph{ctrlVs} &\texttt{Vector<int[]>} &controlling vectors (for greedy mode)\\[1em]
\end{tabular}}

\noindent Where a \texttt{\bf GeostObject} is defined by:

\centerline{\begin{tabular}{p{3cm}p{4cm}p{7cm}}
attribute &type &description \\
\hline
\emph{dim} &\texttt{int} &dimension\\
\emph{objectId} &\texttt{int} &object id\\
\emph{shapeId} &\texttt{IntegerVariable} &shape id\\
\emph{coordinates} &\texttt{IntegerVariable[$dim$]} &coordinates of the origin\\
\emph{startTime} &\texttt{IntegerVariable} &starting time\\
\emph{durationTime} &\texttt{IntegerVariable} &duration\\
\emph{endTime} &\texttt{IntegerVariable} &finishing time\\[1em]
\end{tabular}}

\noindent Where a \texttt{\bf ShiftedBox} is a $dim$-box defined by the shape it belongs to, its origin (the coordinates of the lower left corner of the box) and its lengths in every dimensions:

\centerline{\begin{tabular}{p{3cm}p{4cm}p{7cm}}
attribute &type &description \\
\hline
\emph{sid} &\texttt{int} &shape id\\
\emph{offset} &\texttt{int[$dim$]} &coordinates of the offset (lower left corner)\\
\emph{size} &\texttt{int[$dim$]} &lengths in every dimensions\\[1em]
\end{tabular}}

\noindent Where an \texttt{\bf ExternalConstraint} contains informations and functionality common to all external constraints and is defined by:

\centerline{\begin{tabular}{p{3cm}p{4cm}p{7cm}}
attribute &type &description \\
\hline
 \emph{ectrID} &\texttt{int} &constraint id\\
 \emph{dimensions} &\texttt{int[]} &list of dimensions that the external constraint is active for\\
 \emph{objectIdentifiers} &\texttt{int[]} &list of object ids that this external constraint affects.\\[1em]
\end{tabular}}

\begin{notedef}
For further informations, visit the following \hyperlink{geostdescription:placementanduseofthegeostconstraint}{page}.
\end{notedef}

\textbf{Example}:
\lstinputlisting{java/cgeost.j2t}
