%\part{setunion}
\label{setunion}
\hypertarget{setunion}{}

\section{setUnion (constraint)}\label{setunion:setunionconstraint}\hypertarget{setunion:setunionconstraint}{}
\begin{notedef}
  \texttt{setUnion}$(s_1,s_2,s_3)$ states that the third set $s_3$ is exactly the union of the two first sets:
$$s_1\cup s_2=s_3$$
\end{notedef}

\begin{itemize}
	\item \textbf{API} : \mylst{setUnion(SetVariable s1, SetVariable s2, SetVariable s3)}
	\item \textbf{return type} : \texttt{Constraint}
	\item \textbf{options} :\emph{n/a}
	\item \textbf{favorite domain} : \emph{to complete}
\end{itemize}

\textbf{Example}:
\begin{lstlisting}
	Model m = new CPModel();
	Solver s = new CPSolver();
	setVar x = makeSetVar("X", 1, 3);
	SetVar y = makeSetVar("Y", 3, 5);
	SetVar z = makeSetVar("Z", 0, 6);
	Constraint c1 = setUnion(x, y, z);
	m.addConstraint(c1);
	s.read(m);
	s.solveAll();
\end{lstlisting} 
