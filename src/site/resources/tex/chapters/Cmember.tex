%\part{member}
\label{member}
\hypertarget{member}{}

\section{member (constraint)}\label{member:memberconstraint}\hypertarget{member:memberconstraint}{}

\begin{notedef}
  \texttt{member}$(x,s)$ states that integer $x$ is contained in set
  $s$:
$$x\in s.$$
\end{notedef}

\begin{itemize}
	\item \textbf{API} :
	\begin{itemize}
		\item \mylst{member(int x, SetVariable s)}
		\item \mylst{member(SetVariable s, int x)}
		\item \mylst{member(SetVariable s, IntegerVariable x)}
		\item \mylst{member(IntegerVariable x, SetVariable s)}
	\end{itemize}
	\item \textbf{return type} : \texttt{Constraint}
	\item \textbf{options} :\emph{n/a}
	\item \textbf{favorite domain} : \emph{to complete}
	\item \textbf{references} :\\
      global constraint catalog: \href{http://www.emn.fr/x-info/sdemasse/gccat/Cin_set.html}{in\_set}
\end{itemize}

\textbf{Example}:
\begin{lstlisting}
	Model m = new CPModel();
	Solver s = new CPSolver();
	int x = 3;
	int card = 2;
	SetVariable y = makeSetVar("y", 2, 4);
	m.addConstraint(member(y, x));
	m.addConstraint(eqCard(y, card));
	s.read(m);
	s.solveAll();
\end{lstlisting} 
