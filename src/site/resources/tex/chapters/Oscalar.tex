%\part{scalar}
\label{scalar}
\hypertarget{scalar}{}

\section{scalar (operator)}\label{scalar:scalaroperator}\hypertarget{scalar:scalaroperator}{}
Return an integer expression that corresponds to the scalar product of coefficients array and variables array (\(c_1*x_1+c_2*x_2+...+c_n*x_n\)).

\begin{itemize}
	\item \textbf{API} :
	\begin{itemize}
		\item scalar(int[] c, IntegerVariable[] x)
		\item scalar(IntegerVariable[] x, int[] c)
	\end{itemize}
	\item \textbf{return type} : IntegerExpressionVariable
	\item \textbf{options} : \emph{n/a}
	\item \textbf{favorite domain} : \emph{to complete}
\end{itemize}

\textbf{Example}:

\begin{lstlisting}
	Model m = new CPModel();
	Solver s = new CPSolver();
	
	IntegerVariable[] vars = new IntegerVariable[n * n];
	for (int i = 0; i < n; i++)
	      for (int j = 0; j < n; j++) {
	           vars[i * n + j] = makeIntVar("C" + i + "_" + j, 1, n * n);
	      }
	IntegerVariable sum = makeIntVar("S", 1, n * n * (n * n + 1) / 2);
	
	m.addConstraint(eq(sum, n * (n * n + 1) / 2));
	for (int i = 0; i < n * n; i++)
	    for (int j = 0; j < i; j++)
	        m.addConstraint(neq(vars[i], vars[j]));
	int[] coeffs = new int[n];
	for (int i = 0; i < n; i++) {
	    coeffs[i] = 1;
	}
	
	for (int i = 0; i < n; i++) {
	    IntegerVariable[] col = new IntegerVariable[n];    
	    IntegerVariable[] row = new IntegerVariable[n];
	    for (int j = 0; j < n; j++) {
	        col[j] = vars[i * n + j];
	        row[j] = vars[j * n + i];
	    } 
	    m.addConstraint(eq(scalar(coeffs, row), sum));
	    m.addConstraint(eq(scalar(coeffs, col), sum));
	}
	s.read(m);
	s.solve();
	//System.out.println("" + pretty());
	for (int i = 0; i < n; i++) {
	    for (int j = 0; j < n; j++) {   
	        System.out.print("" + s.getVar(vars[i * n + j]).getVal());
	        if (s.getVar(vars[i * n + j]).getVal() > 9) System.out.print(" ");
	           else System.out.print("  ");
	        }
	    System.out.println("");
	}
\end{lstlisting}
