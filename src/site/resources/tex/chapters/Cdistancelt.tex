%\part{distancelt}
\label{distancelt}
\hypertarget{distancelt}{}

\section{distanceLT (constraint)}\label{distancelt:distanceltconstraint}\hypertarget{distancelt:distanceltconstraint}{}
\begin{notedef}
  \texttt{distanceLT}$(x_1,x_2,x_3,c)$ states that $x_3$ plus an offset $c$ (by default $c=0$) is strictly smaller than the distance between $x_1$ and $x_2$:
$$ x_3 + c < | x_1 - x_2 |$$
\end{notedef}

\begin{itemize}
	\item \textbf{API} :
	\begin{itemize}
		\item \mylst{distanceLT(IntegerVariable x1, IntegerVariable x2, int x3)}
		\item \mylst{distanceLT(IntegerVariable x1, IntegerVariable x2, IntegerVariable x3)}
		\item \mylst{distanceLT(IntegerVariable x1, IntegerVariable x2, IntegerVariable x3, int c)}
	\end{itemize}
	\item \textbf{return type}: \texttt{Constraint}
	\item \textbf{options} : \emph{n/a}
	\item \textbf{favorite domain} : \emph{to complete}
\end{itemize}

\textbf{Example}:
\begin{lstlisting}
	Model m = new CPModel();
	Solver s = new CPSolver();
	IntegerVariable v0 = makeIntVar("v0", 0, 5);
	IntegerVariable v1 = makeIntVar("v1", 0, 5);
	IntegerVariable v2 = makeIntVar("v2", 0, 5);
	m.addConstraint(distanceLT(v0, v1, v2, 0));
	s.read(m);
	s.solveAll();
\end{lstlisting}
