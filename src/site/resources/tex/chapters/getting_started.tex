%\part{getting started}
\label{gettingstarted}
\hypertarget{gettingstarted}{}

\chapter{Getting started: welcome to Choco}\label{gettingstarted:gettingstarted:welcometochoco}\hypertarget{gettingstarted:gettingstarted:welcometochoco}{}
This introduction covers the basics of writing a program in Choco

Choco is a java library for constraint satisfaction problems (CSP), constraint programming (CP) and explanation-based constraint solving (e-CP). It is built on a event-based propagation mechanism with backtrackable structures. 

\section{Before starting}\label{gettingstarted:beforestarting}\hypertarget{gettingstarted:beforestarting}{}

Before doing anything, you have to be sure that 
\begin{itemize}
	\item you have at least \href{http://java.sun.com/javase/6/}{Java6} installed on your environment.
	\item you have a IDE (like \href{http://www.jetbrains.com/idea/}{IntelliJ IDEA} or \href{http://www.eclipse.org/}{Eclipse}).
\end{itemize}

To install Java6 or your IDE, please refer to its specific documentation. We now assume that you have the previously defined environment.

You need to create a \textbf{New Project...} on your favorite IDE (\href{http://www.jetbrains.com/idea/training/demos.html}{create a new project on IntelliJ}, \href{https://eclipse-tutorial.dev.java.net/eclipse-tutorial/part1.html}{create a new project on Eclipse}). Our project name is \emph{ChocoProgram}.
Create a new class, named \emph{MyFirstChocoProgram}, with a main method.
\begin{lstlisting}
	public class MyFirstChocoProgram {
	
	    public static void main(String[] args) {
	        
	    }
	}
\end{lstlisting}

\section{Download Choco}\label{gettingstarted:downloadchoco}\hypertarget{gettingstarted:downloadchoco}{}
Now, before doing anything else, you need to download the last stable version of Choco. See the \href{http://choco.emn.fr}{download page}. 
Once you have download choco-2.0.0.\emph{X}.jar (\emph{X} is the last stable version indicator), you need to add it to the classpath of your project.

Now you are ready to create you first Choco program.

If you want a short introduction on what is constraint programming, you can find some informations on the \hyperlink{doc:introduction}{introduction}. If you prefer start with an example, please refer to \hyperlink{introduction:myfirstchocoprogram}{introduction\#my\ first\ choco\ program}.
When you feel ready, solve your own problem! And if you need more tries, please take a look at the \hyperlink{exercises}{exercises}. %and \hyperlink{examples}{examples}. 

\chapter{First Example: Magic square}\label{gettingstarted:firstexample:magicsquare}\hypertarget{gettingstarted:firstexample:magicsquare}{}
A simple magic square of order 3 can be seen as the ``Hello world!'' program in Choco. First of all, we need to agree on the definition of a magic square of order 3.
\href{http://en.wikipedia.org/wiki/Magic_square}{Wikipedia} tells us that :
\begin{myquote}
A \textbf{magic square} of order $n$ is an arrangement of $n^2$ numbers, usually distinct integers, in a square, such that the $n$ numbers in all rows, all columns, and both diagonals sum to the same constant. A normal magic square contains the integers from 1 to $n^2$.
\end{myquote}

So we are going to solve a problem where unknows are cells value, knowing that each cell can take its value between 1 and $n^2$, is different from the others and columns, diagonals and rows are equal to the same constant M (which is equal to $n * (n^2 + 1) / 2$).

We have the definition, let see how to add some Choco in it.

\section{First, the model}\label{gettingstarted:first,themodel}\hypertarget{gettingstarted:first,themodel}{}
To define our problem, we need to create a Model object. As we want to solve our problem with constraint programming (of course, we do), we need to create a CPModel.
\begin{lstlisting}
	//constants of the problem:
	int n = 3;
	int M = n*(n*n+1)/2;
	
	// Our model
	Model m = new CPModel();
\end{lstlisting}
These objects require to import the following classes:
\begin{lstlisting}
	import choco.cp.model.CPModel;
	import choco.kernel.model.Model;
\end{lstlisting}

At the begining, our model is empty, no problem has been defined explicitly. A model is composed of variables and constraints, and constraints link variables to each others.

\begin{itemize}
\item 
\textbf{Variables}\label{gettingstarted:variables}\hypertarget{gettingstarted:variables}{}
A variable is an object defined by a name, a type and a domain. We know that our unknowns are cells of the magic square. So:
\mylst{IntegerVariable cell = Choco.makeIntVar("aCell", 1, n*n);}
which means that \emph{aCell} is an integer variable, and its domain is defined from \emph{1} to \emph{n*n}.
But we need $n^2$ variables, so the easiest way to define them is:
\begin{lstlisting}
	IntegerVariable[][] cells = new IntegerVariable[n][n];
	for(int i = 0; i < n; i++){
	   for(int j = 0; j < n; j++){
	      cells[i][j] = Choco.makeIntVar("cell"+j, 1, n*n); 
	      m.addVariables(cells[i][j]);
	   }
	}
\end{lstlisting}
This code requires to import the following classes:
\begin{lstlisting}
	import choco.kernel.model.variables.integer.IntegerVariable;
	import choco.Choco;
\end{lstlisting}
We add each variables to our model:
\mylst{m.addVariables(cells[i][j]);}\\
Now that our variables are defined, we have to define the constraints between variables.
\item
\textbf{Constraints over the rows}\label{gettingstarted:constraintsovertherows}\hypertarget{gettingstarted:constraintsovertherows}{}
The sum of ach rows is equal to a constant $M$.
So we need a sum operator and and equality constraint. The both are provides by the \texttt{Choco.java} class.
\begin{lstlisting}
	//Constraints
	// ... over rows
	Constraint[] rows = new Constraint[n];
	for(int i = 0; i < n; i++){
	   rows[i] = Choco.eq(Choco.sum(cells[i]), M);
	}
\end{lstlisting}
This part of code requires the following import:
\begin{lstlisting}
  import choco.kernel.model.constraints.Constraint;
\end{lstlisting}
After the creation of the constraints, we need to add them to the model:
\begin{lstlisting}
  m.addConstraints(rows);
\end{lstlisting}
\item
\textbf{Constraints over the columns}\label{gettingstarted:constraintsoverthecolumns}\hypertarget{gettingstarted:constraintsoverthecolumns}{}
Now, we need to declare the equality between the sum of each column and $M$.
But, the way we have declare our variables matrix does not allow us to deal easily with it in the column case. So we create the transposed matrix (a $90^o$ rotation of the matrix) of \emph{cells}.
\begin{note}
We do not introduce new variables. We just reorder the matrix to see the \emph{column point of view}.
\end{note}
\begin{lstlisting}
	//... over columns
	// first, get the columns, with a temporary array
	IntegerVariable[][] cellsDual = new IntegerVariable[n][n];
	for(int i = 0; i < n; i++){
	   for(int j = 0; j < n; j++){
	      cellsDual[i][j] = cells[j][i];
	   }
	}
\end{lstlisting}
Now, we can declare the constraints as before:
\begin{lstlisting}
	Constraint[] cols = new Constraint[n];
	for(int i = 0; i < n; i++){
	   cols[i] = Choco.eq(Choco.sum(cellsDual[i]), M);
	}
\end{lstlisting}
And we add them to the model:
\begin{lstlisting}
  m.addConstraints(cols);
\end{lstlisting}
\item
\textbf{Constraints over the diagonals}\label{gettingstarted:constraintsoverthediagonals}\hypertarget{gettingstarted:constraintsoverthediagonals}{}
Now, we get the two diagonals array \emph{diags}, reordering the required \emph{cells} variables, like in the previous step.
\begin{lstlisting}
	//... over diagonals                                  
	IntegerVariable[][] diags = new IntegerVariable[2][n];
	for(int i = 0; i < n; i++){                           
	    diags[0][i] = cells[i][i];                        
	    diags[1][i] = cells[i][(n-1)-i];                  
	}
\end{lstlisting} 
And we add the constraints to the model (in one step this time).
\begin{lstlisting}
	m.addConstraint(Choco.eq(Choco.sum(diags[0]), M));    
	m.addConstraint(Choco.eq(Choco.sum(diags[1]), M));
\end{lstlisting}
\item
\textbf{Constraints of variables AllDifferent}\label{gettingstarted:constraintsofvariablesalldifferent}\hypertarget{gettingstarted:constraintsofvariablesalldifferent}{}
Finally, we add the AllDifferent constraints, stating that each \emph{cells} variables takes a unique value. 
One more time, we have to reorder the variables, introducing temporary array.
\begin{lstlisting}
	//All cells are differents from each other           
	IntegerVariable[] allVars = new IntegerVariable[n*n];
	for(int i = 0; i < n; i++){                          
	    for(int j = 0; j < n; j++){                      
	        allVars[i*n+j] = cells[i][j];                
	    }                                                
	}                                                    
	m.addConstraint(Choco.allDifferent(allVars));
\end{lstlisting}
\end{itemize}

\section{Then, the solver}\label{gettingstarted:then,thesolver}\hypertarget{gettingstarted:then,thesolver}{}
Our model is established, it does not require any other information, we can focus on the way to solve it.
The first step is to create a Solver;
\begin{lstlisting}
	//Our solver              
	Solver s = new CPSolver();
\end{lstlisting}
This part requires the following imports:
\begin{lstlisting}
	import choco.kernel.solver.Solver;
	import choco.cp.solver.CPSolver;
\end{lstlisting}

After that, the model and the solver have to be linked, thus the solver \emph{read} the model, to extract informations:
\begin{lstlisting}
	//read the model
	s.read(m);
\end{lstlisting}

Once it is done, we just need to solve it:
\begin{lstlisting}
	//solve the problem
	s.solve();
\end{lstlisting}
And print the information
\begin{lstlisting}
	//Print the values                                           
	for(int i = 0; i < n; i++){                                  
	    for(int j = 0; j < n; j++){                              
	        System.out.print(s.getVar(cells[i][j]).getVal()+" ");
	    }                                                        
	    System.out.println();                                    
	}
\end{lstlisting}

\section{Conclusion}\label{gettingstarted:conclusion}\hypertarget{gettingstarted:conclusion}{}
We have seen, in a few steps, how to solve a basic problem using constraint programming and Choco. Now, you are ready to solve your own problem, and if you need more tries, please take a look at the \hyperlink{exercises}{exercises}. %and \hyperlink{examples}{examples}. 
You can download the java file of the introduction: \href{media/zip/myfirstchocoprogram.zip}{myfirstchocoprogram.zip}
