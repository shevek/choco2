\label{disjoint}
\hypertarget{disjoint}{}

\section{disjoint (constraint)}\label{disjoint:disjointconstraint}\hypertarget{disjoint:disjointconstraint}{}

\subsection{disjoint values}\label{disjoint:disjointvalues}\hypertarget{disjoint:disjointvalues}{}

\begin{notedef}
  \texttt{disjoint}$(x, V)$ states that the variable $x$ takes its value out of $V$:
 $$x \cup V = \emptyset$$  
\end{notedef}

\begin{itemize}
	\item \textbf{API}: \mylst{disjoint(IntegerVariable x, int[] v)}
	\item \textbf{return type}: \texttt{Constraint}
\end{itemize}

\textbf{Example}:
\lstinputlisting{java/cdisjoint1.j2t}

\subsection{disjoint tasks}\label{disjoint:disjointtasks}\hypertarget{disjoint:disjointtasks}{}
\begin{notedef}
  \texttt{disjoint}$(t1, t2)$ states that each tasks $t_1$ should not overlap anu tasks $t_2$:
 $$x \cup V = \emptyset$$  
\end{notedef}

CHOCO only provides a decomposition with reified precedences because the coloured cumulative is not available.

\begin{itemize}
	\item \textbf{API}: \mylst{disjoint(TaskVariable[] t1, TaskVariable[] t2)}
	\item \textbf{return type}: \texttt{Constraint[]}
	\item \textbf{favorite domain} : \emph{n/a}.
	\item \textbf{references} :
      \begin{itemize}
      \item global constraint catalog: \href{http://www.emn.fr/x-info/sdemasse/gccat/Cdisjoint_tasks.html}{\tt disjoint\_tasks}
      \end{itemize}
\end{itemize}
\textbf{Example}:
\lstinputlisting{java/cdisjoint2.j2t}
