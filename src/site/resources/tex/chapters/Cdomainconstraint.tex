%\part{domainconstraint}
\label{domainconstraint}
\hypertarget{domainconstraint}{}

\section{domainconstraint (constraint)}\label{domainconstraint:domainconstraintconstraint}\hypertarget{domainconstraint:domainconstraintconstraint}{}
\begin{notedef}  
\texttt{domainConstraint}$(bVar,values)$ states that $values[i]$ is equal to 1 if and only if $bVar$ is equal to $i$ (0 otherwise):
$$values[i]=1\quad\iff\quad (bVar=i)$$ 
\end{notedef}

It makes the link between a domain variable $bVar$ and those 0-1 variables that are associated with each potential value of $bVar$: the 0-1 variable associated with the value that is taken by variable $bVar$ is equal to 1, while the remaining 0-1 variables are all equal to 0.

\begin{itemize}
	\item \textbf{API} : \mylst{domainConstraint(IntegerVariable bVar, IntegerVariable[] values)}
	\item \textbf{return type} : \texttt{Constraint}
	\item \textbf{options} : \emph{n/a}
	\item \textbf{favorite domain} : enumerated for $bVar$
	\item \textbf{references} :\\
	  global constraint catalog: \href{http://www.emn.fr/x-info/sdemasse/gccat/Cdomain_constraint.html}{\tt domainConstraint}
\end{itemize}

\textbf{Example}:
\lstinputlisting{java/cdomainconstraint.j2t}
