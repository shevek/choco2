%\part{setdisjoint}
\label{setdisjoint}
\hypertarget{setdisjoint}{}

\section{setDisjoint (constraint)}\label{setdisjoint:setdisjointconstraint}\hypertarget{setdisjoint:setdisjointconstraint}{}
\begin{notedef}
  \texttt{setDisjoint}$(s_1,s_2)$ states that the two set arguments are disjoint:
$$s_1\cap s_2=\emptyset$$
\end{notedef}

\begin{itemize}
	\item \textbf{API} : \mylst{setDisjoint(SetVariable s1, SetVariable s2)}
	\item \textbf{return type} : \texttt{Constraint}
	\item \textbf{options} :\emph{n/a}
	\item \textbf{favorite domain} : \emph{to complete}
\end{itemize}

\textbf{Example}:
\lstinputlisting{java/csetdisjoint.j2t}
