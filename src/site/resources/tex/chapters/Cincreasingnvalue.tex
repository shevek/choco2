%\part{increasingnvalue}
\label{increasingnvalue}
\hypertarget{increasingnvalue}{}

\section{increasingnvalue (constraint)}\label{increasingnvalue:increasingnvalueconstraint}\hypertarget{increasingnvalue:increasingnvalueconstraint}{}
\begin{notedef}
  \texttt{increasing\_nvalue}$(nval, variables)$ states that $variables$ are increasing. In addition, $nval$ is the number of distinct values taken by $variables$.
\end{notedef}

\begin{itemize}
	\item \textbf{API} : \mylst{increasing\_nvalue(IntegerVariable nval, IntegerVariable[] variables)}
	\item \textbf{return type} : \texttt{Constraint}
	\item \textbf{options} :
	\begin{itemize}
		\item \emph{no option} filter on lower bound and on lower bound
		\item \texttt{cp:atleast} filter on lower bound only
		\item \texttt{cp:atmost} filter on upper bound only
		\item \texttt{cp:both} \textit{--default value--} filter on lower bound and on upper bound
	\end{itemize}
	\item \textbf{favorite domain} : \emph{to complete}
	\item \textbf{references} :\\
      global constraint catalog: \href{http://www.emn.fr/x-info/sdemasse/gccat/Cincreasing_nvalue.html}{increasing\_nvalue}
\end{itemize}

\textbf{Example}:
\lstinputlisting{java/cincreasingnvalue.j2t} 
