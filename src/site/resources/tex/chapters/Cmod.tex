%\part{mod}
\label{mod}
\hypertarget{mod}{}

\section{mod (constraint)}\label{mod:modconstraint}\hypertarget{mod:modconstraint}{}
\begin{notedef}
  \texttt{mod}$(x_1,x_2,x_3)$ states that $x_1$ is congruent to $x_2$
  modulo $x_3$:
$$x_1 \equiv x_2 \mod x_3$$
\end{notedef}
\begin{itemize}
	\item \textbf{API} : \mylst{mod(IntegerVariable x1, IntegerVariable x2, int x3)}
	\item \textbf{return type} : \texttt{Constraint}
	\item \textbf{options} : \emph{n/a}
	\item \textbf{favorite domain} : \emph{n/a}
\end{itemize}

\textbf{Example}:
\begin{lstlisting}
	Model m = new CPModel();
	Solver s = new CPSolver();
	
	IntegerVariable x = makeIntVar("x", 0, 10);
	IntegerVariable w = makeIntVar("w", 0, 10);
	
	m.addConstraint(mod(w,x, 1));
	
	s.read(m);
	s.solve();
\end{lstlisting}
