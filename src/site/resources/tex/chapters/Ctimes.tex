%\part{times}
\label{times}
\hypertarget{times}{}

\section{times (constraint)}\label{times:timesconstraint}\hypertarget{times:timesconstraint}{}
\begin{notedef}
  \texttt{times}$(x_1, x_2, x_3)$ states that the third argument is equal to the product of the two arguments:
$$x_3=x_1\times x_2.$$
\end{notedef}

\begin{itemize}
	\item \textbf{API}:
	\begin{itemize}
		\item \mylst{times(IntegerVariable x1, IntegerVariable x2, IntegerVariable x3)}
		\item \mylst{times(int x1, IntegerVariable x2, IntegerVariable x3)}
		\item \mylst{times(IntegerVariable x1, int x2, IntegerVariable x3)}
	\end{itemize}
	\item \textbf{return type} : \texttt{Constraint}
	\item \textbf{option} : \emph{n/a}
	\item \textbf{favorite domain}: bound
\end{itemize}

\textbf{Example}:
\begin{lstlisting}
	Model m = new CPModel();
	IntegerVariable x = makeIntVar("x", 1, 2);
	IntegerVariable y = makeIntVar("y", 3, 5);
	IntegerVariable z = makeIntVar("z", 3, 10);
	m.addConstraint(times(x, y, z));
	s.setVarIntSelector(new RandomIntVarSelector(s, i));
	s.setValIntSelector(new RandomIntValSelector(i + 1));
	s.read(m);
	s.solve();
\end{lstlisting}

