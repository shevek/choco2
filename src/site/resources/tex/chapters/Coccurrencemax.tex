%\part{occurrencemax}
\label{occurrencemax}
\hypertarget{occurrencemax}{}

\section{occurrenceMax (constraint)}\label{occurrencemax:occurrencemaxconstraint}\hypertarget{occurrencemax:occurrencemaxconstraint}{}
\begin{notedef}
  \texttt{occurrenceMax}$(v,z,x)$ states that $z$ is at most equal to the number of elements in $x$ with value $v$:
$$z\le|\{i\ |\ x_i=v\}|$$   
\end{notedef}
  This is a specialization of the \texttt{globalCardinality} constraint.

\begin{itemize}
	\item \textbf{API}: \mylst{occurrenceMax(int v, IntegerVariable z, IntegerVariable... x)}
	\item \textbf{return type} : \texttt{Constraint}
	\item \textbf{options} :\emph{n/a}
	\item \textbf{favorite domain} : \emph{to complete}
	\item \textbf{references} :\\
      global constraint catalog: \href{http://www.emn.fr/x-info/sdemasse/gccat/Ccount.html}{count}
\end{itemize}

\textbf{Example}:
\begin{lstlisting}
	Model m = new CPModel();
	Solver s = new CPSolver();
	 
	IntegerVariable x1 = makeIntVar("X1", 0, 10);
	IntegerVariable x2 = makeIntVar("X2", 0, 10);
	IntegerVariable x3 = makeIntVar("X3", 0, 10);
	IntegerVariable x4 = makeIntVar("X4", 0, 10);
	IntegerVariable x5 = makeIntVar("X5", 0, 10);
	IntegerVariable x6 = makeIntVar("X6", 0, 10);
	IntegerVariable x7 = makeIntVar("X7", 0, 10);
	IntegerVariable y1 = makeIntVar("Y1", 0, 10);
	 
	m.addConstraint(occurrenceMax(3, y1, new IntegerVariable[]{x1, x2, x3, x4, x5, x6, x7}));
	 
	s.read(m);
	s.solve();
\end{lstlisting} 
