%\part{neq}
\label{neq}
\hypertarget{neq}{}

\section{neq (constraint)}\label{neq:neqconstraint}\hypertarget{neq:neqconstraint}{}

\begin{notedef}
  \texttt{neq} states that the two arguments are different:
$$x \neq y.$$
\end{notedef}
\begin{itemize}
	\item \textbf{API} :
	\begin{itemize}
		\item \mylst{neq(IntegerExpressionVariable x, IntegerExpressionVariable y)}
		\item \mylst{neq(IntegerExpressionVariable x, int y)}
		\item \mylst{neq(int x, IntegerExpressionVariable y)}
	\end{itemize}
	\item \textbf{return type} : \texttt{Constraint}
	\item \textbf{options} : \emph{n/a}
	\item \textbf{favorite domain} : \emph{to complete}.
	\item \textbf{references} :\\
      global constraint catalog: \href{http://www.emn.fr/x-info/sdemasse/gccat/Cneq.html}{neq}
\end{itemize}

\textbf{Examples:}
\begin{itemize}
	\item example1:
\end{itemize}

\begin{lstlisting}
	Model m = new CPModel();
	Solver s = new CPSolver();
	int c = 1;
	IntegerVariable v = makeIntVar("v", 0, 2);
	m.addConstraint(neq(v, c));
	s.read(m);
	s.solve();
\end{lstlisting}
\begin{itemize}
	\item example2
\end{itemize}

\begin{lstlisting}
	Model m = new CPModel();
	Solver s = new CPSolver();
	IntegerVariable v1 = makeIntVar("v1", 0, 2);
	IntegerVariable v2 = makeIntVar("v2", 0, 2);
	IntegerExpressionVariable w1 = plus(v1, 1);
	IntegerExpressionVariable w2 = minus(v2, 1);
	m.addConstraint(neq(w1, w2));
	s.read(m);
	s.solve();
\end{lstlisting}
