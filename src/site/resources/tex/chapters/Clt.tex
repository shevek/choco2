%\part{lt}
\label{lt}
\hypertarget{lt}{}

\section{lt (constraint)}\label{lt:ltconstraint}\hypertarget{lt:ltconstraint}{}
\begin{notedef}
  \texttt{lt}$(x,y)$ states that $x$ is strictly smaller than $y$:
$$x<y$$
\end{notedef}

\begin{itemize}
	\item \textbf{API} :
	\begin{itemize}
		\item \mylst{lt(IntegerExpressionVariable x, IntegerExpressionVariable y)}
		\item \mylst{lt(IntegerExpressionVariable x, int y)}
		\item \mylst{lt(int x, IntegerExpressionVariable y)}
	\end{itemize}
	\item \textbf{return type} : \texttt{Constraint}
	\item \textbf{options} : \emph{n/a}
	\item \textbf{favorite domain} : \emph{to complete}.
	\item \textbf{references} :\\
      global constraint catalog: \href{http://www.emn.fr/x-info/sdemasse/gccat/Clt.html}{lt}
\end{itemize}

\textbf{Example:}
\begin{lstlisting}
	Model m = new CPModel();
	Solver s = new CPSolver();
	int c = 1;
	IntegerVariable v = makeIntVar("v", 0, 2);
	m.addConstraint(lt(v, c));
	s.read(m);
	s.solve();
\end{lstlisting}
