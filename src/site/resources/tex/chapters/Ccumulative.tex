%\part{cumulative}
\label{cumulative}
\hypertarget{cumulative}{}

\section{cumulative (constraint)}\label{cumulative:cumulativeconstraint}\hypertarget{cumulative:cumulativeconstraint}{}
\begin{notedef}
  \texttt{cumulative(start,duration,height,capacity)} states that a set of tasks (defined by their starting times, finishing dates, durations and heights (or consumptions)) are executed on a cumulative resource of limited capacity. That is, the total height of the tasks which are executed at any time $t$ does not exceed the capacity of the resource:
$$\sum_{\{i\ |\ \mathtt{start}[i]\le t < \mathtt{start}[i]+\mathtt{duration}[i]\}} \mathtt{height}[i] \le \mathtt{capacity},\quad (\forall \text{ time } t)$$
\end{notedef}

The notion of task does not exist yet in Choco. The \texttt{cumulative} takes therefore as input, several arrays of integer variables (of same size $n$) denoting the starting, duration, and height of each task. When the array of finishing times is also specified, the constraint ensures that \texttt{start[i] + duration[i] = end[i]} for all task $i$.
As usual, a task is executed in the interval \texttt{[start,end-1]}.

A tutorial on the use of this constraint is available \hyperlink{schedulinganduseofthecumulative:schedulinganduseofthecumulativeconstraint}{here}

\begin{itemize}
	\item \textbf{API} :
	\begin{itemize}
		\item \mylst{cumulative(IntegerVariable[] start, IntegerVariable[] end, IntegerVariable[] duration, IntegerVariable[] height, IntegerVariable capa, String... options)}
		\item \mylst{cumulative(IntegerVariable[] start, IntegerVariable[] end, IntegerVariable[] duration, int[] height, int capa, String... options)}
		\item \mylst{cumulative(IntegerVariable[] start, IntegerVariable[] duration, IntegerVariable[] height, IntegerVariable capa, String... options)}
	\end{itemize}
	\item \textbf{return type} : \texttt{Constraint}
	\item \textbf{options} :\emph{n/a}
	\item \textbf{favorite domain} : \emph{to complete}
	\item \textbf{references} :
      \begin{itemize}
      \item  \cite{BeldiceanuCP02} \emph{A new multi-resource cumulatives constraint with negative heights}
      \item global constraint catalog: \href{http://www.emn.fr/x-info/sdemasse/gccat/Ccumulative.html}{\tt cumulative}
      \end{itemize}
\end{itemize}

\begin{lstlisting}
  CPModel m = new CPModel();
	
  // data
  int n = 11 + 3; //number of tasks (include the three fake tasks)
  int[] heights_data = new int[]{2, 1, 4, 2, 3, 1, 5, 6, 2, 1, 3, 1, 1, 2};
  int[] durations_data = new int[]{1, 1, 1, 2, 1, 3, 1, 1, 3, 4, 2, 3, 1, 1};

  // variables
  IntegerVariable capa = constant(7);
  IntegerVariable[] starts = makeIntVarArray("start", n, 0, 5, "cp:bound");
  IntegerVariable[] ends = makeIntVarArray("end", n, 0, 6, "cp:bound");
  IntegerVariable[] duration = new IntegerVariable[n];
  IntegerVariable[] height = new IntegerVariable[n];
  for (int i = 0; i < height.length; i++) {
      duration[i] = constant(durations_data[i]);
      height[i] = makeIntVar("height " + i, new int[]{0, heights_data[i]});
  }
  IntegerVariable[] bool = makeIntVarArray("taskIn?", n, 0, 1);
  IntegerVariable obj = makeIntVar("obj", 0, n, "cp:bound", "cp:objective");
	
  //post the cumulative
  m.addConstraint(cumulative(starts, ends, duration, height, capa, ""));
	
  //post the channeling to know if the task is scheduled or not
  for (int i = 0; i < n; i++) {
      m.addConstraint(boolChanneling(bool[i], height[i], heights_data[i]));
  }

  //state the objective function
  m.addConstraint(eq(sum(bool), obj));
	
  CPSolver s = new CPSolver();
  s.read(m);
	
  //set the fake tasks to establish the profile capacity of the ressource
  try {
      s.getVar(starts[0]).setVal(1); s.getVar(ends[0]).setVal(2); s.getVar(height[0]).setVal(2);
      s.getVar(starts[1]).setVal(2); s.getVar(ends[1]).setVal(3); s.getVar(height[1]).setVal(1); 
      s.getVar(starts[2]).setVal(3); s.getVar(ends[2]).setVal(4); s.getVar(height[2]).setVal(4);
  } catch (ContradictionException e) {
      System.out.println("error, no contradiction expected at this stage");
  }
  // maximize the number of tasks placed in this profile	
  s.maximize(s.getVar(obj),false);
  System.out.println("Objective : " + (s.getVar(obj).getVal() - 3));
  for (int i = 3; i < starts.length; i++) {
      if (s.getVar(height[i]).getVal() != 0)
      System.out.println("[" + s.getVar(starts[i]).getVal() + " - " 
                             + (s.getVar(ends[i]).getVal() - 1) + "]:"
                             + s.getVar(height[i]).getVal());
  }
\end{lstlisting} 
