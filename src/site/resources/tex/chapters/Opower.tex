%\part{power}
\label{power}
\hypertarget{power}{}

\section{power (operator)}\label{power:poweroperator}\hypertarget{power:poweroperator}{}
Returns an expression variable that represents the first argument raised to the power of the second argument (\(x^y\)).

\begin{itemize}
	\item \textbf{API} :
	\begin{itemize}
		\item \mylst{power(IntegerExpressionVariable x, IntegerExpressionVariable y)}
		\item \mylst{power(int x, IntegerExpressionVariable y)}
		\item \mylst{power(IntegerExpressionVariable x, int y)}
		\item \mylst{power(RealExpressionVariable x, int y)}
	\end{itemize}
	\item \textbf{return type}:
	\begin{itemize}
		\item \texttt{IntegerExpressionVariable}, if parameters are \texttt{IntegerExpressionVariable}
		\item \texttt{RealExpressionVariable}, if parameters are \texttt{RealExpressionVariable}
	\end{itemize}
	\item \textbf{option} : \emph{n/a}
	\item \textbf{favorite domain} : \emph{to complete}
\end{itemize}

\textbf{Example} : 
\begin{lstlisting}
	Model m = new CPModel();
	Solver s = new CPSolver();
	IntegerVariable x = makeIntVar("x", 0, 10);
	IntegerVariable y = makeIntVar("y", 2, 4);
	IntegerVariable z = makeIntVar("z", 28, 80);
	m.addConstraint(eq(z, power(x, y)));
	s.read(m);
	s.solve();
\end{lstlisting}
