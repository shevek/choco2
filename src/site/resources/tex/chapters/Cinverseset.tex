%\part{inversechanneling}
\label{inverseset}
\hypertarget{inverseset}{}

\section{inverseset (constraint)}\label{inverseset:inversesetconstraint}\hypertarget{inverseset:inversesetconstraint}{}
\begin{notedef}
  \texttt{inverseset}$(iv,sv)$ states a channeling between an array  $iv$ of integer variables and an array $sv$ of set variables. It enforces that value $j$ belongs to $sv[i]$ if and only if $iv[j]$ is equal to $i$ and conversely:
  %It enforces that if the $i$-th element of $x$ is equal to $j$ then the $j$-th element of $y$ is equal to $i$ and conversely:
$$sv_i = j\quad\iff\quad iv_i = j$$
\end{notedef}
\begin{itemize}
	\item \textbf{API} : \mylst{inverseSet(IntegerVariable[] iv, SetVariable[] sv)}
	\item \textbf{return type} : \texttt{Constraint}
	\item \textbf{options} : \emph{no options}
	\item \textbf{favorite domain} : enumerated for iv
	\item \textbf{references} :\\
      global constraint catalog: \href{http://www.emn.fr/x-info/sdemasse/gccat/Cinverse_set.html}{inverse\_set}
\end{itemize}

\textbf{Example}:
\lstinputlisting{java/cinverseset.j2t}
