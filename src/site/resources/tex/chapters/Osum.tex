%\part{sum}
\label{sum}
\hypertarget{sum}{}

\section{sum (operator)}\label{sum:sumoperator}\hypertarget{sum:sumoperator}{}
Return an integer expression that corresponds to the sum of the variables given in argument (\(x_1+x_2+...+x_n\)).

\begin{itemize}
	\item \textbf{API}: sum(IntegerVariable... lv)
	\item \textbf{return type} : IntegerExpressionVariable
	\item \textbf{options} : \emph{n/a}
	\item \textbf{favorite domain} : \emph{to complete}
\end{itemize}

\textbf{Example} :
\begin{lstlisting}
	CPModel pb = new CPModel();
	IntegerVariable[] vs = new IntegerVariable[n];
	for (int i = 0; i < n; i++) {
	     vs[i] = makeIntVar("" + i, 0, n - 1);
	}
	for (int i = 0; i < n; i++) {
	    pb.addConstraint(occurrence(i, vs[i], vs));
	}
	pb.addConstraint(eq(sum(vs), n));     // contrainte redondante 1
	int[] coeff2 = new int[n - 1];
	IntegerVariable[] vs2 = new IntegerVariable[n - 1];
	for (int i = 1; i < n; i++) {
	    coeff2[i - 1] = i;
	    vs2[i - 1] = vs[i];
	}
	pb.addConstraint(eq(scalar(coeff2, vs2), n)); // contrainte redondante 2
	s.read(pb);
	s.solve();
	do {
	    for (int i = 0; i < vs.length; i++) {
	        System.out.print(s.getVar(vs[i]).getVal() + " ");
	    }
	    System.out.println("");
	} while (s.nextSolution() == Boolean.TRUE);
\end{lstlisting}
