%\part{globalcardinality}
\label{globalcardinality}
\hypertarget{globalcardinality}{}

\section{globalCardinality (constraint)}\label{globalcardinality:globalcardinalityconstraint}\hypertarget{globalcardinality:globalcardinalityconstraint}{}
\begin{notedef}
  \texttt{globalCardinality}$(\collec{x_1}{x_n},\collec{l_0}{l_{M-m}}, \collec{u_0}{u_{M-m}}, m)$ states lower bounds $l$ and upper bounds $u$ on the occurrence numbers of the values in collection $x$ according to offset $m$: 
$$l_{j-m}\ \le\ |\{i=1..n\ ,\ x_i=j\}|\ \le\ u_{j-m},\quad\forall j=m..M$$   
  \texttt{globalCardinality}$(\collec{x_1}{x_n},\collec{z_0}{z_{M-m}}, o)$ states that $z$ are the occurrence numbers of the values in collection $x$ according to offset $m$: 
$$z_{j-m} = |\{i=1..n\ ,\ x_i=j\}|,\quad\forall j=m..M$$   
\end{notedef}
Note that the length of the bound tables $l$ and $u$ are equal to $M+m-1$ where $m$, the offset, is the minimum counted value and $M$ is the maximum counted value.

%offset is the minimum value over all variables in $x$
Several APIs exist:
\begin{itemize}
	\item \emph{constant bounds on cardinalities} \collec{l_0}{l_{M-m}} and \collec{u_0}{u_{M-m}}: use the propagator of \cite{ReginAAAI96} or of \cite{QuimperCP03} depending on the set options and the nature of the domain variables.
	\item \emph{variable cardinalities} \collec{z_0}{z_{M-m}}: use the propagator of \cite{QuimperCP03} that:      
      \begin{itemize}
      \item enforces Bound Consistency over $x$ regarding the lower and upper bounds of $z$, 
      \item maintains the upper bound of $z_j$ by counting the variables that may be instantiated to $j$, 
      \item maintains the lower bound of $z_j$ by counting the variables instantiated to $j$, 
      \item enforces $z_0 + \cdots + z_{M-m} = n$
      \end{itemize}
\end{itemize}

The APIs are duplicated to define options. 

\begin{itemize}
	\item \textbf{API} :
      \begin{itemize}
	\item \mylst{globalCardinality(IntegerVariable[] x, int[] low, int[] up, int offset)}
	\item \mylst{globalCardinality(String options, IntegerVariable[] x, int[] low, int[] up, int offset)}
	\item \mylst{globalCardinality(IntegerVariable[] x, IntegerVariable[] card, int offset)}
      \end{itemize}
	\item \textbf{return type} : \texttt{Constraint}
	\item \textbf{options}:
	\begin{itemize}
		\item \emph{no option}: 
          if $x$ have \emph{bounded} domains or if the cardinalities are variable $z$, use the propagator of~\cite{QuimperCP03} for BC, otherwise use the propagator of \cite{ReginAAAI96};
		\item \hyperlink{cgccac:cgccacoptions}{\tt Options.C\_GCC\_AC} : for \cite{ReginAAAI96} implementation of arc consistency
		\item \hyperlink{cgccbc:cgccbcoptions}{\tt Options.C\_GCC\_BC} : for  \cite{QuimperCP03} implementation of bound consistency
	\end{itemize}
	\item \textbf{favorite domain} : \emph{enumerated} for arc consistency, \emph{bounded} for bound consistency.
	\item \textbf{references} :
      \begin{itemize}
      \item \cite{ReginAAAI96}: \emph{Generalized arc consistency for global cardinality constraint},
      \item \cite{QuimperCP03}: \emph{An efficient bounds consistency algorithm for the global cardinality constraint}
      \item global constraint catalog: \href{http://www.emn.fr/x-info/sdemasse/gccat/Cglobal_cardinality.html}{global\_cardinality}
      \end{itemize}
\end{itemize}

\textbf{Examples:}
\begin{itemize}
	\item example1:
\end{itemize}

\lstinputlisting{java/cglobalcardinality1.j2t}

\begin{itemize}
	\item example2:
\end{itemize}

\lstinputlisting{java/cglobalcardinality2.j2t}
