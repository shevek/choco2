%\part{infeastupleac}
\label{infeastupleac}
\hypertarget{infeastupleac}{}

\section{infeasTupleAC (constraint)}\label{infeastupleac:infeastupleacconstraint}\hypertarget{infeastupleac:infeastupleacconstraint}{}
\begin{notedef}
  \texttt{infeasTupleAC}$(\collec{x_1}{x_n},feasTuples)$ states an extensional constraint on \collec{x_1}{x_n} defined by the table $infeasTuples$ of compatible tuples of values, and then enforces arc consistency:
      $$\forall \text{ tuple } i\ |\quad \collec{x_1}{x_n}\neq infeasTuples[i]$$
\end{notedef}

The API is duplicated to define options.
\begin{itemize}
	\item \textbf{API} :
	\begin{itemize}
		\item \mylst{infeasTupleAC(List<int[]> infeasTuples, IntegerVariable... x)}
		\item \mylst{infeasTupleAC(String options, List<int[]> infeasTuples, IntegerVariable... x)}
	\end{itemize}
	\item \textbf{return type}: \texttt{Constraint}
	\item \textbf{options} :
	\begin{itemize}
		\item \emph{no option}: use AC32 (default arc consistency)
		\item \hyperlink{cext32:cext32options}{\tt Options.C\_EXT\_AC32}: to get AC3rm algorithm (maintaining the current support of each value in a non backtrackable way)
		\item \hyperlink{cext2001:cext2001options}{\tt Options.C\_EXT\_AC2001}: to get AC2001 algorithm (maintaining the current support of each value)
		\item \hyperlink{cext2008:cext2008options}{\tt Options.C\_EXT\_AC2008}: to get AC2008 algorithm (maintained by STR)
	\end{itemize}
	\item \textbf{favorite domain} : \emph{to complete}
\end{itemize}

\textbf{Example}:
\lstinputlisting{java/cinfeastupleac.j2t}
