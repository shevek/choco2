%\part{feastupleac}
\label{feastupleac}
\hypertarget{feastupleac}{}

\section{feasTupleAC (constraint)}\label{feastupleac:feastupleacconstraint}\hypertarget{feastupleac:feastupleacconstraint}{}
\begin{notedef}
  \texttt{feasTupleAC}$(x,feasTuples)$ states an extensional constraint on $(x_1,\ldots,x_n)$ defined by the table $feasTuples$ of compatible tuples of values, and then enforces arc consistency:
      $$\exists \text{ tuple } i\ |\quad (x_1,\ldots,x_n)=feasTuples[i]$$
\end{notedef}

The API is duplicated to define options.
\begin{itemize}
	\item \textbf{API} :
	\begin{itemize}
		\item \mylst{feasTupleAC(List<int[]> feasTuples, IntegerVariable... x)}
		\item \mylst{feasTupleAC(String options, List<int[]> feasTuples, IntegerVariable... x)}
	\end{itemize}
	\item \textbf{return type}: \texttt{Constraint}
	\item \textbf{options} :
	\begin{itemize}
		\item \emph{no option}: use AC32 (default arc consistency)
		\item \hyperlink{cext32:cext32options}{\tt Options.C\_EXT\_AC32}: to get AC3rm algorithm (maintaining the current support of each value in a non backtrackable way)
		\item \hyperlink{cext2001:cext2001options}{\tt Options.C\_EXT\_AC2001}: to get AC2001 algorithm (maintaining the current support of each value)
		\item \hyperlink{cext2008:cext2008options}{\tt Options.C\_EXT\_AC2008}: to get AC2008 algorithm (maintained by STR)
	\end{itemize}
	\item \textbf{favorite domain} : \emph{to complete}
	\item \textbf{references} :\\
      global constraint catalog: \href{http://www.emn.fr/x-info/sdemasse/gccat/Cin_relation.html}{in\_relation}
\end{itemize}

\textbf{Example}:
\lstinputlisting{java/cfeastupleac.j2t}
