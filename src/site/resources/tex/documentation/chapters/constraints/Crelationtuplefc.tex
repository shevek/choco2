%\part{relationtuplefc}
\label{relationtuplefc}
\hypertarget{relationtuplefc}{}

\section{relationTupleFC (constraint)}\label{relationtuplefc:relationtuplefcconstraint}\hypertarget{relationtuplefc:relationtuplefcconstraint}{}
\begin{notedef}
  \texttt{relationTupleFC}$(x,rel)$ states an extensional constraint on $(x_1,\ldots,x_n)$ defined by the $n$-ary relation $rel$, and then enforces forward checking:
$$(x_1,\ldots,x_n)\in rel$$
\end{notedef}
Many constraints of the same kind often appear in a model. Relations can therefore often be shared among many constraints to spare memory.

\begin{itemize}
	\item \textbf{API}: \mylst{relationTupleFC(IntegerVariable[] x, LargeRelation rel)}
	\item \textbf{return type}: \texttt{Constraint}
	\item \textbf{options} : \emph{n/a}
	\item \textbf{favorite domain} : \emph{to complete}
\end{itemize}

\textbf{Example} :
\lstinputlisting{java/cnotallequal.j2t}
\lstinputlisting{java/crelationtuplefc.j2t}
