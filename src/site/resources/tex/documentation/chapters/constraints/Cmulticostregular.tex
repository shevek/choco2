%\part{multicostregular}
\label{multicostregular}
\hypertarget{multicostregular}{}

\section{multiCostRegular (constraint)}\label{multicostregular:multicostregularconstraint}\hypertarget{multicostregular:multicostregularconstraint}{}
\begin{notedef}
  \texttt{multiCostRegular}$(\collec{z_1}{z_p}, \collec{x_1}{x_n},\mathcal{L}(\Pi), \coll{c_{i,j,k}})$ states that sequence \collec{x_1}{x_n} is a word belonging to the regular language $\mathcal{L}(\Pi)$ and that each $z_k$ is its cost computed as the sum of the individual symbol weights $c_{i,x_i,k}$: 
$$ \collec{x_1}{x_n} \in \mathcal{L}(\Pi)\quad\land\quad \sum_{i=1}^n c_{i,x_i,k} = z_k,\ \forall k=1..p.$$ 
\end{notedef}
\texttt{multiCostRegular} models the conjunction of $p$ \hyperlink{costregular}{\texttt{costRegular}} constraints, or the conjunction of a \hyperlink{regular}{\texttt{regular}} constraint with $p$ assignment cost functions. Like them, it is useful for modelling sequencing rules in personnel scheduling and rostering problems. Furthermore it allows to handle together several linear counters and costs on the sequence $x$. For example, one may count all the value occurrences like with a \hyperlink{globalcardinality}{\texttt{globalCardinality}} constraint. Counters can also model assignment costs in optimization problems or violation costs in over-constrained problems. For example, counting the occurrence number of a pattern allows to determine the violation cost of a soft forbidden pattern rule.

The filtering algorithm~\cite{MenanaCPAIOR09} of \texttt{multiCostRegular} does not achieve GAC (as it would be NP-hard) but it dominates the decompositions in \hyperlink{regular}{\texttt{regular}} or \hyperlink{costregular}{\texttt{costRegular}} constraints. 

The accepting language is specified by a deterministic finite automaton (DFA) $\Pi$ encoded as an object of class \texttt{Automaton} (see \hyperlink{costregular}{\texttt{costRegular}} for a short API).
The cost functions are vectors of weights on the transitions of $\Pi$. They are encoded as one matrix \texttt{int c[n][m][p][pi.getNbStates()]} such that
\texttt{c[i][j][k][s]} is the cost of assigning variable $x_i$ to value $j$ at state $s$ on dimension $k$. When the transition costs are independent of their initial states, a second API allows to specify a cost matrix \texttt{int c[n][m][p]}.

\begin{itemize}
	\item \textbf{API} : 
      \begin{itemize}
      \item \mylst{multiCostRegular(IntegerVariable[] z, IntegerVariable[] x, FiniteAutomaton pi, int[][][] c)}
      \item \mylst{multiCostRegular(IntegerVariable[] z, IntegerVariable[] x, FiniteAutomaton pi, int[][][][] c)}
      \end{itemize}
	\item \textbf{return type} : \texttt{Constraint}
	\item \textbf{options} : n/a
%      \begin{itemize}
%      \item \texttt{MultiCostRegular.DATA\_STRUCT} is  \texttt{MultiCostRegular.BITSET} or \texttt{MultiCostRegular.LIST}: a parameter stating which backtrable data structure to use for storing the outgoing arcs of the layered digraph. The observed behaviour is until $1000$ arcs the bipartite list is much more efficient, afterwards the memory efficiency of the bitset representation allow faster operations. 
%      \item \texttt{MultiCostRegular.U0}, \texttt{MultiCostRegular.R0}, \texttt{MultiCostRegular.MAXNONIMPROVEITER}, and \texttt{MultiCostRegular.MAXBOUNDITER} are value parameters of the subgradient algorithm used for solving the lagrangean relaxation.
%      \item \texttt{MultiCostRegular.D\_PREC} is a double parameter stating the precision of float computation. It is set by default to $10^{-5}$.
%      \end{itemize}
	\item \textbf{favorite domain} : \emph{to complete}
	\item \textbf{references} :\\
       \cite{MenanaCPAIOR09}: \emph{Sequencing and Counting with the {\tt multicost-regular} Constraint}
\end{itemize}
%\begin{notedef}
%  For further informations, see the multicost-regular description.
%\end{notedef}

\textbf{Example}:
\lstinputlisting{java/cmulticosteregular_import.j2t}
\lstinputlisting{java/cmulticostregular.j2t}

