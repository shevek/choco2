%\part{inversechanneling}
\label{inversechanneling}
\hypertarget{inversechanneling}{}

\section{inverseChanneling (constraint)}\label{inversechanneling:inversechannelingconstraint}\hypertarget{inversechanneling:inversechannelingconstraint}{}
\begin{notedef}
  \texttt{inverseChanneling}$(x,y)$ states a channeling between two arrays  $x$ and $y$ of integer variables with the same domain.It enforces that if the $i$-th element of $x$ is equal to $j$ then the $j$-th element of $y$ is equal to $i$ and conversely:
$$x_i = j\quad\iff\quad y_j = i$$
\end{notedef}
\begin{itemize}
	\item \textbf{API} : \mylst{inverseChanneling(IntegerVariable[] x, IntegerVariable[] y)}
	\item \textbf{return type} : \texttt{Constraint}
	\item \textbf{options} : \emph{no options}
	\item \textbf{favorite domain} : enumerated for x
	\item \textbf{references} :\\
      global constraint catalog: \href{http://www.emn.fr/x-info/sdemasse/gccat/Cinverse.html}{inverse}
\end{itemize}

\textbf{Example}:
\lstinputlisting{java/cinversechanneling.j2t}
