%\part{nth}
\label{nth}
\hypertarget{nth}{}

\section{nth (constraint)}\label{nth:nthconstraint}\hypertarget{nth:nthconstraint}{}
\texttt{nth} is the well known \emph{element} constraint.
Several APIs are available: 
\begin{notedef}
\begin{itemize}
\item \texttt{nth}$(i,x,y)$ ensures that $x[i]=y$
\item \texttt{nth}$(i,x,y,o)$ ensures that $x[i+o]=y$ ($o$ is an \emph{offset} for shifting values)
\item \texttt{nth}$(i,j,x,y)$ ensures that $x[i][j]=y$
\end{itemize}
\end{notedef}

\begin{itemize}
	\item \textbf{API} :
	\begin{itemize}
		\item \mylst{nth(IntegerVariable i, int[] x, IntegerVariable y)}
		\item \mylst{nth(String option, IntegerVariable i, int[] x, IntegerVariable y)}
		\item \mylst{nth(IntegerVariable i, IntegerVariable[] x, IntegerVariable y)}
		\item \mylst{nth(IntegerVariable i, int[] x, IntegerVariable y, int offset)}
		\item \mylst{nth(String option, IntegerVariable i, int[] x, IntegerVariable y, int offset)}		
		\item \mylst{nth(IntegerVariable i, IntegerVariable[] x, IntegerVariable y, int offset)}
		\item \mylst{nth(String option, IntegerVariable i, IntegerVariable[] x, IntegerVariable y, int offset)}
		\item \mylst{nth(IntegerVariable i, IntegerVariable j, int[][] x, IntegerVariable y)}
	\end{itemize}
	\item \textbf{return type} : \texttt{Constraint}
	\item \textbf{options} :
	\begin{itemize}
		\item \emph{no option} 
		\item \hyperlink{cnthg:cnthgoptions}{\tt Options.C\_NTH\_G} for global consistency
	\end{itemize}
	\item \textbf{favorite domain} : \emph{to complete}
	\item \textbf{references} :\\
      global constraint catalog: \href{http://www.emn.fr/x-info/sdemasse/gccat/Celement.html}{element}
\end{itemize}

\textbf{Example}:
\lstinputlisting{java/cnth.j2t} 
