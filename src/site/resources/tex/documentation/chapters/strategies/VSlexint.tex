\section{LexIntVarSelector (Variable selector)}\label{lexintvarselector:lexintvarselectorvarselector}\hypertarget{lexintvarselector:lexintvarselectorvarselector}{}
\begin{notedef}
  \texttt{LexIntVarSelector} applies two heuristics lexicographically for selecting a variable: a first heuristic is applied finding the best variables, ties are broken with the second heuristic 
\end{notedef}

\begin{itemize}
	\item \textbf{Constructor} :\mylst{LexIntVarSelector(TiedIntVarSelector h1, HeuristicIntVarSelector h2)}
	\begin{itemize}
\item \texttt{TiedIntVarSelector} is an \texttt{interface} implemented by : \hyperlink{mindomain:mindomainvarselector}{MinDomain}, \hyperlink{maxdomain:maxdomainvarselector}{MaxDomain}, \hyperlink{maxregret:maxregretvarselector}{MaxRegret}, \hyperlink{maxvaldomain:maxvaldomainvarselector}{MaxValueDomain}, \hyperlink{minvaldomain:minvaldomainvarselector}{MinValueDomain}, \hyperlink{mostconstrained:mostconstrainedvarselector}{MostConstrained}. //TODO to complete

\item \texttt{HeuristicIntVarSelector} is an \texttt{abstract class}  implemented by: \hyperlink{mindomain:mindomainvarselector}{MinDomain}, \hyperlink{maxdomain:maxdomainvarselector}{MaxDomain}, \hyperlink{maxregret:maxregretvarselector}{MaxRegret}, \hyperlink{maxvaldomain:maxvaldomainvarselector}{MaxValueDomain}, \hyperlink{minvaldomain:minvaldomainvarselector}{MinValueDomain}, \hyperlink{mostconstrained:mostconstrainedvarselector}{MostConstrained}. //TODO to complete

\end{itemize}

	\item \textbf{type of variable} : integer
	\item \textbf{references} : \emph{n/a}
\end{itemize}

\textbf{Example}:
%\lstinputlisting{java/cabs.j2t}

