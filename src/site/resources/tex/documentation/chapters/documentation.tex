\part{Documentation}\label{ch:doc}\hypertarget{ch:doc}{}

The documentation of Choco is organized as follows:
\begin{itemize}
\item 
The concise \hyperlink{doc:introduction}{introduction} provides some informations \hyperlink{introduction:aboutconstraintprogramming}{about constraint programming} concepts and a ``Hello world''-like \hyperlink{introduction:myfirstchocoprogram}{first Choco program}.
\item 
The \hyperlink{doc:model}{model} section gives informations on \hyperlink{doc:model}{how to create a model} and introduces \hyperlink{model:variables}{variables} and \hyperlink{model:constraints}{constraints}.
\item 
The \hyperlink{doc:solver}{solver} section gives informations on \hyperlink{doc:solver}{how to create a solver}, to \hyperlink{doc:solver}{read a model}, to define a \hyperlink{solver:searchstrategy}{search strategy}, and finally to \hyperlink{solver:solveaproblem}{solve a problem}.
\item 
The \hyperlink{doc:advanced}{advanced use} section explains how to define your own \hyperlink{advanced:defineyourownlimitsearchspace}{limit search space}, \hyperlink{advanced:defineyourownsearchstrategy}{search strategy}, \hyperlink{advanced:defineyourownconstraint}{constraint}, \hyperlink{advanced:defineyourownoperator}{operator}, \hyperlink{advanced:defineyourownvariable}{variable}, \hyperlink{advanced:backtrackablestructures}{backtrackable structure} and write \hyperlink{advanced:howtowriteloggingstatements}{logging statements}.
\item 
%The \hyperlink{doc:applications}{applications} section shows the use of Choco defined global constraints on \hyperlink{schedulinganduseofthecumulative:schedulinganduseofthecumulativeconstraint}{scheduling} or \hyperlink{geostdescription:placementanduseofthegeostconstraint}{placement} problems.
%\item 
%Lastly, the catalog of Choco defined \hyperlink{ch:constraints}{constraints} is presented.
\end{itemize}

%\section*{Beta}\label{documentation:beta}\hypertarget{documentation:beta}{}
%Here you can find short documentation concerning futur works, only available on the beta version or extension of the current jar:
%\begin{itemize}
%%	\item \hyperlink{chocoandgraphviz}{Choco and Graphviz} \emph{not yet available. Has been included in \hyperlink{chocoandvisu}{Choco and Visu}.}
%	\item \hyperlink{chocoandvisu}{Choco and Visu}
%\end{itemize}

%\section*{Material}\label{documentation:material}\hypertarget{documentation:material}{}
%Here you can find some materials at your disposal. If you create any, you can also send it to us, so we can add it to the list.

%\subsection{Presentations}\label{documentation:presentations}\hypertarget{documentation:presentations}{}
%\begin{itemize}
%	\item \textbf{August 2008} : The \emph{CHOCO : an Open Source Java Constraint Constraint Programming} white paper presentation send to the \href{http://www.cril.univ-artois.fr/cpai08/}{CSP'08 competition}. \href{media/pdf/choco-presentation.pdf}{PDF} of white paper presentation of Choco
%	\item \textbf{june 2008} : \emph{The CHOCO constraint programming solver} presentation held at the \href{https://projects.coin-or.org/events/wiki/cpaior2008}{Workshop on Open-Source Software for Integer and Constraint Programming} during the last \href{http://contraintes.inria.fr/cpaior08/}{CPAIOR} conference. \href{media/slides/cpaior-choco.pdf}{PDF slides} of the presentation given by Guillaume Rochart
%\end{itemize}

\chapter{Introduction to constraint programming and Choco}\label{doc:introduction}\hypertarget{doc:introduction}{}

\section{About constraint programming}\label{introduction:aboutconstraintprogramming}\hypertarget{introduction:aboutconstraintprogramming}{}

\begin{myquote}
Constraint programming represents one of the closest approaches computer science has yet made to the Holy Grail of programming: the user states the problem, the computer solves it.
\begin{flushright}\bf E. C. Freuder, Constraints, 1997.\end{flushright}
\end{myquote}


Fast increasing computing power in the 1960s led to a wealth of works around problem solving, at the root of Operational Research, Numerical Analysis, Symbolic Computing, Scientific Computing, and a large part of Artificial Intelligence and programming languages. Constraint Programming is a discipline that gathers, interbreeds, and unifies ideas shared by all these domains to tackle decision support problems.

Constraint programming has been successfully applied in numerous domains. Recent applications include computer graphics (to express geometric coherence in the case of scene analysis), natural language processing (construction of efficient parsers), database systems (to ensure and/or restore consistency of the data), operations research problems (scheduling, routing), molecular biology (DNA sequencing), business applications (option trading), electrical engineering (to locate faults), circuit design (to compute layouts), etc.

Current research in this area deals with various fundamental issues, with implementation aspects and with new applications of constraint programming.

\subsection{Constraints}\label{introduction:constraints}\hypertarget{introduction:constraints}{}
A constraint is simply a logical relation among several unknowns (or variables), each taking a value in a given domain. A constraint thus restricts the possible values that variables can take, it represents some partial information about the variables of interest. For instance, the circle is inside the square relates two objects without precisely specifying their positions, i.e., their coordinates. Now, one may move the square or the circle and he or she is still able to maintain the relation between these two objects. Also, one may want to add another object, say a triangle, and to introduce another constraint, say the square is to the left of the triangle. From the user (human) point of view, everything remains absolutely transparent.

Constraints naturally meet several interesting properties:
\begin{itemize}
	\item constraints may specify partial information, i.e. constraint need not uniquely specify the values of its variables,
	\item constraints are non-directional, typically a constraint on (say) two variables $X, Y$ can be used to infer a constraint on $X$ given a constraint on $Y$ and vice versa,
	\item constraints are declarative, i.e. they specify what relationship must hold without specifying a computational procedure to enforce that relationship,
	\item constraints are additive, i.e. the order of imposition of constraints does not matter, all that matters at the end is that the conjunction of constraints is in effect,
	\item constraints are rarely independent, typically constraints in the constraint store share variables.
\end{itemize}

Constraints arise naturally in most areas of human endeavor. The three angles of a triangle sum to 180 degrees, the sum of the currents floating into a node must equal zero, the position of the scroller in the window scrollbar must reflect the visible part of the underlying document, these are some examples of constraints which appear in the real world. Thus, constraints are a natural medium for people to express problems in many fields. 

\subsection{Constraint Programming}\label{introduction:constraintprogramming}\hypertarget{introduction:constraintprogramming}{}
Constraint programming is the study of computational systems based on constraints. The idea of constraint programming is to solve problems by stating constraints (conditions, properties) which must be satisfied by the solution.

Work in this area can be tracked back to research in Artificial Intelligence and Computer Graphics in the sixties and seventies. Only in the last decade, however, has there emerged a growing realization that these ideas provide the basis for a powerful approach to programming, modeling and problem solving and that different efforts to exploit these ideas can be unified under a common conceptual and practical framework, constraint programming. 

\begin{note}
If you know \textbf{sudoku}, then you know \textbf{constraint programming}. See why \hyperlink{sudokuandcp}{here}.
\end{note}


\section{Modeling with Constraint programming}\label{introduction:modelingwithconstraintprogramming}\hypertarget{introduction:modelingwithconstraintprogramming}{}
The formulation and the resolution of combinatorial problems are the two main goals of the constraint programming domain. This is an essential way to solve many interesting industrial problems such as scheduling, planning or design of timetables. The main interest of constraint programming is to propose to the user to model a problem without being interested in the way the problem is solved.

\subsection{The Constraint Satisfaction Problem}\label{introduction:csp}\hypertarget{introduction:csp}{}

Constraint programming allows to solve combinatorial problems modeled by a Constraint Satisfaction Problem (CSP). Formally, a CSP is defined by a triplet $(X,D,C)$:
\begin{itemize}
	\item \textbf{Variables}: $X = \{X_1,X_2,\ldots,X_n\}$ is the set of variables of the problem.
	\item \textbf{Domains}: $D$ is a function which associates to each variable $X_i$ its domain $D(X_i)$, i.e. the set of possible values that can be assigned to $X_i$. The domain of a variable is usually a finite set of integers: $D(X_i)\subset\Z$ (\emph{integer variable}). But a domain can also be continuous ($D(X_i)\subseteq\R$ for a \emph{real variable}) or made of discrete set values ($D(X_i)\subseteq\mathcal{P}(\Z)$ for a \emph{set variable}).
	\item \textbf{Constraints}: $C = \{C_1,C_2,\ldots,C_m\}$ is the set of constraints. A constraint $C_j$ is a relation defined on a subset $X^j = \{X^j_1,X^j_2,\ldots,X^j_{n^j}\}\subseteq X$ of variables which restricts the possible tuples of values $(v_1,\ldots,v_{n^j})$ for these variables:
$$(v_1,\ldots,v_{n^j})\in C_j\cap (D(X^j_1)\times D(X^j_2)\times\cdots\times D(X^j_{n^j})).$$
Such a relation can be defined explicitely (ex: $(X_1,X_2)\in\{(0,1),(1,0)\}$) or implicitely (ex: $X_1+X_2\le 1$).
\end{itemize}

Solving a CSP is to find a tuple $v=(v_1,\ldots,v_{n})\in D(X)$ on the set of variables which satisfies all the constraints:
$$(v_1,\ldots,v_{n^j})\in C_j,\quad\forall j\in\{1,\ldots,m\}.$$

For optimization problems, one need to define an \textbf{objective function} $f:D(X)\rightarrow\R$. An optimal solution is then a solution tuple of the CSP that minimizes (or maximizes) function $f$.

\subsection{Examples of CSP models}\label{introduction:examples}\hypertarget{introduction:examples}{}
This part provides three examples using different types of variables in different problems. These examples are used throughout this tutorial to illustrate their modeling with Choco.

\subsubsection{Example 1: the n-queens problem.}\label{introduction:example1:nqueens}\hypertarget{introduction:example1:nqueens}{}
Let us consider a chess board with $n$ rows and $n$ columns. A queen can move as far as she pleases, horizontally, vertically, or diagonally. The standard $n$-queens problem asks how to place $n$ queens on an $n$-ary chess board so that none of them can hit any other in one move.

The $n$-queens problem can be modeled by a CSP in the following way:
\begin{itemize}
	\item \textbf{Variables}: $X = \{X_{i}\ |\ i\in [1,n]\}$.
	\item \textbf{Domain}: for all variable $X_{i}\in X$, $D(X_{i}) = \{1,2,\ldots, n\}$.
	\item \textbf{Constraints}: the set of constraints is defined by the union of the three following constraints,
	\begin{itemize}
		\item queens have to be on distinct lines:
		\begin{itemize}
			\item $C_{lines} = \{X_{i}\neq X_{j}\ |\ i,j\in [1,n], i\neq j\}$.
		\end{itemize}
		\item queens have to be on distinct diagonals:
		\begin{itemize}
			\item $C_{diag1} = \{X_{i}\neq X_{j+j-i}\ |\ i,j\in [1,n], i\neq j\}$.
			\item $C_{diag2} = \{X_{i}\neq X_{j+i-j}\ |\ i,j\in [1,n], i\neq j\}$.
		\end{itemize}
	\end{itemize}
\end{itemize}

\subsubsection{Example 2: the ternary Steiner problem.}\label{introduction:example2:theternarysteinerproblem}\hypertarget{introduction:example2:theternarysteinerproblem}{}
A ternary Steiner system of order $n$ is a set of $n*(n-1)/6$ triplets of distinct elements taking their values in $[1,n]$, such that all the pairs included in two distinct triplets are different.
See \url{http://mathworld.wolfram.com/SteinerTripleSystem.html} for details. 

The ternary Steiner problem can be modeled by a CSP in the following way:
\begin{itemize}
	\item let $t = n*(n-1)/6$.
	\item \textbf{Variables}: $X = \{X_{i}\ |\ i\in [1,t]\}$.
	\item \textbf{Domain}: for all $i\in [1,t]$, $D(X_{i}) = \{1,...,n\}$.
	\item \textbf{Constraints}:
	\begin{itemize}
		\item every set variable $X_i$ has a cardinality of 3:
		\begin{itemize}
			\item for all $i\in [1,t]$, $|X_{i}| = 3$.
		\end{itemize}
		\item the cardinality of the intersection of every two distinct sets must not exceed 1:
		\begin{itemize}
			\item for all $i,j\in [1,t]$, $i\neq j$, $|X_{i}\cap X_{j}|\le 1$.
		\end{itemize}
	\end{itemize}
\end{itemize}

\subsubsection{Example 3: the CycloHexane problem.}\label{introduction:example3:thecyclohexaneproblem}\hypertarget{introduction:example3:thecyclohexaneproblem}{}
The problem consists in finding the 3D configuration of a cyclohexane molecule. It is described with a system of three non linear equations:
\begin{itemize}
	\item \textbf{Variables}: $x,y,z$.
	\item \textbf{Domain}: $]-\infty;+\infty[$.
	\item \textbf{Constraints}:
	\begin{align*}
		y^{2} * (1 + z^{2}) + z * (z - 24 * y) &= -13\\
		x^{2} * (1 + y^{2}) + y * (y - 24 * x) &= -13\\
		z^{2} * (1 + x^{2}) + x * (x - 24 * z) &= -13
	\end{align*}
\end{itemize}

\section{My first Choco program: the magic square}\label{introduction:myfirstchocoprogram}\hypertarget{introduction:myfirstchocoprogram}{}

\subsection{The magic square problem}\label{introduction:amagicsquareproblem}\hypertarget{introduction:amagicsquareproblem}{}
In the following, we will address the magic square problem of order 3 to illustrate step-by-step how to model and solve this problem using choco. 

\subsubsection{Definition:}
A magic square of order $n$ is an arrangement of $n^{2}$ numbers, usually distinct integers, in a square, such that the $n$ numbers in all rows, all columns, and both diagonals sum to the same constant. A standard magic square contains the integers from 1 to $n^{2}$.

The constant sum in every row, column and diagonal is called the magic constant or magic sum $M$. The magic constant of a classic magic square depends only on $n$ and has the value:
$M(n)=n(n^2 +1)/2$.

\href{http://en.wikipedia.org/wiki/magicsquare}{More details on the magic square problem.}


\subsection{A mathematical model}\label{introduction:mathematicalmodeling}\hypertarget{introduction:mathematicalmodeling}{}

Let $x_{ij}$ be the variable indicating the value of the $j^{th}$ cell of row $i$. 
Let $C$ be the set of constraints modeling the magic square as:
\begin{align*}
&x_{ij} \in [1,n^2],\ &&\forall i,j \in [1, n]\\
&x_{ij}\ne x_{kl},\ &&\forall i,j,k,l \in [1,n], i\ne k, j\ne l\\
&\sum_{j=1}^{n} x_{ij} = n^2,\ &&\forall i \in [1,n]\\
&\sum_{i=1}^{n} x_{ij} = n^2,\ &&\forall j \in [1,n]\\
&\sum_{i=1}^{n} x_{ii} = n^2&&\\
&\sum_{i=n}^{1} x_{i(n-i)} = n^2&&\\
\end{align*}

We have all the required information to model the problem with Choco.
\begin{note}
	For the moment, we just talk about \emph{model translation} from a mathematical representation to Choco.
	Choco can be used as a \emph{black box}, that means we just need to define the problem without knowing the way it will be solved. We can therefore focus on the modeling not on the solving.
\end{note}

\subsection{To Choco...}\label{introduction:inchoco}\hypertarget{introduction:inchoco}{}

First, we need to know some of the basic Choco objects:
\begin{itemize}
\item 
The \textbf{model} (object \texttt{Model} in Choco) is one of the central elements of a Choco program. Variables and constraints are associated to it.
\item
The \textbf{variables} (objects \texttt{IntegerVariable}, \texttt{SetVariable}, and \texttt{RealVariable} in Choco) are the \emph{unknown} of the problem. Values of variables are taken from a \textbf{domain} which is defined by a set of values or quite often simply by a lower bound and an upper bound of the allowed values. The domain is given when creating the variable.
\begin{note}
Do not forget that we manipulate \textbf{variables} in the mathematical sense (as opposed to classical computer science). Their effective value will be known only once the problem has been solved.
\end{note}
\item
The \textbf{constraints} define relations to be satisfied between variables and constants.
In our first model, we only use the following constraints provided by Choco:
\begin{itemize}
	\item \texttt{eq(var1, var2)} which ensures that \texttt{var1} equals \texttt{var2}.
	\item \texttt{neq(var1, var2)} which ensures that \texttt{var1} is not equal to \texttt{var2}.
	\item \texttt{sum(var[])} which returns expression \texttt{var[0]+var[1]+...+var[n]}.
\end{itemize}
\end{itemize}

\subsection{The program}\label{introduction:theprogram}\hypertarget{introduction:theprogram}{}
After having created your java class file, import the Choco class to use the API:
\begin{lstlisting}
  import choco.Choco;
\end{lstlisting}
First of all, let's create a Model:
\lstinputlisting{java/imagicsquare1.j2t}
We create an instance of \texttt{CPModel()} for \textbf{C}onstraint \textbf{P}rogramming Model.
Do not forget to add the following imports:
\begin{lstlisting}
  import choco.cp.model.CPModel;
\end{lstlisting}
Then we declare the variables of the problem:
\lstinputlisting{java/imagicsquare2.j2t}
Add the import:
\begin{lstlisting}
  import choco.kernel.model.variables.integer.IntegerVariable;
\end{lstlisting}
We have defined the variable using the \texttt{makeIntVar} method which creates an enumerated domain: all the values are stored in the java object (beware, it is usually not necessary to store all the values and it is less efficient than to create a bounded domain).

\noindent Now, we are going to state a constraint ensuring that all variables must have a different value:
\lstinputlisting{java/imagicsquare3.j2t}
Add the import:
\begin{lstlisting}
  import choco.kernel.model.constraints.Constraint;
\end{lstlisting}
Then, we add the constraint ensuring that the magic sum is respected:
\lstinputlisting{java/imagicsquare4.j2t}
Then we define the constraint ensuring that each column is equal to the magic sum.
Actually, \texttt{var} just denotes the rows of the square. So we have to declare a temporary array of variables that defines the columns.
\lstinputlisting{java/imagicsquare5.j2t}
It is sometimes useful to define some temporary variables to keep the model simple or to reorder array of variables. That is why we also define two other temporary arrays for diagonals.
\lstinputlisting{java/imagicsquare6.j2t}
Now, we have defined the model. The next step is to solve it.
For that, we build a Solver object:
\lstinputlisting{java/imagicsquare7.j2t}

with the imports:
\begin{lstlisting}
  import choco.cp.solver.CPSolver;
\end{lstlisting}
We create an instance of \texttt{CPSolver()} for Constraint Programming Solver.
Then, the solver reads (translates) the model and solves it:
\lstinputlisting{java/imagicsquare8.j2t}
The only variables that need to be printed are the ones in \texttt{var} (all the others are only references to these ones). 
\begin{note}
We have to use the Solver to get the value of each variable of the model. The Model only declares the objects, the Solver finds their value.
\end{note}
We are done, we have created our first Choco program. 
The complete source code can be found here: \href{media/zip/exmagicsquare.zip}{ExMagicSquare.zip}


\subsection{In summary}\label{introduction:whatisimportant}\hypertarget{introduction:whatisimportant}{}
\begin{itemize}
	\item A Choco Model is defined by a set of Variables with a given domain and a set of Constraints that link Variables:
it is necessary to add both Variables and Constraints to the Model.
	\item temporary Variables are useful to keep the Model readable, or necessary when reordering arrays.
	\item The value of a Variable can be known only once the Solver has found a solution.
	\item To keep the code readable, you can avoid the calls to the static methods of the Choco classes, by importing the static classes, i.e. instead of:
\begin{lstlisting}
  import choco.Choco;
  ...
  IntegerVariable v = Choco.makeIntVar("v", 1, 10);
  ...
  Constraint c = Choco.eq(v, 5);
\end{lstlisting}
you can use:
\begin{lstlisting}
  import static choco.Choco.*;
  ...
  IntegerVariable v = makeIntVar("v", 1, 10);
  ...
  Constraint c = eq(v, 5);
\end{lstlisting}
\end{itemize}

\section{Complete examples}\label{model:completeexamples}\hypertarget{model:completeexamples}{}
We provide now the complete Choco model for the three examples \hyperlink{introduction:examples}{previously described}.

\subsection{Example 1: the n-queens problem with Choco}\label{model:example1:nqueenschoco}\hypertarget{model:example1:nqueenschoco}{}
This first model for the \hyperlink{introduction:example1:nqueens}{n-queens problem} only involves binary constraints of differences between integer variables. One can immediately recognize the 4 main elements of any Choco code. First of all, create the model object. Then create the variables by using the Choco API (One variable per queen giving the row (or the column) where the queen will be placed). Finally, add the constraints and solve the problem. 

\lstinputlisting{java/inqueen.j2t}

\subsection{Example 2: the ternary Steiner problem with Choco}\label{model:example2:ternarysteinerchoco}\hypertarget{model:example2:ternarysteinerchoco}{}
The \hyperlink{introduction:example2:theternarysteinerproblem}{ternary Steiner problem} is entirely modeled using set variables and set constraints. 
\lstinputlisting{java/iternarysteiner.j2t}

\subsection{Example 3: the CycloHexane problem with Choco}\label{model:example3:thecyclohexaneproblemwithchoco}\hypertarget{model:example3:thecyclohexaneproblemwithchoco}{}
Real variables are illustrated on the problem of finding the 3D configuration of a cyclohexane molecule. 
\lstinputlisting{java/icyclohexane.j2t}



\chapter{The model}\label{doc:model}\hypertarget{doc:model}{}

The {\tt Model}, along with the {\tt Solver}, is one of the two key elements of any Choco program. The Choco {\tt Model} allows to describe a problem in an easy and declarative way: it simply records the variables and the constraints defining the problem.

This section describes the large API provided by Choco to create different types of \hyperlink{model:variables}{variables} and \hyperlink{model:constraints}{constraints}.

%\begin{note}
\textbf{Note that a static import is required to use the Choco API:}
\begin{lstlisting}
  import static choco.Choco.*;
\end{lstlisting}
%It is mandatory in order to compile !
%\end{note}

%\section{How to create a model}\label{model:howtocreateamodel}\hypertarget{model:howtocreateamodel}{}
First of all, a {\tt Model} object is created as follows:
\begin{lstlisting}
Model model = new CPModel();
\end{lstlisting}
In that specific case, a Constraint Programming (CP) {\tt Model} object has been created. 


%%%%%%%%%%%%%%%%%%%%%%%%%%%%%%%%%%%%%%%%%%%%%%%%%%%%%%%%%%%%%%%%%%%%%%%%%%%%%%%%%%%%%%%%%%%%%%%%%%%%%%%%%%%%%%%%%%%%%%%%%%%%%%%%%%%%%%%%%%%%%%%%%%%
%%%%%%%%%%%%%%%%%%%%%%%%%%%%%%%%%%%%%%%%%%%%%%%%%%%%% VARIABLE %%%%%%%%%%%%%%%%%%%%%%%%%%%%%%%%%%%%%%%%%%%%%%%%%%%%%%%%%%%%%%%%%%%%%%%%%%%%%%%%%%%%
%%%%%%%%%%%%%%%%%%%%%%%%%%%%%%%%%%%%%%%%%%%%%%%%%%%%%%%%%%%%%%%%%%%%%%%%%%%%%%%%%%%%%%%%%%%%%%%%%%%%%%%%%%%%%%%%%%%%%%%%%%%%%%%%%%%%%%%%%%%%%%%%%%%


\section{Variables}\label{model:variables}\hypertarget{model:variables}{}

%Choco provides a large API to create different types of variables : \textbf{integer}, \textbf{real} and \textbf{set}. 

A Variable is defined by a type (\hyperlink{integervariable}{integer}, \hyperlink{realvariable}{real}, or \hyperlink{setvariable}{set} variable), a name, and the values of its domain. When creating a simple variable, some options can be set to specify its domain representation (ex: enumerated or bounded) within the {\tt Solver}.
%Some kinds of variables have options for their domain, it may have an effect on what kind of specific object is created when the model is read by the solver.
\begin{note}
The choice of the domain should be considered. The efficiency of the solver often depends on judicious choice of the domain type.
\end{note}
Variables can be combined as \hyperlink{model:expressionvariables}{expression variables} using operators.

One or more variables can be added to the model using the following methods of the \texttt{Model} class:
\lstinputlisting{java/mvariabledeclaration1.j2t}

\begin{note}
Explictly addition of variables is not mandatory. See \hyperlink{model:constraints}{\tt Constraint} for more details.
\end{note}

Specific role of variables \emph{var} can be defined with \emph{options}:  \hyperlink{model:decisionvariables}{non-decision} variables or  \hyperlink{model:objectivevariable}{objective} variable;
\lstinputlisting{java/mvariabledeclaration2.j2t}

%%%%%%%%%%%%%%%%%%%%%%%%%%%%%%%%%%%%%%%%%%%%%%%%%%%%%%%%%%%%%%%%%%%%%%%%%%%%%%%%%%%%%%%%%%%%%%%%%%%%%%%%%%%%%%%%%%%%%%%%%%%%%%%%%%%%%%%%%%%%%%%%%%%
%%%%%%%%%%%%%%%%%%%%%%%%%%%%%%%%%%%%%%%%%%%%%%%%%%%%% SIMPLE VARIABLE  %%%%%%%%%%%%%%%%%%%%%%%%%%%%%%%%%%%%%%%%%%%%%%%%%%%%%%%%%%%%%%%%%%%%%%%%%%%%

\subsection{Simple Variables}\label{model:simplevariables}\hypertarget{model:simplevariables}{}
See Section \hyperlink{ch:vars}{Variables} for details:

\begin{notedef}\tt
\hyperlink{integervariable}{IntegerVariable}, \hyperlink{setvariable}{SetVariable}, \hyperlink{realvariable}{RealVariable}
\end{notedef}

%%%%%%%%%%%%%%%%%%%%%%%%%%%%%%%%%%%%%%%%%%%%%%%%%%%%%%%%%%%%%%%%%%%%%%%%%%%%%%%%%%%%%%%%%%%%%%%%%%%%%%%%%%%%%%%%%%%%%%%%%%%%%%%%%%%%%%%%%%%%%%%%%%%
%%%%%%%%%%%%%%%%%%%%%%%%%%%%%%%%%%%%%%%%%%%%%%%%%%%%% CONSTANT VARIABLE  %%%%%%%%%%%%%%%%%%%%%%%%%%%%%%%%%%%%%%%%%%%%%%%%%%%%%%%%%%%%%%%%%%%%%%%%%%

\subsection{Constants}\label{model:constants}\hypertarget{model:constants}{}
A constant is a variable with a fixed domain. An \hyperlink{integervariable}{\tt IntegerVariable} declared with a unique value is automatically set as constant. A constant declared twice or more is only stored once in a model.

\lstinputlisting{java/mconstant.j2t}

%%%%%%%%%%%%%%%%%%%%%%%%%%%%%%%%%%%%%%%%%%%%%%%%%%%%%%%%%%%%%%%%%%%%%%%%%%%%%%%%%%%%%%%%%%%%%%%%%%%%%%%%%%%%%%%%%%%%%%%%%%%%%%%%%%%%%%%%%%%%%%%%%%%
%%%%%%%%%%%%%%%%%%%%%%%%%%%%%%%%%%%%%%%%%%%%%%%%%%%%% EXPRESSION VARIABLE  %%%%%%%%%%%%%%%%%%%%%%%%%%%%%%%%%%%%%%%%%%%%%%%%%%%%%%%%%%%%%%%%%%%%%%%%
\subsection{Expression variables and operators}\label{model:expressionvariables}\hypertarget{model:expressionvariables}{}
Expression variables represent the result of combinations between variables of the same type made by operators. Two types of expression variables exist : 
\begin{notedef}
\textbf{\tt IntegerExpressionVariable} and \textbf{\tt RealExpressionVariable}.
\end{notedef}
One can define a buffered expression variable to make a constraint easy to read, for example:
\lstinputlisting{java/mexpressionvariable.j2t}

%\section{Operators}\label{model:operators}\hypertarget{model:operators}{}

To construct expressions of variables, simple operators can be used. Each returns a \texttt{ExpressionVariable} object:
\begin{notedef}\tt
\hyperlink{abs:absoperator}{abs}, \hyperlink{cos:cosoperator}{cos}, \hyperlink{disteq:disteqoperator}{distEq}, \hyperlink{distgt:distgtoperator}{distGt}, \hyperlink{distlt:distltoperator}{distLt}, \hyperlink{distneq:distneqoperator}{distNeq}, \hyperlink{div:divoperator}{div}, \hyperlink{ifthenelse:ifthenelseoperator}{ifThenElse}, \hyperlink{max:maxoperator}{max}, \hyperlink{min:minoperator}{min}, \hyperlink{minus:minusoperator}{minus}, \hyperlink{mod:modoperator}{mod}, \hyperlink{mult:multoperator}{mult}, \hyperlink{neg:negoperator}{neg}, \hyperlink{plus:plusoperator}{plus}, \hyperlink{power:poweroperator}{power}, \hyperlink{scalar:scalaroperator}{scalar}, \hyperlink{sin:sinoperator}{sin}, \hyperlink{sum:sumoperator}{sum}.
\end{notedef}
Note that these operators are not considered as constraints: they do not return a \texttt{Constraint} objet but a \texttt{Variable} object.

%%%%%%%%%%%%%%%%%%%%%%%%%%%%%%%%%%%%%%%%%%%%%%%%%%%%%%%%%%%%%%%%%%%%%%%%%%%%%%%%%%%%%%%%%%%%%%%%%%%%%%%%%%%%%%%%%%%%%%%%%%%%%%%%%%%%%%%%%%%%%%%%%%%
%%%%%%%%%%%%%%%%%%%%%%%%%%%%%%%%%%%%%%%%%%%%%%%%%%%%% MULTIPLE VARIABLE  %%%%%%%%%%%%%%%%%%%%%%%%%%%%%%%%%%%%%%%%%%%%%%%%%%%%%%%%%%%%%%%%%%%%%%%%%%

\subsection{MultipleVariables}\label{model:multiplevariables}\hypertarget{model:multiplevariables}{}
These are syntaxic sugar. To make their declaration easier, \hyperlink{tree:treeconstraint}{\tt tree}, \hyperlink{geost:geostconstraint}{\tt geost}, and scheduling constraints allow or require to use multiple variables, like \texttt{TreeParametersObject}, \texttt{GeostObject} or \hyperlink{taskvariable}{\tt TaskVariable}.
See also the code examples for these constraints.

%%%%%%%%%%%%%%%%%%%%%%%%%%%%%%%%%%%%%%%%%%%%%%%%%%%%%%%%%%%%%%%%%%%%%%%%%%%%%%%%%%%%%%%%%%%%%%%%%%%%%%%%%%%%%%%%%%%%%%%%%%%%%%%%%%%%%%%%%%%%%%%%%%%
%%%%%%%%%%%%%%%%%%%%%%%%%%%%%%%%%%%%%%%%%%%%%%%%%%%%% OPTIONS %%%%%%%%%%%%%%%%%%%%%%%%%%%%%%%%%%%%%%%%%%%%%%%%%%%%%%%%%%%%%%%%%%%%%%%%%%%%%%%%%%%%%

\subsection{Decision/non-decision variables}\label{model:decisionvariables}\hypertarget{model:decisionvariables}{}

By default, each variable added to a model is a decision variable, \textit{i.e.} is included in the default search strategy. A variable can be stated as a non decision one if its value can be computed by side-effect. To specify non decision variables, one can 
\begin{itemize}
\item exclude them from its search strategies (see \hyperlink{solver:searchstrategy}{search strategy} for more details);
\item specify non-decision variables (adding \hyperlink{vnodecision:vnodecisionoptions}{\tt Options.V\_NO\_DECISION} to their options) and keep the default search strategy.
\end{itemize}
\lstinputlisting{java/mnodecision1.j2t}
Each of these options can also be set within a single instruction for a group of variables, as follows: 
\lstinputlisting{java/mnodecision2.j2t}

\begin{note}
 The declaration of a \hyperlink{solver:searchstrategy}{search strategy} will erase setting \hyperlink{vnodecision:vnodecisionoptions}{\tt Options.V\_NO\_DECISION}.
\end{note}
  \todo{more precise: user-defined/pre-defined, variable and/or value heuristics ?}

\subsection{Objective variable}\label{model:objectivevariable}\hypertarget{model:objectivevariable}{}
You can define an objective variable directly within the model, by using option \hyperlink{vobjective:vobjectiveoptions}{\tt Options.V\_OBJECTIVE}:
\lstinputlisting{java/mobjective.j2t}

Only one variable can be defined as an objective. If more than one objective variable is declared, then only the last one will be taken into account.

Note that optimization problems can be declared without defining an objective variable within the model (see the \hyperlink{solver:optimization}{optimization example}.)

%%%%%%%%%%%%%%%%%%%%%%%%%%%%%%%%%%%%%%%%%%%%%%%%%%%%%%%%%%%%%%%%%%%%%%%%%%%%%%%%%%%%%%%%%%%%%%%%%%%%%%%%%%%%%%%%%%%%%%%%%%%%%%%%%%%%%%%%%%%%%%%%%%%
%%%%%%%%%%%%%%%%%%%%%%%%%%%%%%%%%%%%%%%%%%%%%%%%%%%%% CONSTRAINT  %%%%%%%%%%%%%%%%%%%%%%%%%%%%%%%%%%%%%%%%%%%%%%%%%%%%%%%%%%%%%%%%%%%%%%%%%%%%%%%%%
%%%%%%%%%%%%%%%%%%%%%%%%%%%%%%%%%%%%%%%%%%%%%%%%%%%%%%%%%%%%%%%%%%%%%%%%%%%%%%%%%%%%%%%%%%%%%%%%%%%%%%%%%%%%%%%%%%%%%%%%%%%%%%%%%%%%%%%%%%%%%%%%%%%

\section{Constraints}\label{model:constraints}\hypertarget{model:constraints}{}
Choco provides a large number of simple and global constraints and allows the user to easily define its own new constraint.
% Either basic, global (a \hyperlink{constraints}{large set of global constraints} are available) or \hyperlink{advanced:defineyourownconstraint}{user-defined} constraints, they are used to specify conditions to be held on variables to the model. 
A constraint deals with one or more variables of the model and specify conditions to be held on these variables. 
A constraint is stated into the model by using the following methods available from the \texttt{Model} API: 

\lstinputlisting{java/mconstraintdeclaration1.j2t}

\begin{note}\
Adding a constraint automatically adds its variables to the model (explicit declaration of variables addition is not mandatory).
\end{note}


\subsubsection{Example:} adding a difference (disequality) constraint between two variables of the model

\lstinputlisting{java/mconstraintdeclaration2.j2t}

Available \emph{options} depend on the kind of constraint \emph{c} to add: they allow, for example, to choose the filtering algorithm to run during propagation. See \hyperlink{optionssettings}{Section options ans settings} for more details, specific APIs exist for declaring options constraints.

This section presents the constraints available in the Choco API sorted by type or by domain. Related sections:
\begin{itemize}
\item a detailed description (with options, examples, references) of each constraint is given in Section \hyperlink{ch:constraints}{constraints}
\item Section \hyperlink{doc:applications}{applications} shows how to apply some specific global constraints
\item Section \hyperlink{advanced:defineyourownconstraint}{user-defined constraint} explains how to create its own constraint.
\end{itemize}

%%%%%%%%%%%%%%%%%%%%%%%%%%%%%%%%%%%%%%%%%%%%%%%%%%%%%%%%%%%%%%%%%%%%%%%%%%%%%%%%%%%%%%%%%%%%%%%%%%%%%%%%%%%%%%%%%%%%%%%%%%%%%%%%%%%%%%%%%%%%%%%%%%%
%%%%%%%%%%%%%%%%%%%%%%%%%%%%%%%%%%%%%%%%%%%%%%%%%%%%% BINARY CONSTRAINT  %%%%%%%%%%%%%%%%%%%%%%%%%%%%%%%%%%%%%%%%%%%%%%%%%%%%%%%%%%%%%%%%%%%%%%%%%%

\subsection{Binary constraints}\label{model:comparisonconstraints}\hypertarget{model:comparisonconstraints}{}
%The simplest constraints are comparisons which are defined over expressions of variables such as linear combinations. The following comparison constraints can be accessed through the \texttt{Model} API:
Constraints involving two integer variables
\begin{notedef}\tt
  \begin{itemize}
  \item \hyperlink{eq:eqconstraint}{eq}, \hyperlink{geq:geqconstraint}{geq}, \hyperlink{gt:gtconstraint}{gt}, \hyperlink{leq:leqconstraint}{leq}, \hyperlink{lt:ltconstraint}{lt}, \hyperlink{neq:neqconstraint}{neq}
  \item \hyperlink{abs:absconstraint}{abs}, \hyperlink{oppositesign:oppositesignconstraint}{oppositeSign}, \hyperlink{samesign:samesignconstraint}{sameSign}
  \end{itemize}
\end{notedef}

%%%%%%%%%%%%%%%%%%%%%%%%%%%%%%%%%%%%%%%%%%%%%%%%%%%%%%%%%%%%%%%%%%%%%%%%%%%%%%%%%%%%%%%%%%%%%%%%%%%%%%%%%%%%%%%%%%%%%%%%%%%%%%%%%%%%%%%%%%%%%%%%%%%
%%%%%%%%%%%%%%%%%%%%%%%%%%%%%%%%%%%%%%%%%%%%%%%%%%%%% TERNARY CONSTRAINT  %%%%%%%%%%%%%%%%%%%%%%%%%%%%%%%%%%%%%%%%%%%%%%%%%%%%%%%%%%%%%%%%%%%%%%%%%

\subsection{Ternary constraints}\label{model:ternaryconstraints}\hypertarget{model:ternaryconstraints}{}
Constraints involving three integer variables
\begin{notedef}\tt
  \begin{itemize}
  \item \hyperlink{distanceeq:distanceeqconstraint}{distanceEQ}, \hyperlink{distanceneq:distanceneqconstraint}{distanceNEQ}, \hyperlink{distancegt:distancegtconstraint}{distanceGT}, \hyperlink{distancelt:distanceltconstraint}{distanceLT}
  \item \hyperlink{intdiv:intdivconstraint}{intDiv}, \hyperlink{mod:modconstraint}{mod}, \hyperlink{times:timesconstraint}{times}
  \end{itemize}
\end{notedef}

%%%%%%%%%%%%%%%%%%%%%%%%%%%%%%%%%%%%%%%%%%%%%%%%%%%%%%%%%%%%%%%%%%%%%%%%%%%%%%%%%%%%%%%%%%%%%%%%%%%%%%%%%%%%%%%%%%%%%%%%%%%%%%%%%%%%%%%%%%%%%%%%%%%
%%%%%%%%%%%%%%%%%%%%%%%%%%%%%%%%%%%%%%%%%%%%%%%%%%%%% REAL CONSTRAINT  %%%%%%%%%%%%%%%%%%%%%%%%%%%%%%%%%%%%%%%%%%%%%%%%%%%%%%%%%%%%%%%%%%%%%%%%%%%%

\subsection{Constraints involving real variables}\label{model:realconstraints}\hypertarget{model:realconstraints}{}
%The simplest constraints are comparisons which are defined over expressions of variables such as linear combinations. The following comparison constraints can be accessed through the \texttt{Model} API:
Constraints involving two real variables
\begin{notedef}\tt
  \begin{itemize}
  \item \hyperlink{eq:eqconstraint}{eq}, \hyperlink{geq:geqconstraint}{geq}, \hyperlink{leq:leqconstraint}{leq}
  \end{itemize}
\end{notedef}

%%%%%%%%%%%%%%%%%%%%%%%%%%%%%%%%%%%%%%%%%%%%%%%%%%%%%%%%%%%%%%%%%%%%%%%%%%%%%%%%%%%%%%%%%%%%%%%%%%%%%%%%%%%%%%%%%%%%%%%%%%%%%%%%%%%%%%%%%%%%%%%%%%%
%%%%%%%%%%%%%%%%%%%%%%%%%%%%%%%%%%%%%%%%%%%%%%%%%%%%% SET CONSTRAINT  %%%%%%%%%%%%%%%%%%%%%%%%%%%%%%%%%%%%%%%%%%%%%%%%%%%%%%%%%%%%%%%%%%%%%%%%%%%%%

\subsection{Constraints involving set variables}\label{model:setconstraints}\hypertarget{model:setconstraints}{}
%The simplest constraints are comparisons which are defined over expressions of variables such as linear combinations. The following comparison constraints can be accessed through the \texttt{Model} API:
Set constraints are illustrated on the \hyperlink{model:example2:ternarysteinerchoco}{ternary Steiner problem}. 
\begin{notedef}\tt
  \begin{itemize}
  \item \hyperlink{eqcard:eqcardconstraint}{eqCard}, \hyperlink{geqcard:geqcardconstraint}{geqCard}, \hyperlink{leqcard:leqcardconstraint}{leqCard}
  \item \hyperlink{member:memberconstraint}{member}, \hyperlink{notmember:notmemberconstraint}{notMember}
  \item \hyperlink{isincluded:isincludedconstraint}{isIncluded}, \hyperlink{isnotincluded:isnotincludedconstraint}{isNotIncluded}, \hyperlink{setdisjoint:setdisjointconstraint}{setDisjoint}
  \item \hyperlink{setinter:setinterconstraint}{setInter}, \hyperlink{setunion:setunionconstraint}{setUnion}
  \item \hyperlink{max:maxofaset}{max}, \hyperlink{min:minofaset}{min}
  \item \hyperlink{pack:packconstraint}{pack}
  \end{itemize}
\end{notedef}

%\hyperlink{max:maxconstraint}{max}, \hyperlink{min:minconstraint}{min},

%%%%%%%%%%%%%%%%%%%%%%%%%%%%%%%%%%%%%%%%%%%%%%%%%%%%%%%%%%%%%%%%%%%%%%%%%%%%%%%%%%%%%%%%%%%%%%%%%%%%%%%%%%%%%%%%%%%%%%%%%%%%%%%%%%%%%%%%%%%%%%%%%%%
%%%%%%%%%%%%%%%%%%%%%%%%%%%%%%%%%%%%%%%%%%%%%%%%%%%%% CHANNELING CONSTRAINT  %%%%%%%%%%%%%%%%%%%%%%%%%%%%%%%%%%%%%%%%%%%%%%%%%%%%%%%%%%%%%%%%%%%%%%

\subsection{Channeling constraints}\label{model:channelingconstraints}\hypertarget{model:channelingconstraints}{}
The use of a redundant model is a frequent technique to strengthen propagation or to get more freedom to design dedicated search heuristics. The following constraints allow to ensure integrity of different models:
\begin{notedef}\tt
  \begin{itemize}
  \item \hyperlink{inversechanneling:inversechannelingconstraint}{inverseChanneling}, \hyperlink{boolchanneling:boolchannelingconstraint}{boolChanneling}, \hyperlink{domainconstraint:domainconstraintconstraint}{domainConstraint}
  \end{itemize}
\end{notedef}
More complex channeling can be done using reified constraints (see Section \hyperlink{model:reifiedconstraints}{reification}) although they are less efficient. For example, to ensure that two variables are equal or not, one can reify the equality into a boolean variables :
\lstinputlisting{java/cchannelingreified.j2t}

%%%%%%%%%%%%%%%%%%%%%%%%%%%%%%%%%%%%%%%%%%%%%%%%%%%%%%%%%%%%%%%%%%%%%%%%%%%%%%%%%%%%%%%%%%%%%%%%%%%%%%%%%%%%%%%%%%%%%%%%%%%%%%%%%%%%%%%%%%%%%%%%%%%
%%%%%%%%%%%%%%%%%%%%%%%%%%%%%%%%%%%%%%%%%%%%%%%%%%%%% EXTENSIONS CONSTRAINT  %%%%%%%%%%%%%%%%%%%%%%%%%%%%%%%%%%%%%%%%%%%%%%%%%%%%%%%%%%%%%%%%%%%%%%

\subsection{Constraints in extension and relations}\label{model:arbitraryconstraintsinextension}\hypertarget{model:arbitraryconstraintsinextension}{}
Choco supports the statement of constraints defining arbitrary relations over two or more variables.
Such a relation may be defined by three means:
\begin{itemize}
	\item \textbf{feasible table:} the list of allowed tuples of values (that belong to the relation),
	\item \textbf{infeasible table:} the list of forbidden tuples of values (that do not belong to the relation),
	\item \textbf{predicate:} a method to be called in order to check whether a tuple of values belongs or not to the relation.
\end{itemize}
On the one hand, constraints based on tables may be rather memory consuming in case of large domains, although one relation table may be shared by several constraints. On the other hand, predicate constraints require little memory as they do not cache truth values, but imply some run-time overhead for calling the feasibility test. Table constraints are thus well suited for constraints over small domains; while predicate constraints are well suited for situations with large domains. 

Different levels of consistency can be enforce on constraints in extension: 
\begin{itemize}
\item several arc-consistency (AC) algorithms for binary relations
\item two AC algorithms for n-ary relations dedicated either to positive or to negative tables (relation defined by the allowed or forbidden tuples)
\item a weaker forward-checking (FC) algorithm for n-ary relations.
\end{itemize}

The Choco API for creating constraints in extension are as follows:
\begin{notedef}\tt
  \begin{itemize}
  \item \hyperlink{feaspairac:feaspairacconstraint}{feasPairAC}, \hyperlink{infeaspairac:infeaspairacconstraint}{infeasPairAC}, \hyperlink{relationpairac:relationpairacconstraint}{relationPairAC}
  \item \hyperlink{feastupleac:feastupleacconstraint}{feasTupleAC}, \hyperlink{infeastupleac:infeastupleacconstraint}{infeasTupleAC}, \hyperlink{relationtupleac:relationtupleacconstraint}{relationTupleAC}
  \item \hyperlink{feastuplefc:feastuplefcconstraint}{feasTupleFC}, \hyperlink{infeastuplefc:infeastuplefcconstraint}{infeasTupleFC}, \hyperlink{relationtuplefc:relationtuplefcconstraint}{relationTupleFC}
  \end{itemize}
\end{notedef}

\subsubsection{Relations.}
A same relation might be shared among several constraints, in this case it is highly recommended to create it first and then use the \hyperlink{relationpairac:relationpairacconstraint}{relationPairAC}, \hyperlink{relationtupleac:relationtupleacconstraint}{relationTupleAC}, or \hyperlink{relationtuplefc:relationtuplefcconstraint}{relationTupleFC} API  on the same relation for each constraint.

For binary relations, the following Choco API is provided:
\mylst{makeBinRelation(int[] min, int[] max, List<int[]>pairs, boolean feas)}

It returns a \texttt{BinRelation} giving a list of compatible (\texttt{feas=true}) or incompatible (\texttt{feas=false}) pairs of values. This relation can be applied to any pair of variables $(x_1,x_2)$ whose domains are included in the \texttt{min/max} intervals, i.e. such that:
$$\mathtt{min}[i] \le x_i.\mathtt{getInf}() \le x_i.\mathtt{getSup}() \le  \mathtt{max}[i],\quad \forall i.$$
Bounds \texttt{min/max} are mandatory in order to allow to compute the opposite of the relation if needed.

For n-ary relations, the corresponding Choco API is:
\mylst{makeLargeRelation(int[] min, int[] max, List<int[]> tuples, boolean feas);}
It returns a \texttt{LargeRelation}. If \texttt{feas=true}, the returned relation matches also the \texttt{IterLargeRelation} interface which provides constant time iteration abilities over tuples (for compatibility with the GAC algorithm used over feasible tuples).
\lstinputlisting{java/mlargerelation.j2t}

Lastly, some specific relations can be defined without storing the tuples, as in the following example (\texttt{TuplesTest} extends \texttt{LargeRelation}):
\lstinputlisting{java/mnotallequal.j2t}
Then, a \emph{NotAllEqual} constraint can be stated within the problem by:
\lstinputlisting{java/mrelationtuplefc.j2t}
%Again, for compatibility with the GAC algorithm invoked by relationTupleAC, such a relation has to match the \texttt{IterLargeRelation} interface for feasible tuples.


%%%%%%%%%%%%%%%%%%%%%%%%%%%%%%%%%%%%%%%%%%%%%%%%%%%%%%%%%%%%%%%%%%%%%%%%%%%%%%%%%%%%%%%%%%%%%%%%%%%%%%%%%%%%%%%%%%%%%%%%%%%%%%%%%%%%%%%%%%%%%%%%%%%
%%%%%%%%%%%%%%%%%%%%%%%%%%%%%%%%%%%%%%%%%%%%%%%%%%%%% REIFIED CONSTRAINT  %%%%%%%%%%%%%%%%%%%%%%%%%%%%%%%%%%%%%%%%%%%%%%%%%%%%%%%%%%%%%%%%%%%%%%%%%

\subsection{Reified constraints}\label{model:reifiedconstraints}\hypertarget{model:reifiedconstraints}{}
Constraints involved in another constraint are usually called reified constraints. Typical examples of reified constraints are
 constraints combined with logical operators, such as $(x \neq y) \lor (z \le 9)$.

%\subsubsection{To reify a constraint into a boolean variable.}\label{model:toreifyaconstraintintoabooleanvariable}\hypertarget{model:toreifyaconstraintintoabooleanvariable}{}
Choco provides a generic constraint to reify any constraints on integer variables or set variables into a boolean variable expressing its truth value:
\begin{notedef}\tt
  \begin{itemize}
  \item \hyperlink{reifiedconstraint:reifiedconstraintconstraint}{reifiedConstraint}, \hyperlink{reifiedand:reifiedandconstraint}{reifiedAnd}, \hyperlink{reifiedleftimp:reifiedleftimpconstraint}{reifiedLeftImp}, \hyperlink{reifiednot:reifiednotconstraint}{reifiedNot}, \hyperlink{reifiedor:reifiedorconstraint}{reifiedOr}, \hyperlink{reifiedrightimp:reifiedrightimpconstraint}{reifiedRightImp}, \hyperlink{reifiedxnor:reifiedxnorconstraint}{reifiedXnor}, \hyperlink{reifiedxor:reifiedxorconstraint}{reifiedXor}
  \end{itemize}
\end{notedef}
This mechanism can be used for example to model MaxCSP problems where the number of satisfied constraints has to be maximized.
It is also intended to give the freedom to the user to build complex reified constraints. However, Choco provides a more simple and direct API to build complex expressions using boolean operators:
\begin{notedef}\tt
  \begin{itemize}
  \item \hyperlink{and:andconstraint}{and}, \hyperlink{or:orconstraint}{or}, \hyperlink{implies:impliesconstraint}{implies}, \hyperlink{ifonlyif:ifonlyifconstraint}{ifOnlyIf}, \hyperlink{ifthenelse:ifthenelseconstraint}{ifThenElse}, \hyperlink{not:notconstraint}{not}
  \end{itemize}
\end{notedef}
Such an expression is represented as a tree of operators. The leaves of this tree are made of variables, constants or even traditional constraints. Variables and constants can be combined as \texttt{ExpressionVariable} using \hyperlink{model:expressionvariables}{operators} (e.g, \texttt{mult(10,abs(w))}), or using simple constraints (e.g., \texttt{leq(z,9)}), or even using global constraints (e.g, \texttt{alldifferent(vars)}).
The language available on expressions is therefore slightly richer and matches the language used in the \href{http://cpai.ucc.ie/08/}{Constraint Solver Competition 2008} of the CPAI workshop.

For example, the following expression
$$((x = 10 * |y|) \lor (z \le 9))\quad \iff\quad \texttt{alldifferent}(a,b,c)$$
could be represented by :
\begin{lstlisting}
	Constraint exp = ifOnlyIf( or( eq(x, mult(10, abs(y))), leq(z, 9) ), 
                               alldifferent(new IntegerVariable[]{a,b,c}) );
\end{lstlisting}


\subsubsection{Handling complex expressions.}\label{model:handlingcomplexexpressions}\hypertarget{model:handlingcomplexexpressions}{}
Expressions offer a more powerful modeling language than the one available via standard constraints. However, they 
can not be handled as efficiently as the standard constraints that embed a dedicated propagation algorithm. We therefore
recommend you to carefully check that you can not model the expression using the intensional constraints of Choco before using
expressions.
Inside the solver, expressions can be represented in two different ways that can be decided at the modeling level, using the following {\tt Model} API:
\begin{lstlisting}
  setDefaultExpressionDecomposition(boolean decomp);
\end{lstlisting}
or the option \hyperlink{edecomp:edecompoptions}{\tt Options.E\_DECOMP}.
\begin{itemize}
\item The first way (\texttt{decomp=false}) is to handle them as \hyperlink{model:arbitraryconstraintsinextension}{constraints in extension}. The expression is then used to check a tuple in a dynamic way just like a n-ary relation that is defined without listing all the possible tuples. The expression is then propagated using the GAC3rm algorithm. This is very powerful as arc-consistency is obtained on the corresponding constraints.
\item The second way (\texttt{decomp=true}) is to decompose the expression automatically by introducing intermediate variables and eventually the generic \hyperlink{reifiedintconstraint:reifiedintconstraintconstraint}{\tt reifiedIntConstraint}. By doing so, the level of pruning decreases but expressions of larger arity involving large domains can be represented.
\end{itemize}

%\subsubsection{Tell the solver how to consider an expression.}
%The default representation of expressions can be enforced using the following API  on the model object: 
%Parameter \emph{decomp} tells the solver whether expressions shoud be considered as extensional constraints (\texttt{decomp=false}) or decomposed 
Once the default representation is chosen, one can also make exception for a particular expression using options on \texttt{addConstraint}. 
For example, the following code tells the solver to decompose e1 and not e2 :
\begin{lstlisting}
	model.setDefaultExpressionDecomposition(false);
	IntegerVariable x = makeIntVar("x", 1, 3, Options.V_BOUND);
	IntegerVariable y = makeIntVar("y", 1, 3, Options.V_BOUND);
	IntegerVariable z = makeIntVar("z", 1, 3, Options.V_BOUND);

	Constraint e1 = or(lt(x, y), lt(y, x));
	model.addConstraint(Options.E_DECOMP, e1);
	
	Constraint e2 = or(lt(y, z), lt(z, y));
	model.addConstraint(e2);
\end{lstlisting}

\subsubsection{When and how should I use expressions ?}\label{model:whenshouldiuseexpressions}\hypertarget{model:whenshouldiuseexpressions}{}
An expression (represented in extension) should be used in the case of a complex logical relationship that involves \textbf{few different variables}, each of \textbf{small domain}, and if \textbf{arc consistency} is desired on those variables.
In such a case, an expression can even be more powerful than a model using intermediate variables and intensional constraints.
Imagine the following ``crazy'' example :
\begin{lstlisting}
 or( and( eq( abs(sub(div(x,50),div(y,50))),1), eq( abs(sub(mod(x,50),mod(y,50))),2)),
     and( eq( abs(sub(div(x,50),div(y,50))),2), eq( abs(sub(mod(x,50),mod(y,50))),1)))
\end{lstlisting}
This expression has a small arity: it involves only two variables $x$ and $y$.
Let assume that their domains has no more than 300 values, then such an expression should typically not be decomposed. Indeed, arc consistency will create many holes in the domains and filter much more than if the relation was decomposed.

Conversely, an expression should be decomposed as soon as it involves a large number of variables, or at least one variable with a large domain.

%%%%%%%%%%%%%%%%%%%%%%%%%%%%%%%%%%%%%%%%%%%%%%%%%%%%%%%%%%%%%%%%%%%%%%%%%%%%%%%%%%%%%%%%%%%%%%%%%%%%%%%%%%%%%%%%%%%%%%%%%%%%%%%%%%%%%%%%%%%%%%%%%%%
%%%%%%%%%%%%%%%%%%%%%%%%%%%%%%%%%%%%%%%%%%%%%%%%%%%%% GLOBAL CONSTRAINT  %%%%%%%%%%%%%%%%%%%%%%%%%%%%%%%%%%%%%%%%%%%%%%%%%%%%%%%%%%%%%%%%%%%%%%%%%%

\subsection{Global constraints}\label{model:advancedconstraints}\hypertarget{model:advancedconstraints}{}
Choco includes several \href{http://www.emn.fr/x-info/sdemasse/gccat/}{global constraints}. Those constraints accept any number of variables and offer dedicated filtering algorithms which are able to make deductions where a decomposed model would not.
For instance, constraint \texttt{alldifferent}$(a,b,c,d)$ with $a,b\in[1,4]$ and $c,d\in[3,4]$ allows to deduce that $a$ and $b$ cannot be instantiated to $3$ or $4$; such rule cannot be inferred by simple binary constraints. 

The up-to-date list of global constraints available in Choco can be found within the Javadoc API.
Most of these global constraints are listed below according to their application fields.
Details and examples can be found in Section \hyperlink{ch:constraints}{Elements of Choco/Constraints}.
\subsubsection{Value constraints}\label{model:valueconstraints}\hypertarget{model:valueconstraints}{}
Constraints that put a restriction on how values can be assigned to usually one or several collections of variables.
See also in Global Constraint Catalog: \href{http://www.emn.fr/x-info/sdemasse/gccat/Kvalue_constraint.html}{value constraint}.

\vspace{1em}\noindent\begin{notedef}\tt
  \begin{itemize}
  \item counting distinct values: 
\hyperlink{alldifferent:alldifferentconstraint}{allDifferent}, 
\hyperlink{atmostnvalue:atmostnvalueconstraint}{atMostNValue},
\hyperlink{increasingnvalue:increasingnvalueconstraint}{increasingnvalue},
  \item counting values: 
\hyperlink{occurrence:occurrenceconstraint}{occurrence},
\hyperlink{occurrencemax:occurrencemaxconstraint}{occurrenceMax},
\hyperlink{occurrencemin:occurrenceminconstraint}{occurrenceMin},
\hyperlink{globalcardinality:globalcardinalityconstraint}{globalCardinality},
  \item indexing values: 
\hyperlink{nth:nthconstraint}{nth} (element),
\hyperlink{max:maxconstraint}{max},
\hyperlink{min:minconstraint}{min},
  \item ordering: 
\hyperlink{sorting:sortingconstraint}{sorting},
\hyperlink{increasingnvalue:increasingnvalueconstraint}{increasingnvalue},
\hyperlink{lex:lexconstraint}{lex}, 
\hyperlink{lexeq:lexeqconstraint}{lexeq},
\hyperlink{leximin:leximinconstraint}{leximin},
\hyperlink{lexchain:lexchainconstraint}{lexChain},
\hyperlink{lexchaineq:lexchaineqconstraint}{lexChainEq},
  \item tuple matching: 
\hyperlink{feastupleac:feastupleacconstraint}{feasTupleAC},
\hyperlink{feastuplefc:feastuplefcconstraint}{feasTupleFC},
\hyperlink{infeastupleac:infeastupleacconstraint}{infeasTupleAC},
\hyperlink{infeastuplefc:infeastuplefcconstraint}{infeasTupleFC},
\hyperlink{relationtupleac:relationtupleacconstraint}{relationTupleAC},
\hyperlink{relationtuplefc:relationtuplefcconstraint}{relationTupleFC},
  \item pattern matching: 
\hyperlink{regular:regularconstraint}{regular},
\hyperlink{costregular:costregularconstraint}{costRegular},
\hyperlink{multicostregular:multicostregularconstraint}{multiCostRegular}, 
\hyperlink{stretchcyclic:stretchcyclicconstraint}{stretchCyclic}, 
\hyperlink{stretchpath:stretchpathconstraint}{stretchPath}, 
\hyperlink{tree:treeconstraint}{tree},
  \end{itemize}
\end{notedef}

\subsubsection{Boolean constraints}\label{model:logicconstraints}\hypertarget{model:logicconstraints}{}
Logical operations on boolean expressions.
See also in Global Constraint Catalog: \href{http://www.emn.fr/x-info/sdemasse/gccat/KBoolean_constraint.html}{boolean constraint}.

\vspace{1em}\noindent\begin{notedef}\tt
\hyperlink{and:andconstraint}{and},
\hyperlink{or:orconstraint}{or},
\hyperlink{clause:clauseconstraint}{clause},
\end{notedef}

\subsubsection{Channelling constraints}\label{model:channellingconstraints}\hypertarget{model:channellingconstraints}{}
Constraints linking two collections of variables (many-to-many) or indexing one among many variables (one-to-many).
See also in Global Constraint Catalog: \href{http://www.emn.fr/x-info/sdemasse/gccat/Kchannelling_constraint.html}{channelling constraint}.

 \vspace{1em}\noindent\begin{notedef}\tt
   \begin{itemize}
   \item one-to-many: 
 \hyperlink{domainconstraint:domainconstraintconstraint}{domainConstraint},
 \hyperlink{nth:nthconstraint}{nth} (element),
 \hyperlink{max:maxconstraint}{max},
 \hyperlink{min:minconstraint}{min},
   \item many-to-many: 
 \hyperlink{inversechanneling:inversechannelingconstraint}{inverseChanneling},
 \hyperlink{inverseset:inversesetconstraint}{inverseset},
 \hyperlink{sorting:sortingconstraint}{sorting},
 \end{itemize}
 \end{notedef}

\subsubsection{Optimization constraints}\label{model:optimizationconstraints}\hypertarget{model:optimizationconstraints}{}
Constraints channelling a variable to the sum of the weights of a collection of variable-value assignments.
See also in Global Constraint Catalog: \href{http://www.emn.fr/x-info/sdemasse/gccat/Kcost_filtering_constraint.html}{cost-filtering constraint}.
\vspace{1em}\noindent\begin{notedef}\tt
 \begin{itemize}
  \item one cost: 
\hyperlink{occurrence:occurrenceconstraint}{occurrence},
\hyperlink{occurrencemax:occurrencemaxconstraint}{occurrenceMax},
\hyperlink{occurrencemin:occurrenceminconstraint}{occurrenceMin},
\hyperlink{knapsackproblem:knapsackproblemconstraint}{knapsackProblem},
\hyperlink{equation:equationconstraint}{equation},
\hyperlink{costregular:costregularconstraint}{costRegular},
\hyperlink{tree:treeconstraint}{tree},
 \item several costs:
\hyperlink{globalcardinality:globalcardinalityconstraint}{globalCardinality},
\hyperlink{multicostregular:multicostregularconstraint}{multiCostRegular}, 
 \end{itemize}
\end{notedef}

\subsubsection{Packing constraints (capacitated resources)}\label{model:packingconstraints}\hypertarget{model:packingconstraints}{}
Constraints involving items to be packed in bins without overlapping. More generaly, any constraints modelling the concurrent assignment of objects to one or several capacitated resources.
See also in Global Constraint Catalog: \href{http://www.emn.fr/x-info/sdemasse/gccat/Kresource_constraint.html}{resource constraint}.

\vspace{1em}\noindent\begin{notedef}\tt
   \begin{itemize}
   \item packing problems: 
\hyperlink{equation:equationconstraint}{equation},
\hyperlink{knapsackproblem:knapsackproblemconstraint}{knapsackProblem},
\hyperlink{pack:packconstraint}{pack} (bin-packing),
   \item geometric placement problems: 
\hyperlink{geost:geostconstraint}{geost}, 
   \item scheduling problems: 
\hyperlink{disjunctive:disjunctiveconstraint}{disjunctive}, 
\hyperlink{cumulative:cumulativeconstraint}{cumulative}, 
   \item timetabling problems: 
\hyperlink{costregular:costregularconstraint}{costRegular},
\hyperlink{multicostregular:multicostregularconstraint}{multiCostRegular}, 
 \end{itemize}
\end{notedef}

\subsubsection{Scheduling constraints (time assignment)}\label{model:schedulingconstraints}\hypertarget{model:schedulingconstraints}{}
Constraints involving tasks to be scheduled over a time horizon.
See also \hyperlink{schedulinganduseofthecumulative:schedulinganduseofthecumulativeconstraint}{scheduling application} and in Global Constraint Catalog: \href{http://www.emn.fr/x-info/sdemasse/gccat/Kscheduling_constraint.html}{scheduling constraint}.

\vspace{1em}\noindent\begin{notedef}\tt
   \begin{itemize}
   \item temporal constraints: 
\hyperlink{preceding:precedingconstraint}{preceding}, 
\hyperlink{precedencedisjoint:precedencedisjointconstraint}{precedenceDisjoint}, 
\hyperlink{precedenceimplied:precedenceimpliedconstraint}{precedenceImplied}, 
\hyperlink{precedencereified:precedencereifiedconstraint}{precedenceReified},
\hyperlink{forbiddeninterval:forbiddenintervalconstraint}{forbiddenInterval},
\hyperlink{tree:treeconstraint}{tree},
   \item resource constraints: 
\hyperlink{cumulative:cumulativeconstraint}{cumulative}, 
\hyperlink{disjunctive:disjunctiveconstraint}{disjunctive}, 
\hyperlink{geost:geostconstraint}{geost}, 
 \end{itemize}
\end{notedef}

%\part{solver}
\label{solver}
\hypertarget{solver}{}


\chapter{The solver}\label{solver:thesolver}\hypertarget{solver:thesolver}{}

%\section{How to create a solver}\label{solver:howtocreateasolver}\hypertarget{solver:howtocreateasolver}{}

To create a {\tt Solver}, one just needs to create a new object as follow:
\begin{lstlisting}
Solver solver = new CPSolver();
\end{lstlisting}
By this, a Constraint Programming (CP) {\tt Solver} object is created. 

%\section{Read a model}\label{solver:readamodel}\hypertarget{solver:readamodel}{}
The solver gives an API to read a model. The reading of a model is compulsory and must be done after the entire definition of the model. 
\begin{lstlisting}
solver.read(model);
\end{lstlisting}
The reading is divided in 2 parts: \hyperlink{solver:variablesreading}{variables reading} and \hyperlink{solver:constraintsreading}{constraints reading}.

\section{Variables reading}\label{solver:variablesreading}\hypertarget{solver:variablesreading}{}
The solver iterates over the variables of the Model to create solver-specific variables and domains (as defined in the model). 
Thus, three types of variables can be created: integer variables, real variables and set variables. 
Depending on the constructor, the correct domain is created (like bounded domain or enumerated domain for integer variables). 

\begin{note}
\textbf{Bound variables} are related to large domains which are only represented by their lower and upper bounds. The domain is encoded in a space efficient way and propagation events only concern bound updates. Value removals between the bounds are therefore ignored (\emph{holes} are not considered). The level of consistency achieved by most constraints on these variables is called \emph{bound-consistency}.

On the contrary, the domain of an \textbf{enumerated variable} is explicitly represented and every value is considered while pruning. Basic constraints are therefore often able to achieve \emph{arc-consistency} on enumerated variables (except for NP global constraint such as the cumulative constraint). Remember that switching from an enumerated variable to a bounded variables decrease the level of propagation achieved by the system.
\end{note}

%\begin{note}
Model variables and Solver variables are distinct. Solver variables are solver representation of the model variables. One can't access to variable value directly from the model variable. To access to a model variable thanks to the solver, use the following \texttt{Solver} API: \mylst{getVar(Variable v);}
%\end{note}

\subsection{Solver and IntegerVariables}\label{solver:solverandintegervariables}\hypertarget{solver:solverandintegervariables}{}

A model integer variable can be accessed by the method \textbf{\tt getVar(IntegerVariable v)} which returns a \textbf{\tt IntDomainVar} object:
\begin{lstlisting}
  IntegerVariable x = makeEnumIntVar("x", 1, 100);  // model variable
  IntDomainVar xOnSolver = solver.getVar(x);  // solver variable
\end{lstlisting}

The state of an \texttt{IntDomainVar} can be accessed through the main following public methods :

\noindent\begin{tabular}{p{.3\linewidth}p{.7\linewidth}}
  \hline
  \texttt{IntDomainVar} API &  description \\
  \hline
	\mylst{hasEnumeratedDomain()} &checks if the variable is an enumerated or a bound one\\
	\mylst{getInf()} &returns the lower bound of the variable\\
	\mylst{getSup()} &returns the upper bound of the variable\\
	\mylst{getVal()} &returns the value if it is instantiated\\
	\mylst{isInstantiated()} &checks if the domain is reduced to a singleton\\
	\mylst{canBeInstantiatedTo(int v)} &checks if the value \emph{v} is contained in the domain of the variable\\
	\mylst{getDomainSize()} &returns the current size of the domain\\
  \hline\\
\end{tabular}

For more informations on advanced uses of such \texttt{IntDomainVar}, see \hyperlink{advanced}{advanced uses}.

\subsection{Solver and SetVariables}\label{solver:solverandsetvariables}\hypertarget{solver:solverandsetvariables}{}

A model set variable can be access by the method \textbf{\tt getVar(SetVariable v)} which returns a \textbf{\tt SetVar} object:
\begin{lstlisting}
	SetVariable x = makeBoundSetVar("x", 1, 40); // model variable
	SetVar xOnSolver = solver.getVar(x); // solver variable
\end{lstlisting}
A set variable on integer values between $[1,n]$ has $2^{n}$ values (every possible subsets of $\{1..n\}$). This makes an exponential number of values and the domain is represented with two bounds corresponding to the intersection of all possible sets (called the kernel) and the union of all possible sets (called the envelope) which are the possible candidate values for the variable.

The state of a \texttt{SetVar} can be accessed through the main following public methods on the SetVar class:

\noindent\begin{tabular}{p{.3\linewidth}p{.7\linewidth}}
  \hline
  \texttt{SetVar} API &  description \\
  \hline
	\mylst{getCard()} &returns the \texttt{IntDomainVar} representing the cardinality of the set variable\\
	\mylst{isInDomainKernel(int v)} &checks if value \emph{v} is contained in the current kernel\\
	\mylst{isInDomainEnveloppe(int v)} &checks if value \emph{v} is contained in the current envelope\\
	\mylst{getDomain()} &returns the domain of the variable as a \texttt{SetDomain}. Iterators on envelope or kernel can than be called\\
	\mylst{getKernelDomainSize()} &returns the size of the kernel\\
	\mylst{getEnveloppeDomainSize()} &returns the size of the envelope\\
	\mylst{getEnveloppeInf()} &returns the first available value of the envelope\\
	\mylst{getEnveloppeSup()} &returns the last available value of the envelope\\
	\mylst{getKernelInf()} &returns the first available value of the kernel\\
	\mylst{getKernelSup()} &returns the last available value of the kernel\\
	\mylst{getValue()} &returns a table of integers \texttt{int[]} containing the current domain\\
  \hline\\
\end{tabular}


For more informations on advanced uses of such \texttt{SetVar}, see \hyperlink{advanced}{advanced uses}.

\subsection{Solver and RealVariables}\label{solver:solverandrealvariables}\hypertarget{solver:solverandrealvariables}{}

\begin{note}
\emph{Real variables are still under development but can be used to solve toy problems such as small systems of equations.}
\end{note}
 
A model real variable can be access by the method \textbf{\tt getVar(RealVariable v)} which returns a \texttt{RealVar} object:
\begin{lstlisting}
	RealVariable x = makeRealVar("x", 1.0, 3.0); // model variable
	RealVar xOnSolver = s.getVar(x); // solver variable
\end{lstlisting}

Continuous variables are useful for non linear equation systems which are encountered in physics for example.

\noindent\begin{tabular}{p{.3\linewidth}p{.7\linewidth}}
  \hline
  \texttt{RealVar} API &  description \\
  \hline
	\mylst{getInf()} &returns the lower bound of the variable (\texttt{double})\\
	\mylst{getSup()} &returns the upper bound of the variable (\texttt{double})\\
	\mylst{isInstantiated()} &checks if the domain of a variable is reduced to a canonical interval. A canonical interval indicates that the domain has reached the precision given by the user or the solver\\
  \hline\\
\end{tabular}


For more informations on advanced uses of such \texttt{RealVar}, see \hyperlink{advanced}{advanced uses}.

\section{Constraints reading}\label{solver:constraintsreading}\hypertarget{solver:constraintsreading}{}
After variables, the Solver iterates over the constraints added to the Model. It creates Solver constraints that encapsulates a filtering algorithm which are called when a propagation step occur or when external events happen on the variables belonging to the constraint, such as value removals or bounds modifications. And it add it to the constraint network. 

\section{Search Strategy}\label{solver:searchstrategy}\hypertarget{solver:searchstrategy}{}

A key ingredient of any constraint approach is a clever branching strategy. The construction of the search tree is done according to a series of \textit{branching objects} (that plays the role of achieving intermediate goals in logic programming). The user may specify the sequence of branching objects to be used to build the search tree. A common way to branch in CP is by assigning variables to values. We will present in this section how to define your branching strategies with existing variables and values selectors/iterators. But first...

\subsection{Override the default search stragtegy}\label{solver:overridethedefaultsearchstrategy}\hypertarget{solver:overridethedefaultsearchstrategy}{}


Basically, a search strategy is the composition of three objects: a branching strategy, a variable selector and a value selector. Some branching simply assign a selected value to a selected variable, like \hyperlink{assignvar:assignvarbranchstrat}{AssignVar}, others branching strategies embed the variable selector, like \hyperlink{domoverwdeg:domoverwdegbranchstrat}{DomOverWDegBranchingNew}, or more, like  \hyperlink{impact:impactbranchstrat}{ImpactBasedBranching}.

The default branchings are: 

\noindent\begin{tabular}{p{.4\linewidth}p{.6\linewidth}}
\hline
Variable &  Default strategy \\
\hline
Integer & \hyperlink{domoverwdeg:domoverwdegbranchstrat}{DomOverWDegBranchingNew} +\hyperlink{increasingdomain:increasingdomainvaliterator}{IncreasingDomain}\\
Set &   \hyperlink{assignsetvar:assignsetvarbranchstrat}{AssignSetVar} + \hyperlink{mindomset:mindomsetvarselector}{MinDomainSet} + \hyperlink{minenv:minenvvalselector}{MinEnv} \\
 Real &  \hyperlink{assigninterval:assignintervalbranchstrat}{AssignInterval} + \hyperlink{cyclicrealvarselector:cyclicrealvarselectorvarselector}{CyclicRealVarSelector}+ \hyperlink{realincreasingdomain:realincreasingdomainvaliterator}{RealIncreasingDomain} \\
\hline\\
\end{tabular}

There are two ways to custom a search strategy: use the ones define in the factory  \texttt{BranchingFactory} or compose it. 
A branching strategy can be set or add to the previous one using the following API (must be done before calling the method \mylst{solve()}):

  \mylst{solver.addGoal(AbstractIntBranchingStrategy branching)} 

\noindent to clear the list of goals, use:

  \mylst{solver.clearGoals()} 

You might want to apply different heuristics to different set of variables of the problem. In that case, the search is viewed as a sequence of branching objects (or goals). Up to now, we only had one branching or one goal including all the variables of the problem but several goals can be used.

Adding a new goal is made through the solver with the \mylst{solver.addGoal(AbstractIntBranchingStrategy branching)} method. 

The following example add three branching objects on integer variables \emph{vars1}, \emph{vars2} and set variables \emph{svars} to solver \emph{s}. The first two branchings are both \texttt{AssignVar} but use two different variable/values selection strategies:
\begin{lstlisting}
  s.addGoal(new AssignVar(new MinDomain(s,s.getVar(vars1)), new IncreasingDomain()));
  s.addGoal(new AssignVar(new DomOverDeg(s,s.getVar(vars2)),new DecreasingDomain());
  s.addGoal(new AssignSetVar(new MinDomSet(s,s.getVar(svars)), new MinEnv(s)));
  s.solve();
\end{lstlisting}

\begin{note}
Strategies are made of \texttt{Solver} variables (not \texttt{Model} variables).
\end{note}

\subsubsection{Branching strategy.}\label{solver:branchstrat}\hypertarget{solver:branchstrat}{}
It defines the way to take a decision in a tree search node.
  
\noindent The \textbf{branching strategies} currently available in Choco are the following: 
\begin{notedef}\tt
\hyperlink{assigninterval:assignintervalbranchstrat}{AssignInterval}, \hyperlink{assignorforbidintvarval:assignorforbidintvarvalbranchstrat}{AssignOrForbidIntVarVal}, \hyperlink{assignorforbidintvarvalpair:assignorforbidintvarvalpairbranchstrat}{AssignOrForbidIntVarValPair}, \hyperlink{assignsetvar:assignsetvarbranchstrat}{AssignSetVar}, \hyperlink{assignvar:assignvarbranchstrat}{AssignVar}, \hyperlink{domoverwdeg:domoverwdegbranchstrat}{DomOverWDegBranchingNew}, \hyperlink{domoverwdegbin:domoverwdegbinbranchstrat}{DomOverWDegBinBranchingNew}, \hyperlink{impact:impactbranchstrat}{ImpactBasedBranching}, \hyperlink{packdynremovals:packdynremovalsbranchstrat}{PackDynRemovals}, \hyperlink{settimes:settimesbranchstrat}{SetTimes}, \hyperlink{taskdomoverwdeg:taskdomoverwdegbranchstrat}{TaskOverWDegBinBranching}.
\end{notedef}    


\subsubsection{Variable selector.}\label{solver:variableselector}\hypertarget{solver:variableselector}{}
It defines the way to choose the non instantiated variable on which the next decision will be made.

\noindent The \textbf{integer variable selectors} currently available in Choco are the following: 
\begin{notedef}\tt
\hyperlink{compositeintvarselector:compositeintvarselectorvarselector}{CompositeIntVarSelector}, \hyperlink{lexintvarselector:lexintvarselectorvarselector}{LexIntVarSelector}, \hyperlink{maxdomain:maxdomainvarselector}{MaxDomain}, \hyperlink{maxregret:maxregretvarselector}{MaxRegret}, \hyperlink{maxvaldomain:maxvaldomainvarselector}{MaxValueDomain}, \hyperlink{mindomain:mindomainvarselector}{MinDomain}, \hyperlink{minvaldomain:minvaldomainvarselector}{MinValueDomain}, \hyperlink{mostconstrained:mostconstrainedvarselector}{MostConstrained},  \hyperlink{randomvarint:randomvarintvarselector}{RandomIntVarSelector},  \hyperlink{staticvarorder:staticvarordervarselector}{StaticVarOrder}
\end{notedef}

\noindent The \textbf{set variable selectors} currently available in Choco are the following: 
\begin{notedef}\tt
\hyperlink{maxdomset:maxdomsetvarselector}{MaxDomainSet}, \hyperlink{maxregretset:maxregretsetvarselector}{MaxRegretSet}, \hyperlink{maxvaldomainset:maxvaldomainsetvarselector}{MaxValueDomainSet}, \hyperlink{mindomset:mindomsetvarselector}{MinDomainSet}, \hyperlink{minvaldomainset:minvaldomainsetvarselector}{MinValueDomainSet}, \hyperlink{mostconstrainedset:mostconstrainedsetvarselector}{MostConstrainedSet},  \hyperlink{randomvarset:randomvarsetvarselector}{RandomSetVarSelector},  \hyperlink{staticsetvarorder:staticsetvarordervarselector}{StaticSetVarOrder}
\end{notedef}

\noindent The \textbf{real variable selector} currently available in Choco is the following: 
\begin{notedef}\tt
\hyperlink{cyclicrealvarselector:cyclicrealvarselectorvarselector}{CyclicRealVarSelector}
\end{notedef}

\subsubsection{Value iterator}\label{solver:valueiterator}\hypertarget{solver:valueiterator}{}
Once the variable has been choosen, the Solver has to compute its value. The first way to do it is to schedule the value once and give an iterator to the solver.

\noindent The \textbf{integer value iterator} currently available in Choco are the following: 
\begin{notedef}\tt
\hyperlink{decreasingdomain:decreasingdomainvaliterator}{DecreasingDomain}, \hyperlink{increasingdomain:increasingdomainvaliterator}{IncreasingDomain}
\end{notedef}

\noindent The \textbf{real value iterator} currently available in Choco is the following: 
\begin{notedef}\tt
\hyperlink{realincreasingdomain:realincreasingdomainvaliterator}{RealIncreasingDomain}
\end{notedef}




\subsubsection{Value selector}\label{solver:valueselector}\hypertarget{solver:valueselector}{}
The second way to do it is to compute the following value at each call.

\noindent The \textbf{integer value selector} currently available in Choco are the following: 
\begin{notedef}\tt
\hyperlink{bestfit:bestfitvalselector}{BestFit}, \hyperlink{costregularvalselector:costregularvalselectorvalselector}{CostRegularValSelector}, \hyperlink{fcostregularvalselector:fcostregularvalselectorvalselector}{FCostRegularValSelector}, \hyperlink{maxval:maxvalvalselector}{MaxVal}, \hyperlink{mcrvalselector:mcrvalselectorvalselector}{MCRValSelector}, \hyperlink{midval:midvalvalselector}{MidVal}, \hyperlink{minval:minvalvalselector}{MinVal}
\end{notedef}

\noindent The \textbf{set value selector} currently available in Choco is the following: 
\begin{notedef}\tt
\hyperlink{minenv:minenvvalselector}{MinEnv}, \hyperlink{randomsetvalselector:randomsetvalselectorvalselector}{RandomSetValSelector}
\end{notedef}

\subsection{Why is it important to define a search strategy ?}\label{solver:whyisitimportanttodefineasearchstrategy}\hypertarget{solver:whyisitimportanttodefineasearchstrategy}{}

In a partial instantiation, when a fix point has been reached, the Solver needs to take a decision to resume the search. The way decisions are chosen has a \textbf{real impact on the resolution step efficient}. 
\begin{note}
\emph{The search strategy should not be overlooked!!}
An adapted search strategy can reduce: the execution time, the number of node expanded, the number of backtrack done.
\end{note}
Let see that small example:
\begin{lstlisting}
	Model m = new CPModel();
        int n = 1000;
        IntegerVariable var = Choco.makeIntVar("var", 0, 2);
        IntegerVariable[] bi = Choco.makeBooleanVarArray("b", n);
        m.addConstraint(Choco.eq(var, Choco.sum(bi)));

        Solver badStrat = new CPSolver();
        badStrat.read(m);
        badStrat.addGoal(
                new AssignVar(
                        new MinDomain(badStrat), 
                        new IncreasingDomain()
                ));
        badStrat.solve();
        badStrat.printRuntimeStatistics();

        Solver goodStrat = new CPSolver();
        goodStrat.read(m);
        goodStrat.addGoal(
                new AssignVar(
                        new MinDomain(goodStrat, goodStrat.getVar(new IntegerVariable[]{var})), 
                        new DecreasingDomain()
                ));
        goodStrat.solve();
        goodStrat.printRuntimeStatistics();
\end{lstlisting}

This model ensures that $var = b_{0} + b_{1} + \ldots + b_{1000}$ where \emph{var} has a small domain and $b_{i}$ is a binary variable. The propagation has no effect on any domain and a fix point is reached at the beginning of the search. So, a decision has to be done choosing a variable and its value. If the variable selector is set to \texttt{MinDomain} (see below), the solver will iterate over the variables, starting by the 1000 binary variables and ending with \emph{var}, and 1001 nodes will be created.

\subsection{Restarts}\label{solver:restarts}\hypertarget{solver:restarts}{}

You can set geometric restarts by using the following API available on the solver:
\begin{lstlisting}
setGeometricRestart(int base, double grow);
setGeometricRestart(int base, double grow, int restartLimit);
\end{lstlisting}
It performs a search with restarts regarding the number of backtrack. An initial allowed number of backtrack is given (parameter base) and once this limit is reached a restart is performed and the new limit imposed to the search is increased by multiplying the previous limit with the parameter grow. restartLimit parameter states the maximum number of restarts. Restart strategies makes really sense with strategies that make choices based on the past experience of the search : \texttt{DomOverWdeg} or Impact based search. It could also be used with a random heuristic
\begin{lstlisting}
	CPSolver s = new CPSolver();
	s.read(model);
	
	s.setGeometricRestart(14, 1.5d);
	s.setFirstSolution(true);
	s.generateSearchStrategy();
	s.attachGoal(new DomOverWDegBranching(s, new IncreasingDomain()));
	s.launch();
\end{lstlisting}

You can also set Luby restarts by using the following API available on the solver:
\begin{lstlisting}
setLubyRestart(int base);
setLubyRestart(int base, int grow);
setLubyRestart(int base, int grow, int restartLimit);
\end{lstlisting}
it performs a search with restarts regarding the number of backtracks. One way to describe this strategy is to say that all run lengths are power of two, and that each time a pair of runs of a given length has been completed, a run of twice that length is immediatly executed. The limit is equals to \emph{length*base}.
\begin{itemize}
	\item \textbf{example with growing factor of 2 : [1, 1, 2, 1, 1, 2, 4, 1, 1, 2, 1, 1, 2, 4, 8, 1,...]}
	\item \textbf{example with growing factor of 3 : [1, 1, 1, 3, 1, 1, 1, 3, 9,...]}
\end{itemize}

\begin{lstlisting}
	CPSolver s = new CPSolver();
	s.read(model);
	
	s.setLubyRestart(50, 2, 100);
	s.setFirstSolution(true);
	s.generateSearchStrategy();
	s.attachGoal(new DomOverWDegBranching(s, new IncreasingDomain()));
	s.launch();
\end{lstlisting}

\section{Limiting Search Space}\label{solver:limitingsearchspace}\hypertarget{solver:limitingsearchspace}{}
The Solver class provides some limits on the search strategy that you can fix or just monitor.
Limits may be imposed on the search algorithm to avoid spending too much time in the exploration. The limits are updated and checked each time a new node is created. It has to be specified before the resolution. 
After having created the solver, you can specify whether or not you want to fix a limit:

\begin{description}
\item[time limit] State a time limit on tree search. When the execution time is equal to the time limit, the search stops whatever a solution is found or not. You can define a time limit with the following API : \mylst{setTimeLimit(int timeLimit)} where unit is millisecond. Or just monitor (or not) the search time with the API : \mylst{monitorTimeLimit(boolean b)}. The default value is set to \texttt{true}. Finally, you can get the time limit, once the solve method has been called, with the API: \mylst{getTimeCount()} 
\item[node limit] State a node limit on tree search. When the number of nodes explored is equal to the node limit, the search stops whatever a solution is found or not. You can define a node limit with the following API: \mylst{setNodeLimit(int nodeLimit)} where unit is the number of nodes. Or just monitor (or not) the number of nodes explored with the API: \mylst{monitorNodeLimit(boolean b)}. The default value is set to \texttt{true}. Finally, you can get the node limit, once the solve method has been called, with the API: \mylst{getNodeCount()} 
\item[backtrack limit] State a backtrack limit on tree search. When the number of backtracks done is equal to the backtrack limit, the search stops whatever a solution is found or not. You can define a backtrack limit with the following API: \mylst{setBackTrackLimit(int backtrackLimit)} where unit is the number of backtracks. Or just monitor (or not) the number of backtrack done with the API: \mylst{monitorBackTrackLimit(boolean b)}. The default value is set to \texttt{false}. Finally, you can get the backtrack limit, once the solve method has been called, with the API: \mylst{getBackTrackCount()} 
\item[fail limit] State a fail limit on tree search. When the number of failure is equal to the fail limit, the search stops whatever a solution is found or not. You can define a fail limit with the following API : \mylst{setFailLimit(int failLimit)} where unit is the number of failure. Or just monitor (or not) the number of failure encountered with the API : \mylst{monitorFailLimit(boolean b)}. The default value is set to \texttt{false}. Finally, you can get the fail limit, once the solve method has been called, with the API : \mylst{getFailCount()} 
%\item[CPU time limit] State a CPU limit on tree search. When the CPU time (user + system) is equal to the CPU time limit, the search stops whatever a solution is found or not. You can define a CPU time limit with the following API: \mylst{setCpuTimeLimit(int timeLimit)} where unit is millisecond. Or just monitor (or not) the search time with the API: \mylst{monitorCpuTimeLimit(boolean b)}. The default value is set to \texttt{false}. Finally, you can get the CPU time limit, once the solve method has been called, with the API: \mylst{getCpuTimeCount()} 
\end{description}

\todo{add example}

\section{Solve a problem}\label{solver:solveaproblem}\hypertarget{solver:solveaproblem}{}
As Solver is the second element of a Choco program, the control of the search process without using predefined tools is made on the Solver.

\noindent\begin{tabular}{p{.4\linewidth}p{.6\linewidth}}
  \hline
  \texttt{Solver} API & description \\
  \hline
      \mylst{solve()} &  Compute the first solution of the Model, if the Model is feasible. \\
      \mylst{solve(boolean all)} &  If \emph{all} is set to true, computes all solutions of the Model, if the Model is feasible. \\
      \mylst{solveAll()} &  Computes all the solution of the Model, if the Model is feasible. \\
      \mylst{propagate()} &  Computes initial propagation of the Model, and reachs the first Fix Point. It reduces variables Domain through constraints linked and other variables domain. Can throw a \texttt{ContradictionException} if the Solver detects a contradiction in the Model. \\
      \mylst{maximize(Var obj, boolean restart)} &  Allows user to find a solution that maximizing the objective varible \emph{obj}. The optimization finds a first solution then finds a new solution that improves \emph{obj} and so on till no other solution can be found that improves \emph{obj}. Parameter \emph{restart} is a boolean indicating whether the Solver will restart the search after each solution found (if set to \texttt{true}) or if it will keep backtracking from the leaf of the last solution found. See \hyperlink{solver:optimization}{example}. \textbf{Beware}: the variable \emph{obj} expected must be a Solver variable and not a Model variable. \\
      \mylst{minimize(Var obj, boolean restart)} &  Allows user to find a solution that minimizing the objective varible \emph{obj}. The optimization finds a first solution then finds a new solution that improves \emph{obj} and so on till no other solution can be found that improves \emph{obj}. Parameter \emph{restart} is a boolean indicating whether the Solver will restart the search after each solution found (if set to \texttt{true}) or if it will keep backtracking from the leaf of the last solution found. See \hyperlink{solver:optimization}{example}. \textbf{Beware}: the variable \emph{obj} expected must be a Solver variable and not a Model variable. \\
      \mylst{nextSolution()} &  Allows the Solver to find the next solution, if one or more solution have already been find with \texttt{solve()} or \texttt{nextSolution()}. \\
      \mylst{isFeasible()} &  Indicates whether or not the Model has at least one solution. \\
      \hline\\
	\end{tabular}

\subsection{Solver settings}\label{solver:solversettings}\hypertarget{solver:solversettings}{}

\subsubsection{Logs}\label{solver:logs}\hypertarget{solver:logs}{}
A logging class is instrumented in order to produce trace statements throughout search: ChocoLogging. The verbosity level of the solver can be set, by the following static method
\begin{lstlisting}
	ChocoLogging.toVerbose();
	// And after solver.solve()
	ChocoLogging.flushLogs();
\end{lstlisting}

The code above ensure that messages are printed in order to describe the construction of the search tree.

Six verbosity levels are available:

\noindent\begin{tabular}{p{.4\linewidth}p{.6\linewidth}}
  \hline
  Level & prints... \\
  \hline
 \texttt{ChocoLogging.toSilent()} & display only severe messages from core loggers and warning messages otherwise\\
 \ \texttt{ChocoLogging.toQuiet()} & display only severe messages from core loggers and info messages otherwise\\
 \texttt{ChocoLogging.toDefault()} & display information about initial and final state of the search\\
 \texttt{ChocoLogging.toVerbose()} & display search information at regular node intervals\\
 \texttt{ChocoLogging.toSolution()} & display all solutions\\
 \texttt{ChocoLogging.toSearch()} & display the search tree\\
\hline\\
\end{tabular}

Note that in the case of a verbosity greater or equals to \texttt{toVerbose()}, the regular search information step is set to 1000, by default. You can change this value, using:
\begin{lstlisting}
  ChocoLogging.setEveryXNodes(20000);
\end{lstlisting}
 

Note that in the case of verbosity \texttt{toSearch()}, trace statements are printed up to a maximal depth in the search tree. The default value is set to 25, but you can change the value of this threshold, say to 10, with the following setter method:
\begin{lstlisting}
  ChocoLogging.setLoggingMaxDepth(10);
\end{lstlisting}

\subsection{Optimization}\label{solver:optimization}\hypertarget{solver:optimization}{}
\todo{to introduce}
\begin{lstlisting}
  Model m = new CPModel();
  IntegerVariable obj1 = makeEnumIntVar("obj1", 0, 7);
  IntegerVariable obj2 = makeEnumIntVar("obj1", 0, 5);
  IntegerVariable obj3 = makeEnumIntVar("obj1", 0, 3);
  IntegerVariable cost = makeBoundIntVar("cout", 0, 1000000);
  int capacity = 34;
  int[] volumes = new int[]{7, 5, 3};
  int[] energy = new int[]{6, 4, 2};
  // capacity constraint
  m.addConstraint(leq(scalar(volumes, new IntegerVariable[]{obj1, obj2, obj3}), capacity));
	
  // objective function
  m.addConstraint(eq(scalar(energy, new IntegerVariable[]{obj1, obj2, obj3}), cost));
  
  Solver s = new CPSolver();
  s.read(m);
  
  s.maximize(s.getVar(cost), false);
\end{lstlisting}
\label{doc:solver}\hypertarget{doc:solver}{}
%%\part{constraints}
\label{constraints}
\hypertarget{constraints}{}



\chapter{Constraints in alphabetical order}\label{constraints:constraintsinalphabeticalorder}\hypertarget{constraints:constraintsinalphabeticalorder}{}

Choco offers a large number of available constraints.
In this part, you will find a description of each defined constraint, with API, options and examples.
\begin{note}
	 \hyperlink{abs:absconstraint}{abs},
	 \hyperlink{alldifferent:alldifferentconstraint}{allDifferent},
	 \hyperlink{and:andconstraint}{and},
	 \hyperlink{atmostnvalue:atmostnvalueconstraint}{atMostNValue},
	 \hyperlink{boolchanneling:boolchannelingconstraint}{boolChanneling},
	 \hyperlink{cumulative:cumulativeconstraint}{cumulative},
	 \hyperlink{disjunctive:disjunctiveconstraint}{disjunctive},
	 \hyperlink{distanceeq:distanceeqconstraint}{distanceEQ},
	 \hyperlink{distancegt:distancegtconstraint}{distanceGT},
	 \hyperlink{distancelt:distanceltconstraint}{distanceLT},
	 \hyperlink{distanceneq:distanceneqconstraint}{distanceNEQ},
	 \hyperlink{eq:eqconstraint}{eq},
	 \hyperlink{eqcard:eqcardconstraint}{eqCard},
	 \hyperlink{equation:equationconstraint}{equation},
	 \hyperlink{false:falseconstraint}{FALSE},
	 \hyperlink{feaspairac:feaspairacconstraint}{feasPairAC},
	 \hyperlink{feastupleac:feastupleacconstraint}{feasTupleAC},
	 \hyperlink{feastuplefc:feastuplefcconstraint}{feasTupleFC},
	 \hyperlink{geost:geostconstraint}{geost},
	 \hyperlink{geq:geqconstraint}{geq},
	 \hyperlink{geqcard:geqcardconstraint}{geqCard},
	 \hyperlink{globalcardinality:globalcardinalityconstraint}{globalCardinality},
	 \hyperlink{gt:gtconstraint}{gt},
	 \hyperlink{ifonlyif:ifonlyifconstraint}{ifOnlyIf},
	 \hyperlink{ifthenelse:ifthenelseconstraint}{ifThenElse},
	 \hyperlink{implies:impliesconstraint}{implies},
	 \hyperlink{infeaspairac:infeaspairacconstraint}{infeasPairAC},
	 \hyperlink{infeastupleac:infeastupleacconstraint}{infeasTupleAC},
	 \hyperlink{infeastuplefc:infeastuplefcconstraint}{infeasTupleFC},
	 \hyperlink{intdiv:intdivconstraint}{intDiv},
	 \hyperlink{inversechanneling:inversechannelingconstraint}{inverseChanneling},
	 \hyperlink{isincluded:isincludedconstraint}{isIncluded},
	 \hyperlink{isnotincluded:isnotincludedconstraint}{isNotIncluded},
	 \hyperlink{leq:leqconstraint}{leq},
	 \hyperlink{leqcard:leqcardconstraint}{leqCard},
	 \hyperlink{lex:lexconstraint}{lex},
	 \hyperlink{lexchain:lexchainconstraint}{lexChain},
	 \hyperlink{lexchaineq:lexchaineqconstraint}{lexChainEq},
	 \hyperlink{lexeq:lexeqconstraint}{lexeq},
	 \hyperlink{leximin:leximinconstraint}{leximin},
	 \hyperlink{lt:ltconstraint}{lt},
	 \hyperlink{max:maxconstraint}{max},
	 \hyperlink{member:memberconstraint}{member},
	 \hyperlink{min:minconstraint}{min},
	 \hyperlink{mod:modconstraint}{mod},
	 \hyperlink{multicostregular:multicostregularconstraint}{multiCostRegular},
	 \hyperlink{neq:neqconstraint}{neq},
	 \hyperlink{neqcard:neqcardconstraint}{neqCard},
	 \hyperlink{not:notconstraint}{not},
	 \hyperlink{notmember:notmemberconstraint}{notMember},
	 \hyperlink{nth:nthconstraint}{nth},
	 \hyperlink{occurrencemax:occurrencemaxconstraint}{occurrenceMax},
	 \hyperlink{occurrencemin:occurrenceminconstraint}{occurrenceMin},
	 \hyperlink{occurrence:occurrenceconstraint}{occurrence},
	 \hyperlink{oppositesign:oppositesignconstraint}{oppositeSign},
	 \hyperlink{or:orconstraint}{or},
	 \hyperlink{pack:packconstraint}{pack},
	 \hyperlink{precedencereified:precedencereifiedconstraint}{precedenceReified},
	 \hyperlink{preceding:precedingconstraint}{preceding},
	 \hyperlink{regular:regularconstraint}{regular},
	 \hyperlink{reifiedintconstraint:reifiedintconstraintconstraint}{reifiedIntConstraint},
	 \hyperlink{relationpairac:relationpairacconstraint}{relationPairAC},
	 \hyperlink{relationtupleac:relationtupleacconstraint}{relationTupleAC},
	 \hyperlink{relationtuplefc:relationtuplefcconstraint}{relationTupleFC},
	 \hyperlink{samesign:samesignconstraint}{sameSign},
	 \hyperlink{setdisjoint:setdisjointconstraint}{setDisjoint},
	 \hyperlink{setinter:setinterconstraint}{setInter},
	 \hyperlink{setunion:setunionconstraint}{setUnion},
	 \hyperlink{sorting:sortingconstraint}{sorting},
	 \hyperlink{stretchpath:stretchpathconstraint}{stretchPath},
	 \hyperlink{times:timesconstraint}{times},
	 \hyperlink{tree:treeconstraint}{tree},
	 \hyperlink{true:trueconstraint}{TRUE}.
   \end{note}


\chapter{Operators in alphabetical order}\label{constraints:operatorsinalphabeticalorder}\hypertarget{constraints:operatorsinalphabeticalorder}{}

  
\begin{note}
	 \hyperlink{abs:absoperator}{abs},
	 \hyperlink{cos:cosoperator}{cos},
	 \hyperlink{div:divoperator}{div},
	 \hyperlink{false:falseoperator}{FALSE},
	 \hyperlink{ifthenelse:ifthenelseoperator}{ifThenElse},
	 \hyperlink{max:maxoperator}{max},
	 \hyperlink{min:minoperator}{min},
	 \hyperlink{minus:minusoperator}{minus},
	 \hyperlink{mod:modoperator}{mod},
	 \hyperlink{mult:multoperator}{mult},
	 \hyperlink{neg:negoperator}{neg},
	 \hyperlink{plus:plusoperator}{plus},
	 \hyperlink{power:poweroperator}{power},
	 \hyperlink{scalar:scalaroperator}{scalar},
	 \hyperlink{sin:sinoperator}{sin},
	 \hyperlink{sum:sumoperator}{sum},
	 \hyperlink{true:trueoperator}{TRUE}. 
\end{note}
\label{doc:constraints}\hypertarget{doc:constraints}{}
%\part{advanced}
\label{advanced}
\hypertarget{advanced}{}


\chapter{Advanced uses of Choco}\label{advanced:advancedusesofchoco}\hypertarget{advanced:advancedusesofchoco}{}

\section{Environment}\label{advanced:environment}\hypertarget{advanced:environment}{}

Environment is a central object of the backtracking system. It defines the notion of \textit{world}. A world contains values of storable objects or operations that permit to \textit{backtrack} to its state. The environment \textit{pushes} and \textit{pops} worlds.

There are \textit{primitive} data types (\texttt{IstateBitSet, IStateBool, IStateDouble, IStateInt, IStateLong}) and \textit{objects} data types (\texttt{IStateBinarytree, IStateIntInterval, IStateIntProcedure, IStateIntVector, IStateObject, IStateVector}).

There are two different environments: \textit{EnvironmentTrailing} and \textit{EnvironmentCopying}.

\subsection{Copying}\label{advanced:copying}\hypertarget{advanced:copying}{}
In that environment, each data type is defined by a value (primitive or object) and a timestamp. Every time a world is pushed, each value is copied in an array (one array per data type), with finite indice. When a world is popped, every value is restored. 

\subsection{Trailing}\label{advanced:trailing}\hypertarget{advanced:trailing}{}
In that environment, data types are defined by its value. Every operation applied to a data type is pushed in a \textit{trailer}. When a world is pushed, the indice of the last operation is stored. When a world is popped, these operations are popped and \textit{unapplied} until reaching the last operation of the previous world.\\\textit{Default one in CPSolver}

\section{Define your own search strategy}\label{advanced:defineyourownsearchstrategy}\hypertarget{advanced:defineyourownsearchstrategy}{}
%A key ingredient of any constraint approach is a clever branching strategy. The construction of the search tree is done according to a series of Branching objects (that plays the role of achieving intermediate goals in logic programming). The user may specify the sequence of branching objects to be used to build the search tree. 
Section~\hyperlink{solver:searchstrategy}{Search strategy} presented the default branching strategies available in Choco and showed how to post them or to compose them as goals.
In this section, we will start with a very simple and common way to branch by choosing values for variables and specially how to define its own variable/value selection strategy. We will then focus on more complex branching such as dichotomic or n-ary choices. Finally we will show how to control the search space in more details with well known strategy such as LDS (Limited discrepancy search).

For integer variables, the variable and value selection strategy objects are based on the following interfaces:
\begin{itemize}
	\item \texttt{AbstractIntBranchingStrategy}: abstract class for the branching strategy,
	\item \texttt{VarSelector<V>} : Interface for the variable selection (V extends Var),
	\item \texttt{ValIterator<V>} : Interface to describes an iteration scheme on the domain of a variable,
	\item \texttt{ValSelector<V>} : Interface for a value selection.
\end{itemize}

Concrete examples of these interfaces are respectively,  \hyperlink{assignvar:assignvarbranchstrat}{AssignVar}, \hyperlink{mindomain:mindomainvarselector}{MinDomain}, \hyperlink{increasingdomain:increasingdomainvaliterator}{IncreasingDomain}, \hyperlink{maxval:maxvalvalselector}{MaxVal}.

\subsection{How to define your own Branching object}\label{advanced:beyondvariable/valueselection,howtodefineyourownbranchingobject}\hypertarget{advanced:beyondvariable/valueselection,howtodefineyourownbranchingobject}{}

Beyond Variable/value selection...

\subsection{Define your own variable selection}\label{advanced:defineyourownvariableselection}\hypertarget{advanced:defineyourownvariableselection}{}
You may extend this small library of branching schemes and heuristics by defining your own concrete classes of \texttt{AbstractIntVarSelector}. We give here an example of an \texttt{IntVarSelector} with the implementation of a static variable ordering :
\begin{lstlisting}
  public class StaticVarOrder extends AbstractIntVarSelector {

      // the sequence of variables that need be instantiated
	  protected IntDomainVar[] vars;
	
	  public StaticVarOrder(IntDomainVar[] vars) {
          this.vars = vars;
	  }
	
	  public IntDomainVar selectIntVar() {
          for (int i = 0; i < vars.length; i++)
              if (!vars[i].isInstantiated())
                  return vars[i];
          return null;
	  }
  }
\end{lstlisting}

Notice on this example that you only need to implement method \texttt{selectIntVar()} which belongs to the contract of \texttt{IntVarSelector}. This method should return a non instantiated variable or \texttt{null}. Once the branching is finished, the next branching (if one exists) is taken by the search algorithm to continue the search, otherwise, the search stops as all variable are instantiated. To avoid the loop over the variables of the branching, a backtrackable integer (\texttt{StoredInt}) could be used to remember the last instantiated variable and to directly select the next one in the table. Notice that backtrackable structures could be used in any of the code presented in this chapter to speedup the computation of dynamic choices.

You can add your variable selector as a part of a search strategy, using \mylst{solver.addGoal()}.

\subsection{Define your own value selection}\label{advanced:defineyourownvalueselection}\hypertarget{advanced:defineyourownvalueselection}{}
You may also define your own concrete classes of \texttt{ValIterator} or \texttt{ValSelector}. 

\subsubsection{Value selector}\label{advanced:valueselector}\hypertarget{advanced:valueselector}{}

\insertGraphique{.5\linewidth}{media/valselector.pdf}{ValSelector interface and implementation}

We give here an example of an \texttt{IntValSelector} with the implementation of a minimum value selecting:
\begin{lstlisting}
  public class MinVal extends AbstractSearchHeuristic implements ValSelector {
	 /**
      * selecting the lowest value in the domain
      * @param x the variable under consideration
      * @return what seems the most interesting value for branching
      */
	  public int getBestVal(IntDomainVar x) {
          return x.getInf();
	  }
  }
\end{lstlisting}
Only \texttt{getBestVal()} method must be implemented, returning the best value \emph{in the domain} according to the heuristic.

You can add your value selector as a part of a search strategy, using \mylst{solver.addGoal()}.

\begin{note}
Using a value selector with bounded domain variable is strongly inadvised, except if it pick up bounds value. If the value selector pick up a value that is not a bound, when it goes up in the tree search, that value could be not removed and picked twice (or more)!
\end{note} 

\subsubsection{Values iterator}\label{advanced:valuesiterator}\hypertarget{advanced:valuesiterator}{}
We give here an example of an \texttt{ValIterator} with the implementation of an increasing domain iterator:
\begin{lstlisting}
  public final class IncreasingDomain implements ValIterator {
	  /**
	   * testing whether more branches can be considered after branch i, 
       * on the alternative associated to variable x
	   * @param x the variable under scrutiny
	   * @param i the index of the last branch explored
	   * @return true if more branches can be expanded after branch i
	   */
       public boolean hasNextVal(Var x, int i) {
           return (i < ((IntDomainVar) x).getSup());
       }
	
	  /**
	   * accessing the index of the first branch for variable x
	   * @param x the variable under scrutiny
	   * @return the index of the first branch: first value to be assigned to x
	   */
       public int getFirstVal(Var x) {
           return ((IntDomainVar) x).getInf();
       }
	
	  /**
	   * generates the index of the next branch after branch i, 
       * on the alternative associated to variable x
	   * @param x the variable under scrutiny
	   * @param i the index of the last branch explored
	   * @return the index of the next branch to be expanded after branch i
	   */
       public int getNextVal(Var x, int i) {
           return ((IntDomainVar) x).getNextDomainValue(i);
       }
   }
\end{lstlisting}
%Works as an basic \texttt{Iterator} object, implementing the three main methods \texttt{hasNextVal()}, \texttt{getFirstVal()} and \texttt{getNextVal()}.

You can add your value iterator as a part of a search strategy, using \mylst{solver.addGoal()}.

\todo{under development} See \href{http://choco-solver.net/index.phptitle=userguide:beyondvariable.2fvalueselection.2chowtodefineyourownbranchingobject}{old version}

\section{Define your own limit search space}\label{advanced:defineyourownlimitsearchspace}\hypertarget{advanced:defineyourownlimitsearchspace}{}

To define your own limits/statistics (notice that a limit object can be used only to get statistics about the search), you can create a limit object by extending the \texttt{AbstractGlobalSearchLimit} class or implementing directly the interface \texttt{IGlobalSearchLimit}. Limits are managed at each node of the tree search and are updated each time a node is open or closed. Notice that limits are therefore time consuming. Implementing its own limit need only to specify to the following interface :

\begin{lstlisting}
	/**
	 * The interface of objects limiting the global search exploration
	 */
	public interface GlobalSearchLimit {

	  /**
	   * resets the limit (the counter run from now on)
	   * @param first true for the very first initialization, false for subsequent ones
	   */
	  public void reset(boolean first);
	
	  /**
	   * notify the limit object whenever a new node is created in the search tree
	   * @param solver the controller of the search exploration, managing the limit
	   * @return true if the limit accepts the creation of the new node, false otherwise
	   */
	  public boolean newNode(AbstractGlobalSearchSolver solver);
	
	  /**
	   * notify the limit object whenever the search closes a node in the search tree
	   * @param solver the controller of the search exploration, managing the limit
	   * @return true if the limit accepts the death of the new node, false otherwise
	   */
	  public boolean endNode(AbstractGlobalSearchSolver solver);
	}
\end{lstlisting}

Look at the following example to see a concrete implementation of the previous interface. We define here a limit on the depth of the search (which is not found by default in choco). The \texttt{getWorldIndex()} is used to get the current world, i.e the current depth of the search or the number of choices which have been done from baseWorld. 

\begin{lstlisting}
	public class DepthLimit extends AbstractGlobalSearchLimit {
	
	  public DepthLimit(AbstractGlobalSearchSolver theSolver, int theLimit) {
	    super(theSolver,theLimit);
	    unit = "deep";
	  }
	
	  public boolean newNode(AbstractGlobalSearchSolver solver) {
	    nb = Math.max(nb, this.getProblem().getWorldIndex() –
	    this.getProblem().getSolver().getSearchSolver().baseWorld);
	    return (nb < nbMax);
	  }
	
	  public boolean endNode(AbstractGlobalSearchSolver solver) {
	    return true;
	  }
	
	  public void reset(boolean first) {
	   if (first) {
	    nbTot = 0;
	   } else {
	    nbTot = Math.max(nbTot, nb);
	   }
	   nb = 0;
	  }
\end{lstlisting}

Once you have implemented your own limit, you need to tell the search solver to take it into account. Instead of using a call to the \texttt{solve()} method, you have to create the search solver by yourself and add the limit to its limits list such as in the following code :
\begin{lstlisting}
	Solver s = new CPSolver();
	s.read(model);
	s.setFirstSolution(true);
	s.generateSearchStrategy();
	s.getSearchStrategy().limits.add(new DepthLimit(s.getSearchStrategy(),10));
	s.launch();
\end{lstlisting}


\subsection{How does a search loop work ?}\label{advanced:howdoesasearchloopwork}\hypertarget{advanced:howdoesasearchloopwork}{}
The seach loop is created when a \texttt{solve()} method is called. It goes down and up in the branches in order to cover the tree search. 

%\subsubsection{Search loop}\label{advanced:searchloop}\hypertarget{advanced:searchloop}{}
\begin{lstlisting}[title={Algorithm of the search loop in Choco}]
  next_move = new node
  WHILE no solution AND in search limit
        IF next_move is new node
	    THEN
	        create a new node : variable/value selection ;
	        IF node exists 
	        THEN
	            next_move <-- go down branch ;            
	        ELSE 
	            next_move <-- go up branch ;
	            solution is found ;
	
	    ELSE IF next_move is go down branch
	        propagate ;
	        IF no contradiction 
	        THEN
	            next_move <-- new node ;                        
	        ELSE
	            next_move <-- go up branch ;
	        
	    ELSE IF next_move is go up branch 
	        find next branch ;
	        propagate ;
	        IF has next branch AND no contradiction
	        THEN
	            next_move <-- go down branch ;                            
	        ELSE
	            next_move <-- go up branch ;
	
	    END IF
	
  END WHILE
\end{lstlisting}
%\subsubsection{Search loop with recomputation}\label{advanced:searchloopwithrecomputation}\hypertarget{advanced:searchloopwithrecomputation}{}

\section{Define your own constraint}\label{advanced:defineyourownconstraint}\hypertarget{advanced:defineyourownconstraint}{}

This section describes how to add you own constraint, with specific propagation algorithms. Note that this section is only useful in case you want to express a constraint for which the basic propagation algorithms (using tables of tuples, or boolean predicates) are not efficient enough to propagate the constraint.

The general process consists in defining a new constraint class and implementing the various propagation methods. We recommend the user to follow the examples of existing constraint classes (for instance, such as \texttt{GreaterOrEqualXYC} for a binary inequality) 

\subsection{The constraint hierarchy}\label{advanced:theconstrainthierarchy}\hypertarget{advanced:theconstrainthierarchy}{}

Each new constraint must be represented by an object implementing the \texttt{\bf SConstraint} interface (\texttt{S} for solver constraint). To help the user defining new constraint classes, several abstract classes defining \texttt{SConstraint} have been implemented. These abstract classes provide the user with a management of the constraint network and the propagation engineering. They should be used as much as possible.

For constraints on integer variables, the easiest way to implement your own constraint is to inherit from one of the following classes, depending of the number of solver integer variables (\texttt{IntDomainVar}) involved:

\centerline{\begin{tabular}{ll}
      \hline
  Default class to implement &  number of solver integer variables \\
  \hline
  \mylst{AbstractUnIntSConstraint} &  \textbf{one} variable \\
  \mylst{AbstractBinIntSConstraint} &  \textbf{two} variables \\
  \mylst{AbstractTernIntSConstraint} &  \textbf{three} variables \\
  \mylst{AbstractLargeIntSConstraint} &  any number of variables. \\
  \hline\\
\end{tabular}}

\noindent Constraints over integers must implement the following methods (grouped in the \texttt{IntSConstraint} interface):

\noindent\begin{tabular}{lp{.6\linewidth}}
  \hline
  Method to implement &  description \\
  \hline
  \mylst{pretty()} &Returns a pretty print of the constraint \\
  \mylst{propagate()} &The main propagation method (propagation from scratch). Propagating the constraint until local consistency is reached. \\
  \mylst{awake()} &Propagating the constraint for the very first time until local consistency is reached. The awake is meant to initialize the data structures contrary to the propagate. Specially, it is important to avoid initializing the data structures in the constructor. \\
  \mylst{awakeOnInst(int x)} &Default propagation on instantiation: full constraint re-propagation. \\
  \mylst{awakeOnBounds(int x)} &Default propagation on improved bounds: propagation on domain revision. \\
  \mylst{awakeOnRemovals(int x, IntIterator v)} &Default propagation on mutliple values removal: propagation on domain revision. The iterator allow to iterate over the values that have been removed. \\
&\\
\hline
\multicolumn{2}{l}{Methods \texttt{awakeOnBounds} and \texttt{awakeOnRemovals} can be replaced by more fine grained methods:}\\
\hline
%Alternative Method &  description \\
%  \hline
  \mylst{awakeOnInf(int x)} &Default propagation on improved lower bound: propagation on domain revision. \\
  \mylst{awakeOnSup(int x)} &Default propagation on improved upper bound: propagation on domain revision. \\
  \mylst{awakeOnRem(int x, int v)} &Default propagation on one value removal: propagation on domain revision.  \\
&\\
  \hline
\multicolumn{2}{l}{To use the constraint in expressions or reification, the following minimum API is mandatory:}\\
  \hline
  \mylst{isSatisfied(int[] x)} &Tests if the constraint is satisfied when the variables are instantiated. \\
	\mylst{isEntailed()} &Checks if the constraint must be checked or must fail. It returns true if the constraint is known to be satisfied whatever happend on the variable from now on, false if it is violated. \\
	\mylst{opposite()} &It returns an AbstractSConstraint that is the opposite of the current constraint. \\
    \hline\\
	\end{tabular}

In the same way, a \textbf{set constraint} can inherit from \texttt{AbstractUnSetSConstraint}, \texttt{AbstractBinSetSConstraint}, \texttt{AbstractTernSetSConstraint} or \texttt{AbstractLargeSetSConstraint}.

A \textbf{real constraint} can inherit from \texttt{AbstractUnRealSConstraint}, \texttt{AbstractBinRealSConstraint} or \texttt{AbstractLargeRealSConstraint}.

A mixed constraint between \textbf{set and integer variables} can inherit from \texttt{AbstractBinSetIntSConstraint} or \texttt{AbstractLargeSetIntSConstraint}.

\begin{note}
A simple way to implement its own constraint is to:
\begin{itemize}
	\item create an empty constraint with only \texttt{propagate()} method implemented and every \texttt{awakeOnXxx()} ones set to \texttt{this.constAwake(false);}
	\item when the propagation filter is sure, separate it into the \texttt{awakeOnXxx()} methods in order to have finer granularity
	\item finally, if necessary, use backtrackables objects to improve the efficient of your constraint
\end{itemize}

\end{note}

\subsubsection{How do I add my constraint to the Model ?}\label{advanced:howdoiaddmyconstrainttothemodel}\hypertarget{advanced:howdoiaddmyconstrainttothemodel}{}

Adding your constraint to the model requires you to definite a specific constraint manager (that can be a inner class of your Constraint).
This manager need to implement:
\begin{lstlisting}
makeConstraint(Solver s, Variable[] vars, Object params, HashSet<String> options)
\end{lstlisting}
This method allows the Solver to create an instance of your constraint, with your parameters and Solver objects.

\begin{note}
If you create your constraint manager as an inner class, you must declare this class as \textbf{public and static}.
If you don't, the solver can't instantiate your manager.
\end{note}

Once this manager has been implemented, you simply add your constraint to the model using the \texttt{addConstraint()} API with a \texttt{ComponentConstraint} object:
\begin{lstlisting}
  model.addConstraint( new ComponentConstraint(MyConstraintManager.class, params, vars) );
  // OR
  model.addConstraint( new ComponentConstraint("package.of.MyConstraint", params, vars) );
\end{lstlisting}
Where \emph{params} is whatever you want (\texttt{Object[], int, String},...) and \emph{vars} is an array of Model Variables (or more specific) objects.

\subsection{Example: implement and add the \texttt{IsOdd} constraint}
One creates the constraint by implementing the \texttt{AbstractUnIntSConstraint} (one integer variable) class:
\lstinputlisting{java/isodd.j2t}

To add the constraint to the model, one creates the following class (or inner class):
\lstinputlisting{java/isoddmanager.j2t}
It calls the constructor of the constraint, with every \emph{vars}, \emph{params} and \emph{options} needed.

Then, the constraint can be added to a model as follows:
\begin{lstlisting}
	// Creation of the model
	Model m = new CPModel();
	
	// Declaration of the variable
	IntegerVariable aVar = Choco.makeIntVar("a_variable", 0, 10);
	
	// Adding the constraint to the model, 1st solution:
	m.addConstraint(new ComponentConstraint(IsOddManager.class, null, new IntegerVariable[]{aVar}));
	// OR 2nd solution:
	m.addConstraint(new ComponentConstraint("myPackage.Constraint.IsOddManager", null, new IntegerVariable[]{aVar}));
	
	Solver s = new CPSolver();
	s.read(m);
	s.solve();
\end{lstlisting}
And that's it!!

\subsection{Example of an empty constraint}\label{advanced:anexempleofemptyconstraint}\hypertarget{advanced:anexempleofemptyconstraint}{}

See the complete code: \href{media/zip/constraintpattern.zip}{ConstraintPattern.zip}

\begin{lstlisting}
  public class ConstraintPattern extends AbstractLargeIntSConstraint {
      
      public ConstraintPattern(IntDomainVar[] vars) {
          super(vars);
      }
	
      /**
      * pretty print. The String is not constant and may depend on the context.
      * @return a readable string representation of the object
      */
      public String pretty() {
          return null;
      }
	
      /**
      * check whether the tuple satisfies the constraint
      * @param tuple values
      * @return true if satisfied
      */
      public boolean isSatisfied(int[] tuple) {
          return false;
      }

      /**
      * propagate until local consistency is reached
      */
      public void propagate() throws ContradictionException {
          // elementary method to implement
      }
	    
      /**
      * propagate for the very first time until local consistency is reached.
      */
      public void awake() throws ContradictionException {
          constAwake(false);        // change if necessary
      }
	
	
      /**
      * default propagation on instantiation: full constraint re-propagation
      * @param var index of the variable to reduce
      */
      public void awakeOnInst(int var) throws ContradictionException {
          constAwake(false);        // change if necessary
      }
	
      /**
      * default propagation on improved lower bound: propagation on domain revision
      * @param var index of the variable to reduce
      */
      public void awakeOnInf(int var) throws ContradictionException {
          constAwake(false);        // change if necessary
      }
	
	
      /**
      * default propagation on improved upper bound: propagation on domain revision
      * @param var index of the variable to reduce
      */
      public void awakeOnSup(int var) throws ContradictionException {
          constAwake(false);        // change if necessary
      }
	
      /**
      * default propagation on improve bounds: propagation on domain revision
      * @param var index of the variable to reduce
      */
      public void awakeOnBounds(int var) throws ContradictionException {
          constAwake(false);        // change if necessary
      }
	
      /**
      * default propagation on one value removal: propagation on domain revision
      * @param var index of the variable to reduce
      * @param val the removed value
      */
      public void awakeOnRem(int var, int val) throws ContradictionException {
          constAwake(false);        // change if necessary
      }
	
      /**
      * default propagation on one value removal: propagation on domain revision
      * @param var index of the variable to reduce
      * @param delta iterator over remove values
      */
      public void awakeOnRemovals(int var, IntIterator delta) throws ContradictionException {
          constAwake(false);        // change if necessary
      }
  }
\end{lstlisting}

The first step to create a constraint in Choco is to implement all \texttt{awakeOn...} methods with \texttt{constAwake(false)} and to put your propagation algorithm in the \texttt{propagate()} method. 

A constraint can choose not to react to fine grained events such as the removal of a value of a given variable but instead delay its propagation at the end of the fix point reached by ``fine grained events'' and fast constraints that deal with them incrementally (that's the purpose of the constraints events queue). 

To do that, you can use \texttt{constAwake(false)} that tells the solver that you want this constraint to be called only once the variables events queue is empty. This is done so that heavy propagators can delay their action after the fast one to avoid doing a heavy processing at each single little modification of domains.

\section{Define your own operator}\label{advanced:defineyourownoperator}\hypertarget{advanced:defineyourownoperator}{}
\todo{to complete}

\section{Define your own variable}\label{advanced:defineyourownvariable}\hypertarget{advanced:defineyourownvariable}{}
\todo{to complete}

\section{Backtrackable structures}\label{advanced:backtrackablestructures}\hypertarget{advanced:backtrackablestructures}{}
\todo{to complete}

\section{Logging System}\label{advanced:loggingsystem}\hypertarget{advanced:loggingsystem}{}

Choco logging system is based on the \texttt{java.util.logging} package and located in the package \texttt{common.logging}.
Most Choco abstract classes or interfaces propose a static field \texttt{LOGGER}.
The following figures present the architecture of the logging system with the default verbosity.

\insertGraphique{.9\linewidth}{media/logger-default.png}{Logger Tree with the default verbosity}

The shape of the node depicts the type of logger:
\begin{itemize}
	\item The \emph{house} loggers represent private loggers. Do not use directly these loggers because their level are low and all messages would always be displayed.
	\item The \emph{octagon} loggers represent critical loggers. These loggers are provided in the variables, constraints and search classes and could have a huge impact on the global performances.
	\item The \emph{box} loggers are provided for dev and users.
\end{itemize}
The color of the node gives its logging level with DEFAULT verbosity:
\texttt{Level.FINEST} (\textcolor{yellow}{gold}),
\texttt{Level.INFO} (\textcolor{orange}{orange}),
\texttt{Level.WARNING} (\textcolor{red}{red}).

\subsubsection{Verbosity and messages.}\label{advanced:verbosityandmessages}\hypertarget{advanced:verbosityandmessages}{}
The following table summarizes the verbosities available in choco: 

\begin{itemize}
	\item \textbf{OFF -- level 0:} Disable logging.
	\item \textbf{SILENT -- level 1:} Display only severe messages.
	\item \textbf{DEFAULT -- level 2:} Display informations on final search state.
		\begin{itemize}
			\item ON START
				\lstset{language={sh},columns=fixed}
\begin{lstlisting}
 ** CHOCO : Constraint Programming Solver
 ** CHOCO v2.1.1 (April, 2009), Copyleft (c) 1999-2010
 \end{lstlisting}
			\item ON COMPLETE SEARCH:
				\begin{lstlisting}
- Search completed -
 [Maximize		: {0},]
 [Minimize		: {1},]
  Solutions		: {2},
  Times (ms)	: {3},
  Nodes			: {4},
  Backtracks	: {5},
  Restarts		: {6}.
  \end{lstlisting}
	brackets [\textit{line}] indicate \textit{line} is optionnal,\\
 	\texttt{Maximize} --resp. \texttt{Minimize}-- indicates the best known value before exiting of the objective value in \textit{maximize()} -- --resp. \textit{minimize()}-- strategy.

			\item ON COMPLETE SEARCH WITHOUT SOLUTIONS :
				\begin{lstlisting}
- Search completed - No solutions
 [Maximize		: {0},]
 [Minimize		: {1},]
  Solutions		: {2},
  Times (ms)	: {3},
  Nodes			: {4},
  Backtracks	: {5},
  Restarts		: {6}.
\end{lstlisting}
	brackets [\textit{line}] indicate \textit{line} is optionnal,\\
 	\texttt{Maximize} --resp. \texttt{Minimize}-- indicates the best known value before exiting of the objective value in \textit{maximize()} -- --resp. \textit{minimize()}-- strategy.

			\item ON INCOMPLETE SEARCH:
				\begin{lstlisting}
- Search incompleted - Exiting on limit reached
  Limit			: {0},
 [Maximize		: {1},]
 [Minimize		: {2},]
  Solutions		: {3},
  Times (ms)	: {4},
  Nodes			: {5},
  Backtracks	: {6},
  Restarts		: {7}.
  
  \end{lstlisting}
	brackets [\textit{line}] indicate \textit{line} is optionnal,\\
 	\texttt{Maximize} --resp. \texttt{Minimize}-- indicates the best known value before exiting of the objective value in \textit{maximize()} -- --resp. \textit{minimize()}-- strategy.
		\end{itemize}			

	\item \textbf{VERBOSE -- level 3:} Display informations on search state.
		\begin{itemize}
			\item EVERY X (default=1000) NODES:
			\begin{lstlisting}
- Partial search - [Objective : {0}, ]{1} solutions, {2} Time (ms), {3} Nodes, {4} Backtracks, {5} Restarts.
			\end{lstlisting}
			\texttt{Objective} indicates the best known value.

			\item ON RESTART : 
			\begin{lstlisting}
- Restarting search - {0} Restarts.
			\end{lstlisting}
		\end{itemize}

	\item \textbf{SOLUTION -- level 4:} display all solutions.
		\begin{itemize}
			\item AT EACH SOLUTION:
			\begin{lstlisting}
- Solution #{0} found. [Objective: {0}, ]{1} Solutions, {2} Time (ms), {3} Nodes, {4} Backtracks, {5} Restarts.
  X_1:v1, x_2:v2...
			\end{lstlisting}
		\end{itemize}

	\item \textbf{SEARCH -- level 5:} Display the search tree.
		\begin{itemize}
			\item AT EACH NODE, ON DOWN BRANCH:
			\begin{lstlisting}
...[w] down branch X==v branch b
			\end{lstlisting}
where \texttt{w} is the current world index, \texttt{X} the branching variable, \texttt{v} the branching value and \texttt{b} the branch index. This message can be adapted on variable type and search strategy.

			\item AT EACH NODE, ON UP BRANCH:
			\begin{lstlisting}
...[w] up branch X==v branch b
			\end{lstlisting}
where \texttt{w} is the current world index, \texttt{X} the branching variable, \texttt{v} the branching value and \texttt{b} the branch index. This message can be adapted on variable type and search strategy.
		\end{itemize}

	\item \textbf{FINEST -- level 6:} display all logs.

\end{itemize}

More precisely, if the verbosity level is greater than DEFAULT, then the verbosity levels of the model and of the solver are increased to INFO, and the verbosity levels of the search and of the branching are slightly modified to display the solution(s) and search messages.

The verbosity level can be changed as follows:
\begin{lstlisting}
	ChocoLogging.setVerbosity(Verbosity.VERBOSE);
\end{lstlisting}


\subsubsection{How to write logging statements ?}\label{advanced:howtowriteloggingstatements}\hypertarget{advanced:howtowriteloggingstatements}{}

\begin{itemize}
	\item Critical Loggers are provided to display error or warning. Displaying too much message really \textbf{impacts the performances}.
	\item Check the logging level before creating arrays or strings.
	\item Avoid multiple calls to \texttt{Logger} functions. Prefer to build a \texttt{StringBuilder} then call the \texttt{Logger} function.
	\item Use the \texttt{Logger.log} function instead of building string in \texttt{Logger.info()}.
\end{itemize}

\subsubsection{Handlers.}\label{advanced:handlers}\hypertarget{advanced:handlers}{}
Logs are displayed on \texttt{System.out} but warnings and severe messages are also displayed on \texttt{System.err}.
\texttt{ChocoLogging.java} also provides utility functions to easily change handlers:
\begin{itemize}
	\item Functions \texttt{set...Handler} remove current handlers and replace them by a new handler.
	\item Functions \texttt{add...Handler} add new handlers but do not touch existing handlers.
\end{itemize}

\subsubsection{Define your own logger.}\label{advanced:defineyourownlogger}\hypertarget{advanced:defineyourownlogger}{}
\begin{lstlisting}
ChocoLogging.makeUserLogger(String suffix);
\end{lstlisting}

% \subsubsection{Figure source}\label{advanced:figuresource}\hypertarget{advanced:figuresource}{}
% \begin{lstlisting}
%   digraph G {
%       node [style=filled, shape=box];
%       choco [shape=house,fillcolor=gold];
	
%       kernel [shape=house,fillcolor=gold];
%       engine [shape=octagon,fillcolor=indianred];
%       search [shape=octagon,fillcolor=darkorange];
%       branching [shape=octagon,fillcolor=indianred];
	
%       api [shape=house,fillcolor=gold];
%       model [fillcolor=indianred];
%       solver [fillcolor=indianred];
%       parser [fillcolor=darkorange];
      
%       user [fillcolor=darkorange];
%       samples [fillcolor=darkorange];
      
%       test [fillcolor=indianred];
      
%       choco -> kernel;
%       choco -> API;
%       choco -> user;
%       choco -> test;
      
%       kernel -> engine;
%       kernel -> search;
      
%       api -> model;
%       api -> solver
%       api -> parser;
      
%       user  -> samples;
      
%       search -> branching;
% 	}
% \end{lstlisting}
\label{doc:advanced}\hypertarget{doc:advanced}{}
