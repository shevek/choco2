\addcontentsline{toc}{chapter}{Preface}
\chapter*{Preface}
Choco is a java library for constraint satisfaction problems (CSP), constraint programming (CP) and explanation-based constraint solving (e-CP). It is built on a event-based propagation mechanism with backtrackable structures.
Choco is an open-source software, distributed under a \textbf{BSD licence} and hosted by \href{http://sourceforge.net/projects/choco/}{sourceforge.net}.
For any requests send a mail to \url{choco@emn.fr}.
\bigskip

\noindent This document is organized as follows:
\begin{itemize}
\item \hyperlink{ch:doc}{Documentation} is the user-guide of Choco. After a short \hypertarget{doc:introduction}{introduction} to constraint programming and to the Choco solver, it presents the basics of \hyperlink{doc:model}{modeling} and \hyperlink{doc:solver}{solving} with Choco, the \hyperlink{doc:advanced}{advanced usages} (customizing propagation and search), some examples of \hyperlink{doc:applications}{Applications}, and a \hyperlink{faq:frequentlyaskedquestions}{FAQ} section.
\item \hyperlink{part:elements}{Elements of Choco} gives a detailed description of the \hyperlink{ch:vars}{variables}, \hyperlink{ch:operators}{operators}, \hyperlink{ch:constraints}{constraints} currently available in Choco.
\item \hyperlink{ch:tut}{Tutorials} provides a \hyperlink{gettingstarted:gettingstarted:welcometochoco}{fast how-to} write a Choco program, a detailed example of a \hyperlink{gettingstarted:firstexample:magicsquare}{simple program}, and several \hyperlink{exercises}{exercises} with their \hyperlink{solutions}{solutions}.
\item \hyperlink{ch:extra}{Extras} presents future works, only available on the beta version or extension of the current jar, such as the \hyperlink{chocoandvisu:chocoandvisu}{visualization module of Choco}. The section dedicated to \hyperlink{sudokuandcp:sudokuandconstraintprogramming}{Sudoku} aims at explaining the basic principles of Constraint Programming (propagation and search) on this famous game.
%\begin{itemize}
%%	\item \hyperlink{chocoandgraphviz}{Choco and Graphviz} \emph{not yet available. Has been included in \hyperlink{chocoandvisu}{Choco and Visu}.}
%	\item 

\end{itemize}

\part{Documentation}\label{ch:doc}\hypertarget{ch:doc}{}

The documentation of Choco is organized as follows:
\begin{itemize}
\item 
The concise \hyperlink{doc:introduction}{introduction} provides some informations \hyperlink{introduction:aboutconstraintprogramming}{about constraint programming} concepts and a ``Hello world''-like \hyperlink{introduction:myfirstchocoprogram}{first Choco program}.
\item 
The \hyperlink{doc:model}{model} section gives informations on \hyperlink{doc:model}{how to create a model} and introduces \hyperlink{model:variables}{variables} and \hyperlink{model:constraints}{constraints}.
\item 
The \hyperlink{doc:solver}{solver} section gives informations on \hyperlink{doc:solver}{how to create a solver}, to \hyperlink{doc:solver}{read a model}, to define a \hyperlink{solver:searchstrategy}{search strategy}, and finally to \hyperlink{solver:solveaproblem}{solve a problem}.
\item 
The \hyperlink{doc:advanced}{advanced use} section explains how to define your own \hyperlink{advanced:defineyourownlimitsearchspace}{limit search space}, \hyperlink{advanced:defineyourownsearchstrategy}{search strategy}, \hyperlink{advanced:defineyourownconstraint}{constraint}, \hyperlink{advanced:defineyourownoperator}{operator}, \hyperlink{advanced:defineyourownvariable}{variable}, or \hyperlink{advanced:backtrackablestructures}{backtrackable structure}.
\item 
The \hyperlink{doc:applications}{applications} section shows the use of Choco defined global constraints on \hyperlink{schedulinganduseofthecumulative:schedulinganduseofthecumulativeconstraint}{scheduling} or \hyperlink{geostdescription:placementanduseofthegeostconstraint}{placement} problems.
%\item 
%Lastly, the catalog of Choco defined \hyperlink{ch:constraints}{constraints} is presented.
\end{itemize}

%\section*{Beta}\label{documentation:beta}\hypertarget{documentation:beta}{}
%Here you can find short documentation concerning futur works, only available on the beta version or extension of the current jar:
%\begin{itemize}
%%	\item \hyperlink{chocoandgraphviz}{Choco and Graphviz} \emph{not yet available. Has been included in \hyperlink{chocoandvisu}{Choco and Visu}.}
%	\item \hyperlink{chocoandvisu}{Choco and Visu}
%\end{itemize}

%\section*{Material}\label{documentation:material}\hypertarget{documentation:material}{}
%Here you can find some materials at your disposal. If you create any, you can also send it to us, so we can add it to the list.

%\subsection{Presentations}\label{documentation:presentations}\hypertarget{documentation:presentations}{}
%\begin{itemize}
%	\item \textbf{August 2008} : The \emph{CHOCO : an Open Source Java Constraint Constraint Programming} white paper presentation send to the \href{http://www.cril.univ-artois.fr/cpai08/}{CSP'08 competition}. \href{media/pdf/choco-presentation.pdf}{PDF} of white paper presentation of Choco
%	\item \textbf{june 2008} : \emph{The CHOCO constraint programming solver} presentation held at the \href{https://projects.coin-or.org/events/wiki/cpaior2008}{Workshop on Open-Source Software for Integer and Constraint Programming} during the last \href{http://contraintes.inria.fr/cpaior08/}{CPAIOR} conference. \href{media/slides/cpaior-choco.pdf}{PDF slides} of the presentation given by Guillaume Rochart
%\end{itemize}
%\Graph{media/banner.png}{width=\linewidth}

\chapter{Introduction to constraint programming and Choco}\label{doc:introduction}\hypertarget{doc:introduction}{}

\section{About constraint programming}\label{introduction:aboutconstraintprogramming}\hypertarget{introduction:aboutconstraintprogramming}{}

\begin{myquote}
Constraint programming represents one of the closest approaches computer science has yet made to the Holy Grail of programming: the user states the problem, the computer solves it.
\begin{flushright}\bf E. C. Freuder, Constraints, 1997.\end{flushright}
\end{myquote}


Fast increasing computing power in the 1960s led to a wealth of works around problem solving, at the root of Operational Research, Numerical Analysis, Symbolic Computing, Scientific Computing, and a large part of Artificial Intelligence and programming languages. Constraint Programming is a discipline that gathers, interbreeds, and unifies ideas shared by all these domains to tackle decision support problems.

Constraint programming has been successfully applied in numerous domains. Recent applications include computer graphics (to express geometric coherence in the case of scene analysis), natural language processing (construction of efficient parsers), database systems (to ensure and/or restore consistency of the data), operations research problems (scheduling, routing), molecular biology (DNA sequencing), business applications (option trading), electrical engineering (to locate faults), circuit design (to compute layouts), etc.

Current research in this area deals with various fundamental issues, with implementation aspects and with new applications of constraint programming.

\subsection{Constraints}\label{introduction:constraints}\hypertarget{introduction:constraints}{}
A constraint is simply a logical relation among several unknowns (or variables), each taking a value in a given domain. A constraint thus restricts the possible values that variables can take, it represents some partial information about the variables of interest. For instance, the circle is inside the square relates two objects without precisely specifying their positions, i.e., their coordinates. Now, one may move the square or the circle and he or she is still able to maintain the relation between these two objects. Also, one may want to add another object, say a triangle, and to introduce another constraint, say the square is to the left of the triangle. From the user (human) point of view, everything remains absolutely transparent.

Constraints naturally meet several interesting properties:
\begin{itemize}
	\item constraints may specify partial information, i.e. constraint need not uniquely specify the values of its variables,
	\item constraints are non-directional, typically a constraint on (say) two variables $X, Y$ can be used to infer a constraint on $X$ given a constraint on $Y$ and vice versa,
	\item constraints are declarative, i.e. they specify what relationship must hold without specifying a computational procedure to enforce that relationship,
	\item constraints are additive, i.e. the order of imposition of constraints does not matter, all that matters at the end is that the conjunction of constraints is in effect,
	\item constraints are rarely independent, typically constraints in the constraint store share variables.
\end{itemize}

Constraints arise naturally in most areas of human endeavor. The three angles of a triangle sum to 180 degrees, the sum of the currents floating into a node must equal zero, the position of the scroller in the window scrollbar must reflect the visible part of the underlying document, these are some examples of constraints which appear in the real world. Thus, constraints are a natural medium for people to express problems in many fields. 

\subsection{Constraint Programming}\label{introduction:constraintprogramming}\hypertarget{introduction:constraintprogramming}{}
Constraint programming is the study of computational systems based on constraints. The idea of constraint programming is to solve problems by stating constraints (conditions, properties) which must be satisfied by the solution.

Work in this area can be tracked back to research in Artificial Intelligence and Computer Graphics in the sixties and seventies. Only in the last decade, however, has there emerged a growing realization that these ideas provide the basis for a powerful approach to programming, modeling and problem solving and that different efforts to exploit these ideas can be unified under a common conceptual and practical framework, constraint programming. 

\begin{note}
If you know \textbf{sudoku}, then you know \textbf{constraint programming}. See why \hyperlink{sudokuandcp}{here}.
\end{note}


\section{Modeling with Constraint programming}\label{introduction:modelingwithconstraintprogramming}\hypertarget{introduction:modelingwithconstraintprogramming}{}
The formulation and the resolution of combinatorial problems are the two main goals of the constraint programming domain. This is an essential way to solve many interesting industrial problems such as scheduling, planning or design of timetables. The main interest of constraint programming is to propose to the user to model a problem without being interested in the way the problem is solved.

\subsection{The Constraint Satisfaction Problem}\label{introduction:csp}\hypertarget{introduction:csp}{}

Constraint programming allows to solve combinatorial problems modeled by a Constraint Satisfaction Problem (CSP). Formally, a CSP is defined by a triplet $(X,D,C)$:
\begin{itemize}
	\item \textbf{Variables}: $X = \{X_1,X_2,\ldots,X_n\}$ is the set of variables of the problem.
	\item \textbf{Domains}: $D$ is a function which associates to each variable $X_i$ its domain $D(X_i)$, i.e. the set of possible values that can be assigned to $X_i$. The domain of a variable is usually a finite set of integers: $D(X_i)\subset\Z$ (\emph{integer variable}). But a domain can also be continuous ($D(X_i)\subseteq\R$ for a \emph{real variable}) or made of discrete set values ($D(X_i)\subseteq\mathcal{P}(\Z)$ for a \emph{set variable}).
	\item \textbf{Constraints}: $C = \{C_1,C_2,\ldots,C_m\}$ is the set of constraints. A constraint $C_j$ is a relation defined on a subset $X^j = \{X^j_1,X^j_2,\ldots,X^j_{n^j}\}\subseteq X$ of variables which restricts the possible tuples of values $(v_1,\ldots,v_{n^j})$ for these variables:
$$(v_1,\ldots,v_{n^j})\in C_j\cap (D(X^j_1)\times D(X^j_2)\times\cdots\times D(X^j_{n^j})).$$
Such a relation can be defined explicitely (ex: $(X_1,X_2)\in\{(0,1),(1,0)\}$) or implicitely (ex: $X_1+X_2\le 1$).
\end{itemize}

Solving a CSP is to find a tuple $v=(v_1,\ldots,v_{n})\in D(X)$ on the set of variables which satisfies all the constraints:
$$(v_1,\ldots,v_{n^j})\in C_j,\quad\forall j\in\{1,\ldots,m\}.$$

For optimization problems, one need to define an \textbf{objective function} $f:D(X)\rightarrow\R$. An optimal solution is then a solution tuple of the CSP that minimizes (or maximizes) function $f$.

\subsection{Examples of CSP models}\label{introduction:examples}\hypertarget{introduction:examples}{}
This part provides three examples using different types of variables in different problems. These examples are used throughout this tutorial to illustrate their modeling with Choco.

\subsubsection{Example 1: the n-queens problem.}\label{introduction:example1:nqueens}\hypertarget{introduction:example1:nqueens}{}
Let us consider a chess board with $n$ rows and $n$ columns. A queen can move as far as she pleases, horizontally, vertically, or diagonally. The standard $n$-queens problem asks how to place $n$ queens on an $n$-ary chess board so that none of them can hit any other in one move.

The $n$-queens problem can be modeled by a CSP in the following way:
\begin{itemize}
	\item \textbf{Variables}: $X = \{X_{i}\ |\ i\in [1,n]\}$.
	\item \textbf{Domain}: for all variable $X_{i}\in X$, $D(X_{i}) = \{1,2,\ldots, n\}$.
	\item \textbf{Constraints}: the set of constraints is defined by the union of the three following constraints,
	\begin{itemize}
		\item queens have to be on distinct lines:
		\begin{itemize}
			\item $C_{lines} = \{X_{i}\neq X_{j}\ |\ i,j\in [1,n], i\neq j\}$.
		\end{itemize}
		\item queens have to be on distinct diagonals:
		\begin{itemize}
			\item $C_{diag1} = \{X_{i}\neq X_{j+j-i}\ |\ i,j\in [1,n], i\neq j\}$.
			\item $C_{diag2} = \{X_{i}\neq X_{j+i-j}\ |\ i,j\in [1,n], i\neq j\}$.
		\end{itemize}
	\end{itemize}
\end{itemize}

\subsubsection{Example 2: the ternary Steiner problem.}\label{introduction:example2:theternarysteinerproblem}\hypertarget{introduction:example2:theternarysteinerproblem}{}
A ternary Steiner system of order $n$ is a set of $n*(n-1)/6$ triplets of distinct elements taking their values in $[1,n]$, such that all the pairs included in two distinct triplets are different.
See \url{http://mathworld.wolfram.com/SteinerTripleSystem.html} for details. 

The ternary Steiner problem can be modeled by a CSP in the following way:
\begin{itemize}
	\item let $t = n*(n-1)/6$.
	\item \textbf{Variables}: $X = \{X_{i}\ |\ i\in [1,t]\}$.
	\item \textbf{Domain}: for all $i\in [1,t]$, $D(X_{i}) = \{1,...,n\}$.
	\item \textbf{Constraints}:
	\begin{itemize}
		\item every set variable $X_i$ has a cardinality of 3:
		\begin{itemize}
			\item for all $i\in [1,t]$, $|X_{i}| = 3$.
		\end{itemize}
		\item the cardinality of the intersection of every two distinct sets must not exceed 1:
		\begin{itemize}
			\item for all $i,j\in [1,t]$, $i\neq j$, $|X_{i}\cap X_{j}|\le 1$.
		\end{itemize}
	\end{itemize}
\end{itemize}

\subsubsection{Example 3: the CycloHexane problem.}\label{introduction:example3:thecyclohexaneproblem}\hypertarget{introduction:example3:thecyclohexaneproblem}{}
The problem consists in finding the 3D configuration of a cyclohexane molecule. It is described with a system of three non linear equations:
\begin{itemize}
	\item \textbf{Variables}: $x,y,z$.
	\item \textbf{Domain}: $]-\infty;+\infty[$.
	\item \textbf{Constraints}:
	\begin{align*}
		y^{2} * (1 + z^{2}) + z * (z - 24 * y) &= -13\\
		x^{2} * (1 + y^{2}) + y * (y - 24 * x) &= -13\\
		z^{2} * (1 + x^{2}) + x * (x - 24 * z) &= -13
	\end{align*}
\end{itemize}

\section{My first Choco program: the magic square}\label{introduction:myfirstchocoprogram}\hypertarget{introduction:myfirstchocoprogram}{}

\subsection{The magic square problem}\label{introduction:amagicsquareproblem}\hypertarget{introduction:amagicsquareproblem}{}
In the following, we will address the magic square problem of order 3 to illustrate step-by-step how to model and solve this problem using choco. 

\subsubsection{Definition:}
A magic square of order $n$ is an arrangement of $n^{2}$ numbers, usually distinct integers, in a square, such that the $n$ numbers in all rows, all columns, and both diagonals sum to the same constant. A standard magic square contains the integers from 1 to $n^{2}$.

The constant sum in every row, column and diagonal is called the magic constant or magic sum $M$. The magic constant of a classic magic square depends only on $n$ and has the value:
$M(n)=n(n^2 +1)/2$.

\href{http://en.wikipedia.org/wiki/magicsquare}{More details on the magic square problem.}


\subsection{A mathematical model}\label{introduction:mathematicalmodeling}\hypertarget{introduction:mathematicalmodeling}{}

Let $x_{ij}$ be the variable indicating the value of the $j^{th}$ cell of row $i$. 
Let $C$ be the set of constraints modeling the magic square as:
\begin{align*}
&x_{ij} \in [1,n^2],\ &&\forall i,j \in [1, n]\\
&x_{ij}\ne x_{kl},\ &&\forall i,j,k,l \in [1,n], i\ne k, j\ne l\\
&\sum_{j=1}^{n} x_{ij} = n^2,\ &&\forall i \in [1,n]\\
&\sum_{i=1}^{n} x_{ij} = n^2,\ &&\forall j \in [1,n]\\
&\sum_{i=1}^{n} x_{ii} = n^2&&\\
&\sum_{i=n}^{1} x_{i(n-i)} = n^2&&\\
\end{align*}

We have all the required information to model the problem with Choco.
\begin{note}
	For the moment, we just talk about \emph{model translation} from a mathematical representation to Choco.
	Choco can be used as a \emph{black box}, that means we just need to define the problem without knowing the way it will be solved. We can therefore focus on the modeling not on the solving.
\end{note}

\subsection{To Choco...}\label{introduction:inchoco}\hypertarget{introduction:inchoco}{}

First, we need to know some of the basic Choco objects:
\begin{itemize}
\item 
The \textbf{model} (object \texttt{Model} in Choco) is one of the central elements of a Choco program. Variables and constraints are associated to it.
\item
The \textbf{variables} (objects \texttt{IntegerVariable}, \texttt{SetVariable}, and \texttt{RealVariable} in Choco) are the \emph{unknown} of the problem. Values of variables are taken from a \textbf{domain} which is defined by a set of values or quite often simply by a lower bound and an upper bound of the allowed values. The domain is given when creating the variable.
\begin{note}
Do not forget that we manipulate \textbf{variables} in the mathematical sense (as opposed to classical computer science). Their effective value will be known only once the problem has been solved.
\end{note}
\item
The \textbf{constraints} define relations to be satisfied between variables and constants.
In our first model, we only use the following constraints provided by Choco:
\begin{itemize}
	\item \texttt{eq(var1, var2)} which ensures that \texttt{var1} equals \texttt{var2}.
	\item \texttt{neq(var1, var2)} which ensures that \texttt{var1} is not equal to \texttt{var2}.
	\item \texttt{sum(var[])} which returns expression \texttt{var[0]+var[1]+...+var[n]}.
\end{itemize}
\end{itemize}

\subsection{The program}\label{introduction:theprogram}\hypertarget{introduction:theprogram}{}
After having created your java class file, import the Choco class to use the API:
\begin{lstlisting}
  import choco.Choco;
\end{lstlisting}
First of all, let's create a Model:
\lstinputlisting{java/imagicsquare1.j2t}
We create an instance of \texttt{CPModel()} for \textbf{C}onstraint \textbf{P}rogramming Model.
Do not forget to add the following imports:
\begin{lstlisting}
  import choco.cp.model.CPModel;
\end{lstlisting}
Then we declare the variables of the problem:
\lstinputlisting{java/imagicsquare2.j2t}
Add the import:
\begin{lstlisting}
  import choco.kernel.model.variables.integer.IntegerVariable;
\end{lstlisting}
We have defined the variable using the \texttt{makeIntVar} method which creates an enumerated domain: all the values are stored in the java object (beware, it is usually not necessary to store all the values and it is less efficient than to create a bounded domain).

\noindent Now, we are going to state a constraint ensuring that all variables must have a different value:
\lstinputlisting{java/imagicsquare3.j2t}
Add the import:
\begin{lstlisting}
  import choco.kernel.model.constraints.Constraint;
\end{lstlisting}
Then, we add the constraint ensuring that the magic sum is respected:
\lstinputlisting{java/imagicsquare4.j2t}
Then we define the constraint ensuring that each column is equal to the magic sum.
Actually, \texttt{var} just denotes the rows of the square. So we have to declare a temporary array of variables that defines the columns.
\lstinputlisting{java/imagicsquare5.j2t}
It is sometimes useful to define some temporary variables to keep the model simple or to reorder array of variables. That is why we also define two other temporary arrays for diagonals.
\lstinputlisting{java/imagicsquare6.j2t}
Now, we have defined the model. The next step is to solve it.
For that, we build a Solver object:
\lstinputlisting{java/imagicsquare7.j2t}

with the imports:
\begin{lstlisting}
  import choco.cp.solver.CPSolver;
\end{lstlisting}
We create an instance of \texttt{CPSolver()} for Constraint Programming Solver.
Then, the solver reads (translates) the model and solves it:
\lstinputlisting{java/imagicsquare8.j2t}
The only variables that need to be printed are the ones in \texttt{var} (all the others are only references to these ones). 
\begin{note}
We have to use the Solver to get the value of each variable of the model. The Model only declares the objects, the Solver finds their value.
\end{note}
We are done, we have created our first Choco program. 
The complete source code can be found here: \href{media/zip/exmagicsquare.zip}{ExMagicSquare.zip}


\subsection{In summary}\label{introduction:whatisimportant}\hypertarget{introduction:whatisimportant}{}
\begin{itemize}
	\item A Choco Model is defined by a set of Variables with a given domain and a set of Constraints that link Variables:
it is necessary to add both Variables and Constraints to the Model.
	\item temporary Variables are useful to keep the Model readable, or necessary when reordering arrays.
	\item The value of a Variable can be known only once the Solver has found a solution.
	\item To keep the code readable, you can avoid the calls to the static methods of the Choco classes, by importing the static classes, i.e. instead of:
\begin{lstlisting}
  import choco.Choco;
  ...
  IntegerVariable v = Choco.makeIntVar("v", 1, 10);
  ...
  Constraint c = Choco.eq(v, 5);
\end{lstlisting}
you can use:
\begin{lstlisting}
  import static choco.Choco.*;
  ...
  IntegerVariable v = makeIntVar("v", 1, 10);
  ...
  Constraint c = eq(v, 5);
\end{lstlisting}
\end{itemize}

\section{Complete examples}\label{model:completeexamples}\hypertarget{model:completeexamples}{}
We provide now the complete Choco model for the three examples \hyperlink{introduction:examples}{previously described}.

\subsection{Example 1: the n-queens problem with Choco}\label{model:example1:nqueenschoco}\hypertarget{model:example1:nqueenschoco}{}
This first model for the \hyperlink{introduction:example1:nqueens}{n-queens problem} only involves binary constraints of differences between integer variables. One can immediately recognize the 4 main elements of any Choco code. First of all, create the model object. Then create the variables by using the Choco API (One variable per queen giving the row (or the column) where the queen will be placed). Finally, add the constraints and solve the problem. 

\lstinputlisting{java/inqueen.j2t}

\subsection{Example 2: the ternary Steiner problem with Choco}\label{model:example2:ternarysteinerchoco}\hypertarget{model:example2:ternarysteinerchoco}{}
The \hyperlink{introduction:example2:theternarysteinerproblem}{ternary Steiner problem} is entirely modeled using set variables and set constraints. 
\lstinputlisting{java/iternarysteiner.j2t}

\subsection{Example 3: the CycloHexane problem with Choco}\label{model:example3:thecyclohexaneproblemwithchoco}\hypertarget{model:example3:thecyclohexaneproblemwithchoco}{}
Real variables are illustrated on the problem of finding the 3D configuration of a cyclohexane molecule. 
\lstinputlisting{java/icyclohexane.j2t}



\chapter{The model}\label{doc:model}\hypertarget{doc:model}{}

The {\tt Model}, along with the {\tt Solver}, is one of the two key elements of any Choco program. The Choco {\tt Model} allows to describe a problem in an easy and declarative way: it simply records the variables and the constraints defining the problem.

This section describes the large API provided by Choco to create different types of \hyperlink{model:variables}{variables} and \hyperlink{model:constraints}{constraints}.

%\begin{note}
\textbf{Note that a static import is required to use the Choco API:}
\begin{lstlisting}
  import static choco.Choco.*;
\end{lstlisting}
%It is mandatory in order to compile !
%\end{note}

%\section{How to create a model}\label{model:howtocreateamodel}\hypertarget{model:howtocreateamodel}{}
First of all, a {\tt Model} object is created as follows:
\begin{lstlisting}
Model model = new CPModel();
\end{lstlisting}
In that specific case, a Constraint Programming (CP) {\tt Model} object has been created. 


%%%%%%%%%%%%%%%%%%%%%%%%%%%%%%%%%%%%%%%%%%%%%%%%%%%%%%%%%%%%%%%%%%%%%%%%%%%%%%%%%%%%%%%%%%%%%%%%%%%%%%%%%%%%%%%%%%%%%%%%%%%%%%%%%%%%%%%%%%%%%%%%%%%
%%%%%%%%%%%%%%%%%%%%%%%%%%%%%%%%%%%%%%%%%%%%%%%%%%%%% VARIABLE %%%%%%%%%%%%%%%%%%%%%%%%%%%%%%%%%%%%%%%%%%%%%%%%%%%%%%%%%%%%%%%%%%%%%%%%%%%%%%%%%%%%
%%%%%%%%%%%%%%%%%%%%%%%%%%%%%%%%%%%%%%%%%%%%%%%%%%%%%%%%%%%%%%%%%%%%%%%%%%%%%%%%%%%%%%%%%%%%%%%%%%%%%%%%%%%%%%%%%%%%%%%%%%%%%%%%%%%%%%%%%%%%%%%%%%%


\section{Variables}\label{model:variables}\hypertarget{model:variables}{}

%Choco provides a large API to create different types of variables : \textbf{integer}, \textbf{real} and \textbf{set}. 

A Variable is defined by a type (\hyperlink{integervariable}{integer}, \hyperlink{realvariable}{real}, or \hyperlink{setvariable}{set} variable), a name, and the values of its domain. When creating a simple variable, some options can be set to specify its domain representation (ex: enumerated or bounded) within the {\tt Solver}.
%Some kinds of variables have options for their domain, it may have an effect on what kind of specific object is created when the model is read by the solver.
\begin{note}
The choice of the domain should be considered. The efficiency of the solver often depends on judicious choice of the domain type.
\end{note}
Variables can be combined as \hyperlink{model:expressionvariables}{expression variables} using operators.

One or more variables can be added to the model using the following methods of the \texttt{Model} class:
\lstinputlisting{java/mvariabledeclaration1.j2t}

\begin{note}
Explictly addition of variables is not mandatory. See \hyperlink{model:constraints}{\tt Constraint} for more details.
\end{note}

Specific role of variables \emph{var} can be defined with \emph{options}:  \hyperlink{model:decisionvariables}{non-decision} variables or  \hyperlink{model:objectivevariable}{objective} variable;
\lstinputlisting{java/mvariabledeclaration2.j2t}

%%%%%%%%%%%%%%%%%%%%%%%%%%%%%%%%%%%%%%%%%%%%%%%%%%%%%%%%%%%%%%%%%%%%%%%%%%%%%%%%%%%%%%%%%%%%%%%%%%%%%%%%%%%%%%%%%%%%%%%%%%%%%%%%%%%%%%%%%%%%%%%%%%%
%%%%%%%%%%%%%%%%%%%%%%%%%%%%%%%%%%%%%%%%%%%%%%%%%%%%% SIMPLE VARIABLE  %%%%%%%%%%%%%%%%%%%%%%%%%%%%%%%%%%%%%%%%%%%%%%%%%%%%%%%%%%%%%%%%%%%%%%%%%%%%

\subsection{Simple Variables}\label{model:simplevariables}\hypertarget{model:simplevariables}{}
See Section \hyperlink{ch:vars}{Variables} for details:

\begin{notedef}\tt
\hyperlink{integervariable}{IntegerVariable}, \hyperlink{setvariable}{SetVariable}, \hyperlink{realvariable}{RealVariable}
\end{notedef}

%%%%%%%%%%%%%%%%%%%%%%%%%%%%%%%%%%%%%%%%%%%%%%%%%%%%%%%%%%%%%%%%%%%%%%%%%%%%%%%%%%%%%%%%%%%%%%%%%%%%%%%%%%%%%%%%%%%%%%%%%%%%%%%%%%%%%%%%%%%%%%%%%%%
%%%%%%%%%%%%%%%%%%%%%%%%%%%%%%%%%%%%%%%%%%%%%%%%%%%%% CONSTANT VARIABLE  %%%%%%%%%%%%%%%%%%%%%%%%%%%%%%%%%%%%%%%%%%%%%%%%%%%%%%%%%%%%%%%%%%%%%%%%%%

\subsection{Constants}\label{model:constants}\hypertarget{model:constants}{}
A constant is a variable with a fixed domain. An \hyperlink{integervariable}{\tt IntegerVariable} declared with a unique value is automatically set as constant. A constant declared twice or more is only stored once in a model.

\lstinputlisting{java/mconstant.j2t}

%%%%%%%%%%%%%%%%%%%%%%%%%%%%%%%%%%%%%%%%%%%%%%%%%%%%%%%%%%%%%%%%%%%%%%%%%%%%%%%%%%%%%%%%%%%%%%%%%%%%%%%%%%%%%%%%%%%%%%%%%%%%%%%%%%%%%%%%%%%%%%%%%%%
%%%%%%%%%%%%%%%%%%%%%%%%%%%%%%%%%%%%%%%%%%%%%%%%%%%%% EXPRESSION VARIABLE  %%%%%%%%%%%%%%%%%%%%%%%%%%%%%%%%%%%%%%%%%%%%%%%%%%%%%%%%%%%%%%%%%%%%%%%%
\subsection{Expression variables and operators}\label{model:expressionvariables}\hypertarget{model:expressionvariables}{}
Expression variables represent the result of combinations between variables of the same type made by operators. Two types of expression variables exist : 
\begin{notedef}
\textbf{\tt IntegerExpressionVariable} and \textbf{\tt RealExpressionVariable}.
\end{notedef}
One can define a buffered expression variable to make a constraint easy to read, for example:
\lstinputlisting{java/mexpressionvariable.j2t}

%\section{Operators}\label{model:operators}\hypertarget{model:operators}{}

To construct expressions of variables, simple operators can be used. Each returns a \texttt{ExpressionVariable} object:
\begin{notedef}\tt
\hyperlink{abs:absoperator}{abs}, \hyperlink{cos:cosoperator}{cos}, \hyperlink{disteq:disteqoperator}{distEq}, \hyperlink{distgt:distgtoperator}{distGt}, \hyperlink{distlt:distltoperator}{distLt}, \hyperlink{distneq:distneqoperator}{distNeq}, \hyperlink{div:divoperator}{div}, \hyperlink{ifthenelse:ifthenelseoperator}{ifThenElse}, \hyperlink{max:maxoperator}{max}, \hyperlink{min:minoperator}{min}, \hyperlink{minus:minusoperator}{minus}, \hyperlink{mod:modoperator}{mod}, \hyperlink{mult:multoperator}{mult}, \hyperlink{neg:negoperator}{neg}, \hyperlink{plus:plusoperator}{plus}, \hyperlink{power:poweroperator}{power}, \hyperlink{scalar:scalaroperator}{scalar}, \hyperlink{sin:sinoperator}{sin}, \hyperlink{sum:sumoperator}{sum}.
\end{notedef}
Note that these operators are not considered as constraints: they do not return a \texttt{Constraint} objet but a \texttt{Variable} object.

%%%%%%%%%%%%%%%%%%%%%%%%%%%%%%%%%%%%%%%%%%%%%%%%%%%%%%%%%%%%%%%%%%%%%%%%%%%%%%%%%%%%%%%%%%%%%%%%%%%%%%%%%%%%%%%%%%%%%%%%%%%%%%%%%%%%%%%%%%%%%%%%%%%
%%%%%%%%%%%%%%%%%%%%%%%%%%%%%%%%%%%%%%%%%%%%%%%%%%%%% MULTIPLE VARIABLE  %%%%%%%%%%%%%%%%%%%%%%%%%%%%%%%%%%%%%%%%%%%%%%%%%%%%%%%%%%%%%%%%%%%%%%%%%%

\subsection{MultipleVariables}\label{model:multiplevariables}\hypertarget{model:multiplevariables}{}
These are syntaxic sugar. To make their declaration easier, \hyperlink{tree:treeconstraint}{\tt tree}, \hyperlink{geost:geostconstraint}{\tt geost}, and scheduling constraints allow or require to use multiple variables, like \texttt{TreeParametersObject}, \texttt{GeostObject} or \hyperlink{taskvariable}{\tt TaskVariable}.
See also the code examples for these constraints.

%%%%%%%%%%%%%%%%%%%%%%%%%%%%%%%%%%%%%%%%%%%%%%%%%%%%%%%%%%%%%%%%%%%%%%%%%%%%%%%%%%%%%%%%%%%%%%%%%%%%%%%%%%%%%%%%%%%%%%%%%%%%%%%%%%%%%%%%%%%%%%%%%%%
%%%%%%%%%%%%%%%%%%%%%%%%%%%%%%%%%%%%%%%%%%%%%%%%%%%%% OPTIONS %%%%%%%%%%%%%%%%%%%%%%%%%%%%%%%%%%%%%%%%%%%%%%%%%%%%%%%%%%%%%%%%%%%%%%%%%%%%%%%%%%%%%

\subsection{Decision/non-decision variables}\label{model:decisionvariables}\hypertarget{model:decisionvariables}{}

By default, each variable added to a model is a decision variable, \textit{i.e.} is included in the default search strategy. A variable can be stated as a non decision one if its value can be computed by side-effect. To specify non decision variables, one can 
\begin{itemize}
\item exclude them from its search strategies (see \hyperlink{solver:searchstrategy}{search strategy} for more details);
\item specify non-decision variables (adding \hyperlink{vnodecision:vnodecisionoptions}{\tt Options.V\_NO\_DECISION} to their options) and keep the default search strategy.
\end{itemize}
\lstinputlisting{java/mnodecision1.j2t}
Each of these options can also be set within a single instruction for a group of variables, as follows: 
\lstinputlisting{java/mnodecision2.j2t}

\begin{note}
 The declaration of a \hyperlink{solver:searchstrategy}{search strategy} will erase setting \hyperlink{vnodecision:vnodecisionoptions}{\tt Options.V\_NO\_DECISION}.
\end{note}
  \todo{more precise: user-defined/pre-defined, variable and/or value heuristics ?}

\subsection{Objective variable}\label{model:objectivevariable}\hypertarget{model:objectivevariable}{}
You can define an objective variable directly within the model, by using option \hyperlink{vobjective:vobjectiveoptions}{\tt Options.V\_OBJECTIVE}:
\lstinputlisting{java/mobjective.j2t}

Only one variable can be defined as an objective. If more than one objective variable is declared, then only the last one will be taken into account.

Note that optimization problems can be declared without defining an objective variable within the model (see the \hyperlink{solver:optimization}{optimization example}.)

%%%%%%%%%%%%%%%%%%%%%%%%%%%%%%%%%%%%%%%%%%%%%%%%%%%%%%%%%%%%%%%%%%%%%%%%%%%%%%%%%%%%%%%%%%%%%%%%%%%%%%%%%%%%%%%%%%%%%%%%%%%%%%%%%%%%%%%%%%%%%%%%%%%
%%%%%%%%%%%%%%%%%%%%%%%%%%%%%%%%%%%%%%%%%%%%%%%%%%%%% CONSTRAINT  %%%%%%%%%%%%%%%%%%%%%%%%%%%%%%%%%%%%%%%%%%%%%%%%%%%%%%%%%%%%%%%%%%%%%%%%%%%%%%%%%
%%%%%%%%%%%%%%%%%%%%%%%%%%%%%%%%%%%%%%%%%%%%%%%%%%%%%%%%%%%%%%%%%%%%%%%%%%%%%%%%%%%%%%%%%%%%%%%%%%%%%%%%%%%%%%%%%%%%%%%%%%%%%%%%%%%%%%%%%%%%%%%%%%%

\section{Constraints}\label{model:constraints}\hypertarget{model:constraints}{}
Choco provides a large number of simple and global constraints and allows the user to easily define its own new constraint.
% Either basic, global (a \hyperlink{constraints}{large set of global constraints} are available) or \hyperlink{advanced:defineyourownconstraint}{user-defined} constraints, they are used to specify conditions to be held on variables to the model. 
A constraint deals with one or more variables of the model and specify conditions to be held on these variables. 
A constraint is stated into the model by using the following methods available from the \texttt{Model} API: 

\lstinputlisting{java/mconstraintdeclaration1.j2t}

\begin{note}\
Adding a constraint automatically adds its variables to the model (explicit declaration of variables addition is not mandatory).
\end{note}


\subsubsection{Example:} adding a difference (disequality) constraint between two variables of the model

\lstinputlisting{java/mconstraintdeclaration2.j2t}

Available \emph{options} depend on the kind of constraint \emph{c} to add: they allow, for example, to choose the filtering algorithm to run during propagation. See \hyperlink{optionssettings}{Section options ans settings} for more details, specific APIs exist for declaring options constraints.

This section presents the constraints available in the Choco API sorted by type or by domain. Related sections:
\begin{itemize}
\item a detailed description (with options, examples, references) of each constraint is given in Section \hyperlink{ch:constraints}{constraints}
\item Section \hyperlink{doc:applications}{applications} shows how to apply some specific global constraints
\item Section \hyperlink{advanced:defineyourownconstraint}{user-defined constraint} explains how to create its own constraint.
\end{itemize}

%%%%%%%%%%%%%%%%%%%%%%%%%%%%%%%%%%%%%%%%%%%%%%%%%%%%%%%%%%%%%%%%%%%%%%%%%%%%%%%%%%%%%%%%%%%%%%%%%%%%%%%%%%%%%%%%%%%%%%%%%%%%%%%%%%%%%%%%%%%%%%%%%%%
%%%%%%%%%%%%%%%%%%%%%%%%%%%%%%%%%%%%%%%%%%%%%%%%%%%%% BINARY CONSTRAINT  %%%%%%%%%%%%%%%%%%%%%%%%%%%%%%%%%%%%%%%%%%%%%%%%%%%%%%%%%%%%%%%%%%%%%%%%%%

\subsection{Binary constraints}\label{model:comparisonconstraints}\hypertarget{model:comparisonconstraints}{}
%The simplest constraints are comparisons which are defined over expressions of variables such as linear combinations. The following comparison constraints can be accessed through the \texttt{Model} API:
Constraints involving two integer variables
\begin{notedef}\tt
  \begin{itemize}
  \item \hyperlink{eq:eqconstraint}{eq}, \hyperlink{geq:geqconstraint}{geq}, \hyperlink{gt:gtconstraint}{gt}, \hyperlink{leq:leqconstraint}{leq}, \hyperlink{lt:ltconstraint}{lt}, \hyperlink{neq:neqconstraint}{neq}
  \item \hyperlink{abs:absconstraint}{abs}, \hyperlink{oppositesign:oppositesignconstraint}{oppositeSign}, \hyperlink{samesign:samesignconstraint}{sameSign}
  \end{itemize}
\end{notedef}

%%%%%%%%%%%%%%%%%%%%%%%%%%%%%%%%%%%%%%%%%%%%%%%%%%%%%%%%%%%%%%%%%%%%%%%%%%%%%%%%%%%%%%%%%%%%%%%%%%%%%%%%%%%%%%%%%%%%%%%%%%%%%%%%%%%%%%%%%%%%%%%%%%%
%%%%%%%%%%%%%%%%%%%%%%%%%%%%%%%%%%%%%%%%%%%%%%%%%%%%% TERNARY CONSTRAINT  %%%%%%%%%%%%%%%%%%%%%%%%%%%%%%%%%%%%%%%%%%%%%%%%%%%%%%%%%%%%%%%%%%%%%%%%%

\subsection{Ternary constraints}\label{model:ternaryconstraints}\hypertarget{model:ternaryconstraints}{}
Constraints involving three integer variables
\begin{notedef}\tt
  \begin{itemize}
  \item \hyperlink{distanceeq:distanceeqconstraint}{distanceEQ}, \hyperlink{distanceneq:distanceneqconstraint}{distanceNEQ}, \hyperlink{distancegt:distancegtconstraint}{distanceGT}, \hyperlink{distancelt:distanceltconstraint}{distanceLT}
  \item \hyperlink{intdiv:intdivconstraint}{intDiv}, \hyperlink{mod:modconstraint}{mod}, \hyperlink{times:timesconstraint}{times}
  \end{itemize}
\end{notedef}

%%%%%%%%%%%%%%%%%%%%%%%%%%%%%%%%%%%%%%%%%%%%%%%%%%%%%%%%%%%%%%%%%%%%%%%%%%%%%%%%%%%%%%%%%%%%%%%%%%%%%%%%%%%%%%%%%%%%%%%%%%%%%%%%%%%%%%%%%%%%%%%%%%%
%%%%%%%%%%%%%%%%%%%%%%%%%%%%%%%%%%%%%%%%%%%%%%%%%%%%% REAL CONSTRAINT  %%%%%%%%%%%%%%%%%%%%%%%%%%%%%%%%%%%%%%%%%%%%%%%%%%%%%%%%%%%%%%%%%%%%%%%%%%%%

\subsection{Constraints involving real variables}\label{model:realconstraints}\hypertarget{model:realconstraints}{}
%The simplest constraints are comparisons which are defined over expressions of variables such as linear combinations. The following comparison constraints can be accessed through the \texttt{Model} API:
Constraints involving two real variables
\begin{notedef}\tt
  \begin{itemize}
  \item \hyperlink{eq:eqconstraint}{eq}, \hyperlink{geq:geqconstraint}{geq}, \hyperlink{leq:leqconstraint}{leq}
  \end{itemize}
\end{notedef}

%%%%%%%%%%%%%%%%%%%%%%%%%%%%%%%%%%%%%%%%%%%%%%%%%%%%%%%%%%%%%%%%%%%%%%%%%%%%%%%%%%%%%%%%%%%%%%%%%%%%%%%%%%%%%%%%%%%%%%%%%%%%%%%%%%%%%%%%%%%%%%%%%%%
%%%%%%%%%%%%%%%%%%%%%%%%%%%%%%%%%%%%%%%%%%%%%%%%%%%%% SET CONSTRAINT  %%%%%%%%%%%%%%%%%%%%%%%%%%%%%%%%%%%%%%%%%%%%%%%%%%%%%%%%%%%%%%%%%%%%%%%%%%%%%

\subsection{Constraints involving set variables}\label{model:setconstraints}\hypertarget{model:setconstraints}{}
%The simplest constraints are comparisons which are defined over expressions of variables such as linear combinations. The following comparison constraints can be accessed through the \texttt{Model} API:
Set constraints are illustrated on the \hyperlink{model:example2:ternarysteinerchoco}{ternary Steiner problem}. 
\begin{notedef}\tt
  \begin{itemize}
  \item \hyperlink{eqcard:eqcardconstraint}{eqCard}, \hyperlink{geqcard:geqcardconstraint}{geqCard}, \hyperlink{leqcard:leqcardconstraint}{leqCard}
  \item \hyperlink{member:memberconstraint}{member}, \hyperlink{notmember:notmemberconstraint}{notMember}
  \item \hyperlink{isincluded:isincludedconstraint}{isIncluded}, \hyperlink{isnotincluded:isnotincludedconstraint}{isNotIncluded}, \hyperlink{setdisjoint:setdisjointconstraint}{setDisjoint}
  \item \hyperlink{setinter:setinterconstraint}{setInter}, \hyperlink{setunion:setunionconstraint}{setUnion}
  \item \hyperlink{max:maxofaset}{max}, \hyperlink{min:minofaset}{min}
  \item \hyperlink{pack:packconstraint}{pack}
  \end{itemize}
\end{notedef}

%\hyperlink{max:maxconstraint}{max}, \hyperlink{min:minconstraint}{min},

%%%%%%%%%%%%%%%%%%%%%%%%%%%%%%%%%%%%%%%%%%%%%%%%%%%%%%%%%%%%%%%%%%%%%%%%%%%%%%%%%%%%%%%%%%%%%%%%%%%%%%%%%%%%%%%%%%%%%%%%%%%%%%%%%%%%%%%%%%%%%%%%%%%
%%%%%%%%%%%%%%%%%%%%%%%%%%%%%%%%%%%%%%%%%%%%%%%%%%%%% CHANNELING CONSTRAINT  %%%%%%%%%%%%%%%%%%%%%%%%%%%%%%%%%%%%%%%%%%%%%%%%%%%%%%%%%%%%%%%%%%%%%%

\subsection{Channeling constraints}\label{model:channelingconstraints}\hypertarget{model:channelingconstraints}{}
The use of a redundant model is a frequent technique to strengthen propagation or to get more freedom to design dedicated search heuristics. The following constraints allow to ensure integrity of different models:
\begin{notedef}\tt
  \begin{itemize}
  \item \hyperlink{inversechanneling:inversechannelingconstraint}{inverseChanneling}, \hyperlink{boolchanneling:boolchannelingconstraint}{boolChanneling}, \hyperlink{domainconstraint:domainconstraintconstraint}{domainConstraint}
  \end{itemize}
\end{notedef}
More complex channeling can be done using reified constraints (see Section \hyperlink{model:reifiedconstraints}{reification}) although they are less efficient. For example, to ensure that two variables are equal or not, one can reify the equality into a boolean variables :
\lstinputlisting{java/cchannelingreified.j2t}

%%%%%%%%%%%%%%%%%%%%%%%%%%%%%%%%%%%%%%%%%%%%%%%%%%%%%%%%%%%%%%%%%%%%%%%%%%%%%%%%%%%%%%%%%%%%%%%%%%%%%%%%%%%%%%%%%%%%%%%%%%%%%%%%%%%%%%%%%%%%%%%%%%%
%%%%%%%%%%%%%%%%%%%%%%%%%%%%%%%%%%%%%%%%%%%%%%%%%%%%% EXTENSIONS CONSTRAINT  %%%%%%%%%%%%%%%%%%%%%%%%%%%%%%%%%%%%%%%%%%%%%%%%%%%%%%%%%%%%%%%%%%%%%%

\subsection{Constraints in extension and relations}\label{model:arbitraryconstraintsinextension}\hypertarget{model:arbitraryconstraintsinextension}{}
Choco supports the statement of constraints defining arbitrary relations over two or more variables.
Such a relation may be defined by three means:
\begin{itemize}
	\item \textbf{feasible table:} the list of allowed tuples of values (that belong to the relation),
	\item \textbf{infeasible table:} the list of forbidden tuples of values (that do not belong to the relation),
	\item \textbf{predicate:} a method to be called in order to check whether a tuple of values belongs or not to the relation.
\end{itemize}
On the one hand, constraints based on tables may be rather memory consuming in case of large domains, although one relation table may be shared by several constraints. On the other hand, predicate constraints require little memory as they do not cache truth values, but imply some run-time overhead for calling the feasibility test. Table constraints are thus well suited for constraints over small domains; while predicate constraints are well suited for situations with large domains. 

Different levels of consistency can be enforce on constraints in extension: 
\begin{itemize}
\item several arc-consistency (AC) algorithms for binary relations
\item two AC algorithms for n-ary relations dedicated either to positive or to negative tables (relation defined by the allowed or forbidden tuples)
\item a weaker forward-checking (FC) algorithm for n-ary relations.
\end{itemize}

The Choco API for creating constraints in extension are as follows:
\begin{notedef}\tt
  \begin{itemize}
  \item \hyperlink{feaspairac:feaspairacconstraint}{feasPairAC}, \hyperlink{infeaspairac:infeaspairacconstraint}{infeasPairAC}, \hyperlink{relationpairac:relationpairacconstraint}{relationPairAC}
  \item \hyperlink{feastupleac:feastupleacconstraint}{feasTupleAC}, \hyperlink{infeastupleac:infeastupleacconstraint}{infeasTupleAC}, \hyperlink{relationtupleac:relationtupleacconstraint}{relationTupleAC}
  \item \hyperlink{feastuplefc:feastuplefcconstraint}{feasTupleFC}, \hyperlink{infeastuplefc:infeastuplefcconstraint}{infeasTupleFC}, \hyperlink{relationtuplefc:relationtuplefcconstraint}{relationTupleFC}
  \end{itemize}
\end{notedef}

\subsubsection{Relations.}
A same relation might be shared among several constraints, in this case it is highly recommended to create it first and then use the \hyperlink{relationpairac:relationpairacconstraint}{relationPairAC}, \hyperlink{relationtupleac:relationtupleacconstraint}{relationTupleAC}, or \hyperlink{relationtuplefc:relationtuplefcconstraint}{relationTupleFC} API  on the same relation for each constraint.

For binary relations, the following Choco API is provided:
\mylst{makeBinRelation(int[] min, int[] max, List<int[]>pairs, boolean feas)}

It returns a \texttt{BinRelation} giving a list of compatible (\texttt{feas=true}) or incompatible (\texttt{feas=false}) pairs of values. This relation can be applied to any pair of variables $(x_1,x_2)$ whose domains are included in the \texttt{min/max} intervals, i.e. such that:
$$\mathtt{min}[i] \le x_i.\mathtt{getInf}() \le x_i.\mathtt{getSup}() \le  \mathtt{max}[i],\quad \forall i.$$
Bounds \texttt{min/max} are mandatory in order to allow to compute the opposite of the relation if needed.

For n-ary relations, the corresponding Choco API is:
\mylst{makeLargeRelation(int[] min, int[] max, List<int[]> tuples, boolean feas);}
It returns a \texttt{LargeRelation}. If \texttt{feas=true}, the returned relation matches also the \texttt{IterLargeRelation} interface which provides constant time iteration abilities over tuples (for compatibility with the GAC algorithm used over feasible tuples).
\lstinputlisting{java/mlargerelation.j2t}

Lastly, some specific relations can be defined without storing the tuples, as in the following example (\texttt{TuplesTest} extends \texttt{LargeRelation}):
\lstinputlisting{java/mnotallequal.j2t}
Then, a \emph{NotAllEqual} constraint can be stated within the problem by:
\lstinputlisting{java/mrelationtuplefc.j2t}
%Again, for compatibility with the GAC algorithm invoked by relationTupleAC, such a relation has to match the \texttt{IterLargeRelation} interface for feasible tuples.


%%%%%%%%%%%%%%%%%%%%%%%%%%%%%%%%%%%%%%%%%%%%%%%%%%%%%%%%%%%%%%%%%%%%%%%%%%%%%%%%%%%%%%%%%%%%%%%%%%%%%%%%%%%%%%%%%%%%%%%%%%%%%%%%%%%%%%%%%%%%%%%%%%%
%%%%%%%%%%%%%%%%%%%%%%%%%%%%%%%%%%%%%%%%%%%%%%%%%%%%% REIFIED CONSTRAINT  %%%%%%%%%%%%%%%%%%%%%%%%%%%%%%%%%%%%%%%%%%%%%%%%%%%%%%%%%%%%%%%%%%%%%%%%%

\subsection{Reified constraints}\label{model:reifiedconstraints}\hypertarget{model:reifiedconstraints}{}
Constraints involved in another constraint are usually called reified constraints. Typical examples of reified constraints are
 constraints combined with logical operators, such as $(x \neq y) \lor (z \le 9)$.

%\subsubsection{To reify a constraint into a boolean variable.}\label{model:toreifyaconstraintintoabooleanvariable}\hypertarget{model:toreifyaconstraintintoabooleanvariable}{}
Choco provides a generic constraint to reify any constraints on integer variables or set variables into a boolean variable expressing its truth value:
\begin{notedef}\tt
  \begin{itemize}
  \item \hyperlink{reifiedconstraint:reifiedconstraintconstraint}{reifiedConstraint}, \hyperlink{reifiedand:reifiedandconstraint}{reifiedAnd}, \hyperlink{reifiedleftimp:reifiedleftimpconstraint}{reifiedLeftImp}, \hyperlink{reifiednot:reifiednotconstraint}{reifiedNot}, \hyperlink{reifiedor:reifiedorconstraint}{reifiedOr}, \hyperlink{reifiedrightimp:reifiedrightimpconstraint}{reifiedRightImp}, \hyperlink{reifiedxnor:reifiedxnorconstraint}{reifiedXnor}, \hyperlink{reifiedxor:reifiedxorconstraint}{reifiedXor}
  \end{itemize}
\end{notedef}
This mechanism can be used for example to model MaxCSP problems where the number of satisfied constraints has to be maximized.
It is also intended to give the freedom to the user to build complex reified constraints. However, Choco provides a more simple and direct API to build complex expressions using boolean operators:
\begin{notedef}\tt
  \begin{itemize}
  \item \hyperlink{and:andconstraint}{and}, \hyperlink{or:orconstraint}{or}, \hyperlink{implies:impliesconstraint}{implies}, \hyperlink{ifonlyif:ifonlyifconstraint}{ifOnlyIf}, \hyperlink{ifthenelse:ifthenelseconstraint}{ifThenElse}, \hyperlink{not:notconstraint}{not}
  \end{itemize}
\end{notedef}
Such an expression is represented as a tree of operators. The leaves of this tree are made of variables, constants or even traditional constraints. Variables and constants can be combined as \texttt{ExpressionVariable} using \hyperlink{model:expressionvariables}{operators} (e.g, \texttt{mult(10,abs(w))}), or using simple constraints (e.g., \texttt{leq(z,9)}), or even using global constraints (e.g, \texttt{alldifferent(vars)}).
The language available on expressions is therefore slightly richer and matches the language used in the \href{http://cpai.ucc.ie/08/}{Constraint Solver Competition 2008} of the CPAI workshop.

For example, the following expression
$$((x = 10 * |y|) \lor (z \le 9))\quad \iff\quad \texttt{alldifferent}(a,b,c)$$
could be represented by :
\begin{lstlisting}
	Constraint exp = ifOnlyIf( or( eq(x, mult(10, abs(y))), leq(z, 9) ), 
                               alldifferent(new IntegerVariable[]{a,b,c}) );
\end{lstlisting}


\subsubsection{Handling complex expressions.}\label{model:handlingcomplexexpressions}\hypertarget{model:handlingcomplexexpressions}{}
Expressions offer a more powerful modeling language than the one available via standard constraints. However, they 
can not be handled as efficiently as the standard constraints that embed a dedicated propagation algorithm. We therefore
recommend you to carefully check that you can not model the expression using the intensional constraints of Choco before using
expressions.
Inside the solver, expressions can be represented in two different ways that can be decided at the modeling level, using the following {\tt Model} API:
\begin{lstlisting}
  setDefaultExpressionDecomposition(boolean decomp);
\end{lstlisting}
or the option \hyperlink{edecomp:edecompoptions}{\tt Options.E\_DECOMP}.
\begin{itemize}
\item The first way (\texttt{decomp=false}) is to handle them as \hyperlink{model:arbitraryconstraintsinextension}{constraints in extension}. The expression is then used to check a tuple in a dynamic way just like a n-ary relation that is defined without listing all the possible tuples. The expression is then propagated using the GAC3rm algorithm. This is very powerful as arc-consistency is obtained on the corresponding constraints.
\item The second way (\texttt{decomp=true}) is to decompose the expression automatically by introducing intermediate variables and eventually the generic \hyperlink{reifiedintconstraint:reifiedintconstraintconstraint}{\tt reifiedIntConstraint}. By doing so, the level of pruning decreases but expressions of larger arity involving large domains can be represented.
\end{itemize}

%\subsubsection{Tell the solver how to consider an expression.}
%The default representation of expressions can be enforced using the following API  on the model object: 
%Parameter \emph{decomp} tells the solver whether expressions shoud be considered as extensional constraints (\texttt{decomp=false}) or decomposed 
Once the default representation is chosen, one can also make exception for a particular expression using options on \texttt{addConstraint}. 
For example, the following code tells the solver to decompose e1 and not e2 :
\begin{lstlisting}
	model.setDefaultExpressionDecomposition(false);
	IntegerVariable x = makeIntVar("x", 1, 3, Options.V_BOUND);
	IntegerVariable y = makeIntVar("y", 1, 3, Options.V_BOUND);
	IntegerVariable z = makeIntVar("z", 1, 3, Options.V_BOUND);

	Constraint e1 = or(lt(x, y), lt(y, x));
	model.addConstraint(Options.E_DECOMP, e1);
	
	Constraint e2 = or(lt(y, z), lt(z, y));
	model.addConstraint(e2);
\end{lstlisting}

\subsubsection{When and how should I use expressions ?}\label{model:whenshouldiuseexpressions}\hypertarget{model:whenshouldiuseexpressions}{}
An expression (represented in extension) should be used in the case of a complex logical relationship that involves \textbf{few different variables}, each of \textbf{small domain}, and if \textbf{arc consistency} is desired on those variables.
In such a case, an expression can even be more powerful than a model using intermediate variables and intensional constraints.
Imagine the following ``crazy'' example :
\begin{lstlisting}
 or( and( eq( abs(sub(div(x,50),div(y,50))),1), eq( abs(sub(mod(x,50),mod(y,50))),2)),
     and( eq( abs(sub(div(x,50),div(y,50))),2), eq( abs(sub(mod(x,50),mod(y,50))),1)))
\end{lstlisting}
This expression has a small arity: it involves only two variables $x$ and $y$.
Let assume that their domains has no more than 300 values, then such an expression should typically not be decomposed. Indeed, arc consistency will create many holes in the domains and filter much more than if the relation was decomposed.

Conversely, an expression should be decomposed as soon as it involves a large number of variables, or at least one variable with a large domain.

%%%%%%%%%%%%%%%%%%%%%%%%%%%%%%%%%%%%%%%%%%%%%%%%%%%%%%%%%%%%%%%%%%%%%%%%%%%%%%%%%%%%%%%%%%%%%%%%%%%%%%%%%%%%%%%%%%%%%%%%%%%%%%%%%%%%%%%%%%%%%%%%%%%
%%%%%%%%%%%%%%%%%%%%%%%%%%%%%%%%%%%%%%%%%%%%%%%%%%%%% GLOBAL CONSTRAINT  %%%%%%%%%%%%%%%%%%%%%%%%%%%%%%%%%%%%%%%%%%%%%%%%%%%%%%%%%%%%%%%%%%%%%%%%%%

\subsection{Global constraints}\label{model:advancedconstraints}\hypertarget{model:advancedconstraints}{}
Choco includes several \href{http://www.emn.fr/x-info/sdemasse/gccat/}{global constraints}. Those constraints accept any number of variables and offer dedicated filtering algorithms which are able to make deductions where a decomposed model would not.
For instance, constraint \texttt{alldifferent}$(a,b,c,d)$ with $a,b\in[1,4]$ and $c,d\in[3,4]$ allows to deduce that $a$ and $b$ cannot be instantiated to $3$ or $4$; such rule cannot be inferred by simple binary constraints. 

The up-to-date list of global constraints available in Choco can be found within the Javadoc API.
Most of these global constraints are listed below according to their application fields.
Details and examples can be found in Section \hyperlink{ch:constraints}{Elements of Choco/Constraints}.
\subsubsection{Value constraints}\label{model:valueconstraints}\hypertarget{model:valueconstraints}{}
Constraints that put a restriction on how values can be assigned to usually one or several collections of variables.
See also in Global Constraint Catalog: \href{http://www.emn.fr/x-info/sdemasse/gccat/Kvalue_constraint.html}{value constraint}.

\vspace{1em}\noindent\begin{notedef}\tt
  \begin{itemize}
  \item counting distinct values: 
\hyperlink{alldifferent:alldifferentconstraint}{allDifferent}, 
\hyperlink{atmostnvalue:atmostnvalueconstraint}{atMostNValue},
\hyperlink{increasingnvalue:increasingnvalueconstraint}{increasingnvalue},
  \item counting values: 
\hyperlink{occurrence:occurrenceconstraint}{occurrence},
\hyperlink{occurrencemax:occurrencemaxconstraint}{occurrenceMax},
\hyperlink{occurrencemin:occurrenceminconstraint}{occurrenceMin},
\hyperlink{globalcardinality:globalcardinalityconstraint}{globalCardinality},
  \item indexing values: 
\hyperlink{nth:nthconstraint}{nth} (element),
\hyperlink{max:maxconstraint}{max},
\hyperlink{min:minconstraint}{min},
  \item ordering: 
\hyperlink{sorting:sortingconstraint}{sorting},
\hyperlink{increasingnvalue:increasingnvalueconstraint}{increasingnvalue},
\hyperlink{lex:lexconstraint}{lex}, 
\hyperlink{lexeq:lexeqconstraint}{lexeq},
\hyperlink{leximin:leximinconstraint}{leximin},
\hyperlink{lexchain:lexchainconstraint}{lexChain},
\hyperlink{lexchaineq:lexchaineqconstraint}{lexChainEq},
  \item tuple matching: 
\hyperlink{feastupleac:feastupleacconstraint}{feasTupleAC},
\hyperlink{feastuplefc:feastuplefcconstraint}{feasTupleFC},
\hyperlink{infeastupleac:infeastupleacconstraint}{infeasTupleAC},
\hyperlink{infeastuplefc:infeastuplefcconstraint}{infeasTupleFC},
\hyperlink{relationtupleac:relationtupleacconstraint}{relationTupleAC},
\hyperlink{relationtuplefc:relationtuplefcconstraint}{relationTupleFC},
  \item pattern matching: 
\hyperlink{regular:regularconstraint}{regular},
\hyperlink{costregular:costregularconstraint}{costRegular},
\hyperlink{multicostregular:multicostregularconstraint}{multiCostRegular}, 
\hyperlink{stretchcyclic:stretchcyclicconstraint}{stretchCyclic}, 
\hyperlink{stretchpath:stretchpathconstraint}{stretchPath}, 
\hyperlink{tree:treeconstraint}{tree},
  \end{itemize}
\end{notedef}

\subsubsection{Boolean constraints}\label{model:logicconstraints}\hypertarget{model:logicconstraints}{}
Logical operations on boolean expressions.
See also in Global Constraint Catalog: \href{http://www.emn.fr/x-info/sdemasse/gccat/KBoolean_constraint.html}{boolean constraint}.

\vspace{1em}\noindent\begin{notedef}\tt
\hyperlink{and:andconstraint}{and},
\hyperlink{or:orconstraint}{or},
\hyperlink{clause:clauseconstraint}{clause},
\end{notedef}

\subsubsection{Channelling constraints}\label{model:channellingconstraints}\hypertarget{model:channellingconstraints}{}
Constraints linking two collections of variables (many-to-many) or indexing one among many variables (one-to-many).
See also in Global Constraint Catalog: \href{http://www.emn.fr/x-info/sdemasse/gccat/Kchannelling_constraint.html}{channelling constraint}.

 \vspace{1em}\noindent\begin{notedef}\tt
   \begin{itemize}
   \item one-to-many: 
 \hyperlink{domainconstraint:domainconstraintconstraint}{domainConstraint},
 \hyperlink{nth:nthconstraint}{nth} (element),
 \hyperlink{max:maxconstraint}{max},
 \hyperlink{min:minconstraint}{min},
   \item many-to-many: 
 \hyperlink{inversechanneling:inversechannelingconstraint}{inverseChanneling},
 \hyperlink{inverseset:inversesetconstraint}{inverseset},
 \hyperlink{sorting:sortingconstraint}{sorting},
 \end{itemize}
 \end{notedef}

\subsubsection{Optimization constraints}\label{model:optimizationconstraints}\hypertarget{model:optimizationconstraints}{}
Constraints channelling a variable to the sum of the weights of a collection of variable-value assignments.
See also in Global Constraint Catalog: \href{http://www.emn.fr/x-info/sdemasse/gccat/Kcost_filtering_constraint.html}{cost-filtering constraint}.
\vspace{1em}\noindent\begin{notedef}\tt
 \begin{itemize}
  \item one cost: 
\hyperlink{occurrence:occurrenceconstraint}{occurrence},
\hyperlink{occurrencemax:occurrencemaxconstraint}{occurrenceMax},
\hyperlink{occurrencemin:occurrenceminconstraint}{occurrenceMin},
\hyperlink{knapsackproblem:knapsackproblemconstraint}{knapsackProblem},
\hyperlink{equation:equationconstraint}{equation},
\hyperlink{costregular:costregularconstraint}{costRegular},
\hyperlink{tree:treeconstraint}{tree},
 \item several costs:
\hyperlink{globalcardinality:globalcardinalityconstraint}{globalCardinality},
\hyperlink{multicostregular:multicostregularconstraint}{multiCostRegular}, 
 \end{itemize}
\end{notedef}

\subsubsection{Packing constraints (capacitated resources)}\label{model:packingconstraints}\hypertarget{model:packingconstraints}{}
Constraints involving items to be packed in bins without overlapping. More generaly, any constraints modelling the concurrent assignment of objects to one or several capacitated resources.
See also in Global Constraint Catalog: \href{http://www.emn.fr/x-info/sdemasse/gccat/Kresource_constraint.html}{resource constraint}.

\vspace{1em}\noindent\begin{notedef}\tt
   \begin{itemize}
   \item packing problems: 
\hyperlink{equation:equationconstraint}{equation},
\hyperlink{knapsackproblem:knapsackproblemconstraint}{knapsackProblem},
\hyperlink{pack:packconstraint}{pack} (bin-packing),
   \item geometric placement problems: 
\hyperlink{geost:geostconstraint}{geost}, 
   \item scheduling problems: 
\hyperlink{disjunctive:disjunctiveconstraint}{disjunctive}, 
\hyperlink{cumulative:cumulativeconstraint}{cumulative}, 
   \item timetabling problems: 
\hyperlink{costregular:costregularconstraint}{costRegular},
\hyperlink{multicostregular:multicostregularconstraint}{multiCostRegular}, 
 \end{itemize}
\end{notedef}

\subsubsection{Scheduling constraints (time assignment)}\label{model:schedulingconstraints}\hypertarget{model:schedulingconstraints}{}
Constraints involving tasks to be scheduled over a time horizon.
See also \hyperlink{schedulinganduseofthecumulative:schedulinganduseofthecumulativeconstraint}{scheduling application} and in Global Constraint Catalog: \href{http://www.emn.fr/x-info/sdemasse/gccat/Kscheduling_constraint.html}{scheduling constraint}.

\vspace{1em}\noindent\begin{notedef}\tt
   \begin{itemize}
   \item temporal constraints: 
\hyperlink{preceding:precedingconstraint}{preceding}, 
\hyperlink{precedencedisjoint:precedencedisjointconstraint}{precedenceDisjoint}, 
\hyperlink{precedenceimplied:precedenceimpliedconstraint}{precedenceImplied}, 
\hyperlink{precedencereified:precedencereifiedconstraint}{precedenceReified},
\hyperlink{forbiddeninterval:forbiddenintervalconstraint}{forbiddenInterval},
\hyperlink{tree:treeconstraint}{tree},
   \item resource constraints: 
\hyperlink{cumulative:cumulativeconstraint}{cumulative}, 
\hyperlink{disjunctive:disjunctiveconstraint}{disjunctive}, 
\hyperlink{geost:geostconstraint}{geost}, 
 \end{itemize}
\end{notedef}

%\part{solver}
\label{solver}
\hypertarget{solver}{}


\chapter{The solver}\label{solver:thesolver}\hypertarget{solver:thesolver}{}

%\section{How to create a solver}\label{solver:howtocreateasolver}\hypertarget{solver:howtocreateasolver}{}

To create a {\tt Solver}, one just needs to create a new object as follow:
\begin{lstlisting}
Solver solver = new CPSolver();
\end{lstlisting}
By this, a Constraint Programming (CP) {\tt Solver} object is created. 

%\section{Read a model}\label{solver:readamodel}\hypertarget{solver:readamodel}{}
The solver gives an API to read a model. The reading of a model is compulsory and must be done after the entire definition of the model. 
\begin{lstlisting}
solver.read(model);
\end{lstlisting}
The reading is divided in 2 parts: \hyperlink{solver:variablesreading}{variables reading} and \hyperlink{solver:constraintsreading}{constraints reading}.

\section{Variables reading}\label{solver:variablesreading}\hypertarget{solver:variablesreading}{}
The solver iterates over the variables of the Model to create solver-specific variables and domains (as defined in the model). 
Thus, three types of variables can be created: integer variables, real variables and set variables. 
Depending on the constructor, the correct domain is created (like bounded domain or enumerated domain for integer variables). 

\begin{note}
\textbf{Bound variables} are related to large domains which are only represented by their lower and upper bounds. The domain is encoded in a space efficient way and propagation events only concern bound updates. Value removals between the bounds are therefore ignored (\emph{holes} are not considered). The level of consistency achieved by most constraints on these variables is called \emph{bound-consistency}.

On the contrary, the domain of an \textbf{enumerated variable} is explicitly represented and every value is considered while pruning. Basic constraints are therefore often able to achieve \emph{arc-consistency} on enumerated variables (except for NP global constraint such as the cumulative constraint). Remember that switching from an enumerated variable to a bounded variables decrease the level of propagation achieved by the system.
\end{note}

%\begin{note}
Model variables and Solver variables are distinct. Solver variables are solver representation of the model variables. One can't access to variable value directly from the model variable. To access to a model variable thanks to the solver, use the following \texttt{Solver} API: \mylst{getVar(Variable v);}
%\end{note}

\subsection{Solver and IntegerVariables}\label{solver:solverandintegervariables}\hypertarget{solver:solverandintegervariables}{}

A model integer variable can be accessed by the method \textbf{\tt getVar(IntegerVariable v)} which returns a \textbf{\tt IntDomainVar} object:
\begin{lstlisting}
  IntegerVariable x = makeEnumIntVar("x", 1, 100);  // model variable
  IntDomainVar xOnSolver = solver.getVar(x);  // solver variable
\end{lstlisting}

The state of an \texttt{IntDomainVar} can be accessed through the main following public methods :

\noindent\begin{tabular}{p{.3\linewidth}p{.7\linewidth}}
  \hline
  \texttt{IntDomainVar} API &  description \\
  \hline
	\mylst{hasEnumeratedDomain()} &checks if the variable is an enumerated or a bound one\\
	\mylst{getInf()} &returns the lower bound of the variable\\
	\mylst{getSup()} &returns the upper bound of the variable\\
	\mylst{getVal()} &returns the value if it is instantiated\\
	\mylst{isInstantiated()} &checks if the domain is reduced to a singleton\\
	\mylst{canBeInstantiatedTo(int v)} &checks if the value \emph{v} is contained in the domain of the variable\\
	\mylst{getDomainSize()} &returns the current size of the domain\\
  \hline\\
\end{tabular}

For more informations on advanced uses of such \texttt{IntDomainVar}, see \hyperlink{advanced}{advanced uses}.

\subsection{Solver and SetVariables}\label{solver:solverandsetvariables}\hypertarget{solver:solverandsetvariables}{}

A model set variable can be access by the method \textbf{\tt getVar(SetVariable v)} which returns a \textbf{\tt SetVar} object:
\begin{lstlisting}
	SetVariable x = makeBoundSetVar("x", 1, 40); // model variable
	SetVar xOnSolver = solver.getVar(x); // solver variable
\end{lstlisting}
A set variable on integer values between $[1,n]$ has $2^{n}$ values (every possible subsets of $\{1..n\}$). This makes an exponential number of values and the domain is represented with two bounds corresponding to the intersection of all possible sets (called the kernel) and the union of all possible sets (called the envelope) which are the possible candidate values for the variable.

The state of a \texttt{SetVar} can be accessed through the main following public methods on the SetVar class:

\noindent\begin{tabular}{p{.3\linewidth}p{.7\linewidth}}
  \hline
  \texttt{SetVar} API &  description \\
  \hline
	\mylst{getCard()} &returns the \texttt{IntDomainVar} representing the cardinality of the set variable\\
	\mylst{isInDomainKernel(int v)} &checks if value \emph{v} is contained in the current kernel\\
	\mylst{isInDomainEnveloppe(int v)} &checks if value \emph{v} is contained in the current envelope\\
	\mylst{getDomain()} &returns the domain of the variable as a \texttt{SetDomain}. Iterators on envelope or kernel can than be called\\
	\mylst{getKernelDomainSize()} &returns the size of the kernel\\
	\mylst{getEnveloppeDomainSize()} &returns the size of the envelope\\
	\mylst{getEnveloppeInf()} &returns the first available value of the envelope\\
	\mylst{getEnveloppeSup()} &returns the last available value of the envelope\\
	\mylst{getKernelInf()} &returns the first available value of the kernel\\
	\mylst{getKernelSup()} &returns the last available value of the kernel\\
	\mylst{getValue()} &returns a table of integers \texttt{int[]} containing the current domain\\
  \hline\\
\end{tabular}


For more informations on advanced uses of such \texttt{SetVar}, see \hyperlink{advanced}{advanced uses}.

\subsection{Solver and RealVariables}\label{solver:solverandrealvariables}\hypertarget{solver:solverandrealvariables}{}

\begin{note}
\emph{Real variables are still under development but can be used to solve toy problems such as small systems of equations.}
\end{note}
 
A model real variable can be access by the method \textbf{\tt getVar(RealVariable v)} which returns a \texttt{RealVar} object:
\begin{lstlisting}
	RealVariable x = makeRealVar("x", 1.0, 3.0); // model variable
	RealVar xOnSolver = s.getVar(x); // solver variable
\end{lstlisting}

Continuous variables are useful for non linear equation systems which are encountered in physics for example.

\noindent\begin{tabular}{p{.3\linewidth}p{.7\linewidth}}
  \hline
  \texttt{RealVar} API &  description \\
  \hline
	\mylst{getInf()} &returns the lower bound of the variable (\texttt{double})\\
	\mylst{getSup()} &returns the upper bound of the variable (\texttt{double})\\
	\mylst{isInstantiated()} &checks if the domain of a variable is reduced to a canonical interval. A canonical interval indicates that the domain has reached the precision given by the user or the solver\\
  \hline\\
\end{tabular}


For more informations on advanced uses of such \texttt{RealVar}, see \hyperlink{advanced}{advanced uses}.

\section{Constraints reading}\label{solver:constraintsreading}\hypertarget{solver:constraintsreading}{}
After variables, the Solver iterates over the constraints added to the Model. It creates Solver constraints that encapsulates a filtering algorithm which are called when a propagation step occur or when external events happen on the variables belonging to the constraint, such as value removals or bounds modifications. And it add it to the constraint network. 

\section{Search Strategy}\label{solver:searchstrategy}\hypertarget{solver:searchstrategy}{}

A key ingredient of any constraint approach is a clever branching strategy. The construction of the search tree is done according to a series of \textit{branching objects} (that plays the role of achieving intermediate goals in logic programming). The user may specify the sequence of branching objects to be used to build the search tree. A common way to branch in CP is by assigning variables to values. We will present in this section how to define your branching strategies with existing variables and values selectors/iterators. But first...

\subsection{Override the default search stragtegy}\label{solver:overridethedefaultsearchstrategy}\hypertarget{solver:overridethedefaultsearchstrategy}{}


Basically, a search strategy is the composition of three objects: a branching strategy, a variable selector and a value selector. Some branching simply assign a selected value to a selected variable, like \hyperlink{assignvar:assignvarbranchstrat}{AssignVar}, others branching strategies embed the variable selector, like \hyperlink{domoverwdeg:domoverwdegbranchstrat}{DomOverWDegBranchingNew}, or more, like  \hyperlink{impact:impactbranchstrat}{ImpactBasedBranching}.

The default branchings are: 

\noindent\begin{tabular}{p{.4\linewidth}p{.6\linewidth}}
\hline
Variable &  Default strategy \\
\hline
Integer & \hyperlink{domoverwdeg:domoverwdegbranchstrat}{DomOverWDegBranchingNew} +\hyperlink{increasingdomain:increasingdomainvaliterator}{IncreasingDomain}\\
Set &   \hyperlink{assignsetvar:assignsetvarbranchstrat}{AssignSetVar} + \hyperlink{mindomset:mindomsetvarselector}{MinDomainSet} + \hyperlink{minenv:minenvvalselector}{MinEnv} \\
 Real &  \hyperlink{assigninterval:assignintervalbranchstrat}{AssignInterval} + \hyperlink{cyclicrealvarselector:cyclicrealvarselectorvarselector}{CyclicRealVarSelector}+ \hyperlink{realincreasingdomain:realincreasingdomainvaliterator}{RealIncreasingDomain} \\
\hline\\
\end{tabular}

There are two ways to custom a search strategy: use the ones define in the factory  \texttt{BranchingFactory} or compose it. 
A branching strategy can be set or add to the previous one using the following API (must be done before calling the method \mylst{solve()}):

  \mylst{solver.addGoal(AbstractIntBranchingStrategy branching)} 

\noindent to clear the list of goals, use:

  \mylst{solver.clearGoals()} 

You might want to apply different heuristics to different set of variables of the problem. In that case, the search is viewed as a sequence of branching objects (or goals). Up to now, we only had one branching or one goal including all the variables of the problem but several goals can be used.

Adding a new goal is made through the solver with the \mylst{solver.addGoal(AbstractIntBranchingStrategy branching)} method. 

The following example add three branching objects on integer variables \emph{vars1}, \emph{vars2} and set variables \emph{svars} to solver \emph{s}. The first two branchings are both \texttt{AssignVar} but use two different variable/values selection strategies:
\begin{lstlisting}
  s.addGoal(new AssignVar(new MinDomain(s,s.getVar(vars1)), new IncreasingDomain()));
  s.addGoal(new AssignVar(new DomOverDeg(s,s.getVar(vars2)),new DecreasingDomain());
  s.addGoal(new AssignSetVar(new MinDomSet(s,s.getVar(svars)), new MinEnv(s)));
  s.solve();
\end{lstlisting}

\begin{note}
Strategies are made of \texttt{Solver} variables (not \texttt{Model} variables).
\end{note}

\subsubsection{Branching strategy.}\label{solver:branchstrat}\hypertarget{solver:branchstrat}{}
It defines the way to take a decision in a tree search node.
  
\noindent The \textbf{branching strategies} currently available in Choco are the following: 
\begin{notedef}\tt
\hyperlink{assigninterval:assignintervalbranchstrat}{AssignInterval}, \hyperlink{assignorforbidintvarval:assignorforbidintvarvalbranchstrat}{AssignOrForbidIntVarVal}, \hyperlink{assignorforbidintvarvalpair:assignorforbidintvarvalpairbranchstrat}{AssignOrForbidIntVarValPair}, \hyperlink{assignsetvar:assignsetvarbranchstrat}{AssignSetVar}, \hyperlink{assignvar:assignvarbranchstrat}{AssignVar}, \hyperlink{domoverwdeg:domoverwdegbranchstrat}{DomOverWDegBranchingNew}, \hyperlink{domoverwdegbin:domoverwdegbinbranchstrat}{DomOverWDegBinBranchingNew}, \hyperlink{impact:impactbranchstrat}{ImpactBasedBranching}, \hyperlink{packdynremovals:packdynremovalsbranchstrat}{PackDynRemovals}, \hyperlink{settimes:settimesbranchstrat}{SetTimes}, \hyperlink{taskdomoverwdeg:taskdomoverwdegbranchstrat}{TaskOverWDegBinBranching}.
\end{notedef}    


\subsubsection{Variable selector.}\label{solver:variableselector}\hypertarget{solver:variableselector}{}
It defines the way to choose the non instantiated variable on which the next decision will be made.

\noindent The \textbf{integer variable selectors} currently available in Choco are the following: 
\begin{notedef}\tt
\hyperlink{compositeintvarselector:compositeintvarselectorvarselector}{CompositeIntVarSelector}, \hyperlink{lexintvarselector:lexintvarselectorvarselector}{LexIntVarSelector}, \hyperlink{maxdomain:maxdomainvarselector}{MaxDomain}, \hyperlink{maxregret:maxregretvarselector}{MaxRegret}, \hyperlink{maxvaldomain:maxvaldomainvarselector}{MaxValueDomain}, \hyperlink{mindomain:mindomainvarselector}{MinDomain}, \hyperlink{minvaldomain:minvaldomainvarselector}{MinValueDomain}, \hyperlink{mostconstrained:mostconstrainedvarselector}{MostConstrained},  \hyperlink{randomvarint:randomvarintvarselector}{RandomIntVarSelector},  \hyperlink{staticvarorder:staticvarordervarselector}{StaticVarOrder}
\end{notedef}

\noindent The \textbf{set variable selectors} currently available in Choco are the following: 
\begin{notedef}\tt
\hyperlink{maxdomset:maxdomsetvarselector}{MaxDomainSet}, \hyperlink{maxregretset:maxregretsetvarselector}{MaxRegretSet}, \hyperlink{maxvaldomainset:maxvaldomainsetvarselector}{MaxValueDomainSet}, \hyperlink{mindomset:mindomsetvarselector}{MinDomainSet}, \hyperlink{minvaldomainset:minvaldomainsetvarselector}{MinValueDomainSet}, \hyperlink{mostconstrainedset:mostconstrainedsetvarselector}{MostConstrainedSet},  \hyperlink{randomvarset:randomvarsetvarselector}{RandomSetVarSelector},  \hyperlink{staticsetvarorder:staticsetvarordervarselector}{StaticSetVarOrder}
\end{notedef}

\noindent The \textbf{real variable selector} currently available in Choco is the following: 
\begin{notedef}\tt
\hyperlink{cyclicrealvarselector:cyclicrealvarselectorvarselector}{CyclicRealVarSelector}
\end{notedef}

\subsubsection{Value iterator}\label{solver:valueiterator}\hypertarget{solver:valueiterator}{}
Once the variable has been choosen, the Solver has to compute its value. The first way to do it is to schedule the value once and give an iterator to the solver.

\noindent The \textbf{integer value iterator} currently available in Choco are the following: 
\begin{notedef}\tt
\hyperlink{decreasingdomain:decreasingdomainvaliterator}{DecreasingDomain}, \hyperlink{increasingdomain:increasingdomainvaliterator}{IncreasingDomain}
\end{notedef}

\noindent The \textbf{real value iterator} currently available in Choco is the following: 
\begin{notedef}\tt
\hyperlink{realincreasingdomain:realincreasingdomainvaliterator}{RealIncreasingDomain}
\end{notedef}




\subsubsection{Value selector}\label{solver:valueselector}\hypertarget{solver:valueselector}{}
The second way to do it is to compute the following value at each call.

\noindent The \textbf{integer value selector} currently available in Choco are the following: 
\begin{notedef}\tt
\hyperlink{bestfit:bestfitvalselector}{BestFit}, \hyperlink{costregularvalselector:costregularvalselectorvalselector}{CostRegularValSelector}, \hyperlink{fcostregularvalselector:fcostregularvalselectorvalselector}{FCostRegularValSelector}, \hyperlink{maxval:maxvalvalselector}{MaxVal}, \hyperlink{mcrvalselector:mcrvalselectorvalselector}{MCRValSelector}, \hyperlink{midval:midvalvalselector}{MidVal}, \hyperlink{minval:minvalvalselector}{MinVal}
\end{notedef}

\noindent The \textbf{set value selector} currently available in Choco is the following: 
\begin{notedef}\tt
\hyperlink{minenv:minenvvalselector}{MinEnv}, \hyperlink{randomsetvalselector:randomsetvalselectorvalselector}{RandomSetValSelector}
\end{notedef}

\subsection{Why is it important to define a search strategy ?}\label{solver:whyisitimportanttodefineasearchstrategy}\hypertarget{solver:whyisitimportanttodefineasearchstrategy}{}

In a partial instantiation, when a fix point has been reached, the Solver needs to take a decision to resume the search. The way decisions are chosen has a \textbf{real impact on the resolution step efficient}. 
\begin{note}
\emph{The search strategy should not be overlooked!!}
An adapted search strategy can reduce: the execution time, the number of node expanded, the number of backtrack done.
\end{note}
Let see that small example:
\begin{lstlisting}
	Model m = new CPModel();
        int n = 1000;
        IntegerVariable var = Choco.makeIntVar("var", 0, 2);
        IntegerVariable[] bi = Choco.makeBooleanVarArray("b", n);
        m.addConstraint(Choco.eq(var, Choco.sum(bi)));

        Solver badStrat = new CPSolver();
        badStrat.read(m);
        badStrat.addGoal(
                new AssignVar(
                        new MinDomain(badStrat), 
                        new IncreasingDomain()
                ));
        badStrat.solve();
        badStrat.printRuntimeStatistics();

        Solver goodStrat = new CPSolver();
        goodStrat.read(m);
        goodStrat.addGoal(
                new AssignVar(
                        new MinDomain(goodStrat, goodStrat.getVar(new IntegerVariable[]{var})), 
                        new DecreasingDomain()
                ));
        goodStrat.solve();
        goodStrat.printRuntimeStatistics();
\end{lstlisting}

This model ensures that $var = b_{0} + b_{1} + \ldots + b_{1000}$ where \emph{var} has a small domain and $b_{i}$ is a binary variable. The propagation has no effect on any domain and a fix point is reached at the beginning of the search. So, a decision has to be done choosing a variable and its value. If the variable selector is set to \texttt{MinDomain} (see below), the solver will iterate over the variables, starting by the 1000 binary variables and ending with \emph{var}, and 1001 nodes will be created.

\subsection{Restarts}\label{solver:restarts}\hypertarget{solver:restarts}{}

You can set geometric restarts by using the following API available on the solver:
\begin{lstlisting}
setGeometricRestart(int base, double grow);
setGeometricRestart(int base, double grow, int restartLimit);
\end{lstlisting}
It performs a search with restarts regarding the number of backtrack. An initial allowed number of backtrack is given (parameter base) and once this limit is reached a restart is performed and the new limit imposed to the search is increased by multiplying the previous limit with the parameter grow. restartLimit parameter states the maximum number of restarts. Restart strategies makes really sense with strategies that make choices based on the past experience of the search : \texttt{DomOverWdeg} or Impact based search. It could also be used with a random heuristic
\begin{lstlisting}
	CPSolver s = new CPSolver();
	s.read(model);
	
	s.setGeometricRestart(14, 1.5d);
	s.setFirstSolution(true);
	s.generateSearchStrategy();
	s.attachGoal(new DomOverWDegBranching(s, new IncreasingDomain()));
	s.launch();
\end{lstlisting}

You can also set Luby restarts by using the following API available on the solver:
\begin{lstlisting}
setLubyRestart(int base);
setLubyRestart(int base, int grow);
setLubyRestart(int base, int grow, int restartLimit);
\end{lstlisting}
it performs a search with restarts regarding the number of backtracks. One way to describe this strategy is to say that all run lengths are power of two, and that each time a pair of runs of a given length has been completed, a run of twice that length is immediatly executed. The limit is equals to \emph{length*base}.
\begin{itemize}
	\item \textbf{example with growing factor of 2 : [1, 1, 2, 1, 1, 2, 4, 1, 1, 2, 1, 1, 2, 4, 8, 1,...]}
	\item \textbf{example with growing factor of 3 : [1, 1, 1, 3, 1, 1, 1, 3, 9,...]}
\end{itemize}

\begin{lstlisting}
	CPSolver s = new CPSolver();
	s.read(model);
	
	s.setLubyRestart(50, 2, 100);
	s.setFirstSolution(true);
	s.generateSearchStrategy();
	s.attachGoal(new DomOverWDegBranching(s, new IncreasingDomain()));
	s.launch();
\end{lstlisting}

\section{Limiting Search Space}\label{solver:limitingsearchspace}\hypertarget{solver:limitingsearchspace}{}
The Solver class provides some limits on the search strategy that you can fix or just monitor.
Limits may be imposed on the search algorithm to avoid spending too much time in the exploration. The limits are updated and checked each time a new node is created. It has to be specified before the resolution. 
After having created the solver, you can specify whether or not you want to fix a limit:

\begin{description}
\item[time limit] State a time limit on tree search. When the execution time is equal to the time limit, the search stops whatever a solution is found or not. You can define a time limit with the following API : \mylst{setTimeLimit(int timeLimit)} where unit is millisecond. Or just monitor (or not) the search time with the API : \mylst{monitorTimeLimit(boolean b)}. The default value is set to \texttt{true}. Finally, you can get the time limit, once the solve method has been called, with the API: \mylst{getTimeCount()} 
\item[node limit] State a node limit on tree search. When the number of nodes explored is equal to the node limit, the search stops whatever a solution is found or not. You can define a node limit with the following API: \mylst{setNodeLimit(int nodeLimit)} where unit is the number of nodes. Or just monitor (or not) the number of nodes explored with the API: \mylst{monitorNodeLimit(boolean b)}. The default value is set to \texttt{true}. Finally, you can get the node limit, once the solve method has been called, with the API: \mylst{getNodeCount()} 
\item[backtrack limit] State a backtrack limit on tree search. When the number of backtracks done is equal to the backtrack limit, the search stops whatever a solution is found or not. You can define a backtrack limit with the following API: \mylst{setBackTrackLimit(int backtrackLimit)} where unit is the number of backtracks. Or just monitor (or not) the number of backtrack done with the API: \mylst{monitorBackTrackLimit(boolean b)}. The default value is set to \texttt{false}. Finally, you can get the backtrack limit, once the solve method has been called, with the API: \mylst{getBackTrackCount()} 
\item[fail limit] State a fail limit on tree search. When the number of failure is equal to the fail limit, the search stops whatever a solution is found or not. You can define a fail limit with the following API : \mylst{setFailLimit(int failLimit)} where unit is the number of failure. Or just monitor (or not) the number of failure encountered with the API : \mylst{monitorFailLimit(boolean b)}. The default value is set to \texttt{false}. Finally, you can get the fail limit, once the solve method has been called, with the API : \mylst{getFailCount()} 
%\item[CPU time limit] State a CPU limit on tree search. When the CPU time (user + system) is equal to the CPU time limit, the search stops whatever a solution is found or not. You can define a CPU time limit with the following API: \mylst{setCpuTimeLimit(int timeLimit)} where unit is millisecond. Or just monitor (or not) the search time with the API: \mylst{monitorCpuTimeLimit(boolean b)}. The default value is set to \texttt{false}. Finally, you can get the CPU time limit, once the solve method has been called, with the API: \mylst{getCpuTimeCount()} 
\end{description}

\todo{add example}

\section{Solve a problem}\label{solver:solveaproblem}\hypertarget{solver:solveaproblem}{}
As Solver is the second element of a Choco program, the control of the search process without using predefined tools is made on the Solver.

\noindent\begin{tabular}{p{.4\linewidth}p{.6\linewidth}}
  \hline
  \texttt{Solver} API & description \\
  \hline
      \mylst{solve()} &  Compute the first solution of the Model, if the Model is feasible. \\
      \mylst{solve(boolean all)} &  If \emph{all} is set to true, computes all solutions of the Model, if the Model is feasible. \\
      \mylst{solveAll()} &  Computes all the solution of the Model, if the Model is feasible. \\
      \mylst{propagate()} &  Computes initial propagation of the Model, and reachs the first Fix Point. It reduces variables Domain through constraints linked and other variables domain. Can throw a \texttt{ContradictionException} if the Solver detects a contradiction in the Model. \\
      \mylst{maximize(Var obj, boolean restart)} &  Allows user to find a solution that maximizing the objective varible \emph{obj}. The optimization finds a first solution then finds a new solution that improves \emph{obj} and so on till no other solution can be found that improves \emph{obj}. Parameter \emph{restart} is a boolean indicating whether the Solver will restart the search after each solution found (if set to \texttt{true}) or if it will keep backtracking from the leaf of the last solution found. See \hyperlink{solver:optimization}{example}. \textbf{Beware}: the variable \emph{obj} expected must be a Solver variable and not a Model variable. \\
      \mylst{minimize(Var obj, boolean restart)} &  Allows user to find a solution that minimizing the objective varible \emph{obj}. The optimization finds a first solution then finds a new solution that improves \emph{obj} and so on till no other solution can be found that improves \emph{obj}. Parameter \emph{restart} is a boolean indicating whether the Solver will restart the search after each solution found (if set to \texttt{true}) or if it will keep backtracking from the leaf of the last solution found. See \hyperlink{solver:optimization}{example}. \textbf{Beware}: the variable \emph{obj} expected must be a Solver variable and not a Model variable. \\
      \mylst{nextSolution()} &  Allows the Solver to find the next solution, if one or more solution have already been find with \texttt{solve()} or \texttt{nextSolution()}. \\
      \mylst{isFeasible()} &  Indicates whether or not the Model has at least one solution. \\
      \hline\\
	\end{tabular}

\subsection{Solver settings}\label{solver:solversettings}\hypertarget{solver:solversettings}{}

\subsubsection{Logs}\label{solver:logs}\hypertarget{solver:logs}{}
A logging class is instrumented in order to produce trace statements throughout search: ChocoLogging. The verbosity level of the solver can be set, by the following static method
\begin{lstlisting}
	ChocoLogging.toVerbose();
	// And after solver.solve()
	ChocoLogging.flushLogs();
\end{lstlisting}

The code above ensure that messages are printed in order to describe the construction of the search tree.

Six verbosity levels are available:

\noindent\begin{tabular}{p{.4\linewidth}p{.6\linewidth}}
  \hline
  Level & prints... \\
  \hline
 \texttt{ChocoLogging.toSilent()} & display only severe messages from core loggers and warning messages otherwise\\
 \ \texttt{ChocoLogging.toQuiet()} & display only severe messages from core loggers and info messages otherwise\\
 \texttt{ChocoLogging.toDefault()} & display information about initial and final state of the search\\
 \texttt{ChocoLogging.toVerbose()} & display search information at regular node intervals\\
 \texttt{ChocoLogging.toSolution()} & display all solutions\\
 \texttt{ChocoLogging.toSearch()} & display the search tree\\
\hline\\
\end{tabular}

Note that in the case of a verbosity greater or equals to \texttt{toVerbose()}, the regular search information step is set to 1000, by default. You can change this value, using:
\begin{lstlisting}
  ChocoLogging.setEveryXNodes(20000);
\end{lstlisting}
 

Note that in the case of verbosity \texttt{toSearch()}, trace statements are printed up to a maximal depth in the search tree. The default value is set to 25, but you can change the value of this threshold, say to 10, with the following setter method:
\begin{lstlisting}
  ChocoLogging.setLoggingMaxDepth(10);
\end{lstlisting}

\subsection{Optimization}\label{solver:optimization}\hypertarget{solver:optimization}{}
\todo{to introduce}
\begin{lstlisting}
  Model m = new CPModel();
  IntegerVariable obj1 = makeEnumIntVar("obj1", 0, 7);
  IntegerVariable obj2 = makeEnumIntVar("obj1", 0, 5);
  IntegerVariable obj3 = makeEnumIntVar("obj1", 0, 3);
  IntegerVariable cost = makeBoundIntVar("cout", 0, 1000000);
  int capacity = 34;
  int[] volumes = new int[]{7, 5, 3};
  int[] energy = new int[]{6, 4, 2};
  // capacity constraint
  m.addConstraint(leq(scalar(volumes, new IntegerVariable[]{obj1, obj2, obj3}), capacity));
	
  // objective function
  m.addConstraint(eq(scalar(energy, new IntegerVariable[]{obj1, obj2, obj3}), cost));
  
  Solver s = new CPSolver();
  s.read(m);
  
  s.maximize(s.getVar(cost), false);
\end{lstlisting}
\label{doc:solver}\hypertarget{doc:solver}{}
%%\part{constraints}
\label{constraints}
\hypertarget{constraints}{}



\chapter{Constraints in alphabetical order}\label{constraints:constraintsinalphabeticalorder}\hypertarget{constraints:constraintsinalphabeticalorder}{}

Choco offers a large number of available constraints.
In this part, you will find a description of each defined constraint, with API, options and examples.
\begin{note}
	 \hyperlink{abs:absconstraint}{abs},
	 \hyperlink{alldifferent:alldifferentconstraint}{allDifferent},
	 \hyperlink{and:andconstraint}{and},
	 \hyperlink{atmostnvalue:atmostnvalueconstraint}{atMostNValue},
	 \hyperlink{boolchanneling:boolchannelingconstraint}{boolChanneling},
	 \hyperlink{cumulative:cumulativeconstraint}{cumulative},
	 \hyperlink{disjunctive:disjunctiveconstraint}{disjunctive},
	 \hyperlink{distanceeq:distanceeqconstraint}{distanceEQ},
	 \hyperlink{distancegt:distancegtconstraint}{distanceGT},
	 \hyperlink{distancelt:distanceltconstraint}{distanceLT},
	 \hyperlink{distanceneq:distanceneqconstraint}{distanceNEQ},
	 \hyperlink{eq:eqconstraint}{eq},
	 \hyperlink{eqcard:eqcardconstraint}{eqCard},
	 \hyperlink{equation:equationconstraint}{equation},
	 \hyperlink{false:falseconstraint}{FALSE},
	 \hyperlink{feaspairac:feaspairacconstraint}{feasPairAC},
	 \hyperlink{feastupleac:feastupleacconstraint}{feasTupleAC},
	 \hyperlink{feastuplefc:feastuplefcconstraint}{feasTupleFC},
	 \hyperlink{geost:geostconstraint}{geost},
	 \hyperlink{geq:geqconstraint}{geq},
	 \hyperlink{geqcard:geqcardconstraint}{geqCard},
	 \hyperlink{globalcardinality:globalcardinalityconstraint}{globalCardinality},
	 \hyperlink{gt:gtconstraint}{gt},
	 \hyperlink{ifonlyif:ifonlyifconstraint}{ifOnlyIf},
	 \hyperlink{ifthenelse:ifthenelseconstraint}{ifThenElse},
	 \hyperlink{implies:impliesconstraint}{implies},
	 \hyperlink{infeaspairac:infeaspairacconstraint}{infeasPairAC},
	 \hyperlink{infeastupleac:infeastupleacconstraint}{infeasTupleAC},
	 \hyperlink{infeastuplefc:infeastuplefcconstraint}{infeasTupleFC},
	 \hyperlink{intdiv:intdivconstraint}{intDiv},
	 \hyperlink{inversechanneling:inversechannelingconstraint}{inverseChanneling},
	 \hyperlink{isincluded:isincludedconstraint}{isIncluded},
	 \hyperlink{isnotincluded:isnotincludedconstraint}{isNotIncluded},
	 \hyperlink{leq:leqconstraint}{leq},
	 \hyperlink{leqcard:leqcardconstraint}{leqCard},
	 \hyperlink{lex:lexconstraint}{lex},
	 \hyperlink{lexchain:lexchainconstraint}{lexChain},
	 \hyperlink{lexchaineq:lexchaineqconstraint}{lexChainEq},
	 \hyperlink{lexeq:lexeqconstraint}{lexeq},
	 \hyperlink{leximin:leximinconstraint}{leximin},
	 \hyperlink{lt:ltconstraint}{lt},
	 \hyperlink{max:maxconstraint}{max},
	 \hyperlink{member:memberconstraint}{member},
	 \hyperlink{min:minconstraint}{min},
	 \hyperlink{mod:modconstraint}{mod},
	 \hyperlink{multicostregular:multicostregularconstraint}{multiCostRegular},
	 \hyperlink{neq:neqconstraint}{neq},
	 \hyperlink{neqcard:neqcardconstraint}{neqCard},
	 \hyperlink{not:notconstraint}{not},
	 \hyperlink{notmember:notmemberconstraint}{notMember},
	 \hyperlink{nth:nthconstraint}{nth},
	 \hyperlink{occurrencemax:occurrencemaxconstraint}{occurrenceMax},
	 \hyperlink{occurrencemin:occurrenceminconstraint}{occurrenceMin},
	 \hyperlink{occurrence:occurrenceconstraint}{occurrence},
	 \hyperlink{oppositesign:oppositesignconstraint}{oppositeSign},
	 \hyperlink{or:orconstraint}{or},
	 \hyperlink{pack:packconstraint}{pack},
	 \hyperlink{precedencereified:precedencereifiedconstraint}{precedenceReified},
	 \hyperlink{preceding:precedingconstraint}{preceding},
	 \hyperlink{regular:regularconstraint}{regular},
	 \hyperlink{reifiedintconstraint:reifiedintconstraintconstraint}{reifiedIntConstraint},
	 \hyperlink{relationpairac:relationpairacconstraint}{relationPairAC},
	 \hyperlink{relationtupleac:relationtupleacconstraint}{relationTupleAC},
	 \hyperlink{relationtuplefc:relationtuplefcconstraint}{relationTupleFC},
	 \hyperlink{samesign:samesignconstraint}{sameSign},
	 \hyperlink{setdisjoint:setdisjointconstraint}{setDisjoint},
	 \hyperlink{setinter:setinterconstraint}{setInter},
	 \hyperlink{setunion:setunionconstraint}{setUnion},
	 \hyperlink{sorting:sortingconstraint}{sorting},
	 \hyperlink{stretchpath:stretchpathconstraint}{stretchPath},
	 \hyperlink{times:timesconstraint}{times},
	 \hyperlink{tree:treeconstraint}{tree},
	 \hyperlink{true:trueconstraint}{TRUE}.
   \end{note}


\chapter{Operators in alphabetical order}\label{constraints:operatorsinalphabeticalorder}\hypertarget{constraints:operatorsinalphabeticalorder}{}

  
\begin{note}
	 \hyperlink{abs:absoperator}{abs},
	 \hyperlink{cos:cosoperator}{cos},
	 \hyperlink{div:divoperator}{div},
	 \hyperlink{false:falseoperator}{FALSE},
	 \hyperlink{ifthenelse:ifthenelseoperator}{ifThenElse},
	 \hyperlink{max:maxoperator}{max},
	 \hyperlink{min:minoperator}{min},
	 \hyperlink{minus:minusoperator}{minus},
	 \hyperlink{mod:modoperator}{mod},
	 \hyperlink{mult:multoperator}{mult},
	 \hyperlink{neg:negoperator}{neg},
	 \hyperlink{plus:plusoperator}{plus},
	 \hyperlink{power:poweroperator}{power},
	 \hyperlink{scalar:scalaroperator}{scalar},
	 \hyperlink{sin:sinoperator}{sin},
	 \hyperlink{sum:sumoperator}{sum},
	 \hyperlink{true:trueoperator}{TRUE}. 
\end{note}
\label{doc:constraints}\hypertarget{doc:constraints}{}
%\part{advanced}
\label{advanced}
\hypertarget{advanced}{}


\chapter{Advanced uses of Choco}\label{advanced:advancedusesofchoco}\hypertarget{advanced:advancedusesofchoco}{}

\section{Environment}\label{advanced:environment}\hypertarget{advanced:environment}{}

Environment is a central object of the backtracking system. It defines the notion of \textit{world}. A world contains values of storable objects or operations that permit to \textit{backtrack} to its state. The environment \textit{pushes} and \textit{pops} worlds.

There are \textit{primitive} data types (\texttt{IstateBitSet, IStateBool, IStateDouble, IStateInt, IStateLong}) and \textit{objects} data types (\texttt{IStateBinarytree, IStateIntInterval, IStateIntProcedure, IStateIntVector, IStateObject, IStateVector}).

There are two different environments: \textit{EnvironmentTrailing} and \textit{EnvironmentCopying}.

\subsection{Copying}\label{advanced:copying}\hypertarget{advanced:copying}{}
In that environment, each data type is defined by a value (primitive or object) and a timestamp. Every time a world is pushed, each value is copied in an array (one array per data type), with finite indice. When a world is popped, every value is restored. 

\subsection{Trailing}\label{advanced:trailing}\hypertarget{advanced:trailing}{}
In that environment, data types are defined by its value. Every operation applied to a data type is pushed in a \textit{trailer}. When a world is pushed, the indice of the last operation is stored. When a world is popped, these operations are popped and \textit{unapplied} until reaching the last operation of the previous world.\\\textit{Default one in CPSolver}

\section{Define your own search strategy}\label{advanced:defineyourownsearchstrategy}\hypertarget{advanced:defineyourownsearchstrategy}{}
%A key ingredient of any constraint approach is a clever branching strategy. The construction of the search tree is done according to a series of Branching objects (that plays the role of achieving intermediate goals in logic programming). The user may specify the sequence of branching objects to be used to build the search tree. 
Section~\hyperlink{solver:searchstrategy}{Search strategy} presented the default branching strategies available in Choco and showed how to post them or to compose them as goals.
In this section, we will start with a very simple and common way to branch by choosing values for variables and specially how to define its own variable/value selection strategy. We will then focus on more complex branching such as dichotomic or n-ary choices. Finally we will show how to control the search space in more details with well known strategy such as LDS (Limited discrepancy search).

For integer variables, the variable and value selection strategy objects are based on the following interfaces:
\begin{itemize}
	\item \texttt{AbstractIntBranchingStrategy}: abstract class for the branching strategy,
	\item \texttt{VarSelector<V>} : Interface for the variable selection (V extends Var),
	\item \texttt{ValIterator<V>} : Interface to describes an iteration scheme on the domain of a variable,
	\item \texttt{ValSelector<V>} : Interface for a value selection.
\end{itemize}

Concrete examples of these interfaces are respectively,  \hyperlink{assignvar:assignvarbranchstrat}{AssignVar}, \hyperlink{mindomain:mindomainvarselector}{MinDomain}, \hyperlink{increasingdomain:increasingdomainvaliterator}{IncreasingDomain}, \hyperlink{maxval:maxvalvalselector}{MaxVal}.

\subsection{How to define your own Branching object}\label{advanced:beyondvariable/valueselection,howtodefineyourownbranchingobject}\hypertarget{advanced:beyondvariable/valueselection,howtodefineyourownbranchingobject}{}

Beyond Variable/value selection...

\subsection{Define your own variable selection}\label{advanced:defineyourownvariableselection}\hypertarget{advanced:defineyourownvariableselection}{}
You may extend this small library of branching schemes and heuristics by defining your own concrete classes of \texttt{AbstractIntVarSelector}. We give here an example of an \texttt{IntVarSelector} with the implementation of a static variable ordering :
\begin{lstlisting}
  public class StaticVarOrder extends AbstractIntVarSelector {

      // the sequence of variables that need be instantiated
	  protected IntDomainVar[] vars;
	
	  public StaticVarOrder(IntDomainVar[] vars) {
          this.vars = vars;
	  }
	
	  public IntDomainVar selectIntVar() {
          for (int i = 0; i < vars.length; i++)
              if (!vars[i].isInstantiated())
                  return vars[i];
          return null;
	  }
  }
\end{lstlisting}

Notice on this example that you only need to implement method \texttt{selectIntVar()} which belongs to the contract of \texttt{IntVarSelector}. This method should return a non instantiated variable or \texttt{null}. Once the branching is finished, the next branching (if one exists) is taken by the search algorithm to continue the search, otherwise, the search stops as all variable are instantiated. To avoid the loop over the variables of the branching, a backtrackable integer (\texttt{StoredInt}) could be used to remember the last instantiated variable and to directly select the next one in the table. Notice that backtrackable structures could be used in any of the code presented in this chapter to speedup the computation of dynamic choices.

You can add your variable selector as a part of a search strategy, using \mylst{solver.addGoal()}.

\subsection{Define your own value selection}\label{advanced:defineyourownvalueselection}\hypertarget{advanced:defineyourownvalueselection}{}
You may also define your own concrete classes of \texttt{ValIterator} or \texttt{ValSelector}. 

\subsubsection{Value selector}\label{advanced:valueselector}\hypertarget{advanced:valueselector}{}

\insertGraphique{.5\linewidth}{media/valselector.pdf}{ValSelector interface and implementation}

We give here an example of an \texttt{IntValSelector} with the implementation of a minimum value selecting:
\begin{lstlisting}
  public class MinVal extends AbstractSearchHeuristic implements ValSelector {
	 /**
      * selecting the lowest value in the domain
      * @param x the variable under consideration
      * @return what seems the most interesting value for branching
      */
	  public int getBestVal(IntDomainVar x) {
          return x.getInf();
	  }
  }
\end{lstlisting}
Only \texttt{getBestVal()} method must be implemented, returning the best value \emph{in the domain} according to the heuristic.

You can add your value selector as a part of a search strategy, using \mylst{solver.addGoal()}.

\begin{note}
Using a value selector with bounded domain variable is strongly inadvised, except if it pick up bounds value. If the value selector pick up a value that is not a bound, when it goes up in the tree search, that value could be not removed and picked twice (or more)!
\end{note} 

\subsubsection{Values iterator}\label{advanced:valuesiterator}\hypertarget{advanced:valuesiterator}{}
We give here an example of an \texttt{ValIterator} with the implementation of an increasing domain iterator:
\begin{lstlisting}
  public final class IncreasingDomain implements ValIterator {
	  /**
	   * testing whether more branches can be considered after branch i, 
       * on the alternative associated to variable x
	   * @param x the variable under scrutiny
	   * @param i the index of the last branch explored
	   * @return true if more branches can be expanded after branch i
	   */
       public boolean hasNextVal(Var x, int i) {
           return (i < ((IntDomainVar) x).getSup());
       }
	
	  /**
	   * accessing the index of the first branch for variable x
	   * @param x the variable under scrutiny
	   * @return the index of the first branch: first value to be assigned to x
	   */
       public int getFirstVal(Var x) {
           return ((IntDomainVar) x).getInf();
       }
	
	  /**
	   * generates the index of the next branch after branch i, 
       * on the alternative associated to variable x
	   * @param x the variable under scrutiny
	   * @param i the index of the last branch explored
	   * @return the index of the next branch to be expanded after branch i
	   */
       public int getNextVal(Var x, int i) {
           return ((IntDomainVar) x).getNextDomainValue(i);
       }
   }
\end{lstlisting}
%Works as an basic \texttt{Iterator} object, implementing the three main methods \texttt{hasNextVal()}, \texttt{getFirstVal()} and \texttt{getNextVal()}.

You can add your value iterator as a part of a search strategy, using \mylst{solver.addGoal()}.

\todo{under development} See \href{http://choco-solver.net/index.phptitle=userguide:beyondvariable.2fvalueselection.2chowtodefineyourownbranchingobject}{old version}

\section{Define your own limit search space}\label{advanced:defineyourownlimitsearchspace}\hypertarget{advanced:defineyourownlimitsearchspace}{}

To define your own limits/statistics (notice that a limit object can be used only to get statistics about the search), you can create a limit object by extending the \texttt{AbstractGlobalSearchLimit} class or implementing directly the interface \texttt{IGlobalSearchLimit}. Limits are managed at each node of the tree search and are updated each time a node is open or closed. Notice that limits are therefore time consuming. Implementing its own limit need only to specify to the following interface :

\begin{lstlisting}
	/**
	 * The interface of objects limiting the global search exploration
	 */
	public interface GlobalSearchLimit {

	  /**
	   * resets the limit (the counter run from now on)
	   * @param first true for the very first initialization, false for subsequent ones
	   */
	  public void reset(boolean first);
	
	  /**
	   * notify the limit object whenever a new node is created in the search tree
	   * @param solver the controller of the search exploration, managing the limit
	   * @return true if the limit accepts the creation of the new node, false otherwise
	   */
	  public boolean newNode(AbstractGlobalSearchSolver solver);
	
	  /**
	   * notify the limit object whenever the search closes a node in the search tree
	   * @param solver the controller of the search exploration, managing the limit
	   * @return true if the limit accepts the death of the new node, false otherwise
	   */
	  public boolean endNode(AbstractGlobalSearchSolver solver);
	}
\end{lstlisting}

Look at the following example to see a concrete implementation of the previous interface. We define here a limit on the depth of the search (which is not found by default in choco). The \texttt{getWorldIndex()} is used to get the current world, i.e the current depth of the search or the number of choices which have been done from baseWorld. 

\begin{lstlisting}
	public class DepthLimit extends AbstractGlobalSearchLimit {
	
	  public DepthLimit(AbstractGlobalSearchSolver theSolver, int theLimit) {
	    super(theSolver,theLimit);
	    unit = "deep";
	  }
	
	  public boolean newNode(AbstractGlobalSearchSolver solver) {
	    nb = Math.max(nb, this.getProblem().getWorldIndex() –
	    this.getProblem().getSolver().getSearchSolver().baseWorld);
	    return (nb < nbMax);
	  }
	
	  public boolean endNode(AbstractGlobalSearchSolver solver) {
	    return true;
	  }
	
	  public void reset(boolean first) {
	   if (first) {
	    nbTot = 0;
	   } else {
	    nbTot = Math.max(nbTot, nb);
	   }
	   nb = 0;
	  }
\end{lstlisting}

Once you have implemented your own limit, you need to tell the search solver to take it into account. Instead of using a call to the \texttt{solve()} method, you have to create the search solver by yourself and add the limit to its limits list such as in the following code :
\begin{lstlisting}
	Solver s = new CPSolver();
	s.read(model);
	s.setFirstSolution(true);
	s.generateSearchStrategy();
	s.getSearchStrategy().limits.add(new DepthLimit(s.getSearchStrategy(),10));
	s.launch();
\end{lstlisting}


\subsection{How does a search loop work ?}\label{advanced:howdoesasearchloopwork}\hypertarget{advanced:howdoesasearchloopwork}{}
The seach loop is created when a \texttt{solve()} method is called. It goes down and up in the branches in order to cover the tree search. 

%\subsubsection{Search loop}\label{advanced:searchloop}\hypertarget{advanced:searchloop}{}
\begin{lstlisting}[title={Algorithm of the search loop in Choco}]
  next_move = new node
  WHILE no solution AND in search limit
        IF next_move is new node
	    THEN
	        create a new node : variable/value selection ;
	        IF node exists 
	        THEN
	            next_move <-- go down branch ;            
	        ELSE 
	            next_move <-- go up branch ;
	            solution is found ;
	
	    ELSE IF next_move is go down branch
	        propagate ;
	        IF no contradiction 
	        THEN
	            next_move <-- new node ;                        
	        ELSE
	            next_move <-- go up branch ;
	        
	    ELSE IF next_move is go up branch 
	        find next branch ;
	        propagate ;
	        IF has next branch AND no contradiction
	        THEN
	            next_move <-- go down branch ;                            
	        ELSE
	            next_move <-- go up branch ;
	
	    END IF
	
  END WHILE
\end{lstlisting}
%\subsubsection{Search loop with recomputation}\label{advanced:searchloopwithrecomputation}\hypertarget{advanced:searchloopwithrecomputation}{}

\section{Define your own constraint}\label{advanced:defineyourownconstraint}\hypertarget{advanced:defineyourownconstraint}{}

This section describes how to add you own constraint, with specific propagation algorithms. Note that this section is only useful in case you want to express a constraint for which the basic propagation algorithms (using tables of tuples, or boolean predicates) are not efficient enough to propagate the constraint.

The general process consists in defining a new constraint class and implementing the various propagation methods. We recommend the user to follow the examples of existing constraint classes (for instance, such as \texttt{GreaterOrEqualXYC} for a binary inequality) 

\subsection{The constraint hierarchy}\label{advanced:theconstrainthierarchy}\hypertarget{advanced:theconstrainthierarchy}{}

Each new constraint must be represented by an object implementing the \texttt{\bf SConstraint} interface (\texttt{S} for solver constraint). To help the user defining new constraint classes, several abstract classes defining \texttt{SConstraint} have been implemented. These abstract classes provide the user with a management of the constraint network and the propagation engineering. They should be used as much as possible.

For constraints on integer variables, the easiest way to implement your own constraint is to inherit from one of the following classes, depending of the number of solver integer variables (\texttt{IntDomainVar}) involved:

\centerline{\begin{tabular}{ll}
      \hline
  Default class to implement &  number of solver integer variables \\
  \hline
  \mylst{AbstractUnIntSConstraint} &  \textbf{one} variable \\
  \mylst{AbstractBinIntSConstraint} &  \textbf{two} variables \\
  \mylst{AbstractTernIntSConstraint} &  \textbf{three} variables \\
  \mylst{AbstractLargeIntSConstraint} &  any number of variables. \\
  \hline\\
\end{tabular}}

\noindent Constraints over integers must implement the following methods (grouped in the \texttt{IntSConstraint} interface):

\noindent\begin{tabular}{lp{.6\linewidth}}
  \hline
  Method to implement &  description \\
  \hline
  \mylst{pretty()} &Returns a pretty print of the constraint \\
  \mylst{propagate()} &The main propagation method (propagation from scratch). Propagating the constraint until local consistency is reached. \\
  \mylst{awake()} &Propagating the constraint for the very first time until local consistency is reached. The awake is meant to initialize the data structures contrary to the propagate. Specially, it is important to avoid initializing the data structures in the constructor. \\
  \mylst{awakeOnInst(int x)} &Default propagation on instantiation: full constraint re-propagation. \\
  \mylst{awakeOnBounds(int x)} &Default propagation on improved bounds: propagation on domain revision. \\
  \mylst{awakeOnRemovals(int x, IntIterator v)} &Default propagation on mutliple values removal: propagation on domain revision. The iterator allow to iterate over the values that have been removed. \\
&\\
\hline
\multicolumn{2}{l}{Methods \texttt{awakeOnBounds} and \texttt{awakeOnRemovals} can be replaced by more fine grained methods:}\\
\hline
%Alternative Method &  description \\
%  \hline
  \mylst{awakeOnInf(int x)} &Default propagation on improved lower bound: propagation on domain revision. \\
  \mylst{awakeOnSup(int x)} &Default propagation on improved upper bound: propagation on domain revision. \\
  \mylst{awakeOnRem(int x, int v)} &Default propagation on one value removal: propagation on domain revision.  \\
&\\
  \hline
\multicolumn{2}{l}{To use the constraint in expressions or reification, the following minimum API is mandatory:}\\
  \hline
  \mylst{isSatisfied(int[] x)} &Tests if the constraint is satisfied when the variables are instantiated. \\
	\mylst{isEntailed()} &Checks if the constraint must be checked or must fail. It returns true if the constraint is known to be satisfied whatever happend on the variable from now on, false if it is violated. \\
	\mylst{opposite()} &It returns an AbstractSConstraint that is the opposite of the current constraint. \\
    \hline\\
	\end{tabular}

In the same way, a \textbf{set constraint} can inherit from \texttt{AbstractUnSetSConstraint}, \texttt{AbstractBinSetSConstraint}, \texttt{AbstractTernSetSConstraint} or \texttt{AbstractLargeSetSConstraint}.

A \textbf{real constraint} can inherit from \texttt{AbstractUnRealSConstraint}, \texttt{AbstractBinRealSConstraint} or \texttt{AbstractLargeRealSConstraint}.

A mixed constraint between \textbf{set and integer variables} can inherit from \texttt{AbstractBinSetIntSConstraint} or \texttt{AbstractLargeSetIntSConstraint}.

\begin{note}
A simple way to implement its own constraint is to:
\begin{itemize}
	\item create an empty constraint with only \texttt{propagate()} method implemented and every \texttt{awakeOnXxx()} ones set to \texttt{this.constAwake(false);}
	\item when the propagation filter is sure, separate it into the \texttt{awakeOnXxx()} methods in order to have finer granularity
	\item finally, if necessary, use backtrackables objects to improve the efficient of your constraint
\end{itemize}

\end{note}

\subsubsection{How do I add my constraint to the Model ?}\label{advanced:howdoiaddmyconstrainttothemodel}\hypertarget{advanced:howdoiaddmyconstrainttothemodel}{}

Adding your constraint to the model requires you to definite a specific constraint manager (that can be a inner class of your Constraint).
This manager need to implement:
\begin{lstlisting}
makeConstraint(Solver s, Variable[] vars, Object params, HashSet<String> options)
\end{lstlisting}
This method allows the Solver to create an instance of your constraint, with your parameters and Solver objects.

\begin{note}
If you create your constraint manager as an inner class, you must declare this class as \textbf{public and static}.
If you don't, the solver can't instantiate your manager.
\end{note}

Once this manager has been implemented, you simply add your constraint to the model using the \texttt{addConstraint()} API with a \texttt{ComponentConstraint} object:
\begin{lstlisting}
  model.addConstraint( new ComponentConstraint(MyConstraintManager.class, params, vars) );
  // OR
  model.addConstraint( new ComponentConstraint("package.of.MyConstraint", params, vars) );
\end{lstlisting}
Where \emph{params} is whatever you want (\texttt{Object[], int, String},...) and \emph{vars} is an array of Model Variables (or more specific) objects.

\subsection{Example: implement and add the \texttt{IsOdd} constraint}
One creates the constraint by implementing the \texttt{AbstractUnIntSConstraint} (one integer variable) class:
\lstinputlisting{java/isodd.j2t}

To add the constraint to the model, one creates the following class (or inner class):
\lstinputlisting{java/isoddmanager.j2t}
It calls the constructor of the constraint, with every \emph{vars}, \emph{params} and \emph{options} needed.

Then, the constraint can be added to a model as follows:
\begin{lstlisting}
	// Creation of the model
	Model m = new CPModel();
	
	// Declaration of the variable
	IntegerVariable aVar = Choco.makeIntVar("a_variable", 0, 10);
	
	// Adding the constraint to the model, 1st solution:
	m.addConstraint(new ComponentConstraint(IsOddManager.class, null, new IntegerVariable[]{aVar}));
	// OR 2nd solution:
	m.addConstraint(new ComponentConstraint("myPackage.Constraint.IsOddManager", null, new IntegerVariable[]{aVar}));
	
	Solver s = new CPSolver();
	s.read(m);
	s.solve();
\end{lstlisting}
And that's it!!

\subsection{Example of an empty constraint}\label{advanced:anexempleofemptyconstraint}\hypertarget{advanced:anexempleofemptyconstraint}{}

See the complete code: \href{media/zip/constraintpattern.zip}{ConstraintPattern.zip}

\begin{lstlisting}
  public class ConstraintPattern extends AbstractLargeIntSConstraint {
      
      public ConstraintPattern(IntDomainVar[] vars) {
          super(vars);
      }
	
      /**
      * pretty print. The String is not constant and may depend on the context.
      * @return a readable string representation of the object
      */
      public String pretty() {
          return null;
      }
	
      /**
      * check whether the tuple satisfies the constraint
      * @param tuple values
      * @return true if satisfied
      */
      public boolean isSatisfied(int[] tuple) {
          return false;
      }

      /**
      * propagate until local consistency is reached
      */
      public void propagate() throws ContradictionException {
          // elementary method to implement
      }
	    
      /**
      * propagate for the very first time until local consistency is reached.
      */
      public void awake() throws ContradictionException {
          constAwake(false);        // change if necessary
      }
	
	
      /**
      * default propagation on instantiation: full constraint re-propagation
      * @param var index of the variable to reduce
      */
      public void awakeOnInst(int var) throws ContradictionException {
          constAwake(false);        // change if necessary
      }
	
      /**
      * default propagation on improved lower bound: propagation on domain revision
      * @param var index of the variable to reduce
      */
      public void awakeOnInf(int var) throws ContradictionException {
          constAwake(false);        // change if necessary
      }
	
	
      /**
      * default propagation on improved upper bound: propagation on domain revision
      * @param var index of the variable to reduce
      */
      public void awakeOnSup(int var) throws ContradictionException {
          constAwake(false);        // change if necessary
      }
	
      /**
      * default propagation on improve bounds: propagation on domain revision
      * @param var index of the variable to reduce
      */
      public void awakeOnBounds(int var) throws ContradictionException {
          constAwake(false);        // change if necessary
      }
	
      /**
      * default propagation on one value removal: propagation on domain revision
      * @param var index of the variable to reduce
      * @param val the removed value
      */
      public void awakeOnRem(int var, int val) throws ContradictionException {
          constAwake(false);        // change if necessary
      }
	
      /**
      * default propagation on one value removal: propagation on domain revision
      * @param var index of the variable to reduce
      * @param delta iterator over remove values
      */
      public void awakeOnRemovals(int var, IntIterator delta) throws ContradictionException {
          constAwake(false);        // change if necessary
      }
  }
\end{lstlisting}

The first step to create a constraint in Choco is to implement all \texttt{awakeOn...} methods with \texttt{constAwake(false)} and to put your propagation algorithm in the \texttt{propagate()} method. 

A constraint can choose not to react to fine grained events such as the removal of a value of a given variable but instead delay its propagation at the end of the fix point reached by ``fine grained events'' and fast constraints that deal with them incrementally (that's the purpose of the constraints events queue). 

To do that, you can use \texttt{constAwake(false)} that tells the solver that you want this constraint to be called only once the variables events queue is empty. This is done so that heavy propagators can delay their action after the fast one to avoid doing a heavy processing at each single little modification of domains.

\section{Define your own operator}\label{advanced:defineyourownoperator}\hypertarget{advanced:defineyourownoperator}{}
\todo{to complete}

\section{Define your own variable}\label{advanced:defineyourownvariable}\hypertarget{advanced:defineyourownvariable}{}
\todo{to complete}

\section{Backtrackable structures}\label{advanced:backtrackablestructures}\hypertarget{advanced:backtrackablestructures}{}
\todo{to complete}

\section{Logging System}\label{advanced:loggingsystem}\hypertarget{advanced:loggingsystem}{}

Choco logging system is based on the \texttt{java.util.logging} package and located in the package \texttt{common.logging}.
Most Choco abstract classes or interfaces propose a static field \texttt{LOGGER}.
The following figures present the architecture of the logging system with the default verbosity.

\insertGraphique{.9\linewidth}{media/logger-default.png}{Logger Tree with the default verbosity}

The shape of the node depicts the type of logger:
\begin{itemize}
	\item The \emph{house} loggers represent private loggers. Do not use directly these loggers because their level are low and all messages would always be displayed.
	\item The \emph{octagon} loggers represent critical loggers. These loggers are provided in the variables, constraints and search classes and could have a huge impact on the global performances.
	\item The \emph{box} loggers are provided for dev and users.
\end{itemize}
The color of the node gives its logging level with DEFAULT verbosity:
\texttt{Level.FINEST} (\textcolor{yellow}{gold}),
\texttt{Level.INFO} (\textcolor{orange}{orange}),
\texttt{Level.WARNING} (\textcolor{red}{red}).

\subsubsection{Verbosity and messages.}\label{advanced:verbosityandmessages}\hypertarget{advanced:verbosityandmessages}{}
The following table summarizes the verbosities available in choco: 

\begin{itemize}
	\item \textbf{OFF -- level 0:} Disable logging.
	\item \textbf{SILENT -- level 1:} Display only severe messages.
	\item \textbf{DEFAULT -- level 2:} Display informations on final search state.
		\begin{itemize}
			\item ON START
				\lstset{language={sh},columns=fixed}
\begin{lstlisting}
 ** CHOCO : Constraint Programming Solver
 ** CHOCO v2.1.1 (April, 2009), Copyleft (c) 1999-2010
 \end{lstlisting}
			\item ON COMPLETE SEARCH:
				\begin{lstlisting}
- Search completed -
 [Maximize		: {0},]
 [Minimize		: {1},]
  Solutions		: {2},
  Times (ms)	: {3},
  Nodes			: {4},
  Backtracks	: {5},
  Restarts		: {6}.
  \end{lstlisting}
	brackets [\textit{line}] indicate \textit{line} is optionnal,\\
 	\texttt{Maximize} --resp. \texttt{Minimize}-- indicates the best known value before exiting of the objective value in \textit{maximize()} -- --resp. \textit{minimize()}-- strategy.

			\item ON COMPLETE SEARCH WITHOUT SOLUTIONS :
				\begin{lstlisting}
- Search completed - No solutions
 [Maximize		: {0},]
 [Minimize		: {1},]
  Solutions		: {2},
  Times (ms)	: {3},
  Nodes			: {4},
  Backtracks	: {5},
  Restarts		: {6}.
\end{lstlisting}
	brackets [\textit{line}] indicate \textit{line} is optionnal,\\
 	\texttt{Maximize} --resp. \texttt{Minimize}-- indicates the best known value before exiting of the objective value in \textit{maximize()} -- --resp. \textit{minimize()}-- strategy.

			\item ON INCOMPLETE SEARCH:
				\begin{lstlisting}
- Search incompleted - Exiting on limit reached
  Limit			: {0},
 [Maximize		: {1},]
 [Minimize		: {2},]
  Solutions		: {3},
  Times (ms)	: {4},
  Nodes			: {5},
  Backtracks	: {6},
  Restarts		: {7}.
  
  \end{lstlisting}
	brackets [\textit{line}] indicate \textit{line} is optionnal,\\
 	\texttt{Maximize} --resp. \texttt{Minimize}-- indicates the best known value before exiting of the objective value in \textit{maximize()} -- --resp. \textit{minimize()}-- strategy.
		\end{itemize}			

	\item \textbf{VERBOSE -- level 3:} Display informations on search state.
		\begin{itemize}
			\item EVERY X (default=1000) NODES:
			\begin{lstlisting}
- Partial search - [Objective : {0}, ]{1} solutions, {2} Time (ms), {3} Nodes, {4} Backtracks, {5} Restarts.
			\end{lstlisting}
			\texttt{Objective} indicates the best known value.

			\item ON RESTART : 
			\begin{lstlisting}
- Restarting search - {0} Restarts.
			\end{lstlisting}
		\end{itemize}

	\item \textbf{SOLUTION -- level 4:} display all solutions.
		\begin{itemize}
			\item AT EACH SOLUTION:
			\begin{lstlisting}
- Solution #{0} found. [Objective: {0}, ]{1} Solutions, {2} Time (ms), {3} Nodes, {4} Backtracks, {5} Restarts.
  X_1:v1, x_2:v2...
			\end{lstlisting}
		\end{itemize}

	\item \textbf{SEARCH -- level 5:} Display the search tree.
		\begin{itemize}
			\item AT EACH NODE, ON DOWN BRANCH:
			\begin{lstlisting}
...[w] down branch X==v branch b
			\end{lstlisting}
where \texttt{w} is the current world index, \texttt{X} the branching variable, \texttt{v} the branching value and \texttt{b} the branch index. This message can be adapted on variable type and search strategy.

			\item AT EACH NODE, ON UP BRANCH:
			\begin{lstlisting}
...[w] up branch X==v branch b
			\end{lstlisting}
where \texttt{w} is the current world index, \texttt{X} the branching variable, \texttt{v} the branching value and \texttt{b} the branch index. This message can be adapted on variable type and search strategy.
		\end{itemize}

	\item \textbf{FINEST -- level 6:} display all logs.

\end{itemize}

More precisely, if the verbosity level is greater than DEFAULT, then the verbosity levels of the model and of the solver are increased to INFO, and the verbosity levels of the search and of the branching are slightly modified to display the solution(s) and search messages.

The verbosity level can be changed as follows:
\begin{lstlisting}
	ChocoLogging.setVerbosity(Verbosity.VERBOSE);
\end{lstlisting}


\subsubsection{How to write logging statements ?}\label{advanced:howtowriteloggingstatements}\hypertarget{advanced:howtowriteloggingstatements}{}

\begin{itemize}
	\item Critical Loggers are provided to display error or warning. Displaying too much message really \textbf{impacts the performances}.
	\item Check the logging level before creating arrays or strings.
	\item Avoid multiple calls to \texttt{Logger} functions. Prefer to build a \texttt{StringBuilder} then call the \texttt{Logger} function.
	\item Use the \texttt{Logger.log} function instead of building string in \texttt{Logger.info()}.
\end{itemize}

\subsubsection{Handlers.}\label{advanced:handlers}\hypertarget{advanced:handlers}{}
Logs are displayed on \texttt{System.out} but warnings and severe messages are also displayed on \texttt{System.err}.
\texttt{ChocoLogging.java} also provides utility functions to easily change handlers:
\begin{itemize}
	\item Functions \texttt{set...Handler} remove current handlers and replace them by a new handler.
	\item Functions \texttt{add...Handler} add new handlers but do not touch existing handlers.
\end{itemize}

\subsubsection{Define your own logger.}\label{advanced:defineyourownlogger}\hypertarget{advanced:defineyourownlogger}{}
\begin{lstlisting}
ChocoLogging.makeUserLogger(String suffix);
\end{lstlisting}

% \subsubsection{Figure source}\label{advanced:figuresource}\hypertarget{advanced:figuresource}{}
% \begin{lstlisting}
%   digraph G {
%       node [style=filled, shape=box];
%       choco [shape=house,fillcolor=gold];
	
%       kernel [shape=house,fillcolor=gold];
%       engine [shape=octagon,fillcolor=indianred];
%       search [shape=octagon,fillcolor=darkorange];
%       branching [shape=octagon,fillcolor=indianred];
	
%       api [shape=house,fillcolor=gold];
%       model [fillcolor=indianred];
%       solver [fillcolor=indianred];
%       parser [fillcolor=darkorange];
      
%       user [fillcolor=darkorange];
%       samples [fillcolor=darkorange];
      
%       test [fillcolor=indianred];
      
%       choco -> kernel;
%       choco -> API;
%       choco -> user;
%       choco -> test;
      
%       kernel -> engine;
%       kernel -> search;
      
%       api -> model;
%       api -> solver
%       api -> parser;
      
%       user  -> samples;
      
%       search -> branching;
% 	}
% \end{lstlisting}
\label{doc:advanced}\hypertarget{doc:advanced}{}
\chapter{Applications with global constraints}\label{doc:applications}\hypertarget{doc:applications}{}
%\part{scheduling and use of the cumulative}
%\label{schedulinganduseofthecumulative}
%\hypertarget{schedulinganduseofthecumulative}{}

\section{Scheduling and use of the cumulative constraint}\label{schedulinganduseofthecumulative:schedulinganduseofthecumulativeconstraint}\hypertarget{schedulinganduseofthecumulative:schedulinganduseofthecumulativeconstraint}{}

\emph{This tutorial is the Choco2 version of \href{http://choco-solver.net/index.phptitle=schedulinganduseofthecumulative}{this one}}

We present a simple example of a scheduling problem solved using the cumulative global constraint.

%The problem is to schedule a given set of tasks on a single resource and in a given time horizon. 
The problem is to maximize the number of tasks that can be scheduled on  a single resource within a given time horizon.
% and to provide a corresponding valid schedule.

The following picture summarizes the instance that we will use as example. 
It shows the resource profile on the left and on the right, the set of tasks to be scheduled. Each task is represented here as a rectangle whose height is the resource consumption of the task and whose length is the duration of the task. Notice that the profile is not a straight line but varies in time. 

\insertGraphique{\linewidth}{media/schedulinstance.jpg}{A cumulative scheduling problem instance}

This tutorial might help you to deal with :

\begin{itemize}
	\item The profile of the resource that varies in time whereas the API of the cumulative only accepts a constant capacity
	\item The objective function that implies optional tasks which is a priori not allowed by cumulative
	\item A search heuristic that will first assign the tasks to the resource and then try to schedule them while maximizing the number of tasks
\end{itemize}

The first point is easy to solve by adding fake tasks at the right position to simulate the consumption of the resource. The second point is possible thanks to the ability of the cumulative to handle variable heights. We shall explain it in more details soon.

Let's have a look at the source code and first start with the representation of the instance. We need three fake tasks to decrease the profile accordindly to the instance capacity. There is otherwise 11 tasks. Their heights and duration are fixed and given in the two following int[] tables. The three first tasks correspond to the fake one are all of duration 1 and heights 2, 1, 4. 
\begin{lstlisting}
  CPModel m = new CPModel();
  // data
  int n = 11 + 3; //number of tasks (include the three fake tasks)
  int[] heights_data = new int[]{2, 1, 4, 2, 3, 1, 5, 6, 2, 1,3, 1, 1, 2};
  int[] durations_data = new int[]{1, 1, 1, 2, 1, 3, 1, 1, 3, 4,2, 3, 1, 1};
\end{lstlisting}

The variables of the problem consist of four variables for each task (start, end, duration, height). We recall here that the scheduling convention is that a task is active on the interval [start, end-1] so that the upper bound of the start and end variables need to be 5 and 6 respectively. Notice that start and end variables are BoudIntVar variables. Indeed, the cumulative is only performing bound reasonning so it would be a waste of efficiency to declare here EnumVariables. Duration and heights are constant in this problem. However, our plan is to simulate the allocation of a task to the resource by using variable height. In other word, we will define the height of the task i as a variable of domain $\{0, \mathtt{height[i]}\}$. The height of the task takes its normal value if the task is assigned to the resource and 0 otherwise. The duration is really constant and is therefore created as a ConstantIntVar.

Moreover, we add a boolean variable per task to specify if the task is assigned to the resource or not. The objective variable is created as a BoundIntVar. 

\begin{lstlisting}
  IntegerVariable capa = constant(7);
  IntegerVariable[] starts = makeIntVarArray("start", n, 0, 5, "cp:bound");
  IntegerVariable[] ends = makeIntVarArray("end", n, 0, 6, "cp:bound");
  
  IntegerVariable[] duration = new IntegerVariable[n];
  IntegerVariable[] height = new IntegerVariable[n];
  for (int i = 0; i < height.length; i++) {
      duration[i] = constant(durations_data[i]);
      height[i] = makeIntVar("height " + i, new int[]{0, heights_data[i]});
  }
  
  IntegerVariable[] bool = makeIntVarArray("taskIn?", n, 0, 1);
  IntegerVariable obj = makeIntVar("obj", 0, n, "cp:bound", "cp:objective");
\end{lstlisting}

We then add the constraints to the model. Three constraints are needed. First, the cumulative ensures that the resource consumption is respected at any time. Then we need to make sure that if a task is assigned to the cumulative, its height can not be null which is done by the use of boolean channeling constraints. Those constraints ensure that :

$$\mathtt{bool[i]} = 1 \quad\iff\quad \mathtt{height[i]} = \mathtt{heights\_data[i]}$$

We state the objective function to be the sum of all boolean variables. 
\begin{lstlisting}
  //post the cumulative
  m.addConstraint(cumulative(starts, ends, duration, height, capa, ""));                                                                            
  //post the channeling to know if the task is scheduled or not
  for (int i = 0; i < n; i++) {
      m.addConstraint(boolChanneling(bool[i], height[i], heights_data[i]));
  }
  
  //state the objective function
  m.addConstraint(eq(sum(bool), obj));
\end{lstlisting}

Finally we fix the fake task at their position to simulate the profil: 
\begin{lstlisting}
  CPSolver s = new CPSolver();
  s.read(m);
  
  //set the fake tasks to establish the profile capacity of the resource
  try {
      s.getVar(starts[0]).setVal(1); s.getVar(ends[0]).setVal(2); s.getVar(height[0]).setVal(2);
      s.getVar(starts[1]).setVal(2); s.getVar(ends[1]).setVal(3); s.getVar(height[1]).setVal(1);
      s.getVar(starts[2]).setVal(3); s.getVar(ends[2]).setVal(4); s.getVar(height[2]).setVal(4);
  } catch (ContradictionException e) {
      System.out.println("error, no contradiction expected at this stage");
  }
\end{lstlisting}
We are now ready to solve the problem. We could call a maximize(false) but we want to add a specific heuristic that first assigned the tasks to the cumulative and then tries to schedule them.
\begin{lstlisting}
  s.maximize(s.getVar(obj),false);
\end{lstlisting}
We now want to print the solution and will use the following code : 
\begin{lstlisting}
	System.out.println("Objective : " + (s.getVar(obj).getVal() - 3));
	for (int i = 3; i < starts.length; i++) {
	   if (s.getVar(height[i]).getVal() != 0)
	      System.out.println("[" + s.getVar(starts[i]).getVal() + " - " 
                                 + (s.getVar(ends[i]).getVal() - 1) + "]:" 
                                 + s.getVar(height[i]).getVal());
	}
\end{lstlisting}
Choco gives the following solution : 
\begin{lstlisting}
	Objective : 9
	[1 - 2]:2
	[0 - 0]:3
	[2 - 4]:1
	[4 - 4]:5
	[5 - 5]:6
	[0 - 2]:2
	[0 - 3]:1
	[3 - 5]:1
	[0 - 0]:1
\end{lstlisting}
This solution could be represented by the following picture :

\insertGraphique{\linewidth}{media/schedulsolution.jpg}{Cumulative profile of a solution.}

Notice that the cumulative gives a necesserary condition for packing (if no schedule exists then no packing exists) but this condition is not sufficient as shown on the picture because it only ensures that the capacity is respected at each time point. Specially, the tasks might be splitted to fit in the profile as in the previous solution.
The complete code can be found \hyperlink{cumulative:cumulativeconstraint}{here}.

%\part{geost description}
\label{geostdescription}
\hypertarget{geostdescription}{}


\section{Placement and use of the Geost constraint}\label{geostdescription:placementanduseofthegeostconstraint}\hypertarget{geostdescription:placementanduseofthegeostconstraint}{}

The global constraint \texttt{\bf geost}($k,O,S,C$) handles in a generic way a variety of geometrical constraints $C$ in space and time between polymorphic $k\in\N$ dimensional objects $O$, each of which taking a shape among a set of shapes $S$ during a given time interval and at a given position in space. Each shape from $S$ is defined as a finite set of shifted boxes, where each shifted box is described by a box in a $k$-dimensional space at the given offset with the given sizes.

More precisely a \emph{shifted box} $s$= \emph{shape(sid,t[],l[])} is an entity defined by a shape id \emph{sid}, an shift offset \emph{s.t[d]}, $0\le d < k$, and a size \emph{s.l[d]}$>0$, $0\le d<k$. All attributes of a shifted box are integer values. Then, a \emph{shape} from $S$ is a collection of shifted boxes sharing all the same shape id. Note that the shifted boxes associated with a given shape may or may not overlap. This sometimes allows a drastic reduction in the number of shifted boxes needed to describe a shape.
Each \emph{object} \emph{o= object( id, sid,x[], start, duration,end)} from $O$ is an entity defined by a unique object id \emph{o.id} (an integer), a shape id \emph{o.sid}, an origin \emph{o.x[d]}, $0\le d<k$, a starting time \emph{o.start}, a duration \emph{o.duration}$>0$, and a finishingxs time \emph{o.end}.

All attributes \emph{sid, x[0],x[1],...,x[k-1], start, duration, end} correspond to domain variables. Typical constraints from the
list of constraints $C$ are for instance the fact that a given subset of objects from $O$ do not pairwise overlap.
Constraints of the list of constraints $C$ have always two first arguments $A_i$ and $O_i$ (followed by possibly some additional arguments) which respectively specify:
\begin{itemize}
	\item A list of dimensions (integers between 0 and k-1), or attributes of the objects of $O$ the constraint considers.
	\item A list of identifiers of the objects to which the constraint applies.
\end{itemize}

\subsection{Example and way to implement it}\label{geostdescription:exampleandwaytoimplementit}\hypertarget{geostdescription:exampleandwaytoimplementit}{}

We will explain how to use \emph{geost} although a 2D example. Consider we have 3 objects $o_0, o_1, o_2$ to place them inside a box $B$ (3x4) such that they don't overlap (see Figure bellow). The first object $o_0$ has two potential shapes while $o_1$ and $o_2$ have one shape. Given that the placement of the objects should be totally inside $B$ this means that the domain of the origins of objects are as follows (we start from 0 this means that the placement space is from 0 to 2 on $x$ and from 0 to 3 on $y$:
\begin{itemize}
	\item $o_0$: $x$ in 0..1, $y$ in 0..1,
	\item $o_1$: $x$ in 0..1, $y$ in 0..1,
	\item $o_2$: $x$ in 0..1, $y$ in 0..3.
\end{itemize}

\insertGraphique{.9\linewidth}{media/exp_geost.png}{Geost objects and shapes}

We describe now how to solve this problem by using Choco.

\subsubsection{Build a CP model.}\label{geostdescription:buildacpmodel}\hypertarget{geostdescription:buildacpmodel}{}
To begin the implementation we build a CP model:
\begin{lstlisting}
  Model m = new CPModel();
\end{lstlisting}
\subsubsection{Set the Dimension.}\label{geostdescription:setthedimension}\hypertarget{geostdescription:setthedimension}{}
Then we first need to specify the dimension k we are working in. This is done by assigning the dimension to a local variable that we will use later:
\begin{lstlisting}
  int dim = 2;
\end{lstlisting}
\subsubsection{Create the Objects.}\label{geostdescription:createtheobjects}\hypertarget{geostdescription:createtheobjects}{}
Then we start by creating the objects and store them in a vector as such:
\begin{lstlisting}
  Vector<GeostObject> objects = new Vector<GeostObject>();
\end{lstlisting}
Now we create the first object $o_0$ by creating all its attributes.
\begin{lstlisting}
  int objectId = 0; // object id
  IntegerVariable shapeId = Choco.makeIntVar("sid", 0, 1); // shape id (2 possible values)
  IntegerVariable coords[] = new IntegerVariable[dim]; // coordinates of the origin 
  coords[0] = Choco.makeIntVar("x", 0, 1);  
  coords[1] =  Choco.makeIntVar("y", 0, 1);
\end{lstlisting}
We need to specify 3 more Integer Domain Variables representing the temporal attributes (start, duration and end), which for the current implementation of \texttt{geost} are not working, however we need to give them dummy values. 
\begin{lstlisting}
  IntegerVariable start = Choco.makeIntVar("start", 0, 0);
  IntegerVariable duration = Choco.makeIntVar("duration", 1, 1);
  IntegerVariable end = Choco.makeIntVar("end", 1, 1);
\end{lstlisting}
Finally we are ready to add the object 0 to our \emph{objects} Vector:
\begin{lstlisting}
  objects.add(new GeostObject(dim, objectId, shapeId, coords, start, duration, end));
\end{lstlisting}
Now we do the same for the other  object $o_1$  and  $o_2$ and add them to our \emph{objects} vector.


\subsubsection{Create the Shifted Boxes.}\label{geostdescription:createtheshiftedboxes}\hypertarget{geostdescription:createtheshiftedboxes}{}
To create the shapes and their shifted boxes we create the shifted boxes and associate them with the corresponding shapeId. This is done as follows, first we create a Vector called \emph{sb} for example 
\begin{lstlisting}
Vector<ShiftedBox> sb = new Vector<ShiftedBox> ();
\end{lstlisting}
To create the shifted boxes for the shape 0 (that corresponds to $o_0$), we start by the first shifted box by creating the sid and 2 arrays one to specify the offset of the box in each dimension and one for the size of the box in each dimension:
\begin{lstlisting}
  int sid = 0; 
  int[] offset = {0,0};
  int[] sizes = {1,3};
\end{lstlisting} 
Now we add our shiftedbox to the \emph{sb} Vector:
\begin{lstlisting}
  sb.add(new ShiftedBox(sid, offset, sizes));
\end{lstlisting} 
We do the same with second shifted box:
\begin{lstlisting}
  sb.add(new ShiftedBox(0, new int[]{0,0}, new int[]{2,1}));
\end{lstlisting} 
By the same way we create the shifted boxes corresponding to second shape $S_1$:
\begin{lstlisting}
  sb.add(new ShiftedBox(1, new int[]{0,0}, new int[]{2,1}));
  sb.add(new ShiftedBox(1, new int[]{1,0}, new int[]{1,3}));
\end{lstlisting}
and the third shape $S_2$ consisting of three shifted boxes:
\begin{lstlisting}
  sb.add(new ShiftedBox(2, new int[]{0,0}, new int[]{2,1})) ;
  sb.add(new ShiftedBox(2, new int[]{1,0}, new int[]{1,3})); 
  sb.add(new ShiftedBox(2, new int[]{0,2}, new int[]{2,1}));
\end{lstlisting}
and finally the last shape $S_3$
\begin{lstlisting}
  sb.add(new ShiftedBox(3, new int[]{0,0}, new int[]{2,1}));
\end{lstlisting}

\subsubsection{Create the constraints.}\label{geostdescription:createtheconstraints}\hypertarget{geostdescription:createtheconstraints}{}
First we create a Vector called ectr that will contain the external constraints. 
\begin{lstlisting}
  Vector <ExternalConstraint> ectr = new Vector <ExternalConstraint>();
\end{lstlisting}
In order to create the non-overlapping constraint we first create an array containing all the dimensions the constraint will be active in (in our example it is all dimensions) and lets name this array \emph{ectrDim} and a list of objects \emph{objOfEctr} that this constraint will apply to (in our example it is all objects).

\begin{note}
Note that in the current implementation of \texttt{geost} only the non-overlapping constraint is available. Moreover, 
\emph{ectrDim} should contain all dimensions and \emph{objOfEctr} should contain all the objects, i.e. the non-overlapping constraint applies to all the objects in all dimensions.
\end{note}
 
After that we add the constraint to a vector \emph{ectr} that contains all the constraints we want to add.
The code for these steps is as follows: 
\begin{lstlisting}
  int[] ectrDim = new int[dim]; 
  for(i = 0; i < dim; i ++)  
      ectrDim[i] = i; 
  int[] objOfEctr = new int[3]; 
  for(i = 0; i < 3; i ++) 
      objOfEctr[i] = objects.elementAt(i).getObjectId();
\end{lstlisting}
All we need to do now is create the non-overlapping constraint and add it to the \emph{ectr} vector
that holds all the constraints. this is done as follows:
\begin{lstlisting}
  //Constants.NON_OVERLAPPING indicates the id of the non-overlapping constraint
  NonOverlapping n = new NonOverlapping(Constants.NON_OVERLAPPING, ectrDim, objOfEctr);
  ectr.add(n);
\end{lstlisting}

\subsubsection{Create the \texttt{geost} constraint and add it to the model.}\label{geostdescription:createthegeostconstraintandaddittothemodel}\hypertarget{geostdescription:createthegeostconstraintandaddittothemodel}{}
\begin{lstlisting}
  Constraint geost = Choco.geost(dim, objects, sb, ectr);
  m.addConstraint(geost);
\end{lstlisting}
	
\subsubsection{Solve the problem.}\label{geostdescription:solvetheproblem}\hypertarget{geostdescription:solvetheproblem}{}
\begin{lstlisting}
  Solver s = new CPSolver();
  s.read(m);
  s.solve();
\end{lstlisting}

The full java code can be found here: \href{media/zip/geostexp.java}{geostexp.java}

\subsection{Support for Greedy Assignment within the geost Kernel}\label{geostdescription:supportforgreedyassignmentwithinthegeostkernel}\hypertarget{geostdescription:supportforgreedyassignmentwithinthegeostkernel}{}

\subsubsection{Motivation and functionality description.}\label{geostdescription:motivationandfunctionalitydescription}\hypertarget{geostdescription:motivationandfunctionalitydescription}{}
Since, for performance reasons, the \texttt{geost} kernel offers a mode where he tries to fix all objects during one single propagation step, we provide a way to specify a preferred order on how to fix all the objects in one single propagation step. This is achieved by:
\begin{itemize}
	\item Fixing the objects according to the order they were passed to the \texttt{geost} kernel.
	\item When considering one object, fixing its shape variable as well as its coordinates:
	\begin{itemize}
		\item According to an order on these variables that can be explicitly specified.
		\item A value to assign that can either be the smallest or the largest value, also specified by the user.
	\end{itemize}
\end{itemize}

\begin{note}
Note that the use of the greedy mode assumes that no other constraint is present in the problem.
\end{note}

	This is encoded by a term that has exactly the same structure as the term associated to an object of  \texttt{geost}. The only difference consists of the fact that a variable is replaced by an expression \_ (\emph{The character \_ denotes the fact that the corresponding attribute is irrelevant, since for instance, we know that it is always fixed}), $\min(I)$ (respectively, $\max(I)$), where $I$ is a strictly positive integer. The meaning is that the corresponding variable should be fixed to its minimum (respectively maximum value) in the order $I$.   We can in fact give a list of vectors  $v_1,v_2,\ldots,v_p$ in order to specify how to fix objects $o_{(1+pa)},o_{(2+pa)},...,o_{(p+pa)}$.

This is illustrated by Figure bellow: for instance, Part(\emph{I}) specifies that we alternatively:
\begin{itemize}
	\item fix the shape variable of an object to its maximum value (i.e., by using max(1) ), fix the $x$-coordinate of an object to its its minimum value (i.e., by using min(2)), fix the $y$-coordinate of an object to its its minimum value (i.e., by using min(3)) and
	\item fix the shape variable of an object to its maximum value (i.e., by using max(1)), fix the $x$-coordinate of an object to its its maximum value (i.e., by using max(2)), fix the $y$-coordinate of an object to its its maximum value (i.e., by using max(3)).
\end{itemize}

In the example associated with Part (I) we successively fix objects $o_1, o_2, o_3, o_4, o_5, o_6$ by alternatively using strategies (1) \mylst{object(_,max(1),x[min(2),min(3)])} and (2) \mylst{object(_,max(1),x[max(2),max(3))}. 

\insertGraphique{\linewidth}{media/greedy.png}{Greedy placement}

\subsubsection{Implementation.}\label{geostdescription:implementation}\hypertarget{geostdescription:implementation}{}
The greedy algorithm for fixing an object o is controlled by a vector \emph{v} of length \emph{k+1} such that:
\begin{itemize}
	\item The shape variable \emph{o.sid} should be set to its minimum possible value if \emph{v}[0]$<0$, and to its maximum possible value otherwise.
	\item abs(\emph{v}[1])-2 is the most significant dimension (the one that varies the slowest) during the sweep. The values are tried in ascending order if \emph{v}[1]$<0$, and in descending order otherwise.
	\item abs(\emph{v}[2])-2 is the next most significant dimension, and its sign indicates the value order, and so on.
\end{itemize}

For example, a term \mylst{object(_,min(1),[max(3),min(4),max(2)])} is encoded as the vector $[-1,4,2,-3]$.
	                                      
\subsubsection{Second example.}\label{geostdescription:secondexample}\hypertarget{geostdescription:secondexample}{}
We will explain although a 2D example how to take into account of greedy mode. Consider we have 12 identical objects $o_0,o_1,\ldots,o_{11}$ having 4 potential shapes and we want to place them in a box $B$ (7x6) (see Figure bellow). Given that the placement of the objects should be totally inside B this means that the domain of the origins of objects are as follows $x\in [0,5]$, $y\in [0,4]$. Moreover, suppose that we want use two strategies when greedy algorithm is called: the term \mylst{object(_,min(1),[min(2),min(3)])} for objects $o_0, o_2, o_4, o_6, o_8, o_{10}$, and the term \mylst{object(_,max(1),[max(2),max(3)])} for objects $o_1, o_3, o_5, o_7, o_9, o_{11}$. These strategies are encoded respectively as $[-1,-2,-3]$ and $[1, 2, 3]$.

\insertGraphique{\linewidth}{media/exp2_geost.png}{A second Geost instance.}

We comment only the additional step w.r.t. the preceding example. In fact we just need to create the list of controlling vectors before creating the geost constraint. Each controlling vector is an array:
\begin{lstlisting}
  Vector <int[]> ctrlVs = new Vector<int[]>();
  int[] v0 = {-1, -2, -3};
  int[] v1 = {1, 2, 3};
  ctrlVs.add(v0);
  ctrlVs.add(v1);
\end{lstlisting}
and then create the \texttt{geost} constraint, by adding the list of controlling vectots as an another argument, as follows:
\begin{lstlisting}
  Constraint geost = Choco.geost(dim, objects, sb, ectr,ctrlVs);
  m.addConstraint(geost);
\end{lstlisting}

The full java code can be reached \href{media/zip/greddyexp.java}{here}.
\lstinputlisting{media/zip/greedyexp.java}

The placement obtained using the preceding strategies is displayed in the following figure (right side).

\insertGraphique{\linewidth}{media/solution_exp2.png}{A solution placement}



%\part{faq}
\label{faq}
\hypertarget{faq}{}

\chapter{Frequently Asked Questions}\label{faq:frequentlyaskedquestions}\hypertarget{faq:frequentlyaskedquestions}{}

\section{Where can I find Choco ?}\label{faq:wherecanifindchoco}\hypertarget{faq:wherecanifindchoco}{}

See the \href{http://choco.emn.fr}{download page} if you want to download a version of Choco library.

\section{What is the required Java version to run Choco ?}\label{faq:whatistherequiredjavaversiontorunchoco}\hypertarget{faq:whatistherequiredjavaversiontorunchoco}{}

Choco requires \href{http://java.sun.com/javase/6/}{java6}.

\begin{note}
If you are working on Mac OS X 10.4 Tiger or if you do not have an Intel processor, you probably can not install java 6 on your OS. Please, take a look at \href{http://landonf.bikemonkey.org/static/soylatte/}{Soy latte}, which goals are ``support for Java 6 Development on Mac OS X 10.4 and 10.5, OpenJDK support for Java 7 on Mac OS X and On-time release of Java 7 for Mac OS X''.
\end{note}

\section{How to add the Choco library to my project?}\label{faq:howtoaddthechocolibrarytomyproject}\hypertarget{faq:howtoaddthechocolibrarytomyproject}{}

% commented out template(flowplay>?640x480 noautoPlay)

You just need to add Choco.X.x.x.jar to your classpath.

\textbf{IntelliJ}:
\begin{itemize}
	\item Go to ``File/Settings''
	\item Select ``Project Settings''
	\item Click on ``Librairies'' then on [+]
	\item Enter ``Choco'' as the library name, and OK
	\item Choose your project, and OK,
	\item Click on ``Attach Jar Directories'' and choose the directory where you put the Choco jar.
\end{itemize}

\todo{fix hyperlink{flowplay>videos:intellij.flv}{How to add choco to IntelliJ in video}}

\textbf{Eclipse}:
\begin{itemize}
	\item Go to ``Project/Properties'',
	\item Select ``Java Build Path'' on the left menu
	\item On the right, select ``Librairies''
	\item Click on ``Add Externals JARs...'' button
	\item Select the Choco jar file
\end{itemize}

\todo{fix hyperlink{flowplay>videos:eclipse.flv}{How to add choco to Eclipse in video}}

\begin{note}
If you work with \textbf{the source and not the jar of Choco}, do not forget to also add the automaton.jar and junit.jar, available in the lib/ directory
\end{note}

\section{Why can't I see the Choco API?}\label{faq:whycan'tiseethechocoapi}\hypertarget{faq:whycan'tiseethechocoapi}{}
To have the Choco API available, you must make the following import in your class file:

\begin{lstlisting}
import static choco.Choco.*;
\end{lstlisting}

\section{How to know the value of my variable in the Solver ?}\label{faq:howtoknowthevalueofmyvariableinthesolver}\hypertarget{faq:howtoknowthevalueofmyvariableinthesolver}{}
There are two different kinds of variables: those associated with the Model (like \texttt{IntegerVariable}, \texttt{SetVariable},...) and those associated with the Solver (like \texttt{IntDomainVar}, \texttt{SetVar},...). The second type is a Solver interpretation of the first one (which is only declarative).
After having defined your model with variables and constraints, it has to be read by the Solver. After that, a Solver object is created. You can access the Variable Model \emph{value} through the Solver using the following method of the Solver:
\mylst{solver.getVar(Variable v);}
where v is a Model variable (or an array of Model variables) and it returns a Solver variable.

\section{How do I use constant value inside constraint ?}\label{faq:howdoiuseconstantevalueinsideconstraint}\hypertarget{faq:howdoiuseconstantevalueinsideconstraint}{}
Some constraints doesn't provide API with java object (like \textbf{int}, \textbf{double} or \textbf{Integer}). 
You can define \emph{constant} variable (ie, variable with one unique value) liek this:
\begin{lstlisting}
	IntegerVariable one = constant(1);
	RealVariable one = constant(1.0);
\end{lstlisting}
And, you can use this \emph{variable} inside the constraint:
\begin{lstlisting}
	Model m = new CPModel();
	
	IntegerVariable x = makeIntVar("x", 0, 10);
	IntegerVariable two = constant(2);
	IntegerVariable maximum = makeIntVar("max", 0, 15);
	
	m.addConstraint(eq(maximum, max(x, two));
\end{lstlisting}

Do not forget that some contraints provide api with java object.

\section{How can I use Choco to solve CSP'08 benchmark ?}\label{faq:howcaniusechocotosolvecsp'08benchmark}\hypertarget{faq:howcaniusechocotosolvecsp'08benchmark}{}
You can easily load an XML file of the CSP'08 competition and solve it with Choco.
To load the file, we use the XMLParser available \href{http://www.cril.univ-artois.fr/\~{}lecoutre/research/tools/tools.html}{here}:
\begin{lstlisting}
	String fileName = "../../ProblemsData/CSPCompet/intension/nonregres/graph1.xml";
	File instance = new File(fileName);
	XmlModel xs = new XmlModel();  // a class to ease loading and solving CSP'08 xml file
	InstanceParser parser = xs.load(instance);  // loading of the CPS'08 xml file
\end{lstlisting}
Once the file has been loaded, a Model object is build from the InstanceParser object:
\begin{lstlisting}
CPModel model = xs.buildModel(parser); // Creation of the model
\end{lstlisting}
At this point, you can choose to solve this model with a pre-processing step.
The pre-processing step analyzes variables and constraints, makes some specific choices to improve the resolution.
Concerning variables, it analyzes domains and constraints and choose what seems to be the best kind of domain (for example, enumerated or bounded domain), or add one variable where large number of variables are equals, ...
Concerning constraints, it detects clique of differences or disjunctions and state the corresponding global constraints, breaks symetries, detects distance...
Then, it can also choose the search strategy.
To do this, use the following code:
\begin{lstlisting}
PreProcessCPSolver s = xs.solve(model); // Build a BlackBoxSolver and solve it.
\end{lstlisting}
Finally, you can print informations concerning the resolution:
\begin{lstlisting}
\lstinline|xs.postAnalyze(instance, parser, s);
\end{lstlisting}

You can easily solve benchmarks of CSP'08 competition, or with your own problem modelize in \href{http://www.cril.univ-artois.fr/cpai08/xcsp21.pdf}{CSP'08 xml format}.

\section{How do I use the build.xml file ?}\label{faq:howdoiusethebuild.xmlfile}\hypertarget{faq:howdoiusethebuild.xmlfile}{}

The choco project provides an ant script \texttt{build.xml} for the most usual tasks of the project.
In this section, we show how to run these tasks with a terminal. Ant is fully integrated in most of Java IDE but we will not talk about it.

First, we are going into the root directory of choco
\begin{lstlisting}
	nono@arrakis:~\$ cd /path/to/choco/
	nono@arrakis:~/workspace/Choco-2.0\$ ls
	bin  build.xml  checkstyle.xml  Choco2.0.0.iml  choco-ruleset.xml  dev  lib  pom.xml
\end{lstlisting}
Then, try a simple 
\begin{lstlisting}
	nono@arrakis:~/workspace/Choco-2.0\$ ant
	Buildfile: build.xml
	
	init:
	     [echo] Ant  version                  : Apache Ant version 1.7.0 compiled on August 29 2007
	     [echo] Java version                  : 1.6.0_06
	     [echo] build of project JChoco : June 30 2008
	
	help:
	     [echo] be careful, there could have a bug in eclipse with this help message.
	     [echo] In this case, type "ant -p popart/build.xml" in a terminal.     
	     [exec] Buildfile: build.xml
	     [exec] 
	     [exec] Main targets:
	     [exec] 
	     [exec]  clean            --> deletes everything that seems useless
	     [exec]  compile          --> compiles everything
	     [exec]  dist             --> makes the distribution package (jar, src.zip, doc.zip)
	     [exec]  doc              --> generates the javadoc
	     [exec]  exec-junit-test  --> executes all junit tests.
	     [exec]  exec-pmd         --> analyzes code with PMD and CPD
	     [exec]  help             --> print this help
	     [exec] Default target: help
	
	BUILD SUCCESSFUL
	Total time: 1 second
\end{lstlisting}

Most of these tasks do not have some special requirements. However, you could need specific settings to run \texttt{exec-junit-test} and \texttt{exec-pmd}.
You need to have \texttt{junit} and \texttt{ant-junit} jars in your classpath to run \texttt{exec\_junit-test}. But, it works for my first attempt without any changes.
You need to set \texttt{pmd} jar in your classpath and probably to update the property \texttt{pmd.xslt} to run \texttt{exec\_junit-test}.
Supposes that you have installed \texttt{pmd} in \texttt{/path/to/pmd}. You have to reset the location of the following property:
\begin{lstlisting}
<property name="pmd.xslt" location="/path/to/pmd/etc/xslt/wz-pmd-report.xslt" />
\end{lstlisting}
Finally you can run the task with :
\begin{lstlisting}
ant -lib /path/to/pmd/lib/pmd-4.1.jar exec-pmd
\end{lstlisting}
It seems that using PMD with your IDE is more effective. The integration offers many filtering options and allows to correct your code on the fly.

\emph{A \href{http://www.emn.fr/x-info/choco-solver/forum/viewtopic.phpf=5&t=4&start=0&st=0&sk=t&sd=a}{post} is opened for feedbacks, for new feature requests, and for any comments about the \texttt{build.xml} file}

\section{Why do I have a error when I add my constraint ?}\label{faq:whydoihaveaerrorwheniaddmyconstraint}\hypertarget{faq:whydoihaveaerrorwheniaddmyconstraint}{}
If you have a error message like this:\\
\centerline{\emph{Component class could not be found: my.package.and.my.Constraint.ConstraintManager}}\\
and if the ConstraintManager is an inner class of your constraint, you must define the name in the component name like this:
\begin{lstlisting}
my.package.and.my.Constraint\$ConstraintManager
\end{lstlisting}

For more details, see \hyperlink{advanced:defineyourownconstraint}{define\ your\ own\ constraint}.

\section{How do I upgrade my program to Choco2.0 ?}\label{faq:howdoiupgrademyprogramtochoco2.0}\hypertarget{faq:howdoiupgrademyprogramtochoco2.0}{}
Without being very precise (see the \hyperlink{ch:doc}{documentation} if you want more details), it is really easy to transpose a program implemented on an old version of Choco to Choco2.0.

\textbf{No Problem!!} The \texttt{Problem} class does not exist anymore. It has been replaced by two new classes: \texttt{CPModel} and \texttt{CPSolver} that implement the interfaces \texttt{Model} and \texttt{Solver.} The model allows you to declare your variables and constraints and the solver allows you to define some search strategies and solve your model. As different kinds of Model and Solver will be available, everything concerning Variables and Constraints is included in the new class \texttt{Choco}. 

Now, let us see in a few steps how to transpose your program. Consider that you created the following program:
\begin{lstlisting}
	// Creation of the problem
	Problem pb = new Problem();
	
	// Declaration of variables
	IntDomainVar v1 = pb.makeEnumIntVar("v1", 1, 10);
	IntDomainVar v2 = pb.makeEnumIntVar("v1", 1, 10);
	
	// Declaration of constraints
	Constraint c1 = pb.neq(v1,v2);
	pb.post(c1);
	// Declaration of a user constraint
	Constraint prime-number = MyConstraint(v1, v2);
	pb.post(prime-number);
	
	// Definition of a search strategy
	pb.getSolver().setVarSelector(new StaticVarOrder(v1, v2));
	pb.getSolver().setValIterator(new IncreasingDomain());
	
	// Resolution of the problem
	pb.solve();
	
	// Print the solution
	System.out.println("v1"+v1.getVal());
	System.out.println("v2"+v2.getVal());
\end{lstlisting}

\begin{itemize}
	\item \textbf{A Problem becomes a Model and a Solver}
\end{itemize}

\begin{lstlisting}
	// Creation of the problem
	Problem pb = new Problem();
\end{lstlisting}
becomes
\begin{lstlisting}
	// Creation of the Model
	Model m = new CPModel();
	//Creation of the Solver
	Solver s = new CPSolver();
\end{lstlisting}

\begin{itemize}
	\item \textbf{Variables are independent of a Problem or a Model}
\end{itemize}

\begin{lstlisting}
	// Declaration of variables
	IntDomainVar v1 = pb.makeEnumIntVar("v1", 1, 10);
	IntDomainVar v2 = pb.makeEnumIntVar("v1", 1, 10);
\end{lstlisting}
becomes
\begin{lstlisting}
	// add import:
	import static choco.Choco.*;
	//...
	// Declaration of variables
	IntegerVariable v1 = makeIntVar("v1", 1, 10);
	IntegerVariable v2 = makeIntVar("v1", 1, 10);
	m.addVariable(CPOptions.V_ENUM, v1, v2);
\end{lstlisting}

\begin{itemize}
	\item \textbf{Easy declaration of constraints}
\end{itemize}

\begin{lstlisting}
	// Declaration of constraints
	Constraint c1 = pb.neq(v1,v2);
	pb.post(c1);
\end{lstlisting}
becomes
\begin{lstlisting}
	// add import (same than Variables):
	import static choco.Choco.*;
	//...
	// Declaration of constraints
	Constraint c1 = neq(v1,v2);
	m.addConstraint(c1);
\end{lstlisting}

\begin{itemize}
	\item \textbf{A specific way to define user constraints}
\end{itemize}

\begin{lstlisting}
	// Declaration of a user constraint
	Constraint prime-number = myConstraint(v1, v2);
	pb.post(prime-number);
\end{lstlisting}
becomes
\begin{lstlisting}
	// Declaration of a user constraint
	m.addConstraint(new ComponentConstraint(MyManager.class, null, v1, v2));
\end{lstlisting}

\begin{itemize}
	\item \textbf{Do not forget to read the model}
\end{itemize}

It is a \emph{\textbf{new step}}, it has to be done! 
\begin{lstlisting}
	// Read the model
	s.read(m);
\end{lstlisting}

\begin{itemize}
	\item \textbf{Clear definition of the search strategy}
\end{itemize}

\begin{lstlisting}
	// Definition of a search strategy
	pb.getSolver().setVarSelector(new StaticVarOrder(v1, v2));
	pb.getSolver().setValIterator(new IncreasingDomain());
\end{lstlisting}
becomes
\begin{lstlisting}
	// Definition of a search strategy
	s.setVarIntSelector(new StaticVarOrder(s.getVar(v1, v2)));
	s.setValIntIterator(new IncreasingDomain());
\end{lstlisting}

\begin{itemize}
	\item \textbf{And the resolution}
\end{itemize}

\begin{lstlisting}
	// Resolution of the problem
	pb.solve();
\end{lstlisting}
becomes
\begin{lstlisting}
	// Resolution of the model
	s.solve();
\end{lstlisting}

\begin{itemize}
	\item \textbf{Printing the solution}
\end{itemize}

\begin{lstlisting}
	// Print the solution
	System.out.println("v1"+v1.getVal());
	System.out.println("v2"+v2.getVal());
\end{lstlisting}
becomes
\begin{lstlisting}
	// Print the solution
	System.out.println("v1"+s.getVar(v1).getVal());
	System.out.println("v2"+s.getVar(v2).getVal());
\end{lstlisting}

And it's done!
We obtain the following code:
\begin{lstlisting}
	import static choco.Choco.*;
	...
	// Creation of the Model
	Model m = new CPModel();
	//Creation of the Solver
	Solver s = new CPSolver();
	
	// Declaration of variables
	IntegerVariable v1 = makeIntVar("v1", 1, 10);
	IntegerVariable v2 = makeIntVar("v2", 1, 10);
	m.addVariable(CPOptions.V_ENUM,v1, v2);
	
	// Declaration of constraints
	Constraint c1 = neq(v1,v2);
	m.addConstraint(c1);
	// Declaration of a user constraint
	m.addConstraint(new ComponentConstraint(MyManager.class, null, v1, v2));
	
	// Read the model
	s.read(m);
	
	// Definition of a search strategy
	s.setVarIntSelector(new StaticVarOrder(s.getVar(v1, v2)));
	s.setValIntIterator(new IncreasingDomain());
	
	// Resolution of the model
	s.solve();
	
	// Print the solution
	System.out.println("v1"+s.getVar(v1).getVal());
	System.out.println("v2"+s.getVar(v2).getVal());
\end{lstlisting}

\section{Are bounds with positive and negative infinity supported within Choco?}\label{faq:areboundswithpositiveandnegativeinfinitysupportedwithinchoco}\hypertarget{faq:areboundswithpositiveandnegativeinfinitysupportedwithinchoco}{}

Integer or Double infinity bounds are not really appreciate by CHOCO :) 
Because, during propagation, a basic test is done on bounds and the following operation can be applied: 
\emph{upper bound +1}.
As \texttt{Integer.MAX\_VALUE+1} is equal to \texttt{Integer.MIN\_VALUE}, it can corrupt the propagation. 

If you really want to have a large domain, a division with 10 should be sufficient: 
\begin{lstlisting}
IntegerVariable v1 = makeIntVar("v1", Integer.MIN_VALUE/10, Integer.MIN_VALUE/10);
RealVariable a1 = makeRealVar("A1", Double.NEGATIVE_INFINITY/10, Double.POSITIVE_INFINITY/10);
\end{lstlisting}


%\chapter{Variables}\label{ch:vars}\hypertarget{ch:vars}{}
%\section{Integer variables}\label{integervariable}\hypertarget{integervariable}{}
\texttt{IntegerVariable} is a variable whose associated domain is made of integer values. 

\subsubsection{constructors:}
      \noindent\begin{tabular}{p{.8\linewidth}p{.15\linewidth}}
        Choco method & return type \\
        \hline
        \mylst{makeIntVar(String name, int lowB, int uppB, String... options)} &\texttt{IntegerVariable}\\
		\mylst{makeIntVar(String name, List<Integer> values, String... options)} &\texttt{IntegerVariable}\\
		\mylst{makeIntVar(String name, int[] values, String... options)} &\texttt{IntegerVariable}\\
        \mylst{makeBooleanVar(String name, String... options)}  &\texttt{IntegerVariable}\\
        \mylst{makeIntVarArray(String name, int dim, int lowB, int uppB, String... options)} &\texttt{IntegerVariable[]}\\
        \mylst{makeIntVarArray(String name, int dim, int[] values, String... options)} &\texttt{IntegerVariable[]}\\
        \mylst{makeBooleanVarArray(String name, int dim, String... options)}  &\texttt{IntegerVariable[]}\\
        \mylst{makeIntVarArray(String name, int dim1, int dim2, int lowB, int uppB, String... options)}  &\texttt{IntegerVariable[][]}\\
        \mylst{makeIntVarArray(String name, int dim1, int dim2, int[] values, String... options)}  &\texttt{IntegerVariable[][]}\\
      \end{tabular}
% 	\begin{itemize}
% 		\item to create an \textbf{IntegerVariable} object:
% 		\begin{itemize}
% 			\item \mylst{makeIntVar(String name, int lowB, int uppB, String... options)}
% 			\item \mylst{makeIntVar(String name, List<Integer> values, String... options)}
% 			\item \mylst{makeIntVar(String name, int[] values, String... options)}
% 		\end{itemize}
% 		\item to create an \textbf{array of IntegerVariable} object:
% 		\begin{itemize}
% 			\item \mylst{makeIntVarArray(String name, int dim, int lowB, int uppB, String... options)}
% 			\item \mylst{makeIntVarArray(String name, int dim, int[] values, String... options)}
% 		\end{itemize}
% 		\item to create a \textbf{matrix of IntegerVariable} object:
% 		\begin{itemize}
% 			\item \mylst{makeIntVarArray(String name, int dim1, int dim2, int lowB, int uppB, String... options)}
% 			\item \mylst{makeIntVarArray(String name, int dim1, int dim2, int[] values, String... options)}
% 		\end{itemize}
% 		\item to create an \textbf{IntegerVariable} object with pre defined domain [0,1]:
% 		\begin{itemize}
% 			\item \mylst{makeBooleanVar(String name, String... options)}
% 		\end{itemize}
% 		\item to create an \textbf{array of IntegerVariable} object with pre defined domain [0,1]:
% 		\begin{itemize}
% 			\item \mylst{makeBooleanVarArray(String name, int dim, String... options)}
% 		\end{itemize}
% 	\end{itemize}
% 	\item \textbf{return type} : \texttt{IntegerVariable} \emph{or} \texttt{IntegerVariable[]} \emph{or} \texttt{IntegerVariable[][]}
\subsubsection{options:}
	\begin{itemize}
		\item \emph{no option} : equivalent to option \texttt{cp:enum}
		\item \texttt{cp:enum} : to force Solver to create enumerated domain for the variable. It is a domain in which holes can be created by the solver. It should be used when discrete and quite small domains are needed and when constraints performing Arc Consistency are added on the corresponding variables. Implemented by a \texttt{BitSet} object.
		\item \texttt{cp:bound} : to force Solver to create bounded domain for the variable. It is a domain where only bound propagation can be done (no holes). It is very well suited when constraints performing only Bound Consistency are added on the corresponding variables. It must be used when large domains are needed. Implemented by two integers.
		\item \texttt{cp:binary} : to force Solver to create binary domain for the variable. It is a [0,1] domain. Some operations are improved and leads to an direct instantiation, if necessary. It must be used with two size domain. Implemented by a shared \texttt{BitSet} object, common to each 32 binary variables (to deal with 64 bits bitset).
		\item \texttt{cp:link} : to force Solver to create linked list domain for the variable. It is an enumerated domain where holes can be done and every values has a link to the previous value and to the next value. It is built by giving its name and its bounds: lower bound and upper bound. It must be used when the very small domains are needed, because although linked list domain consumes more memory than the \texttt{BitSet} implementation, it can provide good performance as iteration over the domain is made in constant time. Implemented by a \texttt{LinkedList} object.
		\item \texttt{cp:btree} : to force Solver to create binary tree domain for the variable. \emph{Under development}.
		\item \texttt{cp:blist} : to force Solver to create bipartite list domain for the variable. It is a domain where unavailable values are placed in the left part of the list, the other one on the right one.
		\item \texttt{cp:decision} : to force variable to be a decisional one
		\item \texttt{cp:no\_decision} : to force variable to be removed from the pool of decisional variables
		\item \texttt{cp:objective} : to define the variable to be the one to optimize
	\end{itemize}
\subsubsection{methods:}
      \begin{itemize}
      \item \mylst{removeVal(int val)}: remove value \emph{val} from the domain of the current variable
      \end{itemize}

A variable with $\{0,1\}$ domain is automatically considered as boolean domain.

\subsubsection{Example:}
\begin{lstlisting}
  IntegerVariable ivar1 = makeIntVar("ivar1", -10, 10);
  IntegerVariable ivar2 = makeIntVar("ivar2", 0, 10000, "cp:bound", "cp:decision");
  IntegerVariable bool = makeIntVar("bool", 0, 1, "cp:binary");
\end{lstlisting} 

%\section{Real variables}\label{realvariable}\hypertarget{realvariable}{}
\texttt{RealVariable} is a variable whose associated domain is made of real values. Only enumerated domain is available for real variables. 

Such domain are memory consuming. In order to minimize the memory use and to have the precision you need, the model offers a way to set a precision (default value is 1.0e-6):
\begin{lstlisting}
	Model m = new CPModel();
	m.setPrecision(0.01);
\end{lstlisting}

\subsubsection{constructor:}
      \noindent\begin{tabular}{p{.8\linewidth}p{.15\linewidth}}
        Choco method & return type \\
        \hline
        \mylst{makeRealVar(String name, double lowB, double uppB, String... options)} &\texttt{RealVariable}\\
      \end{tabular}
%	\begin{itemize}
%		\item to create a \textbf{RealVariable} object:
%		\begin{itemize}
%			\item \mylst{makeRealVar(String name, double lowB, double uppB, String... options)}
%		\end{itemize}
%	\end{itemize}
%	\item \textbf{return type} : \texttt{RealVariable}
\subsubsection{options:}
	\begin{itemize}
		\item \emph{no option} : no particular choice on decision or objective.
		\item \texttt{CPOptions.V_DECISION} : to force variable to be a decisional one
		\item \texttt{cp:no\_decision} : to force variable to be removed from the pool of decisionnal variables
		\item \texttt{CPOptions.V_OBJECTIVE} : to define the variable to be the one to optimize
	\end{itemize}

\subsubsection{Example:}
\begin{lstlisting}
	RealVariable rvar1 = makeRealVar("rvar0", -10.0, 10.0);
	RealVariable rvar2 = makeRealVar("rvar2", 0.0, 100.0,CPOptions.V_DECISION, CPOptions.V_OBJECTIVE);
\end{lstlisting} 

%\section{Set variables}\label{setvariable}\hypertarget{setvariable}{}
\texttt{SetVariable} is high level modeling tool. It allows to represent variable whose values are sets. A SetVariable on integer values between $[1,n]$ has $2*n$ values (every possible subsets of $\{1..n\}$). This makes an exponential number of values and the domain is represented with two bounds corresponding to the intersection of all possible sets (called the kernel) and the union of all possible sets (called the envelope) which are the possible candidate values for the variable. The consistency achieved on SetVariables is therefore a kind of bound consistency.

\subsubsection{constructors:}
      \noindent\begin{tabular}{p{.8\linewidth}p{.15\linewidth}}
        Choco method & return type \\
        \hline
        \mylst{makeSetVar(String name, int lowB, int uppB, String... options)} &\texttt{SetVariable}\\
        \mylst{makeSetVarArray(String name, int dim, int lowB, int uppB, String... options)} &\texttt{SetVariable[]}
      \end{tabular}
%	\begin{itemize}
%		\item to create an \textbf{SetVariable} object:
%		\begin{itemize}
%			\item \mylst{makeSetVar(String name, int lowB, int uppB, String... options)}
%		\end{itemize}
%		\item to create an \textbf{array of SetVariable} object:
%		\begin{itemize}
%			\item \mylst{makeSetVarArray(String name, int dim, int lowB, int uppB, String... options)}
%		\end{itemize}
%	\end{itemize}
%	\item \textbf{return type} : \texttt{SetVariable} \emph{or} \texttt{SetVariable[]}
\subsubsection{options:}
	\begin{itemize}
		\item \emph{no option} : equivalent to option \texttt{cp:enum}
		\item \texttt{cp:enum} : to force Solver to create \texttt{SetVariable} with enumerated domain for the caridinality variable. It is a domain in which holes can be created by the solver. It should be used when set variable cardinality domain is discrete, quite small and constraints performing reasonings on holes in the cardinality are present in the model. Implemeted by two \texttt{BitSets} for upper and lower bounds and an enumerated \texttt{IntegerVariable} for the cardinality.
		\item \texttt{cp:bound} : to force Solver to create \texttt{SetVariable} with bounded cardinality. It is a domain where only bound propagation can be done (no holes). It is very well suited when constraints performing only Bound Consistency are added on the corresponding variables. It must be used when large domains are needed. Implemented by two integers.
		\item \texttt{cp:decision} : to force variable to be a decisional one
		\item \texttt{cp:no\_decision} : to force variable to be removed from the pool of decisionnal variables
		\item \texttt{cp:objective} : to define the variable to be the one to optimize
	\end{itemize}

The variable representing the cardinality can be accessed and constrained using method \texttt{getCard()} that returns an \hyperlink{integervariable}{\tt IntegerVariable} object.

\subsubsection{Example:}
\begin{lstlisting}
  SetVariable svar1 = makeSetVar("svar1", -10, 10);
  setVariable svar2 = makeSetVar("svar2", 0, 10000, "cp:bound", "cp:no_decision");
\end{lstlisting} 

Set variables are illustrated on the \hyperlink{model:example2:ternarysteinerchoco}{ternary Steiner problem}. 




\part{Elements of Choco}\label{part:elements}\hypertarget{part:elements}{}
\chapter{Variables (Model)}\label{ch:vars}\hypertarget{ch:vars}{}
This section describes the three kinds of \hyperlink{model:variables}{variables} that can be used within a Choco Model.
\section{Integer variables}\label{integervariable}\hypertarget{integervariable}{}
\texttt{IntegerVariable} is a variable whose associated domain is made of integer values. 

\subsubsection{constructors:}
      \noindent\begin{tabular}{p{.8\linewidth}p{.15\linewidth}}
        Choco method & return type \\
        \hline
        \mylst{makeIntVar(String name, int lowB, int uppB, String... options)} &\texttt{IntegerVariable}\\
		\mylst{makeIntVar(String name, List<Integer> values, String... options)} &\texttt{IntegerVariable}\\
		\mylst{makeIntVar(String name, int[] values, String... options)} &\texttt{IntegerVariable}\\
        \mylst{makeBooleanVar(String name, String... options)}  &\texttt{IntegerVariable}\\
        \mylst{makeIntVarArray(String name, int dim, int lowB, int uppB, String... options)} &\texttt{IntegerVariable[]}\\
        \mylst{makeIntVarArray(String name, int dim, int[] values, String... options)} &\texttt{IntegerVariable[]}\\
        \mylst{makeBooleanVarArray(String name, int dim, String... options)}  &\texttt{IntegerVariable[]}\\
        \mylst{makeIntVarArray(String name, int dim1, int dim2, int lowB, int uppB, String... options)}  &\texttt{IntegerVariable[][]}\\
        \mylst{makeIntVarArray(String name, int dim1, int dim2, int[] values, String... options)}  &\texttt{IntegerVariable[][]}\\
      \end{tabular}
% 	\begin{itemize}
% 		\item to create an \textbf{IntegerVariable} object:
% 		\begin{itemize}
% 			\item \mylst{makeIntVar(String name, int lowB, int uppB, String... options)}
% 			\item \mylst{makeIntVar(String name, List<Integer> values, String... options)}
% 			\item \mylst{makeIntVar(String name, int[] values, String... options)}
% 		\end{itemize}
% 		\item to create an \textbf{array of IntegerVariable} object:
% 		\begin{itemize}
% 			\item \mylst{makeIntVarArray(String name, int dim, int lowB, int uppB, String... options)}
% 			\item \mylst{makeIntVarArray(String name, int dim, int[] values, String... options)}
% 		\end{itemize}
% 		\item to create a \textbf{matrix of IntegerVariable} object:
% 		\begin{itemize}
% 			\item \mylst{makeIntVarArray(String name, int dim1, int dim2, int lowB, int uppB, String... options)}
% 			\item \mylst{makeIntVarArray(String name, int dim1, int dim2, int[] values, String... options)}
% 		\end{itemize}
% 		\item to create an \textbf{IntegerVariable} object with pre defined domain [0,1]:
% 		\begin{itemize}
% 			\item \mylst{makeBooleanVar(String name, String... options)}
% 		\end{itemize}
% 		\item to create an \textbf{array of IntegerVariable} object with pre defined domain [0,1]:
% 		\begin{itemize}
% 			\item \mylst{makeBooleanVarArray(String name, int dim, String... options)}
% 		\end{itemize}
% 	\end{itemize}
% 	\item \textbf{return type} : \texttt{IntegerVariable} \emph{or} \texttt{IntegerVariable[]} \emph{or} \texttt{IntegerVariable[][]}
\subsubsection{options:}
	\begin{itemize}
		\item \emph{no option} : equivalent to option \texttt{cp:enum}
		\item \texttt{cp:enum} : to force Solver to create enumerated domain for the variable. It is a domain in which holes can be created by the solver. It should be used when discrete and quite small domains are needed and when constraints performing Arc Consistency are added on the corresponding variables. Implemented by a \texttt{BitSet} object.
		\item \texttt{cp:bound} : to force Solver to create bounded domain for the variable. It is a domain where only bound propagation can be done (no holes). It is very well suited when constraints performing only Bound Consistency are added on the corresponding variables. It must be used when large domains are needed. Implemented by two integers.
		\item \texttt{cp:binary} : to force Solver to create binary domain for the variable. It is a [0,1] domain. Some operations are improved and leads to an direct instantiation, if necessary. It must be used with two size domain. Implemented by a shared \texttt{BitSet} object, common to each 32 binary variables (to deal with 64 bits bitset).
		\item \texttt{cp:link} : to force Solver to create linked list domain for the variable. It is an enumerated domain where holes can be done and every values has a link to the previous value and to the next value. It is built by giving its name and its bounds: lower bound and upper bound. It must be used when the very small domains are needed, because although linked list domain consumes more memory than the \texttt{BitSet} implementation, it can provide good performance as iteration over the domain is made in constant time. Implemented by a \texttt{LinkedList} object.
		\item \texttt{cp:btree} : to force Solver to create binary tree domain for the variable. \emph{Under development}.
		\item \texttt{cp:blist} : to force Solver to create bipartite list domain for the variable. It is a domain where unavailable values are placed in the left part of the list, the other one on the right one.
		\item \texttt{cp:decision} : to force variable to be a decisional one
		\item \texttt{cp:no\_decision} : to force variable to be removed from the pool of decisional variables
		\item \texttt{cp:objective} : to define the variable to be the one to optimize
	\end{itemize}
\subsubsection{methods:}
      \begin{itemize}
      \item \mylst{removeVal(int val)}: remove value \emph{val} from the domain of the current variable
      \end{itemize}

A variable with $\{0,1\}$ domain is automatically considered as boolean domain.

\subsubsection{Example:}
\begin{lstlisting}
  IntegerVariable ivar1 = makeIntVar("ivar1", -10, 10);
  IntegerVariable ivar2 = makeIntVar("ivar2", 0, 10000, "cp:bound", "cp:decision");
  IntegerVariable bool = makeIntVar("bool", 0, 1, "cp:binary");
\end{lstlisting} 

\section{Real variables}\label{realvariable}\hypertarget{realvariable}{}
\texttt{RealVariable} is a variable whose associated domain is made of real values. Only enumerated domain is available for real variables. 

Such domain are memory consuming. In order to minimize the memory use and to have the precision you need, the model offers a way to set a precision (default value is 1.0e-6):
\begin{lstlisting}
	Model m = new CPModel();
	m.setPrecision(0.01);
\end{lstlisting}

\subsubsection{constructor:}
      \noindent\begin{tabular}{p{.8\linewidth}p{.15\linewidth}}
        Choco method & return type \\
        \hline
        \mylst{makeRealVar(String name, double lowB, double uppB, String... options)} &\texttt{RealVariable}\\
      \end{tabular}
%	\begin{itemize}
%		\item to create a \textbf{RealVariable} object:
%		\begin{itemize}
%			\item \mylst{makeRealVar(String name, double lowB, double uppB, String... options)}
%		\end{itemize}
%	\end{itemize}
%	\item \textbf{return type} : \texttt{RealVariable}
\subsubsection{options:}
	\begin{itemize}
		\item \emph{no option} : no particular choice on decision or objective.
		\item \texttt{CPOptions.V_DECISION} : to force variable to be a decisional one
		\item \texttt{cp:no\_decision} : to force variable to be removed from the pool of decisionnal variables
		\item \texttt{CPOptions.V_OBJECTIVE} : to define the variable to be the one to optimize
	\end{itemize}

\subsubsection{Example:}
\begin{lstlisting}
	RealVariable rvar1 = makeRealVar("rvar0", -10.0, 10.0);
	RealVariable rvar2 = makeRealVar("rvar2", 0.0, 100.0,CPOptions.V_DECISION, CPOptions.V_OBJECTIVE);
\end{lstlisting} 

\section{Set variables}\label{setvariable}\hypertarget{setvariable}{}
\texttt{SetVariable} is high level modeling tool. It allows to represent variable whose values are sets. A SetVariable on integer values between $[1,n]$ has $2*n$ values (every possible subsets of $\{1..n\}$). This makes an exponential number of values and the domain is represented with two bounds corresponding to the intersection of all possible sets (called the kernel) and the union of all possible sets (called the envelope) which are the possible candidate values for the variable. The consistency achieved on SetVariables is therefore a kind of bound consistency.

\subsubsection{constructors:}
      \noindent\begin{tabular}{p{.8\linewidth}p{.15\linewidth}}
        Choco method & return type \\
        \hline
        \mylst{makeSetVar(String name, int lowB, int uppB, String... options)} &\texttt{SetVariable}\\
        \mylst{makeSetVarArray(String name, int dim, int lowB, int uppB, String... options)} &\texttt{SetVariable[]}
      \end{tabular}
%	\begin{itemize}
%		\item to create an \textbf{SetVariable} object:
%		\begin{itemize}
%			\item \mylst{makeSetVar(String name, int lowB, int uppB, String... options)}
%		\end{itemize}
%		\item to create an \textbf{array of SetVariable} object:
%		\begin{itemize}
%			\item \mylst{makeSetVarArray(String name, int dim, int lowB, int uppB, String... options)}
%		\end{itemize}
%	\end{itemize}
%	\item \textbf{return type} : \texttt{SetVariable} \emph{or} \texttt{SetVariable[]}
\subsubsection{options:}
	\begin{itemize}
		\item \emph{no option} : equivalent to option \texttt{cp:enum}
		\item \texttt{cp:enum} : to force Solver to create \texttt{SetVariable} with enumerated domain for the caridinality variable. It is a domain in which holes can be created by the solver. It should be used when set variable cardinality domain is discrete, quite small and constraints performing reasonings on holes in the cardinality are present in the model. Implemeted by two \texttt{BitSets} for upper and lower bounds and an enumerated \texttt{IntegerVariable} for the cardinality.
		\item \texttt{cp:bound} : to force Solver to create \texttt{SetVariable} with bounded cardinality. It is a domain where only bound propagation can be done (no holes). It is very well suited when constraints performing only Bound Consistency are added on the corresponding variables. It must be used when large domains are needed. Implemented by two integers.
		\item \texttt{cp:decision} : to force variable to be a decisional one
		\item \texttt{cp:no\_decision} : to force variable to be removed from the pool of decisionnal variables
		\item \texttt{cp:objective} : to define the variable to be the one to optimize
	\end{itemize}

The variable representing the cardinality can be accessed and constrained using method \texttt{getCard()} that returns an \hyperlink{integervariable}{\tt IntegerVariable} object.

\subsubsection{Example:}
\begin{lstlisting}
  SetVariable svar1 = makeSetVar("svar1", -10, 10);
  setVariable svar2 = makeSetVar("svar2", 0, 10000, "cp:bound", "cp:no_decision");
\end{lstlisting} 

Set variables are illustrated on the \hyperlink{model:example2:ternarysteinerchoco}{ternary Steiner problem}. 



\chapter{Operators}\label{ch:operators}\hypertarget{ch:operators}{}
This section lists and details the \hyperlink{model:expressionvariables}{operators} that can be used within a Choco Model to combine variables in expressions.
\section{abs (operator)}\label{abs:absoperator}\hypertarget{abs:absoperator}{}
Returns an expression variable that represents the absolute value of the argument (\(|n|\)).

\begin{itemize}
	\item \textbf{API} : abs(IntegerExpressionVariable n)
	\item \textbf{return type} : IntegerExpressionVariable
	\item \textbf{options} : \emph{n/a}
	\item \textbf{favorite domain} : unknown
\end{itemize}

\textbf{Example}:
\begin{lstlisting}
	CPModel m = new CPModel();
	IntegerVariable x = makeIntVar("x", 1, 5, "cp:enum");
	IntegerVariable y = makeIntVar("y", -5, 5, "cp:enum");
	m.addConstraint(eq(abs(x), y));
	CPSolver s = new CPSolver();
	s.read(m);
	s.solve();
	Assert.assertEquals(s.getVar(x).getVal(),Math.abs(s.getVar(y).getVal()));
\end{lstlisting}

%%% Local Variables: 
%%% mode: latex
%%% TeX-master: t
%%% End: 

%\part{cos}
\label{cos}
\hypertarget{cos}{}

\section{cos (operator)}\label{cos:cosoperator}\hypertarget{cos:cosoperator}{}
Returns an expression variable corresponding to the cosinus value of the argument (\(cos(x)\)).

\begin{itemize}
	\item \textbf{API} : \mylst{cos(RealExpressionVariable exp)}
	\item \textbf{return type} : RealExpressionVariable
	\item \textbf{options} : \emph{n/a}
	\item \textbf{favorite domain} : real
\end{itemize}

\textbf{Example}:
\lstinputlisting{java/ocos.j2t}

%\part{div}
\label{div}
\hypertarget{div}{}

\section{div (operator)}\label{div:divoperator}\hypertarget{div:divoperator}{}
Returns an expression variable that represents the \(integer\) \(quotient\) of the division of the first argument variable by the second one (\(n_1/n_2\)).

\begin{itemize}
	\item \textbf{API} :
	\begin{itemize}
		\item \mylst{div(IntegerExpressionVariable n1, IntegerExpressionVariable n2)}
		\item \mylst{div(IntegerExpressionVariable n1, int n2)}
		\item \mylst{div(int n1, IntegerExpressionVariable n2)}
	\end{itemize}
	\item \textbf{return type} : IntegerExpressionVariable
	\item \textbf{options} : \emph{n/a}
	\item \textbf{favorite domain} : \emph{n/a}
\end{itemize}

\textbf{Example}:
\lstinputlisting{java/odiv.j2t}

\section{FALSE (operator)}\label{false:falseoperator}\hypertarget{false:falseoperator}{}
Returns an expression always equals to \emph{false}.

%%% Local Variables: 
%%% mode: latex
%%% TeX-master: t
%%% End: 


\section{ifThenElse (operator)}\label{ifthenelse:ifthenelseoperator}\hypertarget{ifthenelse:ifthenelseoperator}{}
\texttt{ifThenElse}$(c,v_1,v_2)$ states that if the constraint $c$ is satisfied, it returns the second parameter $v_1$, otherwise it returns the third one $v_2$.

\begin{itemize}
	\item \textbf{API} : \mylst{ifThenElse(Constraint c, IntegerExpressionVariable v1, IntegerExpressionVariable v2)}
	\item \textbf{return type} : IntegerExpressionVariable
	\item \textbf{options} : \emph{n/a}
	\item \textbf{favorite domain} : unknown
\end{itemize}

\textbf{Example}:
\lstinputlisting{java/oifthenelse.j2t}

\section{max (operator)}\label{max:maxoperator}\hypertarget{max:maxoperator}{}
Returns an expression variable equals to the greater value of the argument (\(max(x_1, x_2, ..., x_n)\)).

\begin{itemize}
	\item \textbf{API} :
	\begin{itemize}
		\item max(IntegerExpressionVariable x1, IntegerExpressionVariable x2)
		\item max(int x1, IntegerExpressionVariable x2)
		\item max(IntegerExpressionVariable x1, int x2)
		\item max(IntegerExpressionVariable[] x)
	\end{itemize}
	\item \textbf{return type}: IntegerExpressionVariable
	\item \textbf{options} : \emph{n/a}
	\item \textbf{favorite domain} : \emph{to complete}
\end{itemize}

\textbf{Example}:
\begin{lstlisting}
	Model m = new CPModel();
	m.setDefaultExpressionDecomposition(true);
	IntegerVariable[] v = makeIntVarArray("v", 3, -3, 3);
	IntegerVariable minv = makeIntVar("min", -3, 3);
	Constraint c = eq(minv, max(v));
	m.addConstraint(c);
	Solver s = new CPSolver();
	s.read(m);
	s.solveAll();
\end{lstlisting}

%%% Local Variables: 
%%% mode: latex
%%% TeX-master: t
%%% End: 


\section{min (operator)}\label{min:minoperator}\hypertarget{min:minoperator}{}
Returns an expression variable equals to the smaller value of the argument (\(min(x_1, x_2, ..., x_n)\)).

\begin{itemize}
	\item \textbf{API} :
	\begin{itemize}
		\item \mylst{min(IntegerExpressionVariable x1, IntegerExpressionVariable x2)}
		\item \mylst{min(int x1, IntegerExpressionVariable x2)}
		\item \mylst{min(IntegerExpressionVariable x1, int x2)}
		\item \mylst{min(IntegerExpressionVariable[] x)}
	\end{itemize}
	\item \textbf{return type}: IntegerExpressionVariable
	\item \textbf{options} : \emph{n/a}
	\item \textbf{favorite domain} : \emph{to complete}
\end{itemize}

\textbf{Example}:
\lstinputlisting{java/omin.j2t}

%%% Local Variables: 
%%% mode: latex
%%% TeX-master: t
%%% End: 

%\part{minus}
\label{minus}
\hypertarget{minus}{}

\section{minus (operator)}\label{minus:minusoperator}\hypertarget{minus:minusoperator}{}
Returns an expression variable that corresponding to the difference between the two arguments (\(x-y\)).

\begin{itemize}
	\item \textbf{API} :
	\begin{itemize}
		\item \mylst{minus(IntegerExpressionVariable x, IntegerExpressionVariable y)}
		\item \mylst{minus(IntegerExpressionVariable x, int y)}
		\item \mylst{minus(int x, IntegerExpressionVariable y)}
		\item \mylst{minus(RealExpressionVariable x, RealExpressionVariable y)}
		\item \mylst{minus(RealExpressionVariable x, double y)}
		\item \mylst{minus(double x, RealExpressionVariable y)}
	\end{itemize}
	\item \textbf{return type} :
	\begin{itemize}
		\item \texttt{IntegerExpressionVariable}, if parameters are \texttt{IntegerExpressionVariable}
		\item \texttt{RealExpressionVariable}, if parameters are \texttt{RealExpressionVariable}
	\end{itemize}
	\item \textbf{options} : \emph{n/a}
	\item \textbf{favorite domain} : \emph{to complete}
\end{itemize}

\textbf{Example}
\begin{lstlisting}
	Model m = new CPModel();
	Solver s = new CPSolver();
	IntegerVariable a = makeIntVar("a", 0, 4);
	m.addConstraint(eq(minus(a, 1), 2));
	s.read(m);
	s.solve();
\end{lstlisting}


\section{mod (operator)}\label{mod:modoperator}\hypertarget{mod:modoperator}{}
Returns an expression variable that represents the integer remainder of the division of the first argument variable by the second one (\(x_1\%x_2\)).

\begin{itemize}
	\item \textbf{API}:
	\begin{itemize}
		\item mod(IntegerExpressionVariable x1, IntegerExpressionVariable x2)
		\item mod(int x1, IntegerExpressionVariable x2)
		\item mod(IntegerExpressionVariable x1, int x2)
	\end{itemize}
	\item \textbf{return type} : \texttt{IntegerExpressionVariable}
	\item \textbf{options} : \emph{n/a}
	\item \textbf{favorite domain} : \emph{n/a}
\end{itemize}

\textbf{Example}:
\lstinputlisting{java/omod.j2t}

%%% Local Variables: 
%%% mode: latex
%%% TeX-master: t
%%% End: 

%\part{mult}
\label{mult}
\hypertarget{mult}{}

\section{mult (operator)}\label{mult:multoperator}\hypertarget{mult:multoperator}{}
Returns an expression variable that corresponding to the product of variables in argument (\(x*y\)).

\begin{itemize}
	\item \textbf{API} :
	\begin{itemize}
		\item mult(IntegerExpressionVariable x, IntegerExpressionVariable y)
		\item mult(IntegerExpressionVariable x, int y)
		\item mult(int x, IntegerExpressionVariable y)
		\item mult(RealExpressionVariable x, RealExpressionVariable y)
		\item mult(RealExpressionVariable x, double y)
		\item mult(double x, RealExpressionVariable y)
	\end{itemize}
	\item \textbf{return type} :
	\begin{itemize}
		\item \texttt{IntegerExpressionVariable}, if parameters are \texttt{IntegerExpressionVariable}
		\item \texttt{RealExpressionVariable}, if parameters are \texttt{RealExpressionVariable}
	\end{itemize}
	\item \textbf{options} : \emph{n/a}
	\item \textbf{favorite domain} : \emph{to complete}
\end{itemize}

\textbf{Example}
\begin{lstlisting}
	CPModel m = new CPModel();
	IntegerVariable x = makeIntVar("x", -10, 10);
	IntegerVariable z = makeIntVar("z", -10, 10);
	IntegerVariable w = makeIntVar("w", -10, 10);
	m.addVariable(x, z, w);
	
	CPSolver s = new CPSolver();
	// x >= z * w
	Constraint exp = geq(x, mult(z,w));
	
	m.setDefaultExpressionDecomposition(true);
	m.addConstraint(exp);
	
	s.read(m);
	s.solveAll();
\end{lstlisting}

%\part{neg}
\label{neg}
\hypertarget{neg}{}

\section{neg (operator)}\label{neg:negoperator}\hypertarget{neg:negoperator}{}

Returns an expression variable that is the opposite of the expression integer variable in argument (\(-x\)).

\begin{itemize}
	\item \textbf{API} : neg(IntegerExpressionVariable x)
	\item \textbf{return type} : \texttt{IntegerExpressionVariable}
	\item \textbf{options} : \emph{n/a}
	\item \textbf{favorite domain} : \emph{n/a}
\end{itemize}

\textbf{Example}:
\begin{lstlisting}
	Model m = new CPModel();
	Solver s = new CPSolver();
	
	IntegerVariable x = makeIntVar("x", -10, 10);
	IntegerVariable w = makeIntVar("w", -10, 10);
	// -x = w - 20
	m.addConstraint(eq(neg(x), minus(w, 20)));
	
	s.read(m);
	s.solve();
\end{lstlisting}

%\part{plus}
\label{plus}
\hypertarget{plus}{}



\section{plus (operator)}\label{plus:plusoperator}\hypertarget{plus:plusoperator}{}
Returns an expression variable that corresponding to the sum of the two arguments (\(x+y\)).

\begin{itemize}
	\item \textbf{API} :
	\begin{itemize}
		\item \mylst{plus(IntegerExpressionVariable x, IntegerExpressionVariable y)}
		\item \mylst{plus(IntegerExpressionVariable x, int y)}
		\item \mylst{plus(int x, IntegerExpressionVariable y)}
		\item \mylst{plus(RealExpressionVariable x, RealExpressionVariable y)}
		\item \mylst{plus(RealExpressionVariable x, double y)}
		\item \mylst{plus(double x, RealExpressionVariable y)}
	\end{itemize}
	\item \textbf{return type} :
	\begin{itemize}
		\item \texttt{IntegerExpressionVariable}, if parameters are \texttt{IntegerExpressionVariable}
		\item \texttt{RealExpressionVariable}, if parameters are \texttt{RealExpressionVariable}
	\end{itemize}
	\item \textbf{options} : \emph{n/a}
	\item \textbf{favorite domain} : \emph{to complete}
\end{itemize}

\textbf{Example}
% \begin{itemize}
% 	\item example1:
% \end{itemize}

\lstinputlisting{java/oplus.j2t}


%\part{power}
\label{power}
\hypertarget{power}{}

\section{power (operator)}\label{power:poweroperator}\hypertarget{power:poweroperator}{}
Returns an expression variable that represents the first argument raised to the power of the second argument (\(x^y\)).

\begin{itemize}
	\item \textbf{API} :
	\begin{itemize}
		\item \mylst{power(IntegerExpressionVariable x, IntegerExpressionVariable y)}
		\item \mylst{power(int x, IntegerExpressionVariable y)}
		\item \mylst{power(IntegerExpressionVariable x, int y)}
		\item \mylst{power(RealExpressionVariable x, int y)}
	\end{itemize}
	\item \textbf{return type}:
	\begin{itemize}
		\item \texttt{IntegerExpressionVariable}, if parameters are \texttt{IntegerExpressionVariable}
		\item \texttt{RealExpressionVariable}, if parameters are \texttt{RealExpressionVariable}
	\end{itemize}
	\item \textbf{option} : \emph{n/a}
	\item \textbf{favorite domain} : \emph{to complete}
\end{itemize}

\textbf{Example} : 
\lstinputlisting{java/opower.j2t}

%\part{scalar}
\label{scalar}
\hypertarget{scalar}{}

\section{scalar (operator)}\label{scalar:scalaroperator}\hypertarget{scalar:scalaroperator}{}
Return an integer expression that corresponds to the scalar product of coefficients array and variables array (\(c_1*x_1+c_2*x_2+...+c_n*x_n\)).

\begin{itemize}
	\item \textbf{API} :
	\begin{itemize}
		\item scalar(int[] c, IntegerVariable[] x)
		\item scalar(IntegerVariable[] x, int[] c)
	\end{itemize}
	\item \textbf{return type} : IntegerExpressionVariable
	\item \textbf{options} : \emph{n/a}
	\item \textbf{favorite domain} : \emph{to complete}
\end{itemize}

\textbf{Example}:

\lstinputlisting{java/oscalar.j2t}

%\part{sin}
\label{sin}
\hypertarget{sin}{}

\section{sin (operator)}\label{sin:sinoperator}\hypertarget{sin:sinoperator}{}
Returns a real variable that corresponding to the sinus value of the argument (\(sin(x)\)).

\begin{itemize}
	\item \textbf{API} : sin(RealExpressionVariable exp)
	\item \textbf{return type} : RealExpressionVariable
	\item \textbf{options} : \emph{n/a}
	\item \textbf{favorite domain} : real
\end{itemize}

\textbf{Example}:

\emph{No valid example for the moment}

%\part{sum}
\label{sum}
\hypertarget{sum}{}

\section{sum (operator)}\label{sum:sumoperator}\hypertarget{sum:sumoperator}{}
Return an integer expression that corresponds to the sum of the variables given in argument (\(x_1+x_2+...+x_n\)).

\begin{itemize}
	\item \textbf{API}: sum(IntegerVariable... lv)
	\item \textbf{return type} : IntegerExpressionVariable
	\item \textbf{options} : \emph{n/a}
	\item \textbf{favorite domain} : \emph{to complete}
\end{itemize}

\textbf{Example} :
\lstinputlisting{java/osum.j2t}


\section{TRUE (operator)}\label{true:trueoperator}\hypertarget{true:trueoperator}{}
Returns an expression always equals to \emph{true}.

%%% Local Variables: 
%%% mode: latex
%%% TeX-master: t
%%% End: 

\chapter{Constraints}\label{ch:constraints}\hypertarget{ch:constraints}{}
This section lists and details the \hyperlink{model:constraints}{constraints} currently available in Choco.
%\part{abs}
%\label{abs}\hypertarget{abs}{}
\section{abs (constraint)}\label{abs:absconstraint}\hypertarget{abs:absconstraint}{}
\begin{notedef}
  \texttt{abs}$(x,y)$ states that $x$ is the absolute value of $y$:
$$x = |y|$$
\end{notedef}

\begin{itemize}
	\item \textbf{API} : \mylst{abs(IntegerVariable x, IntegerVariable y)}
	\item \textbf{return type} : \texttt{Constraint}
	\item \textbf{options} : \emph{n/a}
	\item \textbf{favorite domain} : enumerated
\end{itemize}

\textbf{Example}:
\begin{lstlisting}
	CPModel m = new CPModel();
	IntegerVariable x = makeIntVar("x", 1, 5, "cp:enum");
	IntegerVariable y = makeIntVar("y", -5, 5, "cp:enum");
	m.addConstraint(abs(x,y));
	CPSolver s = new CPSolver();
	s.read(m);
	s.solve();
	Assert.assertEquals(s.getVar(x).getVal(),Math.abs(s.getVar(y).getVal()));
\end{lstlisting}


%\part{alldifferent}
\label{alldifferent}
\hypertarget{alldifferent}{}

\section{allDifferent (constraint)}\label{alldifferent:alldifferentconstraint}\hypertarget{alldifferent:alldifferentconstraint}{}
\begin{notedef}
  \texttt{allDifferent}$(x_1,\ldots,x_n)$ states that the arguments have pairwise distinct values:
 $$x_i \neq x_j,\quad \forall\ i\neq j$$  
\end{notedef}
This constraint is useful for some matching problems.
Notice that the filtering algorithm used will depend on the nature (enumerated or bounded) of the variables: 
when \emph{enumerated}, the constraint refers to the alldifferent of \cite{ReginAAAI94};
when \emph{bounded}, a dedicated algorithm for bound propagation is used \cite{LopezIJCAI03}.

\begin{itemize}
	\item \textbf{API} :
	\begin{itemize}
		\item \mylst{allDifferent(IntegerVariable... x)}
		\item \mylst{allDifferent(String options, IntegerVariable... x)}
	\end{itemize}
	\item \textbf{return type} : \texttt{Constraint}
	\item \textbf{options} :
	\begin{itemize}
		\item \emph{no option} clever choice made on domains of given variables
		\item \texttt{CPOptions.C_ALLDIFFERENT_AC} for \cite{ReginAAAI94} implementation of arc consistency
		\item \texttt{CPOptions.C_ALLDIFFERENT_BC} for \cite{LopezIJCAI03} implementation of bound consistency
		\item \texttt{CPOptions.C_ALLDIFFERENT_CLIQUE} for propagating the clique of differences
	\end{itemize}
	\item \textbf{favorite domain} : depending of options.
	\item \textbf{references} :
      \begin{itemize}
      \item  \cite{ReginAAAI94}: \emph{A filtering algorithm for constraints of difference in CSPs}
      \item  \cite{LopezIJCAI03}: \emph{A fast and simple algorithm for bounds consistency of the alldifferent constraint}
      \item global constraint catalog: \href{http://www.emn.fr/x-info/sdemasse/gccat/Calldifferent.html}{\tt alldifferent}
      \end{itemize}
\end{itemize}



\textbf{Example}:
\lstinputlisting{java/calldifferent.j2t}

%\part{and}
\label{and}
\hypertarget{and}{}

\section{and (constraint)}\label{and:andconstraint}\hypertarget{and:andconstraint}{}
\begin{notedef}
  \texttt{and}$(c_1,\ldots,c_n)$ states that every constraints in arguments are satisfied:
$$ c_1 \land c_2 \land\ldots\land c_n$$
\end{notedef}

\begin{itemize}
\item \textbf{API} : \mylst{and(Constraint... c)}
\item \textbf{return type} : \texttt{Constraint}
\item \textbf{options} : \emph{n/a}
\item \textbf{favorite domain} : \emph{n/a}
\item \textbf{references} :\\
  global constraint catalog: \href{http://www.emn.fr/x-info/sdemasse/gccat/Cand.html}{\tt and}
\end{itemize}

\textbf{Example}:
\lstinputlisting{java/cand.j2t}

%\part{atmostnvalue}
\label{atmostnvalue}
\hypertarget{atmostnvalue}{}

\section{atMostNValue (constraint)}\label{atmostnvalue:atmostnvalueconstraint}\hypertarget{atmostnvalue:atmostnvalueconstraint}{}
\begin{notedef}
\texttt{atMostNValue}$(x,z)$ states that the number of different values occurring in the array of variables $x$ is at most \emph{z}:
$$z\ge|\{x_1,\ldots,x_n\}|$$  
\end{notedef}

\begin{itemize}
	\item \textbf{API} : \mylst{atMostNValue(IntegerVariable[] x, IntegerVariable z)}
	\item \textbf{return type} : \texttt{Constraint}
	\item \textbf{options} : \emph{n/a}
	\item \textbf{favorite domain} : \emph{n/a}
	\item \textbf{references} :
      \begin{itemize}
      \item  \cite{BessiereCPAIOR05} \emph{Filtering algorithms for the NValue constraint}
      \item global constraint catalog: \href{http://www.emn.fr/x-info/sdemasse/gccat/Catmost_nvalue.html}{\tt atmost\_nvalue}
      \end{itemize}
    \end{itemize}

\textbf{Example}:
\lstinputlisting{java/catmostnvalue.j2t}

%\part{boolchanneling}
\label{boolchanneling}
\hypertarget{boolchanneling}{}

\section{boolChanneling (constraint)}\label{boolchanneling:boolchannelingconstraint}\hypertarget{boolchanneling:boolchannelingconstraint}{}
\begin{notedef}  
\texttt{boolChanneling}$(b,x,v)$ states that boolean $b$ is true if and only if $x$ has value $v$:
$$(b=1)\quad\iff\quad (x=v)$$ 
\end{notedef}

$b$ is an indicator variable acting as an observer of value $v$. 
See also \hyperlink{domainchanneling}{\texttt{domainChanneling}} for observing all the values of $x$.
%Imagine a bin packing problem where variable $x$ tells you on which a given bin object is placed. By stating the boolean channeling, $b$ is true if and only if the object is placed on bin $v$, the knapsack constraint for bin $v$ can then be easily stated as a scalar of the boolean variables.
\begin{itemize}
	\item \textbf{API} : \mylst{boolChanneling(IntegerVariable b, IntegerVariable x, int v)}
	\item \textbf{return type} : \texttt{Constraint}
	\item \textbf{options} : \emph{n/a}
	\item \textbf{favorite domain} : enumerated for $x$
\end{itemize}

\textbf{Example}:
\lstinputlisting{java/cboolchanneling.j2t}

%\input{chapters/Ccos.tex}
%\part{cumulative}
\label{cumulative}
\hypertarget{cumulative}{}

\section{cumulative (constraint)}\label{cumulative:cumulativeconstraint}\hypertarget{cumulative:cumulativeconstraint}{}
\todo{to be cleaned.}

\begin{notedef}
  \texttt{cumulative(start,duration,height,capacity)} states that a set of tasks (defined by their starting times, finishing dates, durations and heights (or consumptions)) are executed on a cumulative resource of limited capacity. That is, the total height of the tasks which are executed at any time $t$ does not exceed the capacity of the resource:
$$\sum_{\{i\ |\ \mathtt{start}[i]\le t < \mathtt{start}[i]+\mathtt{duration}[i]\}} \mathtt{height}[i] \le \mathtt{capacity},\quad (\forall \text{ time } t)$$
\end{notedef}

The notion of task does not exist yet in Choco. The \texttt{cumulative} takes therefore as input, several arrays of integer variables (of same size $n$) denoting the starting, duration, and height of each task. When the array of finishing times is also specified, the constraint ensures that \texttt{start[i] + duration[i] = end[i]} for all task $i$.
As usual, a task is executed in the interval \texttt{[start,end-1]}.

For further informations, see the section devoted to this constraint in the Choco Tutorial document. 
%A tutorial on the use of this constraint is available \hyperlink{schedulinganduseofthecumulative:schedulinganduseofthecumulativeconstraint}{here}

\begin{itemize}
	\item \textbf{API} :
	\begin{itemize}
		\item \mylst{cumulative(IntegerVariable[] start, IntegerVariable[] end, IntegerVariable[] duration, IntegerVariable[] height, IntegerVariable capa, String... options)}
		\item \mylst{cumulative(IntegerVariable[] start, IntegerVariable[] end, IntegerVariable[] duration, int[] height, int capa, String... options)}
		\item \mylst{cumulative(IntegerVariable[] start, IntegerVariable[] duration, IntegerVariable[] height, IntegerVariable capa, String... options)}
	\end{itemize}
	\item \textbf{return type} : \texttt{Constraint}
	\item \textbf{options} :
	\begin{itemize}
		\item \emph{no option}
		\item \hyperlink{ccumulativeti:ccumulativetioptions}{SettingType.TASK\_INTERVAL.getOptionName()} for fast task intervals
		\item \hyperlink{ccumulativesti:ccumulativestioptions}{SettingType.SLOW\_TASK\_INTERVAL.getOptionName()} for slow task intervals
		\item \hyperlink{ccumulativecef:ccumulativecefoptions}{SettingType.VILIM\_CEF\_ALGO.getOptionName()} for Vilim theta lambda tree + lazy computation of the inner maximization of the edge finding rule of Van hentenrick and Mercier
		\item \hyperlink{ccumulativescef:ccumulativescefoptions}{SettingType.VHM\_CEF\_ALGO\_N2K.getOptionName()} for Simple $n^2 * k$ algorithm (lazy for R) (CalcEF -- Van Hentenrick)
	\end{itemize}
	\item \textbf{favorite domain} : \emph{n/a}
	\item \textbf{references} :
      \begin{itemize}
      \item  \cite{BeldiceanuCP02} \emph{A new multi-resource cumulatives constraint with negative heights}
      \item global constraint catalog: \href{http://www.emn.fr/x-info/sdemasse/gccat/Ccumulative.html}{\tt cumulative}
      \end{itemize}
\end{itemize}

\textbf{Example}:
\lstinputlisting{java/ccumulative.j2t} 

%\part{disjunctive}
\label{disjunctive}
\hypertarget{disjunctive}{}

\section{disjunctive (constraint)}\label{disjunctive:disjunctiveconstraint}\hypertarget{disjunctive:disjunctiveconstraint}{}

\begin{notedef}
  \texttt{disjunctive(start,duration)} states that a set of tasks (defined by their starting times and durations) are executed on a ddisjunctive resource, i.e. they do not overlap in time:
$$|\{i\ |\ \mathtt{start}[i]\le t < \mathtt{start}[i]+\mathtt{duration}[i]\}| \le 1,\quad (\forall \text{ time } t)$$
\end{notedef}

The notion of task does not exist yet in Choco. The \texttt{disjunctive} takes therefore as input arrays of integer variables (of same size $n$) denoting the starting and duration of each task. When the array of finishing times is also specified, the constraint ensures that \texttt{start[i] + duration[i] = end[i]} for all task $i$.
As usual, a task is executed in the interval \texttt{[start,end-1]}.

\begin{itemize}
	\item \textbf{API} :
	\begin{itemize}
		\item \mylst{disjunctive(IntegerVariable[] start, int[] duration, String...options)}
		\item \mylst{disjunctive(IntegerVariable[] start, IntegerVariable[] duration, String... options)}
		\item \mylst{disjunctive(IntegerVariable[] start, IntegerVariable[] end, IntegerVariable[] duration, String... options)}
		\item \mylst{disjunctive(IntegerVariable[] start, IntegerVariable[] end, IntegerVariable[] duration, IntegerVariable uppBound, String... options)}
	\end{itemize}
	\item \textbf{return type} : \texttt{Constraint}
	\item \textbf{options} :\emph{n/a}
	\item \textbf{favorite domain} : \emph{to complete}
	\item \textbf{references} :\\
      global constraint catalog: \href{http://www.emn.fr/x-info/sdemasse/gccat/Cdisjunctive.html}{\tt disjunctive}
\end{itemize}

\textbf{Example}:
%\lstinputlisting{java/cdisjunctive.j2t}
\mylst{//TODO: complete} 

%\part{distanceeq}
\label{distanceeq}
\hypertarget{distanceeq}{}

\section{distanceEQ (constraint)}\label{distanceeq:distanceeqconstraint}\hypertarget{distanceeq:distanceeqconstraint}{}
\begin{notedef}
  \texttt{distanceEQ}$(x_1,x_2,x_3,c)$ states that $x_3$ plus an offset $c$ (by default $c=0$) is equal to the distance between $x_1$ and $x_2$:
$$ x_3 + c = | x_1 - x_2 |$$
\end{notedef}

\begin{itemize}
	\item \textbf{API} :
	\begin{itemize}
		\item \mylst{distanceEQ(IntegerVariable x1, IntegerVariable x2, int x3)}
		\item \mylst{distanceEQ(IntegerVariable x1, IntegerVariable x2, IntegerVariable x3)}
		\item \mylst{distanceEQ(IntegerVariable x1, IntegerVariable x2, IntegerVariable x3, int c)}
	\end{itemize}
	\item \textbf{return type}: \texttt{Constraint}
	\item \textbf{options} : \emph{n/a}
	\item \textbf{favorite domain} : \emph{to complete}
	\item \textbf{references} :\\
      global constraint catalog: \href{http://www.emn.fr/x-info/sdemasse/gccat/Call_min_dist.html}{\tt all\_min\_dist} (variant)
\end{itemize}

\textbf{Example}:
\lstinputlisting{java/cdistanceeq.j2t}

%\part{distancegt}
\label{distancegt}
\hypertarget{distancegt}{}

\section{distanceGT (constraint)}\label{distancegt:distancegtconstraint}\hypertarget{distancegt:distancegtconstraint}{}
\begin{notedef}
  \texttt{distanceGT}$(x_1,x_2,x_3,c)$ states that $x_3$ plus an offset $c$ (by default $c=0$) is strictly greater than the distance between $x_1$ and $x_2$:
$$ x_3 + c > | x_1 - x_2 |$$
\end{notedef}

\begin{itemize}
	\item \textbf{API} :
	\begin{itemize}
		\item \mylst{distanceGT(IntegerVariable x1, IntegerVariable x2, int x3)}
		\item \mylst{distanceGT(IntegerVariable x1, IntegerVariable x2, IntegerVariable x3)}
		\item \mylst{distanceGT(IntegerVariable x1, IntegerVariable x2, IntegerVariable x3, int c)}
	\end{itemize}
	\item \textbf{return type}: \texttt{Constraint}
	\item \textbf{options} : \emph{n/a}
	\item \textbf{favorite domain} : \emph{to complete}
	\item \textbf{references} :\\
      global constraint catalog: \href{http://www.emn.fr/x-info/sdemasse/gccat/Call_min_dist.html}{\tt all\_min\_dist} (variant)
\end{itemize}

\textbf{Example}:
\begin{lstlisting}
	Model m = new CPModel();
	Solver s = new CPSolver();
	IntegerVariable v0 = makeIntVar("v0", 0, 5);
	IntegerVariable v1 = makeIntVar("v1", 0, 5);
	IntegerVariable v2 = makeIntVar("v2", 0, 5);
	m.addConstraint(distanceGT(v0, v1, v2, 0));
	s.read(m);
	s.solveAll();
\end{lstlisting}

%\part{distancelt}
\label{distancelt}
\hypertarget{distancelt}{}

\section{distanceLT (constraint)}\label{distancelt:distanceltconstraint}\hypertarget{distancelt:distanceltconstraint}{}
\begin{notedef}
  \texttt{distanceLT}$(x_1,x_2,x_3,c)$ states that $x_3$ plus an offset $c$ (by default $c=0$) is strictly smaller than the distance between $x_1$ and $x_2$:
$$ x_3 + c < | x_1 - x_2 |$$
\end{notedef}

\begin{itemize}
	\item \textbf{API} :
	\begin{itemize}
		\item \mylst{distanceLT(IntegerVariable x1, IntegerVariable x2, int x3)}
		\item \mylst{distanceLT(IntegerVariable x1, IntegerVariable x2, IntegerVariable x3)}
		\item \mylst{distanceLT(IntegerVariable x1, IntegerVariable x2, IntegerVariable x3, int c)}
	\end{itemize}
	\item \textbf{return type}: \texttt{Constraint}
	\item \textbf{options} : \emph{n/a}
	\item \textbf{favorite domain} : \emph{to complete}
\end{itemize}

\textbf{Example}:
\begin{lstlisting}
	Model m = new CPModel();
	Solver s = new CPSolver();
	IntegerVariable v0 = makeIntVar("v0", 0, 5);
	IntegerVariable v1 = makeIntVar("v1", 0, 5);
	IntegerVariable v2 = makeIntVar("v2", 0, 5);
	m.addConstraint(distanceLT(v0, v1, v2, 0));
	s.read(m);
	s.solveAll();
\end{lstlisting}

%\part{distanceneq}
\label{distanceneq}
\hypertarget{distanceneq}{}

\section{distanceNEQ (constraint)}\label{distanceneq:distanceneqconstraint}\hypertarget{distanceneq:distanceneqconstraint}{}
\begin{notedef}
  \texttt{distanceNEQ}$(x_1,x_2,x_3,c)$ states that $x_3$ plus an offset $c$ (by default $c=0$) is not equal to the distance between $x_1$ and $x_2$:
$$ x_3 + c \neq | x_1 - x_2 |$$
\end{notedef}

\begin{itemize}
	\item \textbf{API} :
	\begin{itemize}
		\item \mylst{distanceNEQ(IntegerVariable x1, IntegerVariable x2, int x3)}
		\item \mylst{distanceNEQ(IntegerVariable x1, IntegerVariable x2, IntegerVariable x3)}
		\item \mylst{distanceNEQ(IntegerVariable x1, IntegerVariable x2, IntegerVariable x3, int c)}
	\end{itemize}
	\item \textbf{return type}: \texttt{Constraint}
	\item \textbf{options} : \emph{n/a}
	\item \textbf{favorite domain} : \emph{to complete}
	\item \textbf{references} :\\
      global constraint catalog: \href{http://www.emn.fr/x-info/sdemasse/gccat/Call_min_dist.html}{\tt all\_min\_dist} (variant)
\end{itemize}

\textbf{Example}:
\begin{lstlisting}
	Model m = new CPModel();
	Solver s = new CPSolver();
	IntegerVariable v0 = makeIntVar("v0", 0, 5);
	IntegerVariable v1 = makeIntVar("v1", 0, 5);
	IntegerVariable v2 = makeIntVar("v2", 0, 5);
	m.addConstraint(distanceNEQ(v0, v1, v2, 0));
	s.read(m);
	s.solveAll();
\end{lstlisting}

%\input{chapters/Cdiv.tex}
%\part{domainchanneling}
\label{domainchanneling}
\hypertarget{domainchanneling}{}

\section{domainChanneling (constraint)}\label{domainchanneling:domainchannelingconstraint}\hypertarget{domainchanneling:domainchannelingconstraint}{}
\begin{notedef}  
\texttt{domainChanneling}$(x, \collec{b_1}{b_n})$ states that boolean $b_j$ is true if and only if $x$ has value $j$:
$$b_j=1\quad\iff\quad x=j,\qquad\forall j=1..n$$ 
\end{notedef}

It makes the link between a domain variable $x$ and those 0-1 variables $b$ that are associated with each potential value of $x$: the 0-1 variable $b_j$ associated with the value $j$ taken by $x$ is equal to 1, while the remaining 0-1 variables $b_i$ ($i\neq j$) are all equal to 0.

\begin{itemize}
	\item \textbf{API} : \mylst{domainChanneling(IntegerVariable x, IntegerVariable[] b)}
	\item \textbf{return type} : \texttt{Constraint}
	\item \textbf{options} : \emph{n/a}
	\item \textbf{favorite domain} : enumerated for $x$
	\item \textbf{references} :\\
	  global constraint catalog: \href{http://www.emn.fr/x-info/sdemasse/gccat/Cdomain_constraint.html}{\tt domain\_constraint}
\end{itemize}

\textbf{Example}:
\lstinputlisting{java/cdomainchanneling.j2t}

%\part{eq}
\label{eq}
\hypertarget{eq}{}

\section{eq (constraint)}\label{eq:eqconstraint}\hypertarget{eq:eqconstraint}{}
\begin{notedef}
  \texttt{eq}$(x,y)$ states that the two arguments are equal:
$$x = y$$
\end{notedef}

\begin{itemize}
	\item \textbf{API} :
	\begin{itemize}
		\item \mylst{eq(IntegerExpressionVariable x, IntegerExpressionVariable y)}
		\item \mylst{eq(IntegerExpressionVariable x, int y)}
		\item \mylst{eq(int x, IntegerExpressionVariable y)}
		\item \mylst{eq(SetVariable x, SetVariable y)}
		\item \mylst{eq(RealExpressionVariable x, RealExpressionVariable y)}
		\item \mylst{eq(RealExpressionVariable x, double y)}
		\item \mylst{eq(double x, RealExpressionVariable y)}
		\item \mylst{eq(IntegerVariable x, RealVariable y)}
		\item \mylst{eq(RealVariable x, IntegerVariable y)}
	\end{itemize}
	\item \textbf{return type} : \texttt{Constraint}
	\item \textbf{options} : \emph{n/a}
	\item \textbf{favorite domain} : \emph{to complete}.
	\item \textbf{references} :\\
      global constraint catalog: \href{http://www.emn.fr/x-info/sdemasse/gccat/Ceq.html}{\tt eq} (on domain variables) and \href{http://www.emn.fr/x-info/sdemasse/gccat/Ceq_set.html}{\tt eq\_set} (on set variables). 
\end{itemize}

\textbf{Examples:}
\begin{itemize}
	\item example1:
\end{itemize}
\lstinputlisting{java/ceq1.j2t}
\begin{itemize}
	\item example2
\end{itemize}

\lstinputlisting{java/ceq2.j2t}

%\part{eqcard}
\label{eqcard}
\hypertarget{eqcard}{}

\section{eqCard (constraint)}\label{eqcard:eqcardconstraint}\hypertarget{eqcard:eqcardconstraint}{}
\begin{notedef}
  \texttt{eqCard}$(s,x)$ states that the cardinality of set $s$ is equal to $x$:
$$|s| = x$$
\end{notedef}

\begin{itemize}
	\item \textbf{API} :
	\begin{itemize}
		\item \mylst{eqCard(SetVariable s, IntegerVariable x)}
		\item \mylst{eqCard(SetVariable s, int x)}
	\end{itemize}
	\item \textbf{return type} : \texttt{Constraint}
	\item \textbf{options} : \emph{n/a}
	\item \textbf{favorite domain} : \emph{to complete}
\end{itemize}

\textbf{Example}:
\lstinputlisting{java/ceqcard.j2t}

%\part{equation}
\label{equation}
\hypertarget{equation}{}

\section{equation (constraint)}\label{equation:equationconstraint}\hypertarget{equation:equationconstraint}{}
\begin{notedef}
  \texttt{equation}$(z, \collec{x_1}{x_n},\collec{c_1}{c_n})$ states that $z$ is the weighted sum of $x$ by $c$:
$$c_1x_1+c_2x_2+...+c_nx_n = z$$
\end{notedef}
See also \hyperlink{knapsackproblem:knapsackproblemconstraint}{knapsackProblem}.

\begin{itemize}
	\item \textbf{API} :
	\begin{itemize}
		\item equation(int z, IntegerVariable[] x, int[] c)
		\item equation(String option, int z, IntegerVariable[] x, int[] c)
		\item equation(IntegerVariable z, IntegerVariable[] x, int[] c)
		\item equation(String option, IntegerVariable z, IntegerVariable[] x, int[] c)
	\end{itemize}
	\item \textbf{return type} : \texttt{Constraint}
	\item \textbf{options} :
	\begin{itemize}
		\item \emph{no option} or \texttt{"cp:ac"}: to enforce GAC  using \hyperlink{regular}{\texttt{regular}}
		\item \texttt{"cp:bc"}: to enforce bound consistency using \hyperlink{eq}{\texttt{eq}}$(z,$\hyperlink{scalar}{\texttt{scalar}}$(x,c))$
	\end{itemize}
	\item \textbf{favorite domain} : \emph{to complete}
	\item \textbf{global constraint catalog} : \href{http://www.emn.fr/z-info/sdemasse/gccat/Cscalar_product.html}{scalar\_product}
\end{itemize}

\textbf{Example}:
\lstinputlisting{java/cequation.j2t}

%\part{false}
%\label{false}\hypertarget{false}{}
\section{FALSE (constraint)}\label{false:falseconstraint}\hypertarget{false:falseconstraint}{}
\(FALSE\) always returns \emph{false}.

%\part{feaspairac}
\label{feaspairac}
\hypertarget{feaspairac}{}

\section{feasPairAC (constraint)}\label{feaspairac:feaspairacconstraint}\hypertarget{feaspairac:feaspairacconstraint}{}

\begin{notedef}
  \texttt{feasPairAC}$(x,y,feasTuples)$ states an extensional binary constraint on $(x,y)$ defined by the table $feasTuples$ of compatible pairs of values, and then enforces arc consistency. Two APIs are available to define the compatible pairs:
\begin{itemize}
	\item if $feasTuples$ is encoded as a list of pairs \texttt{List<int[2]>}, then:
      $$\exists \text{ tuple } i\ |\quad (x,y)=feasTuples[i]$$
	\item if $feasTuples$ is encoded as a boolean matrix \texttt{boolean[][]}, let $\underline{x}$ and  $\underline{y}$ be the initial minimum values of $x$ and $y$, then:
      $$\exists (u,v)\ |\quad (x,y)=(u+\underline{x},v+\underline{y})\ \land\ feasTuples[u][v]$$
\end{itemize}
\end{notedef}

The two APIs are duplicated to allow definition of options. 
\begin{itemize}
	\item \textbf{API} :
	\begin{itemize}
		\item \mylst{feasPairAC(IntegerVariable x, IntegerVariable y, List<int[]> feasTuples)}
		\item \mylst{feasPairAC(String options, IntegerVariable x, IntegerVariable y, List<int[]> feasTuples)}
		\item \mylst{feasPairAC(IntegerVariable x, IntegerVariable y, boolean[][] feasTuples)}
		\item \mylst{feasPairAC(String options, IntegerVariable x, IntegerVariable y, boolean[][] feasTuples)}
	\end{itemize}
	\item \textbf{return type} : \texttt{Constraint}
	\item \textbf{options} :
	\begin{itemize}
		\item \emph{no option}: use AC3 (default arc consistency)
		\item \texttt{cp:ac3}: to get AC3 algorithm (searching from scratch for supports on all values)
		\item \texttt{cp:ac2001}: to get AC2001 algorithm (maintaining the current support of each value)
		\item \texttt{cp:ac32}: to get AC3rm algorithm (maintaining the current support of each value in a non backtrackable way)
		\item \texttt{cp:ac322}: to get AC3 with the used of \texttt{BitSet} to know if a support still exists
	\end{itemize}
	\item \textbf{favorite domain} : \emph{to complete}
	\item \textbf{references} :\\
      global constraint catalog: \href{http://www.emn.fr/x-info/sdemasse/gccat/Celem.html}{elem}
\end{itemize}



\textbf{Example}:
\begin{lstlisting}
	Model m = new CPModel();
	Solver s = new CPSolver();
	
	ArrayList couples2 = new ArrayList();
	couples2.add(new int[]{1, 2});
	couples2.add(new int[]{1, 3});
	couples2.add(new int[]{2, 1});
	couples2.add(new int[]{3, 1});
	couples2.add(new int[]{4, 1});
	
	IntegerVariable v1 = makeIntVar("v1", 1, 4);
	IntegerVariable v2 = makeIntVar("v2", 1, 4);
	m.addConstraint(feasPairAC("cp:ac32",v1, v2, couples2));
	s.read(m);
	s.solveAll();
\end{lstlisting} 

%\part{feastupleac}
\label{feastupleac}
\hypertarget{feastupleac}{}

\section{feasTupleAC (constraint)}\label{feastupleac:feastupleacconstraint}\hypertarget{feastupleac:feastupleacconstraint}{}
\begin{notedef}
  \texttt{feasTupleAC}$(x,feasTuples)$ states an extensional constraint on $(x_1,\ldots,x_n)$ defined by the table $feasTuples$ of compatible tuples of values, and then enforces arc consistency:
      $$\exists \text{ tuple } i\ |\quad (x_1,\ldots,x_n)=feasTuples[i]$$
\end{notedef}

The API is duplicated to define options.
\begin{itemize}
	\item \textbf{API} :
	\begin{itemize}
		\item \mylst{feasTupleAC(List<int[]> feasTuples, IntegerVariable... x)}
		\item \mylst{feasTupleAC(String options, List<int[]> feasTuples, IntegerVariable... x)}
	\end{itemize}
	\item \textbf{return type}: \texttt{Constraint}
	\item \textbf{options} :
	\begin{itemize}
		\item \emph{no option}: use AC32 (default arc consistency)
		\item \texttt{cp:ac32}: to get AC3rm algorithm (maintaining the current support of each value in a non backtrackable way)
		\item \texttt{cp:ac2001}: to get AC2001 algorithm (maintaining the current support of each value)
		\item \texttt{cp:ac2008}: to get AC2008 algorithm (maintained by STR)
	\end{itemize}
	\item \textbf{favorite domain} : \emph{to complete}
	\item \textbf{references} :\\
      global constraint catalog: \href{http://www.emn.fr/x-info/sdemasse/gccat/Cin_relation.html}{in\_relation}
\end{itemize}

\textbf{Example}:
\begin{lstlisting}
	Model m = new CPModel();
	Solver s = new CPSolver();
	IntegerVariable v1 = makeIntVar("v1", 0, 2);
	IntegerVariable v2 = makeIntVar("v2", 0, 4);
	
	ArrayList feasTuple = new ArrayList();
	feasTuple.add(new int[]{1, 1}); // x*y = 1
	feasTuple.add(new int[]{2, 4}); // x*y = 1
	
	m.addConstraint(feasTupleAC("cp:ac2001",feasTuple, new IntegerVariable[]{v1, v2}));
	
	s.read(m);
	s.solve();
\end{lstlisting}

%\part{feastuplefc}
\label{feastuplefc}
\hypertarget{feastuplefc}{}

\section{feasTupleFC (constraint)}\label{feastuplefc:feastuplefcconstraint}\hypertarget{feastuplefc:feastuplefcconstraint}{}
\begin{notedef}
  \texttt{feasTupleFC}$(\collec{x_1}{x_n},feasTuples)$ states an extensional constraint on \collec{x_1}{x_n} defined by the table $feasTuples$ of compatible tuples of values, and then performs Forward Checking:
      $$\exists \text{ tuple } i\ |\quad \collec{x_1}{x_n}=feasTuples[i]$$
\end{notedef}


\begin{itemize}
	\item \textbf{API} : \mylst{feasTupleFC(List<int[]> tuples, IntegerVariable... x)}
	\item \textbf{return type}: \texttt{Constraint}
	\item \textbf{options} : \emph{n/a}
	\item \textbf{favorite domain}: \emph{to complete}
	\item \textbf{references} :\\
      global constraint catalog: \href{http://www.emn.fr/x-info/sdemasse/gccat/Cin_relation.html}{in\_relation}
\end{itemize}

\textbf{Example}:
\lstinputlisting{java/cfeastuplefc.j2t}

%\part{geost}
\label{geost}
\hypertarget{geost}{}

\section{geost (constraint)}\label{geost:geostconstraint}\hypertarget{geost:geostconstraint}{}
\todo{to be cleaned}

\begin{notedef}
\texttt{geost} is a global constraint that generically handles a variety of geometrical placement problems. 
It handles geometrical constraints (non-overlapping, distance, etc.) between polymorphic objects (ex: polymorphism can be used for representing rotation) in any dimension.
%The \texttt{geost}$(K, O, S, C)$ constraint is given set of parameters which will define the environment of \texttt{geost}. The parameters are as follows:
The parameters of \texttt{geost}$(dim, objects, shiftedBoxes, eCtrs)$ are respectively:
the space dimension, the list of geometrical objects, the set of boxes that compose the shapes of the objects, the set of geometrical constraints.
The greedy mode should be used without external constraints to have safe results, because it excludes external constraints from its exploration and look for instanciation of variables involved in geost which respect the geost constraint.
For further informations, see the section devoted to this constraint in the Choco Tutorial document. 
%visit the following \hyperlink{geostdescription:placementanduseofthegeostconstraint}{page}.
\end{notedef}

\begin{itemize}
	\item \textbf{API} :\\
\mylst{geost(int dim, Vector<GeostObject> objects, Vector<ShiftedBox> shiftedBoxes, Vector<ExternalConstraint> eCtrs)}\\
\mylst{geost(int dim, Vector<GeostObject> objects, Vector<ShiftedBox> shiftedBoxes, Vector<ExternalConstraint> eCtrs, Vector<int[]> ctrlVs)}
	\item \textbf{return type} : \texttt{Constraint}
	\item \textbf{options} :\emph{n/a}
	\item \textbf{favorite domain} : \emph{to complete}
	\item \textbf{references} :\\
      global constraint catalog: \href{http://www.emn.fr/x-info/sdemasse/gccat/Cgeost.html}{geost}
\end{itemize}

The geost constraint requires the creation of different objects:

\centerline{\begin{tabular}{p{3cm}p{5cm}p{6cm}}
parameter &type &description \\
\hline
\emph{objects} &\texttt{Vector<GeostObject>} &geometrical objects\\
\emph{shiftedBoxes} &\texttt{Vector<ShiftedBox>} &boxes that compose the object shapes\\
\emph{eCtrs} &\texttt{Vector<ExternalConstraint>} &geometrical constraints\\
\emph{ctrlVs} &\texttt{Vector<int[]>} &controlling vectors (for greedy mode)\\[1em]
\end{tabular}}

\noindent Where a \texttt{\bf GeostObject} is defined by:

\centerline{\begin{tabular}{p{3cm}p{4cm}p{7cm}}
attribute &type &description \\
\hline
\emph{dim} &\texttt{int} &dimension\\
\emph{objectId} &\texttt{int} &object id\\
\emph{shapeId} &\texttt{IntegerVariable} &shape id\\
\emph{coordinates} &\texttt{IntegerVariable[$dim$]} &coordinates of the origin\\
\emph{startTime} &\texttt{IntegerVariable} &starting time\\
\emph{durationTime} &\texttt{IntegerVariable} &duration\\
\emph{endTime} &\texttt{IntegerVariable} &finishing time\\[1em]
\end{tabular}}

\noindent Where a \texttt{\bf ShiftedBox} is a $dim$-box defined by the shape it belongs to, its origin (the coordinates of the lower left corner of the box) and its lengths in every dimensions:

\centerline{\begin{tabular}{p{3cm}p{4cm}p{7cm}}
attribute &type &description \\
\hline
\emph{sid} &\texttt{int} &shape id\\
\emph{offset} &\texttt{int[$dim$]} &coordinates of the offset (lower left corner)\\
\emph{size} &\texttt{int[$dim$]} &lengths in every dimensions\\[1em]
\end{tabular}}

\noindent Where an \texttt{\bf ExternalConstraint} contains informations and functionality common to all external constraints and is defined by:

\centerline{\begin{tabular}{p{3cm}p{4cm}p{7cm}}
attribute &type &description \\
\hline
 \emph{ectrID} &\texttt{int} &constraint id\\
 \emph{dimensions} &\texttt{int[]} &list of dimensions that the external constraint is active for\\
 \emph{objectIdentifiers} &\texttt{int[]} &list of object ids that this external constraint affects.\\[1em]
\end{tabular}}

\textbf{Example}:
\lstinputlisting{java/cgeost.j2t}

%\part{geq}
\label{geq}
\hypertarget{geq}{}

\section{geq (constraint)}\label{geq:geqconstraint}\hypertarget{geq:geqconstraint}{}
\begin{notedef}
  \texttt{geq}$(x,y)$ states that $x$ is greater than or equal to $y$:
$$x\ge y$$
\end{notedef}

\begin{itemize}
	\item \textbf{API} :
	\begin{itemize}
		\item \mylst{geq(IntegerExpressionVariable x, IntegerExpressionVariable y)}
		\item \mylst{geq(IntegerExpressionVariable x, int y)}
		\item \mylst{geq(int x, IntegerExpressionVariable y)}
		\item \mylst{geq(RealExpressionVariable x, RealExpressionVariable y)}
		\item \mylst{geq(RealExpressionVariable x, double y)}
		\item \mylst{geq(double x, RealExpressionVariable y)}
	\end{itemize}
	\item \textbf{return type} : \texttt{Constraint}
	\item \textbf{options} : \emph{n/a}
	\item \textbf{favorite domain} : \emph{to complete}.
	\item \textbf{references} :\\
      global constraint catalog: \href{http://www.emn.fr/x-info/sdemasse/gccat/Cgeq.html}{geq}
\end{itemize}

\textbf{Examples:}
\begin{itemize}
	\item example1:
\end{itemize}

\begin{lstlisting}
	Model m = new CPModel();
	Solver s = new CPSolver();
	int c = 1;
	IntegerVariable v = makeIntVar("v", 0, 2);
	m.addConstraint(geq(v, c));
	s.read(m);
	s.solve();
\end{lstlisting}
\begin{itemize}
	\item example2
\end{itemize}

\begin{lstlisting}
	Model m = new CPModel();
	Solver s = new CPSolver();
	IntegerVariable v1 = makeIntVar("v1", 0, 2);
	IntegerVariable v2 = makeIntVar("v2", 0, 2);
	IntegerExpressionVariable w1 = plus(v1, 1);
	IntegerExpressionVariable w2 = minus(v2, 1);
	m.addConstraint(geq(w1, w2));
	s.read(m);
	s.solve();
\end{lstlisting}

%\part{geqcard}
\label{geqcard}
\hypertarget{geqcard}{}

\section{geqCard (constraint)}\label{geqcard:geqcardconstraint}\hypertarget{geqcard:geqcardconstraint}{}
\begin{notedef}
  \texttt{geqCard}$(s,x)$ states that the cardinality of set $s$ is greater than or equal to $x$:
$$|s| \ge x$$
\end{notedef}

\begin{itemize}
	\item \textbf{API} :
	\begin{itemize}
		\item \mylst{geqCard(SetVariable s, IntegerVariable x)}
		\item \mylst{geqCard(SetVariable s, int x)}
	\end{itemize}
	\item \textbf{return type} : \texttt{Constraint}
	\item \textbf{options} : \emph{n/a}
	\item \textbf{favorite domain} : \emph{to complete}
\end{itemize}

\textbf{Example}:
\begin{lstlisting}
	Model m = new CPModel();
	Solver s = new CPSolver();
	SetVariable s = makeSetVar("s", 1, 5);
	IntegerVariable i = makeIntVar("card", 2, 3);
	m.addConstraint(member(x, 3));
	m.addConstraint(geqCard(x, i));
	s.read(m);
	s.solve();
\end{lstlisting}

%\part{globalcardinality}
\label{globalcardinality}
\hypertarget{globalcardinality}{}

\section{globalCardinality (constraint)}\label{globalcardinality:globalcardinalityconstraint}\hypertarget{globalcardinality:globalcardinalityconstraint}{}
\begin{notedef}
  \texttt{globalCardinality}$(x,low, up)$ states bounds on the occurrence numbers of any value $v$ in $x$ (here, offset $min$ is the minimum value over all variables in $x$) :
$$low[v-min]\le|\{i\ |\ x_i=v\}|\le up[v-min],\quad\forall \text{ value } v$$   
\end{notedef}

Mulitple APIs exist:
\begin{itemize}
%	\item \emph{bounds on cardinalities:} Given an array of variables $x$, $min$ the minimal value over all variables, and $max$ the maximal value over all variables, the constraint ensures that the number of occurrences of value $v$ among the variables is between $low[v-min]$ and $up[v-min]$. Note that the length of $low$ and $up$ should be $max - min + 1$. Use the propagator of \cite{QuimperCP03}.
	\item Given an array of variables $x$ and an $offset$, the constraint ensures that the number of occurrences of the value $v$ in $x$ is between $low[v-offset]$ and $up[v-offset]$. Use the propagator of \cite{QuimperCP03}.
	\item \emph{variable cardinalities:} Given an array of variables $x$, an array of variables $card$ to represent the cardinalities and an $offset$, the constraint ensures that the number of occurrences of the value $v$ among the variables is equal to $card[v-offset]$. This constraint:
      \begin{itemize}
      \item enforces Bound Consistency over $x$ regarding the lower and upper bounds of $card$, 
      \item maintains the upper bounds of $card$ by counting the number of variables in which each value can occur, 
      \item maintains the lower bounds of $card$ by counting the number of variables instantiated to each value, 
      \item enforces $card[0] + \cdots + card[m] = n$, where \emph{n} is the number of variables and \emph{m} the number of values.
      \end{itemize}
	Use the propagator of \cite{QuimperCP03}.      

\end{itemize}

The APIs are duplicated to define options. 

\begin{itemize}
	\item \textbf{API} :
      \begin{itemize}
	\item \mylst{globalCardinality(IntegerVariable[] x, int[] low, int[] up, int offset)}
	\item \mylst{globalCardinality(String options, IntegerVariable[] x, int[] low, int[] up, int offset)}
	\item \mylst{globalCardinality(IntegerVariable[] x, IntegerVariable[] card, int offset)}
      \end{itemize}
	\item \textbf{return type} : \texttt{Constraint}
	\item \textbf{options}:
	\begin{itemize}
		\item \emph{no option} :
		\item \hyperlink{cgccac:cgccacoptions}{\tt Options.C\_GCC\_AC} : for \cite{ReginAAAI96} implementation of arc consistency
		\item \hyperlink{cgccbc:cgccbcoptions}{\tt Options.C\_GCC\_BC} : for  \cite{QuimperCP03} implementation of bound consistency
	\end{itemize}
	\item \textbf{favorite domain} : \emph{to complete}
	\item \textbf{references} :
      \begin{itemize}
      \item \cite{ReginAAAI96}: \emph{Generalized arc consistency for global cardinality constraint},
      \item \cite{QuimperCP03}: \emph{An efficient bounds consistency algorithm for the global cardinality constraint}
      \item global constraint catalog: \href{http://www.emn.fr/x-info/sdemasse/gccat/Cglobal_cardinality.html}{global\_cardinality}
      \end{itemize}
\end{itemize}

\textbf{Examples:}
\begin{itemize}
	\item example1:
\end{itemize}

\lstinputlisting{java/cglobalcardinality1.j2t}

\begin{itemize}
	\item example2
\end{itemize}

\lstinputlisting{java/cglobalcardinality2.j2t}

%\part{gt}
\label{gt}
\hypertarget{gt}{}

\section{gt (constraint)}\label{gt:gtconstraint}\hypertarget{gt:gtconstraint}{}
\begin{notedef}
  \texttt{gt}$(x,y)$ states that $x$ is strictly greater than $y$:
$$x>y$$
\end{notedef}

\begin{itemize}
	\item \textbf{API} :
	\begin{itemize}
		\item \mylst{gt(IntegerExpressionVariable x, IntegerExpressionVariable y)}
		\item \mylst{gt(IntegerExpressionVariable x, int y)}
		\item \mylst{gt(int x, IntegerExpressionVariable y)}
	\end{itemize}
	\item \textbf{return type} : \texttt{Constraint}
	\item \textbf{options} : \emph{n/a}
	\item \textbf{favorite domain} : \emph{to complete}.
	\item \textbf{references} :\\
      global constraint catalog: \href{http://www.emn.fr/x-info/sdemasse/gccat/Cgt.html}{gt}
\end{itemize}

\textbf{Example:}
\begin{lstlisting}
	Model m = new CPModel();
	Solver s = new CPSolver();
	int c = 1;
	IntegerVariable v = makeIntVar("v", 0, 2);
	m.addConstraint(gt(v, c));
	s.read(m);
	s.solve();
\end{lstlisting}

%\part{ifonlyif}
\label{ifonlyif}
\hypertarget{ifonlyif}{}

\section{ifOnlyIf (constraint)}\label{ifonlyif:ifonlyifconstraint}\hypertarget{ifonlyif:ifonlyifconstraint}{}
\begin{notedef}
  \texttt{ifOnlyIf}$(c_1,c_2)$ states that $c_1$ holds if and only if $c_2$ holds:
$$c_1\iff c_2$$
\end{notedef}

\begin{itemize}
	\item \textbf{API} : \mylst{ifOnlyIf(Constraint c1, Constraint c2)}
	\item \textbf{return type} : \texttt{Constraint}
	\item \textbf{options} : \emph{n/a}
	\item \textbf{favorite domain} : \emph{n/a}
\end{itemize}

\textbf{Example}:
\lstinputlisting{java/cifonlyif.j2t}

%\part{ifthenelse}
\label{ifthenelse}
\hypertarget{ifthenelse}{}

\section{ifThenElse (constraint)}\label{ifthenelse:ifthenelseconstraint}\hypertarget{ifthenelse:ifthenelseconstraint}{}
\begin{notedef}
  \texttt{ifThenElse}$(c_1,c_2,c_3)$ states that if $c_1$ holds then $c_2$ holds, otherwise $c_3$ holds:
  $$(c_1\land c_2) \lor (\neg c_1 \land c_3)$$
\end{notedef}

\begin{itemize}
	\item \textbf{API} :\mylst{ifThenElse(Constraint c1, Constraint c2, Constraint c3)}
	\item \textbf{return type} : \texttt{Constraint}
	\item \textbf{options} : \emph{n/a}
	\item \textbf{favorite domain} : \emph{n/a}
\end{itemize}

\textbf{Example}:
\lstinputlisting{java/cifthenelse.j2t}

%\part{implies}
\label{implies}
\hypertarget{implies}{}

\section{implies (constraint)}\label{implies:impliesconstraint}\hypertarget{implies:impliesconstraint}{}
\begin{notedef}
  \texttt{implies}$(c_1,c_2)$ states that if $c_1$ holds then $c_2$ holds:
$$c_1\implies c_2$$
\end{notedef}

\begin{itemize}
	\item \textbf{API} : \mylst{implies(Constraint c1, Constraint c2)}
	\item \textbf{return type} : \texttt{Constraint}
	\item \textbf{options} : \emph{n/a}
	\item \textbf{favorite domain} : \emph{n/a}
\end{itemize}

\textbf{Example}:
\lstinputlisting{java/cimplies.j2t}

%\part{infeaspairac}
\label{infeaspairac}
\hypertarget{infeaspairac}{}

\section{infeasPairAC (constraint)}\label{infeaspairac:infeaspairacconstraint}\hypertarget{infeaspairac:infeaspairacconstraint}{}
\begin{notedef}
  \texttt{infeasPairAC}$(x,y,infeasTuples)$ states an extensional binary constraint on $(x,y)$ defined by the table $infeasTuples$ of forbidden pairs of values, and then enforces arc consistency. Two APIs are available to define the forbidden pairs:
\begin{itemize}
	\item if $infeasTuples$ is encoded as a list of pairs \texttt{List<int[2]>}, then:
      $$\forall \text{ tuple } i\ |\quad (x,y)\neq infeasTuples[i]$$
	\item if $infeasTuples$ is encoded as a boolean matrix \texttt{boolean[][]}, let $\underline{x}$ and  $\underline{y}$ be the initial minimum values of $x$ and $y$, then:
      $$\forall (u,v)\ |\quad (x,y)=(u+\underline{x},v+\underline{y})\ \lor\ \neg infeasTuples[u][v]$$
\end{itemize}
\end{notedef}

The two APIs are duplicated to allow definition of options.
\begin{itemize}
	\item \textbf{API} :
	\begin{itemize}
		\item \mylst{infeasPairAC(IntegerVariable x, IntegerVariable y, List<int[]> infeasTuples)}
		\item \mylst{infeasPairAC(String options, IntegerVariable x, IntegerVariable y, List<int[]> infeasTuples)}
		\item \mylst{infeasPairAC(IntegerVariable x, IntegerVariable y, boolean[][] infeasTuples)}
		\item \mylst{infeasPairAC(String options, IntegerVariable x, IntegerVariable y, boolean[][] infeasTuples)}
	\end{itemize}
	\item \textbf{return type} : \texttt{Constraint}
	\item \textbf{options} :
	\begin{itemize}
		\item \emph{no option}: use AC3 (default arc consistency)
		\item \texttt{cp:ac3}: to get AC3 algorithm (searching from scratch for supports on all values)
		\item \texttt{cp:ac2001}: to get AC2001 algorithm (maintaining the current support of each value)
		\item \texttt{cp:ac32}: to get AC3rm algorithm (maintaining the current support of each value in a non backtrackable way)
		\item \texttt{cp:ac322}: to get AC3 with the used of \texttt{BitSet} to know if a support still exists
	\end{itemize}
	\item \textbf{favorite domain} : \emph{to complete}
\end{itemize}

\textbf{Example}:
\begin{lstlisting}
	Model m = new CPModel();
	Solver s = new CPSolver();
	
	boolean[][] matrice2 = new boolean[][]{
	                {false, true, true, false},
	                {true, false, false, false},
	                {false, false, true, false},
	                {false, true, false, false}};
	
	IntegerVariable v1 = makeIntVar("v1", 1, 4);
	IntegerVariable v2 = makeIntVar("v2", 1, 4);
	m.addConstraint(feasPairAC("cp:ac32",v1, v2, matrice2));
	s.read(m);
	s.solveAll();
\end{lstlisting} 

%\part{infeastupleac}
\label{infeastupleac}
\hypertarget{infeastupleac}{}

\section{infeasTupleAC (constraint)}\label{infeastupleac:infeastupleacconstraint}\hypertarget{infeastupleac:infeastupleacconstraint}{}
\begin{notedef}
  \texttt{infeasTupleAC}$(x,feasTuples)$ states an extensional constraint on $(x_1,\ldots,x_n)$ defined by the table $infeasTuples$ of compatible tuples of values, and then enforces arc consistency:
      $$\forall \text{ tuple } i\ |\quad (x_1,\ldots,x_n)\neq infeasTuples[i]$$
\end{notedef}

The API is duplicated to define options.
\begin{itemize}
	\item \textbf{API} :
	\begin{itemize}
		\item \mylst{infeasTupleAC(List<int[]> infeasTuples, IntegerVariable... x)}
		\item \mylst{infeasTupleAC(String options, List<int[]> infeasTuples, IntegerVariable... x)}
	\end{itemize}
	\item \textbf{return type}: \texttt{Constraint}
	\item \textbf{options} :
	\begin{itemize}
		\item \emph{no option}: use AC32 (default arc consistency)
		\item \texttt{cp:ac32}: to get AC3rm algorithm (maintaining the current support of each value in a non backtrackable way)
		\item \texttt{cp:ac2001}: to get AC2001 algorithm (maintaining the current support of each value)
		\item \texttt{cp:ac2008}: to get AC2008 algorithm (maintained by STR)
	\end{itemize}
	\item \textbf{favorite domain} : \emph{to complete}
\end{itemize}

\textbf{Example}:
\lstinputlisting{java/cinfeastupleac.j2t}

%\part{infeastuplefc}
\label{infeastuplefc}
\hypertarget{infeastuplefc}{}

\section{infeasTupleFC (constraint)}\label{infeastuplefc:infeastuplefcconstraint}\hypertarget{infeastuplefc:infeastuplefcconstraint}{}
\begin{notedef}
  \texttt{infeasTupleFC}$(\collec{x_1}{x_n},feasTuples)$ states an extensional constraint on \collec{x_1}{x_n} defined by the table $infeasTuples$ of compatible tuples of values, and then performs Forward Checking:
      $$\forall \text{ tuple } i\ |\quad \collec{x_1}{x_n}\neq infeasTuples[i]$$
\end{notedef}

\begin{itemize}
	\item \textbf{API} : \mylst{infeasTupleFC(List<int[]> infeasTuples, IntegerVariable... x)}
	\item \textbf{return type}: \texttt{Constraint}
	\item \textbf{options} : \emph{n/a}
	\item \textbf{favorite domain}: \emph{to complete}
\end{itemize}

\textbf{Example}:
\lstinputlisting{java/cinfeastuplefc.j2t}

%\part{intdiv}
\label{intdiv}
\hypertarget{intdiv}{}

\section{intDiv (constraint)}\label{intdiv:intdivconstraint}\hypertarget{intdiv:intdivconstraint}{}
\begin{notedef}
  \texttt{intDiv}$(x,y,z)$ states that $z$ is equal to the integer quotient of $x$ by $y$:
$$z = \lfloor x / y \rfloor$$
\end{notedef}

\begin{itemize}
	\item \textbf{API}: \mylst{intDiv(IntegerVariable x, IntegerVariable y, IntegerVariable z)}
	\item \textbf{return type} : \texttt{Constraint}
	\item \textbf{option} : \emph{n/a}
	\item \textbf{favorite domain}: bound
\end{itemize}

\textbf{Example}:
\lstinputlisting{java/cintdiv.j2t}

%\part{inversechanneling}
\label{inversechanneling}
\hypertarget{inversechanneling}{}

\section{inverseChanneling (constraint)}\label{inversechanneling:inversechannelingconstraint}\hypertarget{inversechanneling:inversechannelingconstraint}{}
\begin{notedef}
  \texttt{inverseChanneling}$(x,y)$ states a channeling between two arrays  $x$ and $y$ of integer variables with the same domain.It enforces that if the $i$-th element of $x$ is equal to $j$ then the $j$-th element of $y$ is equal to $i$ and conversely:
$$x_i = j\quad\iff\quad y_j = i$$
\end{notedef}
\begin{itemize}
	\item \textbf{API} : \mylst{inverseChanneling(IntegerVariable[] x, IntegerVariable[] y)}
	\item \textbf{return type} : \texttt{Constraint}
	\item \textbf{options} : \emph{no options}
	\item \textbf{favorite domain} : enumerated for x
	\item \textbf{references} :\\
      global constraint catalog: \href{http://www.emn.fr/x-info/sdemasse/gccat/Cinverse.html}{inverse}
\end{itemize}

\textbf{Example}:
\lstinputlisting{java/cinversechanneling.j2t}

%\part{isincluded}
\label{isincluded}
\hypertarget{isincluded}{}

\section{isIncluded (constraint)}\label{isincluded:isincludedconstraint}\hypertarget{isincluded:isincludedconstraint}{}
\begin{notedef}
  \texttt{isIncluded}$(x,y)$ states that the second set $y$ contains the first set $x$:
 $$x\subseteq y$$
\end{notedef}

\begin{itemize}
	\item \textbf{API} : \mylst{isIncluded(SetVariable x, SetVariable y)}
	\item \textbf{return type} : \texttt{Constraint}
	\item \textbf{options} :\emph{n/a}
	\item \textbf{favorite domain} : \emph{to complete}
\end{itemize}

\textbf{Example}:
\begin{lstlisting}
	Model m = new CPModel();
	Solver s = new CPSolver();
	SetVariable v1 = makeSetVar("v1", 3, 4);
	SetVariable v2 = makeSetVar("v2", 3, 8);
	m.addConstraint(isIncluded(v1, v2));
	s.read(m);
	s.solveAll();
\end{lstlisting} 

%\part{isnotincluded}
\label{isnotincluded}
\hypertarget{isnotincluded}{}

\section{isNotIncluded (constraint)}\label{isnotincluded:isnotincludedconstraint}\hypertarget{isnotincluded:isnotincludedconstraint}{}
\begin{notedef}
  \texttt{isNotIncluded}$(s_1,s_2)$ states that set $s_1$ is not included in set $s_2$:
 $$s_2\not\subseteq s_2$$
\end{notedef}

\begin{itemize}
	\item \textbf{API} : \mylst{isNotIncluded(SetVariable s1, SetVariable s2)}
	\item \textbf{return type} : \texttt{Constraint}
	\item \textbf{options} :\emph{n/a}
	\item \textbf{favorite domain} : \emph{to complete}
\end{itemize}

\textbf{Example}:
\lstinputlisting{java/cisnotincluded.j2t} 

%\part{leq}
\label{leq}
\hypertarget{leq}{}

\section{leq (constraint)}\label{leq:leqconstraint}\hypertarget{leq:leqconstraint}{}
\begin{notedef}
  \texttt{leq}$(x,y)$ states that $x$ is less than or equal to $y$:
$$x \le y$$
\end{notedef}

\begin{itemize}
	\item \textbf{API} :
	\begin{itemize}
		\item \mylst{leq(IntegerExpressionVariable x, IntegerExpressionVariable y)}
		\item \mylst{leq(IntegerExpressionVariable x, int y)}
		\item \mylst{leq(int x, IntegerExpressionVariable y)}
		\item \mylst{leq(RealExpressionVariable x, RealExpressionVariable y)}
		\item \mylst{leq(RealExpressionVariable x, double y)}
		\item \mylst{leq(double x, RealExpressionVariable y)}
	\end{itemize}
	\item \textbf{return type} : \texttt{Constraint}
	\item \textbf{options} : \emph{n/a}
	\item \textbf{favorite domain} : \emph{to complete}.
	\item \textbf{references} :\\
      global constraint catalog: \href{http://www.emn.fr/x-info/sdemasse/gccat/Cleq.html}{leq}
\end{itemize}

\textbf{Example:}
\lstinputlisting{java/cleq.j2t}

%\part{leqcard}
\label{leqcard}
\hypertarget{leqcard}{}

\section{leqCard (constraint)}\label{leqcard:leqcardconstraint}\hypertarget{leqcard:leqcardconstraint}{}
\begin{notedef}
  \texttt{leqCard}$(s,x)$ states that the cardinality of set $s$ is less than or equal to $x$:
$$|s| \le x$$
\end{notedef}

\begin{itemize}
	\item \textbf{API} :
	\begin{itemize}
		\item \mylst{leqCard(SetVariable s, IntegerVariable x)}
		\item \mylst{leqCard(SetVariable s, int x)}
	\end{itemize}
	\item \textbf{return type} : \texttt{Constraint}
	\item \textbf{options} : \emph{n/a}
	\item \textbf{favorite domain} : \emph{to complete}
\end{itemize}

\textbf{Example}:
\lstinputlisting{java/cleqcard.j2t}

%\part{lex}
\label{lex}
\hypertarget{lex}{}

\section{lex (constraint)}\label{lex:lexconstraint}\hypertarget{lex:lexconstraint}{}
\begin{notedef}
  \texttt{lex}$(x,y)$ enforces a strict lexicographic ordering  $x <_{lex} y$ between two arrays of same size $n$:
$$\exists\ j\in\{1,\ldots,n\}\ |\qquad x_j<y_j\quad \land\quad x_i=y_i\ (\forall\  i<j)$$
\end{notedef}

\begin{itemize}
	\item \textbf{API} : \mylst{lex(IntegerVariable[] x, IntegerVariable[] y)}
	\item \textbf{return type} : \texttt{Constraint}
	\item \textbf{options} :\emph{n/a}
	\item \textbf{favorite domain} : \emph{to complete}
	\item \textbf{references} :
      \begin{itemize}
      \item \cite{FrischCP02}: \emph{Global Constraints for Lexicographic Orderings}
      \item global constraint catalog: \href{http://www.emn.fr/x-info/sdemasse/gccat/Clex_less.html}{lex\_less}
      \end{itemize}
\end{itemize}

\textbf{Example}:
\begin{lstlisting}
	Model m = new CPModel();
	Solver s = new CPSolver();
	
	int n1 = 8;
	int k = 2;
	IntegerVariable[] vs1 = new IntegerVariable[n1 / 2];
	IntegerVariable[] vs2 = new IntegerVariable[n1 / 2];
	
	for (int i = 0; i < n1 / 2; i++) {
	   vs1[i] = makeIntVar("" + i, 0, k);
	   vs2[i] = makeIntVar("" + i, 0, k);
	}
	m.addConstraint(lex(vs1, vs2));
	
	s.read(m);
	s.solve();
\end{lstlisting} 

%\part{lexchain}
\label{lexchain}
\hypertarget{lexchain}{}

\section{lexChain (constraint)}\label{lexchain:lexchainconstraint}\hypertarget{lexchain:lexchainconstraint}{}
\begin{notedef}
  \texttt{lexChain}$(\collec{x^1_1}{x^1_n},\ldots,\collec{x^p_1}{x^p_n})$ states a strict lexicographic ordering on a chain of $p$ integer vectors:
$$x^1 <_{lex} x^2 <_{lex}\cdots <_{lex} x^p$$
%where $X^1$ contains up to $n$ variables. 
\end{notedef}

\begin{itemize}
	\item \textbf{API} : \mylst{lexChain(IntegerVariable[]... x)}
	\item \textbf{return type} : \texttt{Constraint}
	\item \textbf{options} : \emph{n/a}
	\item \textbf{favorite domain} : \emph{to complete}
	\item \textbf{references} :
      \begin{itemize}
      \item \cite{BeldiceanuSICS02} \emph{Arc-Consistency for a chain of Lexicographic Ordering Constraints} 
      \item global constraint catalog: \href{http://www.emn.fr/x-info/sdemasse/gccat/Clex_chain_less.html}{lex\_chain\_less}
      \end{itemize}
\end{itemize}

\textbf{Example}:
\lstinputlisting{java/clexchain.j2t}

%\part{lexchaineq}
\label{lexchaineq}
\hypertarget{lexchaineq}{}

\section{lexChainEq (constraint)}\label{lexchaineq:lexchaineqconstraint}\hypertarget{lexchaineq:lexchaineqconstraint}{}
\begin{notedef}
\texttt{lexChainEq}$(x^1 ,x^2 ,x^3,\ldots)$ enforces a lexicographic ordering on a chain of integer vectors:
$$x^1 \le_{lex} x^2 \le_{lex} x^3 \le_{lex}\cdots$$
%where $X^1$ contains up to $n$ variables. 
\end{notedef}

\begin{itemize}
	\item \textbf{API} : \mylst{lexChainEq(IntegerVariable[]... arrayOfVectors)}
	\item \textbf{return type} : \texttt{Constraint}
	\item \textbf{options} : \emph{n/a}
	\item \textbf{favorite domain} : \emph{to complete}
	\item \textbf{references} :
      \begin{itemize}
      \item \cite{BeldiceanuSICS02} \emph{Arc-Consistency for a chain of Lexicographic Ordering Constraints} 
      \item global constraint catalog: \href{http://www.emn.fr/x-info/sdemasse/gccat/Clex_chain_lesseq.html}{lex\_chain\_lesseq}
      \end{itemize}
\end{itemize}

\textbf{Example}:
\lstinputlisting{java/clexchaineq.j2t}
%\part{lexeq}
\label{lexeq}
\hypertarget{lexeq}{}

\section{lexEq (constraint)}\label{lexeq:lexeqconstraint}\hypertarget{lexeq:lexeqconstraint}{}
\begin{notedef}
  \texttt{lexEq}$(\collec{x_1}{x_n},\collec{y_1}{y_n})$ states a lexicographic ordering  $x \le_{lex} y$:
$$\exists\ j=1..n\ |\qquad x_j\le y_j\quad \land\quad x_i=y_i\ (\forall\  i<j)$$
\end{notedef}

\begin{itemize}
	\item \textbf{API} : \mylst{lexEq(IntegerVariable[] x, IntegerVariable[] y)}
	\item \textbf{return type} : \texttt{Constraint}
	\item \textbf{options} :\emph{n/a}
	\item \textbf{favorite domain} : \emph{to complete}
	\item \textbf{references} :
      \begin{itemize}
      \item \cite{FrischCP02}: \emph{Global Constraints for Lexicographic Orderings}
      \item global constraint catalog: \href{http://www.emn.fr/x-info/sdemasse/gccat/Clex_lesseq.html}{lex\_lesseq}
      \end{itemize}
\end{itemize}

\textbf{Example}:
\textbf{Example}:
\lstinputlisting{java/clexeq.j2t}

%\part{leximin}
\label{leximin}
\hypertarget{leximin}{}

\section{leximin (constraint)}\label{leximin:leximinconstraint}\hypertarget{leximin:leximinconstraint}{}

\emph{TODO: verify the specifications of the implemented version.}

\begin{notedef}
Let $x = (x_1,\ldots, x_n)$ and $y = (y_1,\ldots, y_n)$ be two vectors of $n$ integers, and let $x'$ and $y'$ be respectively permutations of vectors $x$ and $y$ sorted by increasing order of the components.
Constraint \texttt{leximin(x, y)} holds if and only if $x'<_{lex} y'$:
$$\exists\ j\in\{1,\ldots,n\}\ |\qquad x'_j<y'_j\quad \land\quad x'_i=y'_i\ (\forall\  i<j)$$
  \end{notedef}

\begin{itemize}
	\item \textbf{API} :
	\begin{itemize}
		\item \mylst{leximin(IntegerVariable[] x, IntegerVariable[] y)}
		\item \mylst{leximin(int[] x, IntegerVariable[] y)}
	\end{itemize}
	\item \textbf{return type} : \texttt{Constraint}
	\item \textbf{options} :\emph{n/a}
	\item \textbf{favorite domain} : \emph{to complete}
	\item \textbf{references} :
      \begin{itemize}
      \item \cite{FrischIJCAI03}: \emph{Multiset ordering constraints} 
      \item global constraint catalog: \href{http://www.emn.fr/x-info/sdemasse/gccat/Clex_lesseq_allperm.html}{lex\_lesseq\_allperm} (variant)
      \end{itemize}
\end{itemize}

\textbf{Example}:
\lstinputlisting{java/cleximin.j2t}

%\part{lt}
\label{lt}
\hypertarget{lt}{}

\section{lt (constraint)}\label{lt:ltconstraint}\hypertarget{lt:ltconstraint}{}
\begin{notedef}
  \texttt{lt}$(x,y)$ states that $x$ is strictly smaller than $y$:
$$x<y$$
\end{notedef}

\begin{itemize}
	\item \textbf{API} :
	\begin{itemize}
		\item \mylst{lt(IntegerExpressionVariable x, IntegerExpressionVariable y)}
		\item \mylst{lt(IntegerExpressionVariable x, int y)}
		\item \mylst{lt(int x, IntegerExpressionVariable y)}
	\end{itemize}
	\item \textbf{return type} : \texttt{Constraint}
	\item \textbf{options} : \emph{n/a}
	\item \textbf{favorite domain} : \emph{to complete}.
	\item \textbf{references} :\\
      global constraint catalog: \href{http://www.emn.fr/x-info/sdemasse/gccat/Clt.html}{lt}
\end{itemize}

\textbf{Example}:
\lstinputlisting{java/clt.j2t}

%\part{max}
\label{max}
\hypertarget{max}{}

\section{max (constraint)}\label{max:maxconstraint}\hypertarget{max:maxconstraint}{}

\subsection{max of a list}\label{max:maxofalist}\hypertarget{max:maxofalist}{}

\begin{notedef}
\texttt{max}$(x,z)$ states that $z$ is equal to the greater element of vector $x$:
$$z = \max(x_1, x_2, ..., x_n)$$  
\end{notedef}

\begin{itemize}
	\item \textbf{API}:
	\begin{itemize}
		\item \mylst{max(IntegerVariable[] x, IntegerVariable z)}
		\item \mylst{max(IntegerVariable x1, IntegerVariable x2, IntegerVariable z)}
		\item \mylst{max(int x1, IntegerVariable x2, IntegerVariable z)}
		\item \mylst{max(IntegerVariable x1, int x2, IntegerVariable z)}
	\end{itemize}
	\item \textbf{return type}: \texttt{Constraint}
	\item \textbf{options} : \emph{n/a}
	\item \textbf{favorite domain} : \emph{to complete}
	\item \textbf{references} :\\
      global constraint catalog: \href{http://www.emn.fr/x-info/sdemasse/gccat/Cmaximum.html}{maximum}
\end{itemize}

\textbf{Example}:
\begin{lstlisting}
	Model m = new CPModel();
	Solver s= new CPSolver();
	IntegerVariable x = makeIntVar("x", 1, 5);
	IntegerVariable y = makeIntVar("y", 1, 5);
	IntegerVariable z = makeIntVar("z", 1, 5);
	m.addVariable("cp:bound", x, y, z);
	m.addConstraint(max(y, z, x));
	s.read(m);
	s.solve();
\end{lstlisting}

\subsection{max of a set}\label{max:maxofaset}\hypertarget{max:maxofaset}{}

\begin{notedef}
\texttt{max}$(s,x,z)$ states that $z$ is equal to the greater element of vector $x$ whose index is in set $s$:
$$z = \max_{i\in s}( x_i )$$
  \end{notedef}

\begin{itemize}
	\item \textbf{API}:
	\begin{itemize}
		\item \mylst{max(SetVariable s, IntegerVariable[] x, IntegerVariable z)}
	\end{itemize}
	\item \textbf{return type}: \texttt{Constraint}
	\item \textbf{options} : \emph{n/a}
	\item \textbf{favorite domain} : \emph{to complete}
\end{itemize}

\begin{lstlisting}
	Model m = new CPModel();
	Solver s= new CPSolver();
	IntegerVariable[] x = constantArray(new int[]{5,7,9,10,12,3,2});
	IntegerVariable max = makeIntVar("max", 1, 100);
	SetVariable set = makeSetVar("set", 0, x.length-1);
	m.addConstraints(max(set, x, max), leqCard(set, constant(5)));
	s.read(m);
	s.solve();
\end{lstlisting}

%\part{member}
\label{member}
\hypertarget{member}{}

\section{member (constraint)}\label{member:memberconstraint}\hypertarget{member:memberconstraint}{}

\begin{notedef}
  \texttt{member}$(x,s)$ states that integer $x$ belongs to set $s$:
$$x\in s$$
\end{notedef}

\begin{itemize}
	\item \textbf{API} :
	\begin{itemize}
		\item \mylst{member(int x, SetVariable s)}
		\item \mylst{member(SetVariable s, int x)}
		\item \mylst{member(SetVariable s, IntegerVariable x)}
		\item \mylst{member(IntegerVariable x, SetVariable s)}
		\item \mylst{member(member(SetVariable sv, IntegerVariable... vars)}
		\item \mylst{member(IntegerVariable x, int[] s)}
		\item \mylst{member(IntegerVariable x, int lower, int upper)}
	\end{itemize}
	\item \textbf{return type} : \texttt{Constraint}
	\item \textbf{options} :\emph{n/a}
	\item \textbf{favorite domain} : \emph{to complete}
	\item \textbf{references} :\\
      global constraint catalog: \href{http://www.emn.fr/x-info/sdemasse/gccat/Cin_set.html}{in\_set}
\end{itemize}

\textbf{Examples}:
1. using a set variable
\lstinputlisting{java/cmember.j2t}

2. using an array of integers
\lstinputlisting{java/camong1.j2t}

%\part{min}
\label{min}
\hypertarget{min}{}

\section{min (constraint)}\label{min:minconstraint}\hypertarget{min:minconstraint}{}

\subsection{min of a list}\label{min:minofalist}\hypertarget{min:minofalist}{}

\begin{notedef}
  \texttt{mix}$(x,z)$ states that $z$ is equal to the smaller element
  of vector $x$:
$$z = \min(x_1, x_2, ..., x_n).$$
\end{notedef}
\begin{itemize}
	\item \textbf{API}:
	\begin{itemize}
		\item \mylst{min(IntegerVariable[] x, IntegerVariable z)}
		\item \mylst{min(IntegerVariable x1, IntegerVariable x2, IntegerVariable z)}
		\item \mylst{min(int x1, IntegerVariable x2, IntegerVariable z)}
		\item \mylst{min(IntegerVariable x1, int x2, IntegerVariable z)}
	\end{itemize}
	\item \textbf{return type}: \texttt{Constraint}
	\item \textbf{options} : \emph{n/a}
	\item \textbf{favorite domain} : \emph{to complete}
	\item \textbf{references} :\\
      global constraint catalog: \href{http://www.emn.fr/x-info/sdemasse/gccat/Cminimum.html}{minimum}
\end{itemize}

\textbf{Example}:
\begin{lstlisting}
	Model m = new CPModel();
	Solver s= new CPSolver();
	IntegerVariable x = makeIntVar("x", 1, 5);
	IntegerVariable y = makeIntVar("y", 1, 5);
	IntegerVariable z = makeIntVar("z", 1, 5);
	m.addVariable("cp:bound", x, y, z);
	m.addConstraint(min(y, z, x));
	s.read(m);
	s.solve();
\end{lstlisting}

\subsection{min of a set}\label{min:minofaset}\hypertarget{min:minofaset}{}

\begin{notedef}
  \texttt{min}$(s,x,z)$ states that $z$ is equal to the smaller
  element of vector $x$ whose index is in set $s$:
$$z = \min_{i\in s}( x_i ).$$
\end{notedef}
\begin{itemize}
	\item \textbf{API}:
	\begin{itemize}
		\item \mylst{min(SetVariable s,IntegerVariable[] x, IntegerVariable z)}
	\end{itemize}
	\item \textbf{return type}: \texttt{Constraint}
	\item \textbf{options} : \emph{n/a}
	\item \textbf{favorite domain} : \emph{to complete}
\end{itemize}

\begin{lstlisting}
	Model m = new CPModel();
	Solver s= new CPSolver();
	IntegerVariable[] x = constantArray(new int[]{5,7,9,10,12,3,2});
	IntegerVariable min = makeIntVar("min", 1, 100);
	SetVariable set = makeSetVar("set", 0, x.length-1);
	m.addConstraints(min(set, x, max), leqCard(set, constant(5)));
	s.read(m);
	s.solve();
\end{lstlisting}

%\input{chapters/Cminus.tex}
%\part{mod}
\label{mod}
\hypertarget{mod}{}

\section{mod (constraint)}\label{mod:modconstraint}\hypertarget{mod:modconstraint}{}
\begin{notedef}
  \texttt{mod}$(x_1,x_2,x_3)$ states that $x_1$ is congruent to $x_2$
  modulo $x_3$:
$$x_1 \equiv x_2 \mod x_3$$
\end{notedef}
\begin{itemize}
	\item \textbf{API} : \mylst{mod(IntegerVariable x1, IntegerVariable x2, int x3)}
	\item \textbf{return type} : \texttt{Constraint}
	\item \textbf{options} : \emph{n/a}
	\item \textbf{favorite domain} : \emph{n/a}
\end{itemize}

\textbf{Example}:
\begin{lstlisting}
	Model m = new CPModel();
	Solver s = new CPSolver();
	
	IntegerVariable x = makeIntVar("x", 0, 10);
	IntegerVariable w = makeIntVar("w", 0, 10);
	
	m.addConstraint(mod(w,x, 1));
	
	s.read(m);
	s.solve();
\end{lstlisting}

%\input{chapters/Cmult.tex}
%\part{multicostregular}
\label{multicostregular}
\hypertarget{multicostregular}{}

\section{multiCostRegular (constraint)}\label{multicostregular:multicostregularconstraint}\hypertarget{multicostregular:multicostregularconstraint}{}
\begin{notedef}
  \texttt{multiCostRegular}$(x,z,\mathcal{L}(\Pi),c)$ states that sequence $x$ is a word belonging to the regular language $\mathcal{L}(\Pi)$,
% recognized by a deterministic finite automaton (DFA) or a multicostregular expression $\Pi$:
$$(x_1,\ldots,x_n)\in\mathcal{L}(\Pi)$$
and that the bounded vector $z$ is equal to the costs of $x$ according to the assigment cost matrix $c$:
$$\sum_{i=1}^{n} c[r][i][x_i]=z[r],\quad \forall r\in\{0,\ldots,R\}$$
\end{notedef}
\texttt{multiCostRegular} is a conjunction of a \hyperlink{regular}{\texttt{regular}} constraint with $R+1$ cost functions.
It may be used in the context of personnel scheduling problems, handling complex work regulations by the mean of regular expressions, together with cardinality or financial constraints by the mean of cost functions.
The filtering algorithm associated with \texttt{multiCostRegular} is based on lagrangian relaxation and computations of shortest/longest pathes in a layered digraph~\cite{MenanaCPAIOR09}. It typically performs more filtering than the conjunction of \texttt{costRegular} and \texttt{globalCardinality} or than multiple \texttt{costRegular}.

The accepting language is specified by a deterministic finite automaton (DFA):
Automaton $\Pi$ is defined on a given \emph{alphabet} $\Sigma\subseteq\Z$ by a set $Q=\{0,\ldots,m\}$ of \emph{states}, a subset $A\subseteq Q$ of \emph{final} or \emph{accepting states} and a table $\Delta\subseteq Q\!\times\!\Sigma\!\times Q$ of \emph{transitions} between states. $\Pi$ is encoded as an object of class \texttt{Automaton}, whose API contains:
\begin{lstlisting}
  Automaton();
  int addState();
  void setStartingState(int state); 
  void setAcceptingState(int state); 
  void addTransition(int state1, int state2, int label);
  int getNbStates();
\end{lstlisting}
The cost functions are encoded as one matrix \texttt{int cost[nTime][nAct][auto.getNbStates()][nRes]} such that
\texttt{cost[i][j][s][r]} is the cost of assigning variable $x_i$ to activity $j$ at state $s$ on dimension $r+1$.

\begin{itemize}
	\item \textbf{API} : \mylst{multiCostRegular(IntegerVariable[] x, IntegerVariable[] z, Automaton P, int[][][][] c)}
	\item \textbf{return type} : \texttt{Constraint}
	\item \textbf{options} :
      \begin{itemize}
      \item \texttt{MultiCostRegular.DATA\_STRUCT} is  \texttt{MultiCostRegular.BITSET} or \texttt{MultiCostRegular.LIST}: a parameter stating which backtrable data structure to use for storing the outgoing arcs of the layered digraph. The observed behaviour is until $1000$ arcs the bipartite list is much more efficient, afterwards the memory efficiency of the bitset representation allow faster operations. 
      \item \texttt{MultiCostRegular.U0}, \texttt{MultiCostRegular.R0}, \texttt{MultiCostRegular.MAXNONIMPROVEITER}, and \texttt{MultiCostRegular.MAXBOUNDITER} are value parameters of the subgradient algorithm used for solving the lagrangean relaxation.
      \item \texttt{MultiCostRegular.D\_PREC} is a double parameter stating the precision of float computation. It is set by default to $10^{-5}$.
      \end{itemize}
	\item \textbf{favorite domain} : \emph{to complete}
	\item \textbf{references} :\\
       \cite{MenanaCPAIOR09}: \emph{Sequencing and Counting with the {\tt multicost-regular} Constraint}
\end{itemize}
%\begin{notedef}
%  For further informations, see the multicost-regular description.
%\end{notedef}

\textbf{Example}:
\lstinputlisting{java/cmulticosteregular_import.j2t}
\lstinputlisting{java/cmulticostregular.j2t}


%\input{chapters/Cneg.tex}
%\part{neq}
\label{neq}
\hypertarget{neq}{}

\section{neq (constraint)}\label{neq:neqconstraint}\hypertarget{neq:neqconstraint}{}

\begin{notedef}
  \texttt{neq} states that the two arguments are different:
$$x \neq y.$$
\end{notedef}
\begin{itemize}
	\item \textbf{API} :
	\begin{itemize}
		\item \mylst{neq(IntegerExpressionVariable x, IntegerExpressionVariable y)}
		\item \mylst{neq(IntegerExpressionVariable x, int y)}
		\item \mylst{neq(int x, IntegerExpressionVariable y)}
	\end{itemize}
	\item \textbf{return type} : \texttt{Constraint}
	\item \textbf{options} : \emph{n/a}
	\item \textbf{favorite domain} : \emph{to complete}.
	\item \textbf{references} :\\
      global constraint catalog: \href{http://www.emn.fr/x-info/sdemasse/gccat/Cneq.html}{neq}
\end{itemize}

\textbf{Examples:}
\begin{itemize}
	\item example1:
\end{itemize}

\lstinputlisting{java/cneq1.j2t}

\begin{itemize}
	\item example2
\end{itemize}

\lstinputlisting{java/cneq2.j2t}
%\part{neqcard}
\label{neqcard}
\hypertarget{neqcard}{}

\section{neqCard (constraint)}\label{neqcard:neqcardconstraint}\hypertarget{neqcard:neqcardconstraint}{}
\begin{notedef}
  \texttt{neqCard}$(s,z)$ states that the cardinality of set $s$ is not equal to $z$:
$$|s| \neq z$$
\end{notedef}

\begin{itemize}
	\item \textbf{API} :
	\begin{itemize}
		\item \mylst{neqCard(SetVariable s, IntegerVariable z)}
		\item \mylst{neqCard(SetVariable s, int z)}
	\end{itemize}
	\item \textbf{return type} : \texttt{Constraint}
	\item \textbf{options} : \emph{n/a}
	\item \textbf{favorite domain} : \emph{to complete}
\end{itemize}

\textbf{Example}:
\lstinputlisting{java/cneqcard.j2t}

%\part{not}
\label{not}
\hypertarget{not}{}

\section{not (constraint)}\label{not:notconstraint}\hypertarget{not:notconstraint}{}
\begin{notedef}
  \texttt{not}$(c)$ holds if and only if constraint $c$ does not hold:
$$\neg c$$
\end{notedef}
\begin{itemize}
	\item \textbf{API} : \mylst{not(Constraint c)}
	\item \textbf{return type} : \texttt{Constraint}
	\item \textbf{options} : \emph{n/a}
	\item \textbf{favorite domain} : \emph{n/a}
\end{itemize}

\textbf{Example} : 
\begin{lstlisting}
	Model m = new CPModel();
	Solver s = new CPSolver();
	IntegerVariable x = makeIntVar("x", 1, 10);
	// x < 3
	m.addConstraint(not(geq(x, 3)));
	
	s.read(m);
	s.solve();
\end{lstlisting}

%\part{notmember}
\label{notmember}
\hypertarget{notmember}{}

\section{notMember (constraint)}\label{notmember:notmemberconstraint}\hypertarget{notmember:notmemberconstraint}{}
\begin{notedef}
\texttt{notMember}$(x,s)$ states that integer $x$ is not contained in set $s$:
$$x\not\in s$$  
\end{notedef}

\begin{itemize}
	\item \textbf{API} :
	\begin{itemize}
		\item \mylst{notMember(int x, SetVariable s)}
		\item \mylst{notMember(SetVariable s, int x)}
		\item \mylst{notMember(SetVariable s, IntegerVariable x)}
		\item \mylst{notMember(IntegerVariable x, SetVariable s)}
	\end{itemize}
	\item \textbf{return type} : \texttt{Constraint}
	\item \textbf{options} :\emph{n/a}
	\item \textbf{favorite domain} : \emph{to complete}
\end{itemize}

\textbf{Example}:
\lstinputlisting{java/cnotmember.j2t}

%\part{nth}
\label{nth}
\hypertarget{nth}{}

\section{nth (constraint)}\label{nth:nthconstraint}\hypertarget{nth:nthconstraint}{}
\texttt{nth} is the well known \emph{element} constraint.
Several APIs are available: 
\begin{notedef}
\begin{itemize}
\item \texttt{nth}$(i,x,y)$ ensures that $x[i]=y$
\item \texttt{nth}$(i,x,y,o)$ ensures that $x[i+o]=y$ ($o$ is an \emph{offset} for shifting values)
\item \texttt{nth}$(i,j,x,y)$ ensures that $x[i][j]=y$
\end{itemize}
\end{notedef}

\begin{itemize}
	\item \textbf{API} :
	\begin{itemize}
		\item \mylst{nth(IntegerVariable i, int[] x, IntegerVariable y)}
		\item \mylst{nth(String option, IntegerVariable i, int[] x, IntegerVariable y)}
		\item \mylst{nth(IntegerVariable i, IntegerVariable[] x, IntegerVariable y)}
		\item \mylst{nth(IntegerVariable i, int[] x, IntegerVariable y, int offset)}
		\item \mylst{nth(String option, IntegerVariable i, int[] x, IntegerVariable y, int offset)}		
		\item \mylst{nth(IntegerVariable i, IntegerVariable[] x, IntegerVariable y, int offset)}
		\item \mylst{nth(String option, IntegerVariable i, IntegerVariable[] x, IntegerVariable y, int offset)}
		\item \mylst{nth(IntegerVariable i, IntegerVariable j, int[][] x, IntegerVariable y)}
	\end{itemize}
	\item \textbf{return type} : \texttt{Constraint}
	\item \textbf{options} :
	\begin{itemize}
		\item \emph{no option} 
		\item \hyperlink{cnthg:cnthgoptions}{\tt Options.C\_NTH\_G} for global consistency
	\end{itemize}
	\item \textbf{favorite domain} : \emph{to complete}
	\item \textbf{references} :\\
      global constraint catalog: \href{http://www.emn.fr/x-info/sdemasse/gccat/Celement.html}{element}
\end{itemize}

\textbf{Example}:
\lstinputlisting{java/cnth.j2t} 

%\part{occurrence}
\label{occurrence}
\hypertarget{occurrence}{}

\section{occurrence (constraint)}\label{occurrence:occurrenceconstraint}\hypertarget{occurrence:occurrenceconstraint}{}
\begin{notedef}
  \texttt{occurrence}$(v,z,x)$ states that $z$ is equal to the number of elements in $x$ with value $v$:
$$z=|\{i\ |\ x_i=v\}|$$   
\end{notedef}
  This is a specialization of the \texttt{globalCardinality} constraint.

\begin{itemize}
	\item \textbf{API}: \mylst{occurrence(int v, IntegerVariable z, IntegerVariable... x)}
	\item \textbf{return type} : \texttt{Constraint}
	\item \textbf{options} :\emph{n/a}
	\item \textbf{favorite domain} : \emph{to complete}
	\item \textbf{references} :\\
      global constraint catalog: \href{http://www.emn.fr/x-info/sdemasse/gccat/Ccount.html}{count}
\end{itemize}

\textbf{Example}:
\begin{lstlisting}
	Model m = new CPModel();
	Solver s = new CPSolver();
	
	IntegerVariable x1 = makeIntVar("X1", 0, 10);
	IntegerVariable x2 = makeIntVar("X2", 0, 10);
	IntegerVariable x3 = makeIntVar("X3", 0, 10);
	IntegerVariable x4 = makeIntVar("X4", 0, 10);
	IntegerVariable x5 = makeIntVar("X5", 0, 10);
	IntegerVariable x6 = makeIntVar("X6", 0, 10);
	IntegerVariable x7 = makeIntVar("X7", 0, 10);
	IntegerVariable y1 = makeIntVar("Y1", 0, 10);
	
	m.addConstraint(occurrence(3, y1, new IntegerVariable[]{x1, x2, x3, x4, x5, x6, x7}));
	
	s.read(m);
	s.solve();
\end{lstlisting} 

%\part{occurrencemax}
\label{occurrencemax}
\hypertarget{occurrencemax}{}

\section{occurrenceMax (constraint)}\label{occurrencemax:occurrencemaxconstraint}\hypertarget{occurrencemax:occurrencemaxconstraint}{}
\begin{notedef}
  \texttt{occurrenceMax}$(v,z,x)$ states that $z$ is at most equal to the number of elements in $x$ with value $v$:
$$z\le|\{i\ |\ x_i=v\}|$$   
\end{notedef}
  This is a specialization of the \texttt{globalCardinality} constraint.

\begin{itemize}
	\item \textbf{API}: \mylst{occurrenceMax(int v, IntegerVariable z, IntegerVariable... x)}
	\item \textbf{return type} : \texttt{Constraint}
	\item \textbf{options} :\emph{n/a}
	\item \textbf{favorite domain} : \emph{to complete}
	\item \textbf{references} :\\
      global constraint catalog: \href{http://www.emn.fr/x-info/sdemasse/gccat/Ccount.html}{count}
\end{itemize}

\textbf{Example}:
\begin{lstlisting}
	Model m = new CPModel();
	Solver s = new CPSolver();
	 
	IntegerVariable x1 = makeIntVar("X1", 0, 10);
	IntegerVariable x2 = makeIntVar("X2", 0, 10);
	IntegerVariable x3 = makeIntVar("X3", 0, 10);
	IntegerVariable x4 = makeIntVar("X4", 0, 10);
	IntegerVariable x5 = makeIntVar("X5", 0, 10);
	IntegerVariable x6 = makeIntVar("X6", 0, 10);
	IntegerVariable x7 = makeIntVar("X7", 0, 10);
	IntegerVariable y1 = makeIntVar("Y1", 0, 10);
	 
	m.addConstraint(occurrenceMax(3, y1, new IntegerVariable[]{x1, x2, x3, x4, x5, x6, x7}));
	 
	s.read(m);
	s.solve();
\end{lstlisting} 

%\part{occurrencemin}
\label{occurrencemin}
\hypertarget{occurrencemin}{}

\section{occurrenceMin (constraint)}\label{occurrencemin:occurrenceminconstraint}\hypertarget{occurrencemin:occurrenceminconstraint}{}
\begin{notedef}
  \texttt{occurrenceMin}$(v,z,x)$ states that $z$ is at least equal to the number of elements in $x$ with value $v$:
$$z\ge|\{i\ |\ x_i=v\}|$$   
\end{notedef}
  This is a specialization of the \texttt{globalCardinality} constraint.

\begin{itemize}
	\item \textbf{API}: \mylst{occurrenceMin(int v, IntegerVariable z, IntegerVariable... x)}
	\item \textbf{return type} : \texttt{Constraint}
	\item \textbf{options} :\emph{n/a}
	\item \textbf{favorite domain} : \emph{to complete}
	\item \textbf{references} :\\
      global constraint catalog: \href{http://www.emn.fr/x-info/sdemasse/gccat/Ccount.html}{count}
\end{itemize}

\textbf{Example}:
\begin{lstlisting}
	Model m = new CPModel();
	Solver s = new CPSolver();
	 
	IntegerVariable x1 = makeIntVar("X1", 0, 10);
	IntegerVariable x2 = makeIntVar("X2", 0, 10);
	IntegerVariable x3 = makeIntVar("X3", 0, 10);
	IntegerVariable x4 = makeIntVar("X4", 0, 10);
	IntegerVariable x5 = makeIntVar("X5", 0, 10);
	IntegerVariable x6 = makeIntVar("X6", 0, 10);
	IntegerVariable x7 = makeIntVar("X7", 0, 10);
	IntegerVariable y1 = makeIntVar("Y1", 0, 10);
	 
	m.addConstraint(occurrenceMin(3, y1, new IntegerVariable[]{x1, x2, x3, x4, x5, x6, x7}));
	 
	s.read(m);
	s.solve();
\end{lstlisting} 

%\part{oppositesign}
\label{oppositesign}
\hypertarget{oppositesign}{}

\section{oppositeSign (constraint)}\label{oppositesign:oppositesignconstraint}\hypertarget{oppositesign:oppositesignconstraint}{}


\begin{notedef}
  \texttt{oppositeSign}$(x,y)$ states that the two arguments have opposite signs:
$$xy\le 0$$
\end{notedef}

\begin{notedef}
  0 is considered as both sign, if one argument is equal to 0, the constraint is not satisfied.
\end{notedef}


\begin{itemize}
	\item \textbf{API} : \mylst{oppositeSign(IntegerExpressionVariable x, IntegerExpressionVariable y)}
	\item \textbf{return type} : \texttt{Constraint}
	\item \textbf{options} :\emph{n/a}
	\item \textbf{favorite domain} : \emph{to complete}
\end{itemize}

\textbf{Example}:
\lstinputlisting{java/coppositesign.j2t} 

%\part{or}
\label{or}
\hypertarget{or}{}

\section{or (constraint)}\label{or:orconstraint}\hypertarget{or:orconstraint}{}
\begin{notedef}
  \texttt{or}$(c_1,\ldots,c_n)$ states that at least one constraint in arguments is satisfied:
$$ c_1 \lor c_2 \lor\ldots\lor c_n$$

  \texttt{or}$(b_1,\ldots,b_n)$ states that at least one boolean variable in argument is true:
$$ (b_1=1) \lor (b_2=1) \lor\ldots\lor (b_n=1)$$
\end{notedef}

\begin{itemize}
\item \textbf{API} : 
\begin{itemize}
\item \mylst{or(Constraint... c)}
\item \mylst{or(IntegerVariable... b)}
\end{itemize}
	\item \textbf{return type} : \texttt{Constraint}
	\item \textbf{options} : \emph{n/a}
	\item \textbf{favorite domain} : \emph{n/a}
	\item \textbf{references} :\\
  global constraint catalog: \href{http://www.emn.fr/z-info/sdemasse/gccat/Cor.html}{\tt or}
\end{itemize}

\textbf{Examples:}
\begin{itemize}
	\item example1:
\end{itemize}
\lstinputlisting{java/cor1.j2t}
\begin{itemize}
	\item example2
\end{itemize}
\lstinputlisting{java/cor2.j2t}

%\part{pack}
\label{pack}
\hypertarget{pack}{}

\section{pack (constraint)}\label{pack:packconstraint}\hypertarget{pack:packconstraint}{}

\begin{notedef}
  \texttt{pack(items, load, bin, size)} states that a collection of items is packed into different bins, such that the total size of the items in each bin does not exceed the bin capacity:
$$ \mathtt{load}[b] = \sum_{i\in\mathtt{items}[b]} \mathtt{size}[i],\quad\forall \text{ bin } b $$
%and
$$ i\in\mathtt{items}[b]\ \iff\ \mathtt{bin}[i]=b,\quad\forall \text{ bin } b,\ \forall \text{ item } i $$
\end{notedef}
%\texttt{pack}$(sizes, n, )$ states a collection of items (each of them having a specific size) is packed into different bins of given capacity such that the total weight of the items in each bin does not exceed the bin capacity.
\texttt{pack} is a \href{http://www.emn.fr/x-info/sdemasse/gccat/Cbin_packing.html}{bin packing constraint} based on \cite{ShawCP04}. 

\begin{itemize}
	\item \textbf{API} :
	\begin{itemize}
		\item \mylst{pack(SetVariable[] items, IntegerVariable[] load, IntegerVariable[] bin, IntegerConstantVariable[] size, String... options)}
		\item \mylst{pack(PackModeler modeler,String... options)}: PackModeler is a high-level modeling object.
		\item \mylst{pack(int[] sizes, int nbBins, int capacity, String... options)}: build instance with PackModeler.
	\end{itemize}
	\item \textbf{Variables}:
	\begin{itemize}
		\item \texttt{SetVariable[] items: items}$[b]$ is the set of items packed into bin $b$.
		\item \texttt{IntegerVariable[] load: load}$[b]$ is the total size of the items packed into bin $b$.
		\item \texttt{IntegerVariable[] bin: bin}$[i]$ is the bin where item $i$ is packed into.
		\item \texttt{IntegerConstantVariable[] size: size}$[i]$ is the size of item $i$.
	\end{itemize}
	\item \textbf{return type} : \texttt{Constraint}
	\item \textbf{options} : 	
      \begin{itemize}
      \item \hyperlink{cpackar:cpackaroptions}{SettingType.ADDITIONAL\_RULES.getOptionName()}: additional filtering rules \emph{recommended}
      \item \hyperlink{cpackdlb:cpackdlboptions}{SettingType.DYNAMIC\_LB.getOptionName()}: feasibility tests based on dynamic lower bounds for 1D-bin packing
      \item \hyperlink{cpackfill:cpackfilloptions}{SettingType.FILL\_BIN.getOptionName()}: dominance rule: fill a bin when an item fit into pertfectly equal-sized items and bins must be equivalent
      \item \hyperlink{cpacklbe:cpacklbeoptions}{SettingType.LAST\_BINS\_EMPTY.getOptionName()}: empty bins are the last ones 
      \end{itemize}
	\item \textbf{favorite domain} : \emph{to complete}
	\item \textbf{references} :
      \begin{itemize}
      \item \cite{ShawCP04}: \emph{A constraint for bin packing}
      \item global constraint catalog: \href{http://www.emn.fr/x-info/sdemasse/gccat/Cbin_packing.html}{bin\_packing} (variant)
      \end{itemize}
\end{itemize}

\textbf{Example}:

Take a look at \emph{samples.pack} to see advanced use of the constraint.
\lstinputlisting{java/cpack_import.j2t}
\lstinputlisting{java/cpack.j2t}

%\input{chapters/Cplus.tex}
%\input{chapters/Cpower.tex}
%\part{precedencereified}
\label{precedencereified}
\hypertarget{precedencereified}{}

\section{precedenceReified (constraint)}\label{precedencereified:precedencereifiedconstraint}\hypertarget{precedencereified:precedencereifiedconstraint}{}
\begin{notedef}
  \texttt{precedenceReified}$(x_1,d,x_2,b)$ states that $x_1$ plus duration $d$ is less than or equal to $x_2$ requires boolean $b$ to be true:
  $$b\quad\iff\quad x_1 + d \le x_2$$
\end{notedef}

\begin{itemize}
	\item \textbf{API} : \mylst{precedenceReified(IntegerVariable x1, int d, IntegerVariable x2, IntegerVariable b)}
	\item \textbf{return type} : \texttt{Constraint}
	\item \textbf{options} :\emph{n/a}
	\item \textbf{favorite domain} : \emph{to complete}
\end{itemize}

\textbf{Example}:
\mylst{//TODO: complete} 

%\part{regular}
\label{regular}
\hypertarget{regular}{}

\section{regular (constraint)}\label{regular:regularconstraint}\hypertarget{regular:regularconstraint}{}
\begin{notedef}
  \texttt{regular}$(x,\mathcal{L}(\Pi))$ states that sequence $x$ is a word belonging to the regular language $\mathcal{L}(\Pi)$:
% recognized by a deterministic finite automaton (DFA) or a regular expression $\Pi$:
$$(x_1,\ldots,x_n)\in\mathcal{L}(\Pi)$$
\end{notedef}

The accepting language can be specified either by a deterministic finite automaton (DFA), a list of feasible or infeasible tuples, or a regular expression:
\begin{description}
\item[DFA:] Automaton $\Pi$ is defined on a given \emph{alphabet} $\Sigma\subseteq\Z$ by a set $Q=\{0,\ldots,m\}$ of \emph{states}, a subset $A\subseteq Q$ of \emph{final} or \emph{accepting states} and a table $\Delta\subseteq Q\!\times\!\Sigma\!\times Q$ of \emph{transitions} between states. $\Delta$ is encoded as \texttt{List<Transition>} where a Transition object $\delta=\texttt{new Transition}(q_i,\sigma,q_j)$ is made of three integers expressing the ingoing state $q_i$, the label $\sigma$, and the outgoing state $q_j$.
Automaton $\Pi$ is a DFA if $\Delta$ is finite and if it has only one initial state (here, state $0$ is considered as the unique initial state) and no two transitions sharing the same ingoing state and the same label.
\item[FiniteAutomaton:] is another API for building a DFA (manually, or from a regular expression, or from a \mylst{dk.brics.Automaton}) and operating on them (intersection, union, complement) in a more flexible way. Using this API leads to another implementation of the constraint: \mylst{FastRegular}. See \hyperlink{costregular}{\texttt{costRegular}} for a short API of \texttt{FiniteAutomaton}. 
\item[feasible tuples:] \emph{regular} can be used as an extensional constraint. Given the list of \emph{feasible} tuples for sequence $x$, this API builds a DFA from the list, and then enforces GAC on the constraint. Using \texttt{regular} can be more efficient than a standard GAC algorithm on tables of tuples if the tuples are structured so that the resulting DFA is compact. The DFA is built from the list of tuples by computing incrementally the minimal DFA after each addition of tuple. 
\item[infeasible tuples:] An another API allows to specify the list of \emph{infeasible} tuples and then builds the corresponding feasible DFA. This operation requires to know the entire alphabet, hence this API has two mandatory table fields \emph{min} and \emph{max} defining the minimum and maximum values of each variable $x_i$.
\item[regular expression:] Finally, the \texttt{regular} constraint can be based on a \href{http://en.wikipedia.org/wiki/regularexpression}{regular expression}, such as \mylst{String regexp = "(1\|2)3\{4\}5*";}. This expression recognizes any sequences starting by one 1 or one 2, then four consecutive 3 followed by any (possibly empty) sequences of 5.
\end{description}

\todo{Warning ! DFA and FiniteAutomaton are both based on the dk.brics library. The construction of these objects is non-deterministic and the order the filtering occur (not the result) may vary at each execution. This may results in different first solutions when branching dynamically using  weighted degrees-base heuristics for example.}

\begin{itemize}
	\item \textbf{API} :
	\begin{itemize}
		\item \mylst{regular(IntegerVariable[] x, FiniteAutomaton pi)}
		\item \mylst{regular(IntegerVariable[] x, DFA pi)}
		\item \mylst{regular(IntegerVariable[] x, List<int[]> feasTuples)}
		\item \mylst{regular(IntegerVariable[] x, List<int[]> infeasTuples, int[] min, int[] max)}
		\item \mylst{regular(IntegerVariable[] x, String regexp)}
	\end{itemize}
	\item \textbf{return type} : \texttt{Constraint}
	\item \textbf{options} :\emph{n/a}
	\item \textbf{favorite domain} : \emph{to complete}
	\item \textbf{references} :\\
       \cite{PesantCP04}: \emph{A regular language membership constraint}
\end{itemize}

\textbf{Examples}:
\begin{itemize}
	\item example with \texttt{FiniteAutomaton}: see \hyperlink{costregular:costregularconstraint}{\texttt{costRegular}}.
	\item example 1 with DFA:
\end{itemize}
\lstinputlisting{java/cregular1_import.j2t}
\lstinputlisting{java/cregular1.j2t}

\begin{itemize}
	\item example 2 with feasible tuples:
\end{itemize}
\lstinputlisting{java/cregular2.j2t}

\begin{itemize}
	\item example 3 with regular expression:
\end{itemize}
\lstinputlisting{java/cregular3.j2t}

%\part{reifiedintconstraint}
\label{reifiedintconstraint}
\hypertarget{reifiedintconstraint}{}

\section{reifiedIntConstraint (constraint)}\label{reifiedintconstraint:reifiedintconstraintconstraint}\hypertarget{reifiedintconstraint:reifiedintconstraintconstraint}{}
\begin{notedef}
  \begin{itemize}
  \item \texttt{reifiedIntConstraint}$(b,c)$ states that boolean $b$ is true if and only if constraint $c$ holds:
  $$b\ \iff\ c$$
  \item \texttt{reifiedIntConstraint}$(b,c_1,c_2)$ states that boolean $b$ is true if and only if $c_1$ holds, and $b$ is false if and only if $c_2$ holds ($c_2$ must be the opposite constraint of $c_1$):
$$(b\land c_1) \lor (\neg b \land c_2)$$
  \end{itemize}
\end{notedef}

\begin{itemize}
	\item \textbf{API} :
	\begin{itemize}
		\item \mylst{reifiedIntConstraint(IntegerVariable b, Constraint c)}
		\item \mylst{reifiedIntConstraint(IntegerVariable b, Constraint c1, Constraint c2)}
	\end{itemize}
	\item \textbf{return type} : \texttt{Constraint}
	\item \textbf{options} : \emph{n/a}
	\item \textbf{favorite domain} : \emph{n/a}
\end{itemize}

Parameter \emph{b} is a boolean variable (enumerated domain with two values $\{0,1\}$) and \emph{c} is a constraint over Integer variables.

The constraint $c$ to reify has to provide its opposite (the opposite is needed for propagation). Most basic constraints of Choco provides their opposite by default, and can then be reified using the first API.
The second API attends to reify user-defined constraints as it allows the user to directly specify the opposite constraint.

\textbf{Example}:

\lstinputlisting{java/creifiedintconstraint.j2t}

%\part{relationpairac}
\label{relationpairac}
\hypertarget{relationpairac}{}

\section{relationPairAC (constraint)}\label{relationpairac:relationpairacconstraint}\hypertarget{relationpairac:relationpairacconstraint}{}
\begin{notedef}
  \texttt{relationPairAC}$(x,y,rel)$ states an extensional binary constraint on $(x,y)$ defined by the binary relation $rel$:
$$(x,y)\in rel$$
\end{notedef}
Many constraints of the same kind often appear in a model. Relations can therefore often be shared among many constraints to spare memory.

The API is duplicated to allow definition of options.

\begin{itemize}
	\item \textbf{API} :
	\begin{itemize}
		\item \mylst{relationPairAC(IntegerVariable x, IntegerVariable y, BinRelation rel)}
		\item \mylst{relationPairAC(String options, IntegerVariable x, IntegerVariable y, BinRelation rel)}
	\end{itemize}
	\item \textbf{return type} : \texttt{Constraint}
	\item \textbf{options} :
	\begin{itemize}
		\item \emph{no option} : use AC3 (default arc consistency)
		\item \texttt{CPOptions.C_EXT_AC3}: to get AC3 algorithm (searching from scratch for supports on all values)
		\item \texttt{CPOptions.C_EXT_AC2001}: to get AC2001 algorithm (maintaining the current support of each value)
		\item \texttt{CPOptions.C_EXT_AC32}: to get AC3rm algorithm (maintaining the current support of each value in a non backtrackable way)
		\item \texttt{CPOptions.C_EXT_AC322}: to get AC3 with the used of \texttt{BitSet} to know if a support still exists
	\end{itemize}
	\item \textbf{favorite domain} : \emph{to complete}
\end{itemize}

\textbf{Example}:
\lstinputlisting{java/crelationpairac_import.j2t}
\lstinputlisting{java/ccoupletest.j2t}
\lstinputlisting{java/crelationpairac.j2t}

%\part{relationtupleac}
\label{relationtupleac}
\hypertarget{relationtupleac}{}

\section{relationTupleAC (constraint)}\label{relationtupleac:relationtupleacconstraint}\hypertarget{relationtupleac:relationtupleacconstraint}{}
\begin{notedef}
  \texttt{relationTupleAC}$(x,rel)$ states an extensional constraint on $(x_1,\ldots,x_n)$ defined by the $n$-ary relation $rel$, and then enforces arc consistency:
$$(x_1,\ldots,x_n)\in rel$$
\end{notedef}
Many constraints of the same kind often appear in a model. Relations can therefore often be shared among many constraints to spare memory.
The API is duplicated to define options.

\begin{itemize}
	\item \textbf{API}:
	\begin{itemize}
		\item \mylst{relationTupleAC(IntegerVariable[] x, LargeRelation rel)}
		\item \mylst{relationTupleAC(String options, IntegerVariable[] x, LargeRelation rel)}
	\end{itemize}
	\item \textbf{return type}: \texttt{Constraint}
	\item \textbf{options} :
	\begin{itemize}
		\item \emph{no option}: use AC32 (default arc consistency)
		\item \texttt{cp:ac32}: to get AC3rm algorithm (maintaining the current support of each value in a non backtrackable way)
		\item \texttt{cp:ac2001}: to get AC2001 algorithm (maintaining the current support of each value)
		\item \texttt{cp:ac2008}: to get AC2008 algorithm (maintained by STR)
	\end{itemize}
	\item \textbf{favorite domain} : \emph{to complete}
\end{itemize}

\textbf{Example} :
\mylst{// TODO : add example}

%\part{relationtuplefc}
\label{relationtuplefc}
\hypertarget{relationtuplefc}{}

\section{relationTupleFC (constraint)}\label{relationtuplefc:relationtuplefcconstraint}\hypertarget{relationtuplefc:relationtuplefcconstraint}{}
\begin{notedef}
  \texttt{relationTupleFC}$(x,rel)$ states an extensional constraint on $(x_1,\ldots,x_n)$ defined by the $n$-ary relation $rel$, and then enforces forward checking:
$$(x_1,\ldots,x_n)\in rel$$
\end{notedef}
Many constraints of the same kind often appear in a model. Relations can therefore often be shared among many constraints to spare memory.

\begin{itemize}
	\item \textbf{API}: \mylst{relationTupleFC(IntegerVariable[] x, LargeRelation rel)}
	\item \textbf{return type}: \texttt{Constraint}
	\item \textbf{options} : \emph{n/a}
	\item \textbf{favorite domain} : \emph{to complete}
\end{itemize}

\textbf{Example} :
\mylst{// TODO : add example}

%\part{samesign}
\label{samesign}
\hypertarget{samesign}{}

\section{sameSign (constraint)}\label{samesign:samesignconstraint}\hypertarget{samesign:samesignconstraint}{}
\todo{verify case 0}

\begin{notedef}
  \texttt{sameSign}$(x,y)$ states that the two arguments have the same sign:
$$xy\ge 0$$
\end{notedef}

\begin{itemize}
	\item \textbf{API} : \mylst{sameSign(IntegerExpressionVariable x, IntegerExpressionVariable y)}
	\item \textbf{return type} : \texttt{Constraint}
	\item \textbf{options} :\emph{n/a}
	\item \textbf{favorite domain} : \emph{to complete}
\end{itemize}

\textbf{Example}:
\lstinputlisting{java/csamesign.j2t}

%\input{chapters/Cscalar.tex}
%\part{setdisjoint}
\label{setdisjoint}
\hypertarget{setdisjoint}{}

\section{setDisjoint (constraint)}\label{setdisjoint:setdisjointconstraint}\hypertarget{setdisjoint:setdisjointconstraint}{}
\begin{notedef}
  \texttt{setDisjoint}$(s_1,s_2)$ states that the two set arguments are disjoint:
$$s_1\cap s_2=\emptyset$$
\end{notedef}

\begin{itemize}
	\item \textbf{API} : \mylst{setDisjoint(SetVariable s1, SetVariable s2)}
	\item \textbf{return type} : \texttt{Constraint}
	\item \textbf{options} :\emph{n/a}
	\item \textbf{favorite domain} : \emph{to complete}
\end{itemize}

\textbf{Example}:
\begin{lstlisting}
	Model m = new CPModel();
	Solver s = new CPSolver();
	SetVar x = makeSetVar("X", 1, 3);
	SetVar y = makeSetVar("Y", 1, 3);
	Constraint c1 = setDisjoint(x, y);
	m.addConstraint(c1);
	s.read(m);
	s.solveAll();
\end{lstlisting} 

%\part{setinter}
\label{setinter}
\hypertarget{setinter}{}

\section{setInter (constraint)}\label{setinter:setinterconstraint}\hypertarget{setinter:setinterconstraint}{}
\begin{notedef}
  \texttt{setInter}$(s_1,s_2,s_3)$ states that the third set $s_3$ is exactly the intersection of the two first sets:
$$s_1\cap s_2=s_3$$
\end{notedef}

\begin{itemize}
	\item \textbf{API} : \mylst{setInter(SetVariable s1, SetVariable s2, SetVariable s3)}
	\item \textbf{return type} : \texttt{Constraint}
	\item \textbf{options} :\emph{n/a}
	\item \textbf{favorite domain} : \emph{to complete}
\end{itemize}

\textbf{Example}:
\lstinputlisting{java/csetinter.j2t}
%\part{setunion}
\label{setunion}
\hypertarget{setunion}{}

\section{setUnion (constraint)}\label{setunion:setunionconstraint}\hypertarget{setunion:setunionconstraint}{}
\begin{notedef}
  \texttt{setUnion}$(sv,s_{union})$ states that the $s_{union}$ set is exactly the union of the sets $sv$:
$$sv_1\cup sv_2 \cup \ldots sv_i \cup sv_{i+1} \ldots \cup sv_n=s_{union}$$
\end{notedef}

\begin{itemize}
	\item \textbf{API} : 
	\begin{itemize}
		\item \mylst{setUnion(SetVariable s1, SetVariable s2, SetVariable union)}
		\item \mylst{setUnion(SetVariable[] sv, SetVariable union)}
	\end{itemize}
	\item \textbf{return type} : \texttt{Constraint}
	\item \textbf{options} :\emph{n/a}
	\item \textbf{favorite domain} : \emph{to complete}
\end{itemize}

\textbf{Example}:
\lstinputlisting{java/csetunion.j2t}
%\input{chapters/Csin.tex}
%\part{sorting}
\label{sorting}
\hypertarget{sorting}{}

\section{sorting (constraint)}\label{sorting:sortingconstraint}\hypertarget{sorting:sortingconstraint}{}

%%\part{stretchcyclic}
\label{stretchcyclic}
\hypertarget{stretchcyclic}{}

\section{stretchCyclic (constraint)}\label{stretchcyclic:stretchcyclicconstraint}\hypertarget{stretchcyclic:stretchcyclicconstraint}{}
\(stretchCyclic\) states that ... \emph{to complete}.

\begin{itemize}
	\item \textbf{API} :
	\item \textbf{return type} : Constraint
	\item \textbf{options} :\emph{n/a}
	\item \textbf{favorite domain} : \emph{to complete}
\end{itemize}

\textbf{Example}:
\mylst{} 

%\part{stretchpath}
\label{stretchpath}
\hypertarget{stretchpath}{}

\section{stretchPath (constraint)}\label{stretchpath:stretchpathconstraint}\hypertarget{stretchpath:stretchpathconstraint}{}


\begin{notedef}
  A \emph{stretch} in a sequence $x$ is a maximum subsequence of (consecutive) identical values.  
  \texttt{stretchPath}$(param,x)$ enforces the minimal and maximal length of the stretches in sequence $x$ of any values given in $param$:
Consider the sequence $x$ as a concatenation of stretches $x^1.x^2\ldots x^k$ with $v^i$ and $l^i$ being respectively the value and the length of stretch $x^i$,
$$\forall i\in\{1,\ldots,k\},\ \forall j,\quad param[j][0]=v^i\quad\implies\quad param[j][1]\le l^i\le param[j][2]$$
\end{notedef}


Useful for Rostering Problems. \texttt{stretchPath} is implemented by a \hyperlink{regular}{\texttt{regular}} constraint that performs GAC. The bounds on the stretch lengths are defined by $param$ a list of triples of integers: $[value, min, max]$ specifying the minimal and maximal lengths of any stretch of the corresponding value. 

This API requires a Java library on automaton available on \href{http://www.brics.dk/automaton/}{http://www.brics.dk/automaton/}. (It is contained in the Choco jar file.)

\begin{itemize}
	\item \textbf{API} : \mylst{stretchPath(List<int[]> param, IntegerVariable... x)}
	\item \textbf{return type} : Constraint
	\item \textbf{options} :\emph{n/a}
	\item \textbf{favorite domain} : \emph{to complete}
	\item \textbf{references} :
      \begin{itemize}
      \item \cite{PesantCP04}: \emph{A regular language membership constraint}
      \item global constraint catalog: \href{http://www.emn.fr/x-info/sdemasse/gccat/Cstretch_path.html}{stretch\_path}
      \end{itemize}
\end{itemize}


\textbf{Example}:
\begin{lstlisting}
  Model m = new CPModel();
  int n = 7;
  IntegerVariable[] vars = makeIntVarArray("v", n, 0, 2);
	
  //define the stretches
  ArrayList<int[]> lgt = new ArrayList<int[]>();
  lgt.add(new int[]{0, 2, 2}); // stretches of value 0 are of length 2
  lgt.add(new int[]{1, 2, 3}); // stretches of value 1 are of length 2 or 3
  lgt.add(new int[]{2, 2, 2}); // stretches of value 2 are of length 2
	
  m.addConstraint(stretchPath(lgt, vars));
	
  Solver s = new CPSolver();
  s.read(m);
  s.solve();
\end{lstlisting}

%\input{chapters/Csum.tex}
%\part{times}
\label{times}
\hypertarget{times}{}

\section{times (constraint)}\label{times:timesconstraint}\hypertarget{times:timesconstraint}{}
\begin{notedef}
  \texttt{times}$(x_1, x_2, x_3)$ states that the third argument is equal to the product of the two arguments:
$$x_3=x_1\times x_2.$$
\end{notedef}

\begin{itemize}
	\item \textbf{API}:
	\begin{itemize}
		\item \mylst{times(IntegerVariable x1, IntegerVariable x2, IntegerVariable x3)}
		\item \mylst{times(int x1, IntegerVariable x2, IntegerVariable x3)}
		\item \mylst{times(IntegerVariable x1, int x2, IntegerVariable x3)}
	\end{itemize}
	\item \textbf{return type} : \texttt{Constraint}
	\item \textbf{option} : \emph{n/a}
	\item \textbf{favorite domain}: bound
\end{itemize}

\textbf{Example}:
\lstinputlisting{java/ctimes.j2t}
%\part{tree}
\label{tree}
\hypertarget{tree}{}

\section{tree (constraint)}\label{tree:treeconstraint}\hypertarget{tree:treeconstraint}{}

Let $G=(V,A)$ be a digraph on $V=\{1,\ldots,n\}$. $G$ can be modeled by a sequence of domain variables $x=(x_1,\dots,x_n)\in V^n$ -- the \emph{successors} variables -- whose respective domains are given by $D_i=\{j\in V\ |\ (i,j)\in A\}$. Conversely, when instantiated, $x$ defines a subgraph $G_x=(V,A_x)$ of $G$ with $A_x=\{(i,x_i)\ |\ i\in V\}\subseteq A$. Such a subgraph has one particularity: any connected component of $G_x$ contains either no loop -- and then it contains a cycle -- or exactly one loop $x_i=i$ and then it is a \emph{tree} of root $i$ (literally, it is an anti-arborescence as there exists a path from each node to $i$ and $i$ has a loop).

\begin{notedef}
  \texttt{tree}$(x,restrictions)$ is a vertex-disjoint graph partitioning constraint. It states that $G_x$ is a forest (its connected components are trees) that satisfies some conditions specified by $restrictions$.
\texttt{tree} deals with several kinds of graph restrictions on:
\begin{itemize}
	\item the number of trees
	\item the number of proper trees (a tree is proper if it contains more than 2 nodes)
    \item the weigth of the partition: the sum of the weights of the edges
	\item incomparability: some nodes in pairs have to belong to distinct trees
	\item precedence: some nodes in pairs have to belong to the same tree in a given order
	\item conditional precedence: some nodes in pairs have to respect a given order if they belong to the same tree
	\item the in-degree of the nodes
	\item the time windows on nodes (given travelling times on arcs)
\end{itemize}
\end{notedef}

Many applications require to partition a graph such that each component contains exactly one \emph{resource} node and several \emph{task} nodes. A typical example is a routing problem where vehicle routes are paths (a path is a special case of tree) starting from a depot and delivering goods to several clients. Another example is a local network where each computer has to be connected to one shared printer. Last, one can cite the problem of reconstructing plylogeny trees.
The constraint \texttt{tree} can handle these kinds of problems with many additional constraints on the structure of the partition.

\begin{itemize}
	\item \textbf{API} : \mylst{tree(TreeParametersObject param)}
	\item \textbf{return type} : \texttt{Constraint}
	\item \textbf{options} :\emph{n/a}
	\item \textbf{favorite domain} : \emph{to complete}
	\item \textbf{references} :
      \begin{itemize}
      \item \cite{beldiceanuCONSTRAINTS08}: \emph{Combining tree partitioning, precedence, and incomparability constraints}
      \item global constraint catalog: \href{http://www.emn.fr/x-info/sdemasse/gccat/Cproper_forest.html}{proper\_forest} (variant)
      \end{itemize}

\end{itemize}

The tree constraint API requires a particular Model object, named \textbf{\tt TreeParametersObject}.
It can be created with the following parameters:

\begin{tabular}{p{3cm}p{3cm}p{7cm}}
parameter &type &description\\
\hline
$n$ &\texttt{int} &number of nodes in the initial graph $G$\\
$nTree$ &\texttt{IntegerVariable} &number of trees in the resulting forest $G_x$\\
$nProper$ &\texttt{IntegerVariable} &number of proper trees in $G_x$\\
$objective$ &\texttt{IntegerVariable} &(bounded) total \todo{cost} of $G_x$\\
%$objective$ &\texttt{IntegerVariable} &(bounded) total weight of $G_x$\\
$graphs$ &\texttt{List<BitSet[]>} &
\begin{minipage}[t]{7cm}
graphs encoded as successor lists,\\
  \texttt{graphs[0]} the initial graph $G$,\\
  \texttt{graphs[1]} a precedence graph,\\
  \texttt{graphs[2]} a conditional precedence graph,\\
  \texttt{graphs[3]} an incomparability graph
\end{minipage}\\
$matrix$ &\texttt{List<int[][]>} &\texttt{matrix[0]} the indegree of each node, and \texttt{matrix[1]} the starting time from each node\\
$travel$ &\texttt{int[][]} &the travel time of each arc
\end{tabular}

\textbf{Example}:
\lstinputlisting{java/ctree_import.j2t}
\lstinputlisting{java/ctree.j2t}

%\part{true}
\label{true}
\hypertarget{true}{}

\section{TRUE (constraint)}\label{true:trueconstraint}\hypertarget{true:trueconstraint}{}
\(TRUE\) always returns \emph{true}.

%\part{xnor}
\label{xnor}
\hypertarget{xnor}{}

\section{xnor (constraint)}\label{xnor:xnorconstraint}\hypertarget{xnor:xnorconstraint}{}
\begin{notedef}
    \texttt{xnor}$(b_1,b_2)$ states that two booleans are either both true, or both false:
$$ (b_1=1)\quad\iff\quad(b_2=1)$$
\end{notedef}

\begin{itemize}
    \item \textbf{API} : \mylst{xnor(IntegerVariable b1, IntegerVariable b2)}
	\item \textbf{return type} : \texttt{Constraint}
	\item \textbf{options} : \emph{n/a}
	\item \textbf{favorite domain} : \emph{n/a}
\end{itemize}

\textbf{Examples:}
\lstinputlisting{java/cxnor.j2t}

%\part{xor}
\label{xor}
\hypertarget{xor}{}

\section{xor (constraint)}\label{xor:xorconstraint}\hypertarget{xor:xorconstraint}{}
\begin{notedef}
    \texttt{xor}$(b_1,b_2)$ states that the 0-1 variables in arguments take distinct value:
$$ (b_1=1 \land b_2=0) \lor (b_1=0 \land b_2=1)$$
\end{notedef}

\begin{itemize}
    \item \textbf{API} : \mylst{xor(IntegerVariable... b)}
	\item \textbf{return type} : \texttt{Constraint}
	\item \textbf{options} : \emph{n/a}
	\item \textbf{favorite domain} : \emph{n/a}
\end{itemize}

\textbf{Examples:}
\lstinputlisting{java/cxor.j2t}


\part{Tutorials}\label{ch:tut}\hypertarget{ch:tut}{}
If you look for an easy step-by-step program in CHOCO, \hyperlink{gettingstarted}{getting\ started} is for you! It introduces to basic concepts of a CHOCO program (Model, Solver, variables and constraints...).

This part also presents a collection of \hyperlink{exercises}{exercises} with their \hyperlink{solutions}{solutions}. It covers simple to advanced uses of CHOCO.

\emph{See also old pages:} \url{http://choco.sourceforge.net/tut\_expl.html}

%!TEX root = ../content-tut.tex
%\part{getting started}
\label{gettingstarted}
\hypertarget{gettingstarted}{}

\chapter{Getting started: welcome to Choco}\label{gettingstarted:gettingstarted:welcometochoco}\hypertarget{gettingstarted:gettingstarted:welcometochoco}{}
%This introduction covers the basics of writing a program in Choco

%Choco is a java library for constraint satisfaction problems (CSP), constraint programming (CP) and explanation-based constraint solving (e-CP). It is built on a event-based propagation mechanism with backtrackable structures. 

\section{Before starting}\label{gettingstarted:beforestarting}\hypertarget{gettingstarted:beforestarting}{}

Before doing anything, you have to be sure that 
\begin{itemize}
	\item you have at least \href{http://java.sun.com/javase/6/}{Java6} installed on your environment.
	\item you have a IDE (like \href{http://www.jetbrains.com/idea/}{IntelliJ IDEA} or \href{http://www.eclipse.org/}{Eclipse}).
\end{itemize}

To install Java6 or your IDE, please refer to its specific documentation. We now assume that you have the previously defined environment.

You need to create a \textbf{New Project...} on your favorite IDE (\href{http://www.jetbrains.com/idea/training/demos.html}{create a new project on IntelliJ}, \href{https://eclipse-tutorial.dev.java.net/eclipse-tutorial/part1.html}{create a new project on Eclipse}). Our project name is \emph{ChocoProgram}.
Create a new class, named \emph{MyFirstChocoProgram}, with a main method.
\begin{lstlisting}
	public class MyFirstChocoProgram {
	
	    public static void main(String[] args) {
	        
	    }
	}
\end{lstlisting}

\section{Download Choco}\label{gettingstarted:downloadchoco}\hypertarget{gettingstarted:downloadchoco}{}
Now, before doing anything else, you need to download the last stable version of Choco. See the \href{http://www.emn.fr/z-info/choco-solver/choco-download.html}{download page}. 
Once you have download choco, you need to add it to the classpath of your project.

Now you are ready to create you first Choco program.

If you want a short introduction on what is constraint programming, you can find some informations in the \href{http://www.emn.fr/z-info/choco-solver/choco-documentation.html}{Documentation of choco}.\\
When you feel ready, solve your own problem! And if you need more tries, please take a look at the \hyperlink{exercises}{exercises}. %and \hyperlink{examples}{examples}. 

\chapter{First Example: Magic square}\label{gettingstarted:firstexample:magicsquare}\hypertarget{gettingstarted:firstexample:magicsquare}{}
A simple magic square of order 3 can be seen as the ``Hello world!'' program in Choco. First of all, we need to agree on the definition of a magic square of order 3.
\href{http://en.wikipedia.org/wiki/Magic_square}{Wikipedia} tells us that :
\begin{myquote}
A \textbf{magic square} of order $n$ is an arrangement of $n^2$ numbers, usually distinct integers, in a square, such that the $n$ numbers in all rows, all columns, and both diagonals sum to the same constant. A normal magic square contains the integers from 1 to $n^2$.
\end{myquote}

So we are going to solve a problem where unknows are cells value, knowing that each cell can take its value between 1 and $n^2$, is different from the others and columns, diagonals and rows are equal to the same constant M (which is equal to $n * (n^2 + 1) / 2$).

We have the definition, let see how to add some Choco in it.

\section{First, the model}\label{gettingstarted:first,themodel}\hypertarget{gettingstarted:first,themodel}{}
To define our problem, we need to create a Model object. As we want to solve our problem with constraint programming (of course, we do), we need to create a CPModel.
\begin{lstlisting}
	//constants of the problem:
	int n = 3;
	int M = n*(n*n+1)/2;
	
	// Our model
	Model m = new CPModel();
\end{lstlisting}
These objects require to import the following classes:
\begin{lstlisting}
	import choco.cp.model.CPModel;
	import choco.kernel.model.Model;
\end{lstlisting}

At the begining, our model is empty, no problem has been defined explicitly. A model is composed of variables and constraints, and constraints link variables to each others.

\begin{itemize}
\item 
\textbf{Variables}\label{gettingstarted:variables}\hypertarget{gettingstarted:variables}{}
A variable is an object defined by a name, a type and a domain. We know that our unknowns are cells of the magic square. So:
\mylst{IntegerVariable cell = Choco.makeIntVar("aCell", 1, n*n);}
which means that \emph{aCell} is an integer variable, and its domain is defined from \emph{1} to \emph{n*n}.
But we need $n^2$ variables, so the easiest way to define them is:
\begin{lstlisting}
	IntegerVariable[][] cells = new IntegerVariable[n][n];
	for(int i = 0; i < n; i++){
	   for(int j = 0; j < n; j++){
	      cells[i][j] = Choco.makeIntVar("cell"+j, 1, n*n); 
	      m.addVariables(cells[i][j]);
	   }
	}
\end{lstlisting}
This code requires to import the following classes:
\begin{lstlisting}
	import choco.kernel.model.variables.integer.IntegerVariable;
	import choco.Choco;
\end{lstlisting}
We add each variables to our model:
\mylst{m.addVariables(cells[i][j]);}\\
Now that our variables are defined, we have to define the constraints between variables.
\item
\textbf{Constraints over the rows}\label{gettingstarted:constraintsovertherows}\hypertarget{gettingstarted:constraintsovertherows}{}
The sum of ach rows is equal to a constant $M$.
So we need a sum operator and and equality constraint. The both are provides by the \texttt{Choco.java} class.
\begin{lstlisting}
	//Constraints
	// ... over rows
	Constraint[] rows = new Constraint[n];
	for(int i = 0; i < n; i++){
	   rows[i] = Choco.eq(Choco.sum(cells[i]), M);
	}
\end{lstlisting}
This part of code requires the following import:
\begin{lstlisting}
  import choco.kernel.model.constraints.Constraint;
\end{lstlisting}
After the creation of the constraints, we need to add them to the model:
\begin{lstlisting}
  m.addConstraints(rows);
\end{lstlisting}
\item
\textbf{Constraints over the columns}\label{gettingstarted:constraintsoverthecolumns}\hypertarget{gettingstarted:constraintsoverthecolumns}{}
Now, we need to declare the equality between the sum of each column and $M$.
But, the way we have declare our variables matrix does not allow us to deal easily with it in the column case. So we create the transposed matrix (a $90^o$ rotation of the matrix) of \emph{cells}.
\begin{note}
We do not introduce new variables. We just reorder the matrix to see the \emph{column point of view}.
\end{note}
\begin{lstlisting}
	//... over columns
	// first, get the columns, with a temporary array
	IntegerVariable[][] cellsDual = new IntegerVariable[n][n];
	for(int i = 0; i < n; i++){
	   for(int j = 0; j < n; j++){
	      cellsDual[i][j] = cells[j][i];
	   }
	}
\end{lstlisting}
Now, we can declare the constraints as before:
\begin{lstlisting}
	Constraint[] cols = new Constraint[n];
	for(int i = 0; i < n; i++){
	   cols[i] = Choco.eq(Choco.sum(cellsDual[i]), M);
	}
\end{lstlisting}
And we add them to the model:
\begin{lstlisting}
  m.addConstraints(cols);
\end{lstlisting}
\item
\textbf{Constraints over the diagonals}\label{gettingstarted:constraintsoverthediagonals}\hypertarget{gettingstarted:constraintsoverthediagonals}{}
Now, we get the two diagonals array \emph{diags}, reordering the required \emph{cells} variables, like in the previous step.
\begin{lstlisting}
	//... over diagonals                                  
	IntegerVariable[][] diags = new IntegerVariable[2][n];
	for(int i = 0; i < n; i++){                           
	    diags[0][i] = cells[i][i];                        
	    diags[1][i] = cells[i][(n-1)-i];                  
	}
\end{lstlisting} 
And we add the constraints to the model (in one step this time).
\begin{lstlisting}
	m.addConstraint(Choco.eq(Choco.sum(diags[0]), M));    
	m.addConstraint(Choco.eq(Choco.sum(diags[1]), M));
\end{lstlisting}
\item
\textbf{Constraints of variables AllDifferent}\label{gettingstarted:constraintsofvariablesalldifferent}\hypertarget{gettingstarted:constraintsofvariablesalldifferent}{}
Finally, we add the AllDifferent constraints, stating that each \emph{cells} variables takes a unique value. 
One more time, we have to reorder the variables, introducing temporary array.
\begin{lstlisting}
	//All cells are differents from each other           
	IntegerVariable[] allVars = new IntegerVariable[n*n];
	for(int i = 0; i < n; i++){                          
	    for(int j = 0; j < n; j++){                      
	        allVars[i*n+j] = cells[i][j];                
	    }                                                
	}                                                    
	m.addConstraint(Choco.allDifferent(allVars));
\end{lstlisting}
\end{itemize}

\section{Then, the solver}\label{gettingstarted:then,thesolver}\hypertarget{gettingstarted:then,thesolver}{}
Our model is established, it does not require any other information, we can focus on the way to solve it.
The first step is to create a Solver;
\begin{lstlisting}
	//Our solver              
	Solver s = new CPSolver();
\end{lstlisting}
This part requires the following imports:
\begin{lstlisting}
	import choco.kernel.solver.Solver;
	import choco.cp.solver.CPSolver;
\end{lstlisting}

After that, the model and the solver have to be linked, thus the solver \emph{read} the model, to extract informations:
\begin{lstlisting}
	//read the model
	s.read(m);
\end{lstlisting}

Once it is done, we just need to solve it:
\begin{lstlisting}
	//solve the problem
	s.solve();
\end{lstlisting}
And print the information
\begin{lstlisting}
	//Print the values                                           
	for(int i = 0; i < n; i++){                                  
	    for(int j = 0; j < n; j++){                              
	        System.out.print(s.getVar(cells[i][j]).getVal()+" ");
	    }                                                        
	    System.out.println();                                    
	}
\end{lstlisting}

\section{Conclusion}\label{gettingstarted:conclusion}\hypertarget{gettingstarted:conclusion}{}
We have seen, in a few steps, how to solve a basic problem using constraint programming and Choco. Now, you are ready to solve your own problem, and if you need more tries, please take a look at the \hyperlink{exercises}{exercises}. %and \hyperlink{examples}{examples}. 

%\part{exercises}
\label{exercises}
\hypertarget{exercises}{}

\chapter{Exercises}\label{exercises:exercises}\hypertarget{exercises:exercises}{}

\section{I'm new to CP}\label{exercises:i'mnewtocp}\hypertarget{exercises:i'mnewtocp}{}

\begin{note}
\textbf{The goal of this practical work is twofold :}
\begin{itemize}
	\item \textbf{problem modelling} with the help of variables and constraints ;
	\item \textbf{mastering the syntax} of Choco in order to tackle basic problems.
\end{itemize}

\textbf{Warning}, constraint modelling should not be solver specific. That is why you are strongly advised to write down your model before starting its implementation within Choco. %The constraint factories in Choco are static methods of the Choco API.

\end{note}

\subsection{Exercise 1.1 (A soft start)}\label{exercises:exercise1.1}\hypertarget{exercises:exercise1.1}{}

Algorithm 1 (below) describes a problem which fits the minimum choco syntax requirements.
\begin{description}
	\item[Question 1] describe the constraint network modelled in Algorithm 1.
	\item[Question 2] give the variable domains after constraint propagation.
\end{description}

\begin{lstlisting}[language={java}, title={\textbf{Algorithm 1} Mysterious model.}]
	// Build a model
	Model m = new CPModel() ;
	
	// Build enumerated domain variables
	IntegerVariable x1 = makeIntVar("var1", 0, 5);
	IntegerVariable x2 = makeIntVar("var2", 0, 5);
	IntegerVariable x3 = makeIntVar("var3", 0, 5);
	
	// Build the constraints
	Constraint C1 = gt(x1, x2) ;
	Constraint C2 = neq(x1, x3) ;
	Constraint C3 = gt(x2, x3) ;
	
	// Add the constraints to the Choco model
	m.addConstraint(C1) ;
	m.addConstraint(C2) ;
	m.addConstraint(C3) ;
	
	// Build a solver
	Solver s = new CPSolver();
	
	// Read the model
	s.read(m);
	
	// Solve the problem
	s.solve() ;
	
	// Print the variable domains
	System.out.println("var1 =" + s.getVar(x1) .getVal()) ;
	System.out.println("var2 =" + s.getVar(x2) .getVal()) ;
	System.out.println("var3 =" + s.getVar(x3) .getVal()) ;
\end{lstlisting}

(\hyperlink{solutions:solutionofexercise1.1}{Solution})

\subsection{Exercise 1.2 (DONALD +  GERALD = ROBERT)}\label{exercises:exercise1.2}\hypertarget{exercises:exercise1.2}{}
Associate a different digit to every letter so that the equation DONALD + GERALD = ROBERT is verified.

(\hyperlink{solutions:solutionofexercise1.2}{Solution})

\subsection{Exercise 1.3 (A famous example. . . a sudoku grid)}\label{exercises:exercise1.3}\hypertarget{exercises:exercise1.3}{}
A sudoku grid is a square composed of nine squares called \emph{blocks}. Each block is itself composed of 3x3 cells
(see figure 1). The purpose of the game is to fill the grid so that each block, column and row contains all the
numbers from 1 to 9 once and only once

\begin{description}
	\item[Question 1] propose a way to model the sudoku problem with difference constraints. Implement your model with Choco.
	\item[Question 2] which global constraint can be used to model such a problem ? Modify your code accordingly.
	\item[Question 3] Test, for both models, the initial propagation step (use Choco \texttt{propagate()} method). What can be noticed ? What is the point in using global constraints ?
\end{description}

\insertGraphique{.5\linewidth}{media/sudoku-grid.jpeg}{An exemple of a Sudoku grid}

(\hyperlink{solutions:solutionofexercise1.3}{Solution})

\subsection{Exercise 1.4 (The knapsack problem)}\label{exercises:exercise1.4}\hypertarget{exercises:exercise1.4}{}
Let us organise a trek. Each hiker carries a knapsack of capacity 34 and can store 3 kinds of food which respectively supply energetic values (6,4,2) for a consumed capacity of (7,5,3). The problem is to find which food is to be put in the knapsack so that the energetic value is maximal.

\begin{description}
	\item[Question 1] In the first place, we will not consider the idea of maximizing the energetic value. Try to find a satisfying solution by modelling and implementing the problem within choco.
	\item[Question 2] Find and use the choco method to \textbf{maximise} the energetic value of the knapsack.
	\item[Question 3] Propose a Value selector heuristic to improve the efficiency of the model.
\end{description}

(\hyperlink{solutions:solutionofexercise1.4}{Solution})

\subsection{Exercise 1.5 (The n-queens problem)}\label{exercises:exercise1.5}\hypertarget{exercises:exercise1.5}{}
The $n$-queens problem aims to place $n$ queens on a chessboard of size $n$ so that no queen can attack one another.
\begin{description}
\item[Question 1] propose and implement a model based on one $L_{i}$ variable for every row. The value of $L_{i}$ indicates the column where a queen is to be put. Use simple difference constraints and confirm that 92 solutions are obtained for $n= 8$.
\item[Question 2] Add a redundant model by considering variables on the columns ($C_{i}$). Continue to use simple difference constraints.
\item[Question 3] Compare the number of nodes created to find the solutions with both models. How can you explain such a difference ?
\item[Question 4] Add to the previously implemented model the following heuristics:
  \begin{itemize}
  \item Select first the line variable $L_I$ which has the smallest domain ;
  \item Select the value $j\in L_i$ so that the associated column variable $C_j$ has the smallest domain.
  \end{itemize}
  Again, compare both approaches in term of number of nodes and solving time to find ONE solution for $n = 75, 90, 95, 105$.
\item[Question 5] what changes are caused by the use of the global constraint \texttt{alldifferent} ?
\end{description}

\insertGraphique{.3\linewidth}{media/nqueen.png}{A solution of the n-queens problem for $n = 8$}

(\hyperlink{solutions:solutionofexercise1.5}{Solution})

\section{I know CP}\label{exercises:iknowcp}\hypertarget{exercises:iknowcp}{}

\subsection{Exercise 2.1 (Bin packing, cumulative and search strategies)}\label{exercises:exercise2.1}\hypertarget{exercises:exercise2.1}{}

Can $n$ objects of a given size fit in $m$ bins of capacity $C$ ? The problem is here stated has a satisfaction problem for the sake of simplicity. Your model and heuristics will be checked by generating random instances for given $n$ and $C$. The random generation must be reproducible.
\begin{description}
	\item[Question 1] Propose a boolean model (0/1 variables).
	\item[Question 2] Let us turn this satisfaction problem into an optimization one. Use your previously stated model but increase regularly the number of containers until a feasible solution is found.
	\item[Question 3] Implement a naive lower bound. This can be done by considering the occupied size globally.
	\item[Question 4] Propose a model with integer variables based on the cumulative constraint (see choco user guide/API for details). Define an objective function to minimize the number of used bins.
	\item[Bonus question] Compare different search strategies (variables/values selector) on this model for $n$ between 10 and 15.
\end{description}

Take a look at the following exercise in the old version of Choco and try to transpose it on new version of Choco.

Here is \textbf{the complete code in Choco1} : \href{media/zip/binpackingv1.zip}{BinPackingv1.zip}.

\begin{lstlisting}
	int[] instance = getRandomPackingPb(n, capaBin, seed); 
	QuickSort sort = new QuickSort(instance); //Sort objects in increasing order
	sort.sort(); 
	Problem pb = new Problem(); 
	IntDomainVar[] debut = new IntDomainVar[n]; 
	IntDomainVar[] duree = new IntDomainVar[n]; 
	IntDomainVar[] fin = new IntDomainVar[n]; 
	 
	int nbBinMin = computeLB(instance, capaBin); 
	for (int i = 0; i < n; i++) { 
	    debut[i] = pb.makeEnumIntVar("debut " + i, 0, n); 
	    duree[i] = pb.makeEnumIntVar("duree " + i, 1, 1); 
	    fin[i] = pb.makeEnumIntVar("fin " + i, 0, n); 
	} 
	IntDomainVar obj = pb.makeEnumIntVar("nbBin ", nbBinMin, n); 
	pb.post(pb.cumulative(debut, fin, duree, instance, capaBin)); 
	for (int i = 0; i < n; i++) { 
	    pb.post(pb.geq(obj, debut[i])); 
	} 
	 
	IntDomainVar[] branchvars = new IntDomainVar[n + 1]; 
	System.arraycopy(debut, 0, branchvars, 0, n); 
	branchvars[n] = obj; 
	 
	//long tps = System.currentTimeMillis(); 
	pb.getSolver().setVarSelector(new StaticVarOrder(branchvars)); 
	Solver.setVerbosity(Solver.SOLUTION); 
	pb.minimize(obj, false); 
	Solver.flushLogs(); 
	// print solution 
	System.out.println("------------------------ " + (obj.getVal() + 1) + " bins"); 
	if (pb.isFeasible() == Boolean.TRUE) { 
	    for (int j = 0; j <= obj.getVal(); j++) { 
	    System.out.print("Bin " + j + ": "); 
	    int load = 0; 
	    for (int i = 0; i < n; i++) { 
	        if (debut[i].isInstantiatedTo(j)) { 
	        System.out.print(i + " "); 
	        load += instance[i]; 
	        } 
	    } 
	    System.out.println(" - load " + load); 
	    } 
	    //System.out.println("tps " + tps + " node " 
        //     + ((NodeLimit) pb.getSolver().getSearchSolver().limits.get(1)).getNbTot()); 
	}
\end{lstlisting}

(\hyperlink{solutions:solutionofexercise2.1}{Solution})

\subsection{Exercise 2.2 (Social golfer)}\label{exercises:exercise2.2}\hypertarget{exercises:exercise2.2}{}

A group of golfers play once a week and are splitted into $k$ groups of size $s$ (there are therefore $ks$ golfers in the club). The objective is to build a game scheduling on $w$ weeks so that no golfer play in the same group than another one more than once (hence the name of the problem: \emph{social golfers}). However, it may happen that two golfers will never play together. The point is only that once they have played together, they cannot play together anymore. 

\begin{note}
You can test your model with the parameters $(w, s, g)$ set to: $\{(11, 6, 2), (13, 7, 2), (9, 8, 8), (9, 8, 4), (4, 7, 3), (3, 6, 4)\}.$
\end{note}

\begin{description}
	\item[Question 1] Propose a boolean model for this problem. Use an heuristic that consists in scheduling a golfer on every week before scheduling a new one. More precisely, a golfer can be put in the first available group of each week before considering the next golfer.
	\item[Question 2] Identify some symmetries of the problem by using every similar elements of the problem. Try to improve your model by breaking those symmetries.
\end{description}

\begin{table}[htbp]
\centering
 	\begin{tabular}{ c c c c c}
		  &  group 1 &  group 2 &  group 3 &  group 4 \\
          \cline{2-5}
		 week 1 &  1 2 3 &  4 5 6 &  7 8 9 &  10 11 12 \\
		 week 1 &  1 4 7 &  10 2 5 &  8 11 3 &  6 9 12 \\
		 week 1 &  1 5 9 &  10 2 6 &  7 11 3 &  4 8 12 \\
	\end{tabular}
\caption{A valid configuration with 4 groups of 3 golfers on 3 weeks.}
\end{table}

Here is \textbf{the complete code in Choco1} : \href{media/zip/socialgolferv1.zip}{SocialGolferv1.zip}

\begin{lstlisting}
  Problem pb = new Problem();
  int numplayers = g * s;

  // golfmat[i][j][k] : is golfer k playing week j in group i ?
  IntDomainVar[][][] golfmat = new IntDomainVar[g][w][numplayers];
  for (int i = 0; i < g; i++) {
      for (int j = 0; j < w; j++)
      	  for (int k = 0; k < numplayers; k++)
          	  golfmat[i][j][k] = pb.makeEnumIntVar("("+i+"_"+j+"_"+k+")", 0, 1);
  }

  //every week, every golfer plays in one group
  for (int i = 0; i < w; i++) {
      for (int j = 0; j < numplayers; j++) {
          IntDomainVar[] vars = new IntDomainVar[g];
          for (int k = 0; k < g; k++) {
              vars[k] = golfmat[k][i][j];
          }
          pb.post(pb.eq(pb.scalar(vars, getOneMatrix(g)), 1));
      }
  }
	
  //every group is of size s
  for (int i = 0; i < w; i++) {
      for (int j = 0; j < g; j++) {
          IntDomainVar[] vars = new IntDomainVar[numplayers];
          System.arraycopy(golfmat[j][i], 0, vars, 0, numplayers);
          pb.post(pb.eq(pb.scalar(vars, getOneMatrix(numplayers)), s));
      }
  }
	
  //every pair of players only meets once
  // Efficient way : use of a ScalarAtMost
  for (int i = 0; i < numplayers; i++) {
      for (int j = i + 1; j < numplayers; j++) {
          IntDomainVar[] vars = new IntDomainVar[w * g * 2];
          int cpt = 0;
          for (int k = 0; k < w; k++) {
              for (int l = 0; l < g; l++) {
                  vars[cpt] = golfmat[l][k][i];
                  vars[cpt + w * g] = golfmat[l][k][j];
                  cpt++;
              } 
          }
          pb.post(new ScalarAtMostv1(vars, w * g, 1));
      }
  }
	
  //break symetries among weeks
  //enforce a lexicographic ordering between every pairs of week
  for (int i = 0; i < w; i++) {
      for (int j = i + 1; j < w; j++) {
          IntDomainVar[] vars1 = new IntDomainVar[numplayers * g];
          IntDomainVar[] vars2 = new IntDomainVar[numplayers * g];
          int cpt = 0;
          for (int k = 0; k < numplayers; k++) {
              for (int l = 0; l < g; l++) {
                  vars1[cpt] = golfmat[l][i][k];
                  vars2[cpt] = golfmat[l][j][k];
                  cpt++;
              }
          }
          pb.post(pb.lex(vars1, vars2));
      }
  }
	
  //break symetries among groups
  for (int i = 0; i < numplayers; i++) {
      for (int j = i + 1; j < numplayers; j++) {
          IntDomainVar[] vars1 = new IntDomainVar[w * g];
          IntDomainVar[] vars2 = new IntDomainVar[w * g];
          int cpt = 0;
          for (int k = 0; k < w; k++) {
              for (int l = 0; l < g; l++) {
                  vars1[cpt] = golfmat[l][k][i];
                  vars2[cpt] = golfmat[l][k][j];
                  cpt++;
              }
          }
          pb.post(pb.lex(vars1, vars2));
      }
  }
	
  //break symetries among players
  for (int i = 0; i < w; i++) {
      for (int j = 0; j < g; j++) {
          for (int p = j + 1; p < g; p++) {
              IntDomainVar[] vars1 = new IntDomainVar[numplayers];
              IntDomainVar[] vars2 = new IntDomainVar[numplayers];
              int cpt = 0;
              for (int k = 0; k < numplayers; k++) {
                  vars1[cpt] = golfmat[j][i][k];
                  vars2[cpt] = golfmat[p][i][k];
                  cpt++;
              }
              pb.post(pb.lex(vars1, vars2));
          }
      }
  }
	
  //gather branching variables
  IntDomainVar[] staticvars = new IntDomainVar[g * w * numplayers];
  int cpt = 0;
  for (int i = 0; i < numplayers; i++) {
      for (int j = 0; j < w; j++) {
          for (int k = 0; k < g; k++) {
              staticvars[cpt] = golfmat[k][j][i];
              cpt++;
          }
      }
  }
  pb.getSolver().setVarSelector(new StaticVarOrder(staticvars));
	
  pb.getSolver().setTimeLimit(120000);
  Solver.setVerbosity(Solver.SOLUTION);
  pb.solve();
  Solver.flushLogs();
\end{lstlisting}

(\hyperlink{solutions:solutionofexercise2.2}{Solution})

\subsection{Exercise 2.3 (Golomb rule)}\label{exercises:exercise2.3}\hypertarget{exercises:exercise2.3}{}

\emph{under development}

(\hyperlink{solutions:solutionofexercise2.3}{Solution})

\section{I know CP and Choco}\label{exercises:iknowcpandchoco}\hypertarget{exercises:iknowcpandchoco}{}

\subsection{Exercise 3.1 (Hamiltonian Cycle Problem Traveling Salesman Problem)}\label{exercises:exercise3.1}\hypertarget{exercises:exercise3.1}{}

Given a graph $G = (V,E)$, an \emph{Hamiltonian cycle} is a cycle that goes through every nodes of G once and only once. This exercise first intorduces a naive model to solve the Hamiltonian Cycle Problem. A second part tackles with the well known Traveling Salesman Problem.

Let $V = \{2,...,n\}$ be a set of cities index to cover, and let $d$ be a single warehouse duplicated into two indices $1$ and $n+1$. Notice the duplication distinguishes the source from the sink while there is only one warehouse. Finally, let us denote by $V_{d} = V \cup \{1,n+1\}$ the set of nodes to cover by a tour. Thus, the two following problems are defined:
\begin{itemize}
	\item find an Hamiltonian path covering all the cities of $V$
	\item find an Hamiltonian cycle of minimum cost that covers all the cities of $V$.
\end{itemize}

\subsubsection{Question 1 [Hamiltonian Cycle Problem]:} We first consider the satisfaction problem. Formally, a directed graph $G = (V,E)$ represents the topology of the cities and the unfolded warehouse. There is an arc $(i,j)\in E$ iff there exists a directed road from $i\in V$ to $j\in V$. Furthermore, every arc $(1,i)$ and $(i,n+1)$, with $i\in V$, belongs to E. Such a problem has to respect the following constraints:
\begin{itemize}
	\item each node of $V_{d}$ is reached exactly once,
	\item there is no subcycle containing nodes of $V$. In other words, the single cycle involved in $G$ is hamiltonian and contains arc $(n+1,1)$.
\end{itemize}
\begin{description}
\item[Question 1.a] The first constraint can directly be modelled using those proposed by Choco. On the other way, the second one requires to implement a constraint. This can be done through the following steps (see the provided skeleton):
\begin{itemize}
	\item strictly specify your constraint signature,
	\item formalise the underlying subproblemand information that need to be maintained,
	\item ignore in a first time the Choco event based mechanism and implement your filtering algorithm directly within the \texttt{propagate()} method,
	\item once your algorithm has been checked, try to reformulate your constraint through an event based implementation with the following methods: \texttt{awakeOnInst()}, \texttt{awakeOnSup()}, \texttt{awakeOnInf()}, \texttt{awakeOnBounds()}, \texttt{awakeOnRem()}, \texttt{awakeOnRemovals()}.
\end{itemize}
\item[Question 1.b] Now, propose a search heuristic (both on variables and values) that incrementaly builds the searched path from the source node. For this purpose, you have to respectively implement java classes that inherit from \texttt{IntVarSelector} and \texttt{ValSelector}.
\end{description}

\subsubsection{Question 2 [Traveling Salesman Problem]:} We now consider the optimisation view of the Hamiltonian Cycle Problem. A quantitative information is now associated with each arc of $G$ given by a cost function $f: E \leftarrow\Z_+$. Then, the graph $G$ is now defined by the triplet $(V_d,E,f)$ and we have to find an Hamiltonian path of minimum cost in $G$. 

For this purpose, we provide a skeleton of a Choco global constraint that dynamically maintains a lower bound evaluation of the searched path cost. Here, an evaluation of a minimum spanning tree of $G$ is proposed. \emph{Be careful}: take into account the partial assignment of the variables associated with the cities. 

\begin{description}
	\item[Question 2.a] find an upper bound on the cost of the Hamiltonian path,
	\item[Question 2.b] back-propagate lower/upper bounds informations on the required/infeasible arcs of $G$.
\end{description}

(\hyperlink{solutions:solutionofexercise3.1}{Solution})

\subsection{Exercise 3.2 (Shop scheduling)}\label{exercises:exercise3.2}\hypertarget{exercises:exercise3.2}{}

Given a set of $n$ tasks $T$ and $m$ disjunctive resources $R$, the problem is to find a plan to assign tasks to resources so that for every instant $t$, each resource $r\in R$ executes at most one task. Each task $T_i\in T$ is defined by:
\begin{itemize}
	\item a starting date $s_i = [s^-_i, s^+_i]\in\Z_+$,
	\item an ending date  $e_i = [e^-_i, e^+_i]\in\Z_+$,
	\item a duration  $d_i = e_i-s_i$, %$[d^-_i, d^+_i]\in\Z_+$,
	\item a resource $r_i = \{res_{1},\ldots, res_{m}\}\subseteq R$,
	\item a set of tasks, $preds_i\subseteq T$ that need to be processed before the start of $T_i$.
\end{itemize}

Let us consider the following satisfaction problem : Can one find a schedule of tasks $T$ on the resources $R$ that
\begin{itemize}
\item satisfies all the precedence constraints:
  $$ e_j\le s_i,\qquad (\forall T_i\in T, \forall T_j\in preds_i)$$
\item the last processed task ends before a given date $D$ ?:
  $$ e_i\le D,\qquad (\forall T_i\in T)$$
\end{itemize}

Then consider the optimization version : We now aim at finding the scheduling that satisfies all the constraints and that minimizes the date of the last task processed.
You will be given for this :
\begin{itemize}
\item a class structure where you have to describe your model (\texttt{AssignmentProblem}),
\item a class structure describing a Task (\texttt{Task}),
\item a class structure (\texttt{BinaryNonOverlapping}) which defines a Choco constraint. This constraint takes two tasks as parameters and has to verify whether at any time $t$ those tasks will be processed by the same resource or not.
\item a class structure (\texttt{MandatoryInterval}) which describes for a given task, the time window it has to be processed in.
\end{itemize}

\begin{description}
	\item[Question 1] How would you model the job scheduling problem ? Make use of the constraint \texttt{BinaryNonOverlapping}.
	\item[Question 2] Implement your model as if the constraint \texttt{BinaryNonOverlapping} was implemented.
	\item[Question 3] Sketch the mandatory processing interval of a task.
	\item[Question 4] Implement the constraint \texttt{BinaryNonOverlapping}:
	\begin{itemize}
		\item Implement the following reasoning : if two tasks have to be processed on the same resource and their mandatory intervals intersect, throw a failure.
		\item Now, implement the condition : if two tasks have a mandatory interval intersection, they must be scheduled on different resources.
		\item Finally, implement the following reasoning : If two tasks have to be processed by the same resource, then the starting and ending dates of every task ought to be updated functions to their mandatory intervals.
	\end{itemize}
	\item[Question 5] Implement an variable selection heuristic on the decision variable of the problem.
	\item[Question 6] Propose a model which minimize the end date of the last assigned task.
	\item[Question 7] Can you find a way to improve the BinaryNonOverlapping constraint.
	\item[Bonus Question] Find a lower bound on the end date of the last processed task.
\end{description}

\hyperlink{solutions:solutionofexercise3.2}{Solution}

%\part{solutions}
\label{solutions}
\hypertarget{solutions}{}

\chapter{Solutions}\label{solutions:solutions}\hypertarget{solutions:solutions}{}

\section{I'm new to CP}\label{solutions:i'mnewtocp}\hypertarget{solutions:i'mnewtocp}{}

\subsection{Solution of Exercise 1.1 (A soft start)}\label{solutions:solutionofexercise1.1}\hypertarget{solutions:solutionofexercise1.1}{}
(\hyperlink{exercises:exercise1.1}{Problem})

\noindent\emph{\textbf{Question 1}: describe the constraint network related to code}

The model is defined as :
\begin{itemize}
	\item $V = \{x_1, x_2, x_3\}$: the set of variables,
	\item $D = \{[0,5], [0,5], [0,5]\}$: the set of domain
	\item $C = \{x_1>x_2, x_1\neq x_3, x_2>x_3\}$: the set of constraints.
\end{itemize}

\noindent\emph{\textbf{Question 2}: give the variable domains after constraint propagation.}

\begin{itemize}
	\item From $x_1 = [0,5]$ and $x_2 = [0,5]$ and $x_1>x_2$, we can deduce tha : the domain of $x_1$ can be reduce to $[1,5]$ and the domain of $x_2$ can be reduce to $[0,4]$.
	\item Then, from $x_2 = [0,4]$ and $x_3 = [0,5]$ and $x_2>x_3$, we can deduce that : the domain of $x_2$ can be reduce to $[1,4]$ and the domain of $x_3$ can be reduce to $[0,3]$.
	\item Then, from $x_1 = [1,5]$ and $x_2 = [1,4]$ and $x_1>x_2$, we can deduce that : the domain of $x_1$ can be reduce to $[2,5]$.
\end{itemize}

We cannot deduce anything else, so we have reached a \textbf{fix point}, and here is the domain of each variables:
$$x_{1} : [2,5],\quad x_{2} : [1,4],\quad x_{3} : [0,3].$$


\subsection{Solution of Exercise 1.2 (DONALD + GERALD = ROBERT)}\label{solutions:solutionofexercise1.2}\hypertarget{solutions:solutionofexercise1.2}{}

(\hyperlink{exercises:exercise1.2}{Problem})

Source code: \href{media/zip/exdonaldgeraldrobert.zip}{ExDonaldGeraldRobert.zip}

\begin{lstlisting}
  // Build model
  Model model = new CPModel();
  
  // Declare every letter as a variable
  IntegerVariable d = makeIntVar("d", 0, 9, "cp:enum");
  IntegerVariable o = makeIntVar("o", 0, 9, "cp:enum");
  IntegerVariable n = makeIntVar("n", 0, 9, "cp:enum");
  IntegerVariable a = makeIntVar("a", 0, 9, "cp:enum");
  IntegerVariable l = makeIntVar("l", 0, 9, "cp:enum");
  IntegerVariable g = makeIntVar("g", 0, 9, "cp:enum");
  IntegerVariable e = makeIntVar("e", 0, 9, "cp:enum");
  IntegerVariable r = makeIntVar("r", 0, 9, "cp:enum");
  IntegerVariable b = makeIntVar("b", 0, 9, "cp:enum");
  IntegerVariable t = makeIntVar("t", 0, 9, "cp:enum");
  
  // Declare every name as a variable  
  IntegerVariable donald = makeIntVar("donald", 0, 1000000,"cp:bound");
  IntegerVariable gerald = makeIntVar("gerald", 0, 1000000,"cp:bound");
  IntegerVariable robert = makeIntVar("robert", 0, 1000000,"cp:bound");
  
  // Array of coefficients
  int[] c = new int[]{100000, 10000, 1000, 100, 10, 1}; 
  
  // Declare every combination of letter as an integer expression
  IntegerExpressionVariable donaldLetters = scalar(new IntegerVariable[]{d,o,n,a,l,d}, c);
  IntegerExpressionVariable geraldLetters = scalar(new IntegerVariable[]{g,e,r,a,l,d}, c);
  IntegerExpressionVariable robertLetters = scalar(new IntegerVariable[]{r,o,b,e,r,t}, c);
  
  // Add equality between name and letters combination
  model.addConstraint(eq(donaldLetters, donald));
  model.addConstraint(eq(geraldLetters, gerald));
  model.addConstraint(eq(robertLetters, robert));
  // Add constraint name sum
  model.addConstraint(eq(plus(donald, gerald), robert));
  // Add constraint of all different letters.
  model.addConstraint(allDifferent(new IntegerVariable[]{d,o,n,a,l,g,e,r,b,t}));
  
  // Build a solver, read the model and solve it
  Solver s = new CPSolver();
  s.read(model);
  s.solve();
  
  // Print name value
  System.out.println("donald = " + s.getVar(donald).getVal());
  System.out.println("gerald = " + s.getVar(gerald).getVal());
  System.out.println("robert = " + s.getVar(robert).getVal());
\end{lstlisting}

\subsection{Solution of Exercise 1.3 (A famous example. . . a sudoku grid)}\label{solutions:solutionofexercise1.3}\hypertarget{solutions:solutionofexercise1.3}{}

(\hyperlink{exercises:exercise1.3}{Problem})

Source code: \href{media/zip/exsudoku.zip}{ExSudoku.zip}

\noindent\emph{\textbf{Question 1}: propose a way to model the sudoku problem with difference constraints. Implement your model with choco solver.}

\begin{lstlisting}
  int n = instance.length;
  // Build Model
  Model m = new CPModel();
  
  // Build an array of integer variables
  IntegerVariable[][] rows = makeIntVarArray("rows", n, n, 1, n,"cp:enum");
	
  // Not equal constraint between each case of a row
  for (int i = 0; i < n; i++) {
      for (int j = 0; j < n; j++)
          for (int k = j; k < n; k++)
              if (k != j) m.addConstraint(neq(rows[i][j], rows[i][k]));
  }
                  
  // Not equal constraint between each case of a column
  for (int j = 0; j < n; j++) {
      for (int i = 0; i < n; i++)
          for (int k = 0; k < n; k++)
              if (k != i)  m.addConstraint(neq(rows[i][j], rows[k][j]));
  }

  // Not equal constraint between each case of a sub region
  for (int ci = 0; ci < n; ci += 3) {
      for (int cj = 0; cj < n; cj += 3)
          // Extraction of disequality of a sub region
          for (int i = ci; i < ci + 3; i++)
              for (int j = cj; j < cj + 3; j++)
                  for (int k = ci; k < ci + 3; k++)
                      for (int l = cj; l < cj + 3; l++)
                          if (k != i || l != j) m.addConstraint(neq(rows[i][j], rows[k][l]));
  }
	
  //...
	
  // Call solver
  Solver s = new CPSolver();
  s.read(m);
  CPSolver.setVerbosity(CPSolver.SOLUTION);
  s.solve();
  CPSolver.flushLogs();
  printGrid(rows, s);
\end{lstlisting}

\noindent\emph{\textbf{Question 2}: which global constraint can be used to model such a problem ? Modify your code to use this constraint.}

The \emph{allDifferent} constraint can be used to remplace every disequality constraint on the first Sudoku model. It improves the efficient of the model and make it more ``readable''.

\begin{lstlisting}
  // Build model
  Model m = new CPModel();
  // Declare variables
  IntegerVariable[][] cols = new IntegerVariable[n][n];
  IntegerVariable[][] rows = makeIntVarArray("rows", n, n, 1, n,"cp:enum");
  
  // Channeling between rows and columns
  for (int i = 0; i < n; i++) {
      for (int j = 0; j < n; j++)
          cols[i][j] = rows[j][i];
  }
	
  // Add alldifferent constraint
  for (int i = 0; i < n; i++) {
      m.addConstraint(allDifferent(cols[i]));
      m.addConstraint(allDifferent(rows[i]));
  }

  // Define sub regions
  IntegerVariable[][] carres = new IntegerVariable[n][n];
  for (int i = 0; i < 3; i++) {
      for (int j = 0; j < 3; j++)
          for (int k = 0; k < 3; k++)
              carres[j + k * 3][i] = rows[0 + k * 3][i + j * 3];
              carres[j + k * 3][i + 3] = rows[1 + k * 3][i + j * 3];
              carres[j + k * 3][i + 6] = rows[2 + k * 3][i + j * 3];
  }
	
  // Add alldifferent on sub regions
  for (int i = 0; i < n; i++) {
      Constraint c = allDifferent(carres[i]);
      m.addConstraint(c);
  }
  
  //...
	
  // Call solver
  Solver s = new CPSolver();
  s.read(m);
  CPSolver.setVerbosity(CPSolver.SOLUTION);
  s.solve();
  printGrid(rows, s);
\end{lstlisting} 

\noindent\emph{\textbf{Question 3}: Test for both model the initial propagation step (use choco} \texttt{propagate()} \emph{method). What can be noticed ? What is the point in using global constraints ?}

The sudoku problem can be solved just with the propagation. \todo{FIXME explanation.}
The global constraint provides a more efficient filter algorithm, due to more complex deduction.

\subsection{Solution of Exercise 1.4 (The knapsack problem)}\label{solutions:solutionofexercise1.4}\hypertarget{solutions:solutionofexercise1.4}{}

(\hyperlink{exercises:exercise1.4}{Problem})

Source code: \href{media/zip/exknapsack.zip}{ExKnapSack.zip}

\noindent\emph{\textbf{Question 1} : In the first place, we will not consider the idea of maximizing the energetic value. Try to find a satisfying solution by modelling and implementing the problem within choco.}

\begin{lstlisting}
	Model m = new CPModel();
	
	obj1 = makeIntVar("obj1", 0, 5,"cp:enum");
	obj2 = makeIntVar("obj2", 0, 7,"cp:enum");
	obj3 = makeIntVar("obj3", 0, 10,"cp:enum");
	c = makeIntVar("cost", 1, 1000000,"cp:bound");
	
	int capacity = 34;
	int[] volumes = new int[]{7, 5, 3};
	int[] energy = new int[]{6, 4, 2};
	
	m.addConstraint(leq(scalar(volumes, new IntegerVariable[]{obj1, obj2, obj3}), capacity));
	m.addConstraint(eq(scalar(energy, new IntegerVariable[]{obj1, obj2, obj3}), c));
	
	Solver s = new CPSolver();
	s.read(m);

	s.solve();
	
	System.out.println("("+s.getVar(obj1).getVal()+","+s.getVar(obj2).getVal()+","
                       +s.getVar(obj3).getVal()+") cost = "+ s.getVar(c).getVal());
\end{lstlisting}

\noindent\emph{\textbf{Question 2} : Find and use the choco method to maximise the energetic value of the knapsack.}
Replace \mylst{s.solve()} by:
\begin{lstlisting}
	s.maximize(s.getVar(c), false);
\end{lstlisting}

\noindent\emph{\textbf{Question 3} : Propose a Value selector heuristic to improve the efficiency of the model.}

It can be improved using the following value selector strategy. It iterates over decreasing values of every domain variables: 
\begin{lstlisting}
  s.setValIntIterator(new DecreasingDomain());
\end{lstlisting}

\subsection{Solution of Exercise 1.5 (The n-queens problem)}\label{solutions:solutionofexercise1.5}\hypertarget{solutions:solutionofexercise1.5}{}
(\hyperlink{exercises:exercise1.5}{Problem})

Source code: \href{media/zip/exqueen.zip}{ExQueen.zip}

\noindent\emph{\textbf{Question 1} : propose and implement a model based on one} $L_{i}$ \emph{variable for every row...}
\begin{lstlisting}
  Model m = new CPModel();
  
  IntegerVariable[] queens = new IntegerVariable[n];
  for (int i = 0; i < n; i++) {
      queens[i] = makeIntVar("Q" + i, 1, n,"cp:enum");
  }
	
  for (int i = 0; i < n; i++) {
      for (int j = i + 1; j < n; j++) {
          int k = j - i;
          m.addConstraint(neq(queens[i], queens[j]));
          m.addConstraint(neq(queens[i], plus(queens[j], k)));  // diagonal
          m.addConstraint(neq(queens[i], minus(queens[j], k))); // diagonal
      }
  }
	
  Solver s = new CPSolver();
  s.read(m);
  CPSolver.setVerbosity(CPSolver.SOLUTION);
  int timeLimit = 60000;
  s.setTimeLimit(timeLimit);
  s.solve();
  CPSolver.flushLogs();
\end{lstlisting}

\noindent\emph{\textbf{Question 2} : Add a redundant model by considering variable on the columns ($C_i$). Continue to use simple difference constraints.}

\begin{lstlisting}
  Model m = new CPModel();
	
  IntegerVariable[] queens = new IntegerVariable[n];
  IntegerVariable[] queensdual = new IntegerVariable[n];
  for (int i = 0; i < n; i++) {
      queens[i] = makeIntVar("Q" + i, 1, n,"cp:enum");
      queensdual[i] = makeIntVar("QD" + i, 1, n,"cp:enum");
  }
	
  for (int i = 0; i < n; i++) {
      for (int j = i + 1; j < n; j++) {
          int k = j - i;
          m.addConstraint(neq(queens[i], queens[j]));
          m.addConstraint(neq(queens[i], plus(queens[j], k)));  // diagonal
          m.addConstraint(neq(queens[i], minus(queens[j], k))); // diagonal
      }
  }

  for (int i = 0; i < n; i++) {
      for (int j = i + 1; j < n; j++) {
          int k = j - i;
          m.addConstraint(neq(queensdual[i], queensdual[j]));
          m.addConstraint(neq(queensdual[i], plus(queensdual[j], k)));  // diagonal
          m.addConstraint(neq(queensdual[i], minus(queensdual[j], k))); // diagonal
      }
  }
  m.addConstraint(inverseChanneling(queens, queensdual));
  
  Solver s = new CPSolver();
  s.read(m);
  
  s.setVarIntSelector(new MinDomain(s,s.getVar(queens)));
  
  CPSolver.setVerbosity(CPSolver.SOLUTION);
  s.setLoggingMaxDepth(50);
  int timeLimit = 60000;
  s.setTimeLimit(timeLimit);
  s.solve();
  CPSolver.flushLogs();
\end{lstlisting}

\noindent\emph{\textbf{Question 3} : Compare the number of nodes created to find the solutions with both models. How can you explain such a difference ?}

The channeling permit to reduce more nodes from the tree search... \todo{FIXME}

\noindent\emph{\textbf{Question 4} : Add to the previous implemented model the following heuristics,
\begin{itemize}
	\item Select first the line variable ($L_i$) which has the smallest domain ;
	\item Select the value $j\in L_i$ so that the associated column variable $C_j$ has the smallest domain.
\end{itemize}
Again, compare both approaches in term of nodes number and solving time to find ONE solution for $n = 75, 90, 95, 105$.}

Add the following lines to your program (after the reading of the model):
\begin{lstlisting}
	s.setVarIntSelector(new MinDomain(s,s.getVar(queens)));
	s.setValIntSelector(new NQueenValueSelector(s.getVar(queensdual)));
\end{lstlisting}
The variable selector strategy (\texttt{MinDomain}) already exists in Choco. It iterates over variables given and returns the variable ordering by creasing domain size. 
The value selector strategy has to be created as follow:
\begin{lstlisting}
  public class NQueenValueSelector implements ValSelector {
	
      // Column variable
      protected IntDomainVar[] dualVar;
	
      // Constructor of the value selector, 
      public NQueenValueSelector(IntDomainVar[] cols) {
          this.dualVar = cols;
      }
	
      // Returns the "best val" that is the smallest column domain size OR -1
      // (-1 is not in the domain of the variables)
      public int getBestVal(IntDomainVar intDomainVar) {
          int minValue = 10000;
          int v0 = -1;
          IntIterator it = intDomainVar.getDomain().getIterator();
          while (it.hasNext()){
              int i = it.next();
              int val = dualVar[i - 1].getDomainSize();
              if (val < minValue)  {
                  minValue = val;
                  v0 = i;
              }
          }
          return v0;
      }
  }
\end{lstlisting}

\noindent\emph{\textbf{Question 5} : what changes are caused by the use of the global constraint \textbf{alldifferent} ?}

\begin{lstlisting}
  Model m = new CPModel();
	
  IntegerVariable[] queens = new IntegerVariable[n];
  IntegerVariable[] queensdual = new IntegerVariable[n];
  IntegerVariable[] diag1 = new IntegerVariable[n];
  IntegerVariable[] diag2 = new IntegerVariable[n];
  IntegerVariable[] diag1dual = new IntegerVariable[n];
  IntegerVariable[] diag2dual = new IntegerVariable[n];
  for (int i = 0; i < n; i++) {
      queens[i] = makeIntVar("Q" + i, 1, n,"cp:enum");
      queensdual[i] = makeIntVar("QD" + i, 1, n,"cp:enum");
      diag1[i] = makeIntVar("D1" + i, 1, 2 * n,"cp:enum");
      diag2[i] = makeIntVar("D2" + i, -n, n,"cp:enum");
      diag1dual[i] = makeIntVar("D1" + i, 1, 2 * n,"cp:enum");
      diag2dual[i] = makeIntVar("D2" + i, -n, n,"cp:enum");
  }
	
  m.addConstraint(allDifferent(queens));
  m.addConstraint(allDifferent(queensdual));
  for (int i = 0; i < n; i++) {
      m.addConstraint(eq(diag1[i], plus(queens[i], i)));
      m.addConstraint(eq(diag2[i], minus(queens[i], i)));
      m.addConstraint(eq(diag1dual[i], plus(queensdual[i], i)));
      m.addConstraint(eq(diag2dual[i], minus(queensdual[i], i)));
  }
  m.addConstraint(inverseChanneling(queens,queensdual));
	
  m.addConstraint(allDifferent(diag1));
  m.addConstraint(allDifferent(diag2));
  m.addConstraint(allDifferent(diag1dual));
  m.addConstraint(allDifferent(diag2dual));
	
  Solver s = new CPSolver();
  s.read(m);
	
  s.setVarIntSelector(new MinDomain(s,s.getVar(queens)));
  s.setValIntSelector(new NQueenValueSelector(s.getVar(queensdual)));
	
  CPSolver.setVerbosity(CPSolver.SOLUTION);
  int timeLimit = 60000;
  s.setTimeLimit(timeLimit);
  s.solve();
  CPSolver.flushLogs();
\end{lstlisting}

\section{I know CP}\label{solutions:iknowcp}\hypertarget{solutions:iknowcp}{}

\subsection{Solution of Exercise 2.1 (Bin packing, cumulative and search strategies)}\label{solutions:solutionofexercise2.1}\hypertarget{solutions:solutionofexercise2.1}{}
(\hyperlink{exercises:exercise2.1}{Problem})

Source code: \href{media/zip/binpackingv2.zip}{BinPackingv2.zip}

\subsection{Solution of Exercise 2.2 (Social golfer)}\label{solutions:solutionofexercise2.2}\hypertarget{solutions:solutionofexercise2.2}{}
(\hyperlink{exercises:exercise2.2}{Problem})

Source code: \href{media/zip/socialgolferv2.zip}{SocialGolferv2.zip}

\subsection{Solution of Exercise 2.3 (Golomb rule)}\label{solutions:solutionofexercise2.3}\hypertarget{solutions:solutionofexercise2.3}{}

\emph{under development}

(\hyperlink{exercises:exercise2.3}{Problem})

\section{I know CP and Choco2.0}\label{solutions:iknowcpandchoco2.0}\hypertarget{solutions:iknowcpandchoco2.0}{}

\subsection{Solution of Exercise 3.1 (Hamiltonian Cycle Problem Traveling Salesman Problem)}\label{solutions:solutionofexercise3.1}\hypertarget{solutions:solutionofexercise3.1}{}

(\hyperlink{exercises:exercise3.1}{Problem})

Source code: \href{media/zip/extsp.zip}{ExTSP.zip}

\subsection{Solution of Exercise 3.2 (Shop scheduling)}\label{solutions:solutionofexercise3.2}\hypertarget{solutions:solutionofexercise3.2}{}

(\hyperlink{exercises:exercise3.2}{Problem})

\emph{under development}


\part{Extras}\label{ch:extra}\hypertarget{ch:extra}{}
%\part{choco and visu}
\label{chocoandvisu}
\hypertarget{chocoandvisu}{}

\chapter{Choco and Visu}\label{chocoandvisu:chocoandvisu}\hypertarget{chocoandvisu:chocoandvisu}{}

\section{Why?}\label{chocoandvisu:why}\hypertarget{chocoandvisu:why}{}
Since few months, it has seemed more and more evident for us that CHOCO needed a way to visualize dynamically the resolution of a problem.
We wanted that visualization to be open, easy to use and not static.
Now, you will find a new package on \textbf{Choco 2.0.1} (the actual beta version) named \emph{visu}.

\section{The visu package}\label{chocoandvisu:thevisupackage}\hypertarget{chocoandvisu:thevisupackage}{}

The \emph{visu} package contains objects to define a visualization of the resolution, domain reduction, constraints propagation, etc. %is build as follow:

% \begin{note}
% \begin{itemize}
% 	\item choco
% 	\begin{itemize}
% 		\item ...
% 		\item visu
% 		\begin{itemize}
% 			\item components
% 			\item searchloop
% 			\item variables
% 		\end{itemize}
% 	\end{itemize}
% \end{itemize}
% \end{note}

Figures~\ref{fig:media/cd_visu.png} depicts the class diagram of the visu package (\emph{powered by \href{http://bouml.free.fr/}{BOUML}}):

\insertGraphique{18cm}{media/cd_visu.png}{Visu classes diagram. The blue classes are examples of implementation and inheritence.}



\section{Steps to use the Visu}\label{chocoandvisu:stepstousethevisu}\hypertarget{chocoandvisu:stepstousethevisu}{}
Only one Visu can be linked to one Solver.

We are going to see a short example of Visu use, based on Sudoku problem.
In our modeling, variables are cells of a sudoku grid, represented by the matrix \emph{rows}.
We want to define a standard visualization where a variable is displayed on a line. Its name is written, and the domain is viewed as an array of colored square.
That representation is known in CHOCO as a \emph{FULLDOMAIN} representation.
\subsection{Visu creation}\label{chocoandvisu:visucreation}\hypertarget{chocoandvisu:visucreation}{}
The first step is to create the Visu object, which is basically a frame with components. We use the static constructor defined in \texttt{Visu.java}:
\begin{itemize}
	\item \mylst{Visu.createFullVisu()}: build a Visu object with default minimum size (width 480 px and heigth 640 px), with \emph{next}, \emph{play},\emph{pause} buttons and the break length slider.
	\item \mylst{Visu.createFullVisu(int width, int height)}: build a Visu object with user defined minimum size (width \emph{width} px and heigth \emph{height} px), with \emph{next}, \emph{play},\emph{pause} buttons and the break length slider.
	\item \mylst{Visu.createVisu(VisuButton... buttons)}: build a Visu object with default minimum size (width 480 px and heigth 640 px), with \emph{buttons} buttons and the break length slider.
	\item \mylst{Visu.createVisu(int width, int height, final VisuButton... buttons)}: build a Visu object with user defined minimum size (width \emph{width} px and heigth \emph{height} px), with \emph{buttons} buttons and the break length slider if necessary (at least, if there.is one button).
\end{itemize}

Parameter \emph{buttons} is an array of VisuButton that can take one of the following values: \emph{NEXT}, \emph{PLAY}. 
\emph{NEXT} add the \emph{next} button to the frame and the slider, \emph{PLAY} add the \emph{play} and \emph{pause} buttons and the slider.

We want to create a simple full Visu:
\begin{lstlisting}
Visu v = Visu.createVisu();
\end{lstlisting}

\subsection{Adding panel}\label{chocoandvisu:addingpanel}\hypertarget{chocoandvisu:addingpanel}{}
Now the frame is defined, we have to add a component: a \texttt{VarChocoPanel}. It is a specified panel, added to a \texttt{TabbedPane}, where one visualization (a \texttt{ChocoPApplet}) can be put.
A \texttt{ChocoPApplet} can be defined in two ways: an existing one, or a user defined one.
Constructors of \texttt{VarChocoPanel} are:
\begin{itemize}
	\item \mylst{VarChocoPanel(final String name, final Variable[] x, final ChocoPApplet applet, final Object params)}: to add a predefined \texttt{ChocoPApplet}. \emph{params} can be null, except for \emph{applet=DOTTYTREESEARCH} (see below).
	\item \mylst{VarChocoPanel(final String name, final Variable[] x, final Class appletclass, Object params)}: like previous, but \texttt{ChocoPApplet} is replaced by \emph{class} which is the class name of the user's \texttt{ChocoPApplet}. Recommanded for use of user's ChocoPApplet.
	\item \mylst{VarChocoPanel(final String name, final Variable[] x, final String appletpath, Object params)}: like previous, but \texttt{ChocoPApplet} is replaced by \emph{path} which is the path of the user's \texttt{ChocoPApplet} in the project.
\end{itemize}

\subsubsection{Existing ChocoPApplet}\label{chocoandvisu:existingchocopapplet}\hypertarget{chocoandvisu:existingchocopapplet}{}
Few ChocoPApplet are defined in Choco:
\begin{itemize}
	\item \textbf{COLORORVALUE} : draw an applet where variables are in columns and where their value is displayed with a colored square (blue: not instantiated, green: instantiated),
	\item \textbf{DOTTYTREESEARCH} : specific applet, which do not display anything, but a \emph{screensaver}. It builds a dot file (name given in parameters) with nodes of the tree search, to represent the tree search. The paramaters are :
	\begin{itemize}
		\item \emph{filename} (\texttt{String}) : output file name
		\item \emph{nbMaxNode} (\texttt{int}): size limit of the tree seach. If there is more than \emph{nbMaxNode} nodes, the dot file will not be printed. The number of nodes has an impact on the file size
		\item \emph{watch} (\texttt{Var}) : the variable to optimize. Can be \texttt{null} if no optimization is performed.
		\item \emph{maximize} (\texttt{Boolean}) : indicating wether the optimization is a maximization (if set to \texttt{true}) or a minimization (if set to \texttt{false}). Can be \texttt{null} if no optimization is performed.
		\item \emph{restart} (\texttt{Boolean}) : indicating wether the search can restart (is set to \texttt{true}) or not (if set to \texttt{false}). Can be \texttt{null} if no optimization is performed.
	\end{itemize}
	\item \textbf{FULLDOMAIN} : draw an applet where variables are in columns. Each line is build with a variable name and a set of colored square (blue: not instantiated, green: instantiated) representing each value of the domain.
	\item \textbf{GRID} : draw an applet with a simple grid, where each cells contains the value of a variable (question mark or value).
	\item \textbf{NAMEORQUESTIONMARK} : draw an applet where a variables are displayed on columns, by a question mark (if not instanciated) or its value (if instanciated).
	\item \textbf{NAMEORVALUE} : draw an applet where a variables are displayed on columns, by its name (if not instanciated) or its value (if instanciated).
	\item \textbf{SUDOKU} : specific applet, draw a sudoku grid where each cell represents the value of a variable or a question mark.
	\item \textbf{TREESEARH} : draw the dynamique construction of the tree search.
\end{itemize}

To add a panel where one of that ChocoPApplet will be drawn, use the following code:
\begin{lstlisting}
	Visu v = Visu.createVisu(
	v.addPanel(new VarChocoPanel("Grid", vars, GRID, null));
	v.addPanel(new VarChocoPanel("TreeSearch", vars, TREESEARCH, null));
	v.addPanel(new VarChocoPanel("Dotty", vars, DOTTYTREESEARCH,
               new Object[]{"/home/choco/treesearch.dot", 100, null, null, null}));
\end{lstlisting}

\subsubsection{User ChocoPApplet}\label{chocoandvisu:userchocopapplet}\hypertarget{chocoandvisu:userchocopapplet}{}

\todo{UNDER DEVELOPMENT}

\section{Examples}\label{chocoandvisu:examples}\hypertarget{chocoandvisu:examples}{}

\todo{UNDER DEVELOPMENT}

%\part{sudoku and cp}
\label{sudokuandcp}
\hypertarget{sudokuandcp}{}

\chapter{Sudoku and Constraint Programming}\label{sudokuandcp:sudokuandconstraintprogramming}\hypertarget{sudokuandcp:sudokuandconstraintprogramming}{}

\section{Sudoku ?!?}\label{sudokuandcp:sudoku!}\hypertarget{sudokuandcp:sudoku!}{}

\insertGraphique{5cm}{media/sudokuillustration.jpg}{A sudoku grid}

Everybody knows those grids that appeared last year in the subway, in wating lounges, on colleague's desks, etc. In Japanese \emph{su} means digit and \emph{doku}, unique. But this game has been discovered by an American ! The first grids appeared in the USA in 1979 (they were hand crafted). \href{http://en.wikipedia.org/wiki/sudoku}{Wikipedia} tells us that they were designed by Howard Garns a retired architect. He died in 1989 well before the success story of sudoku initiated by Wayne Gould, a retired judge from Hong-Kong. The rules are really simple: a 81 cells square grid is divided in 9 smaller blocks of 9 cells (3 x 3). Some of the 81 are filled with one digit. The aim of the puzzle is to fill in the other cells, using digits except 0, such as each digit appears once and only once in each row, each column and each smaller block. The solution is unique.

\subsection{Solving sudokus}\label{sudokuandcp:solvingsudokus}\hypertarget{sudokuandcp:solvingsudokus}{}

Many computer techniques exist to quickly solve a sudoku puzzle. Mainly, they are based on backtracking algorithms. The idea is the following: give a free cell a value and continue as long as choices remain consistent. As soon as an inconsistency is detected, the computer program backtracks to its earliest past choice et tries another value. If no more value is available, the program keeps backtracking until it can go forward again. This systematic technique make it sure to solve a sudoku grid. However, no human player plays this way: this needs too much memory ! 

\begin{note}
see \href{http://en.wikipedia.org/wiki/sudoku}{Wikipedia} for a panel of solving techniques.
\end{note}

\section{Sudoku and Artificial Intelligence}\label{sudokuandcp:sudokuandartificialintelligence}\hypertarget{sudokuandcp:sudokuandartificialintelligence}{}

Many techniques and rules have been designed and discovered to solve sudoku grids. Some are really simple, some need to use some useful tools: pencil and eraser. 

\subsection{Simple rules: single candidate and single position}\label{sudokuandcp:simplerules:singlecandidateandsingleposition}\hypertarget{sudokuandcp:simplerules:singlecandidateandsingleposition}{}

\insertGraphique{5cm}{media/sudokuillustrationscsp.jpg}{Simple rules: single candidates and single position} 

Let consider the grid on Figure~\ref{fig:media/sudokuillustrationscsp.jpg} and the cell with the red dot. In the same line, we find: 3, 4, 6, 7, and 9. In the same column: 2, 3, 5, and 8. In the same block: 2, 7, 8, and 9. There remain only one possibility: \textbf{1}. This is the \textbf{single candidate} rule. This cell should be filled in with \textbf{1}. 

Now let consider a given digit: let's say 4. In the block with a blue dot, there is no 4. Where can it be ? The 4's in the surrounding blocks heavily constrain the problem. There is a \textbf{single position} possible: the blue dot. This another simple rule to apply.

Alternatively using these two rules allows a player to fill in many cells and even solve the simplest grids. 
But, limits are easily reached. More subtle approaches are needed: but an important tool is now needed ... an eraser ! 

\subsection{Human reasoning principles}\label{sudokuandcp:humanreasoningprinciples}\hypertarget{sudokuandcp:humanreasoningprinciples}{}

\insertGraphique{7cm}{media/sudokuillustrationmarks.jpg}{Introducing marks}

Many techniques do exist but a vaste amount of them rely on simple principles. The first one is: do not try to find the value of a cell but instead focus on values that \textbf{will never be assigned} to it. The space of possibility is then reduced. This is where the eraser comes handy. Many players marks the remaining possibilities as in the grid on the left. 

Using this information, rather subtle reasoning is possible. For example, consider the seventh column on the grid on the left. Two cells contain as possible values the two values 5 and 7. This means that those two values cannot appear elsewhere in that very same column. Therefore, the other unassigned cell on the column can only contain a 6. We have \emph{deduced} something.

\noindent\begin{minipage}[b]{0.8\linewidth}
This was an easy to spot inference. This is not always the case. Consider the part of the grid on the right. Let us consider the third column. For cells 4 and 5, only two values are available: 4 and 8. Those values cannot be assigned to any other cell in that column. Therefore, in cell 6 we have a 3, and thus and 7 in cell 2 and finally a 1 in cell 3. This can be a very powerful rule.

Such a reasoning (sometimes called \emph{Naked Pairs}) is easily generalized to any number of cells (always in the same region: row, column or block) presenting this same configuration. This local reasoning can be applied to any region of the grid. It is important to notice that the inferred information can (and should) be used from a region to another.   
\end{minipage}%
\begin{minipage}[m]{0.2\linewidth}
~~\Graph{media/sudokuillustrationpart.jpg}{width=3cm}
\end{minipage}

\noindent The following principles of \emph{human} reasoning can be listed: 
\begin{itemize}
	\item reasoning on \emph{possible} values for a cell (by erasing impossible ones)
	\item systematically applying an evolved local reasoning (such as the \emph{Naked Pairs} rule)
	\item transmitting inferred information from a region to another related through a given a set of cells
\end{itemize}

\subsection{Towards Constraint Programming}\label{sudokuandcp:towardsconstraintprogramming}\hypertarget{sudokuandcp:towardsconstraintprogramming}{}

Those three principles are at the core of \textbf{constraint programming} a recent technique coming from both \emph{artificial intelligence} and \emph{operations research}.

\begin{itemize}
	\item The first principle is called \textbf{domain reduction} or \emph{filtering}
	\item The second considers its region as a \textbf{constraint} (a relation to be verified by the solution of the problem): here we consider an \emph{all different} constraint (all the values must be different in a given region). Constraints are considered \textbf{locally} for reasoning
	\item The third principle is called \textbf{propagation}: constraints (regions) communicate with one another through the available values in variables (cells)
\end{itemize}

Constraint programming is able to solve this problem as a human would do. Moreover, a large majority of the rules and techniques described on the Internet amount to a well-known problem: the alldifferent problem. A \textbf{constraint solver} (as \textbf{Choco}) is therefore able to reason on this problem allowing the solving of sudoku grid as a human would do although it has not be specifically designed to.

Ideally, iterating local reasoning will lead to a solution. However, for exceptionnaly hard grids, an enumerating phase (all constraint solvers provide tools for that) relying on backtracking may be necessary.

\section{See also}\label{sudokuandcp:seealso}\hypertarget{sudokuandcp:seealso}{}

\begin{itemize}
	\item \href{http://njussien.e-constraints.net/sudoku/eng-jouer.html}{SudokuHelper} a sudoku solver and helper applet developed with \emph{Choco}.
	\item \href{http://www.palmsudoku.com}{PalmSudoku} a rather complete list of rules and tips for solving sudokus
\end{itemize}

