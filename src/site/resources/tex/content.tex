\addcontentsline{toc}{chapter}{Preface}
\chapter*{Preface}
Choco is a java library for constraint satisfaction problems (CSP), constraint programming (CP) and explanation-based constraint solving (e-CP). It is built on a event-based propagation mechanism with backtrackable structures.
Choco is an open-source software, distributed under a \textbf{BSD licence} and hosted by \href{http://sourceforge.net/projects/choco/}{sourceforge.net}.
For any requests send a mail to \url{choco@emn.fr}.
\bigskip

\noindent This document is organized as follows:
\begin{itemize}
\item \hyperlink{ch:doc}{Documentation} is the user-guide of Choco. After a short \hypertarget{doc:introduction}{introduction} to constraint programming and to the Choco solver, it presents the basics of \hyperlink{doc:model}{modeling} and \hyperlink{doc:solver}{solving} with Choco, the \hyperlink{doc:advanced}{advanced usages} (customizing propagation and search), some examples of \hyperlink{doc:applications}{Applications}, and a \hyperlink{faq:frequentlyaskedquestions}{FAQ} section.
\item \hyperlink{part:elements}{Elements of Choco} gives a detailed description of the \hyperlink{ch:vars}{variables}, \hyperlink{ch:operators}{operators}, \hyperlink{ch:constraints}{constraints} currently available in Choco.
\item \hyperlink{ch:tut}{Tutorials} provides a \hyperlink{gettingstarted:gettingstarted:welcometochoco}{fast how-to} write a Choco program, a detailed example of a \hyperlink{gettingstarted:firstexample:magicsquare}{simple program}, and several \hyperlink{exercises}{exercises} with their \hyperlink{solutions}{solutions}.
\item \hyperlink{ch:extra}{Extras} presents future works, only available on the beta version or extension of the current jar, such as the \hyperlink{chocoandvisu:chocoandvisu}{visualization module of Choco}. The section dedicated to \hyperlink{sudokuandcp:sudokuandconstraintprogramming}{Sudoku} aims at explaining the basic principles of Constraint Programming (propagation and search) on this famous game.
%\begin{itemize}
%%	\item \hyperlink{chocoandgraphviz}{Choco and Graphviz} \emph{not yet available. Has been included in \hyperlink{chocoandvisu}{Choco and Visu}.}
%	\item 

\end{itemize}

\input{chapters/documentation.tex}
%\part{faq}
\label{faq}
\hypertarget{faq}{}

\chapter{Frequently Asked Questions}\label{faq:frequentlyaskedquestions}\hypertarget{faq:frequentlyaskedquestions}{}

\section{Where can I find Choco ?}\label{faq:wherecanifindchoco}\hypertarget{faq:wherecanifindchoco}{}

See the \href{http://choco.emn.fr}{download page} if you want to download a version of Choco library.

\section{What is the required Java version to run Choco ?}\label{faq:whatistherequiredjavaversiontorunchoco}\hypertarget{faq:whatistherequiredjavaversiontorunchoco}{}

Choco requires \href{http://java.sun.com/javase/6/}{java6}.

\begin{note}
If you are working on Mac OS X 10.4 Tiger or if you do not have an Intel processor, you probably can not install java 6 on your OS. Please, take a look at \href{http://landonf.bikemonkey.org/static/soylatte/}{Soy latte}, which goals are ``support for Java 6 Development on Mac OS X 10.4 and 10.5, OpenJDK support for Java 7 on Mac OS X and On-time release of Java 7 for Mac OS X''.
\end{note}

\section{How to add the Choco library to my project?}\label{faq:howtoaddthechocolibrarytomyproject}\hypertarget{faq:howtoaddthechocolibrarytomyproject}{}

% commented out template(flowplay>?640x480 noautoPlay)

You just need to add Choco.X.x.x.jar to your classpath.

\textbf{IntelliJ}:
\begin{itemize}
	\item Go to ``File/Settings''
	\item Select ``Project Settings''
	\item Click on ``Librairies'' then on [+]
	\item Enter ``Choco'' as the library name, and OK
	\item Choose your project, and OK,
	\item Click on ``Attach Jar Directories'' and choose the directory where you put the Choco jar.
\end{itemize}

\todo{fix hyperlink{flowplay>videos:intellij.flv}{How to add choco to IntelliJ in video}}

\textbf{Eclipse}:
\begin{itemize}
	\item Go to ``Project/Properties'',
	\item Select ``Java Build Path'' on the left menu
	\item On the right, select ``Librairies''
	\item Click on ``Add Externals JARs...'' button
	\item Select the Choco jar file
\end{itemize}

\todo{fix hyperlink{flowplay>videos:eclipse.flv}{How to add choco to Eclipse in video}}

\begin{note}
If you work with \textbf{the source and not the jar of Choco}, do not forget to also add the automaton.jar and junit.jar, available in the lib/ directory
\end{note}

\section{Why can't I see the Choco API?}\label{faq:whycan'tiseethechocoapi}\hypertarget{faq:whycan'tiseethechocoapi}{}
To have the Choco API available, you must make the following import in your class file:

\begin{lstlisting}
import static choco.Choco.*;
\end{lstlisting}

\section{How to know the value of my variable in the Solver ?}\label{faq:howtoknowthevalueofmyvariableinthesolver}\hypertarget{faq:howtoknowthevalueofmyvariableinthesolver}{}
There are two different kinds of variables: those associated with the Model (like \texttt{IntegerVariable}, \texttt{SetVariable},...) and those associated with the Solver (like \texttt{IntDomainVar}, \texttt{SetVar},...). The second type is a Solver interpretation of the first one (which is only declarative).
After having defined your model with variables and constraints, it has to be read by the Solver. After that, a Solver object is created. You can access the Variable Model \emph{value} through the Solver using the following method of the Solver:
\mylst{solver.getVar(Variable v);}
where v is a Model variable (or an array of Model variables) and it returns a Solver variable.

\section{How do I use constant value inside constraint ?}\label{faq:howdoiuseconstantevalueinsideconstraint}\hypertarget{faq:howdoiuseconstantevalueinsideconstraint}{}
Some constraints doesn't provide API with java object (like \textbf{int}, \textbf{double} or \textbf{Integer}). 
You can define \emph{constant} variable (ie, variable with one unique value) liek this:
\begin{lstlisting}
	IntegerVariable one = constant(1);
	RealVariable one = constant(1.0);
\end{lstlisting}
And, you can use this \emph{variable} inside the constraint:
\begin{lstlisting}
	Model m = new CPModel();
	
	IntegerVariable x = makeIntVar("x", 0, 10);
	IntegerVariable two = constant(2);
	IntegerVariable maximum = makeIntVar("max", 0, 15);
	
	m.addConstraint(eq(maximum, max(x, two));
\end{lstlisting}

Do not forget that some contraints provide api with java object.

\section{How can I use Choco to solve CSP'08 benchmark ?}\label{faq:howcaniusechocotosolvecsp'08benchmark}\hypertarget{faq:howcaniusechocotosolvecsp'08benchmark}{}
You can easily load an XML file of the CSP'08 competition and solve it with Choco.
To load the file, we use the XMLParser available \href{http://www.cril.univ-artois.fr/\~{}lecoutre/research/tools/tools.html}{here}:
\begin{lstlisting}
	String fileName = "../../ProblemsData/CSPCompet/intension/nonregres/graph1.xml";
	File instance = new File(fileName);
	XmlModel xs = new XmlModel();  // a class to ease loading and solving CSP'08 xml file
	InstanceParser parser = xs.load(instance);  // loading of the CPS'08 xml file
\end{lstlisting}
Once the file has been loaded, a Model object is build from the InstanceParser object:
\begin{lstlisting}
CPModel model = xs.buildModel(parser); // Creation of the model
\end{lstlisting}
At this point, you can choose to solve this model with a pre-processing step.
The pre-processing step analyzes variables and constraints, makes some specific choices to improve the resolution.
Concerning variables, it analyzes domains and constraints and choose what seems to be the best kind of domain (for example, enumerated or bounded domain), or add one variable where large number of variables are equals, ...
Concerning constraints, it detects clique of differences or disjunctions and state the corresponding global constraints, breaks symetries, detects distance...
Then, it can also choose the search strategy.
To do this, use the following code:
\begin{lstlisting}
PreProcessCPSolver s = xs.solve(model); // Build a BlackBoxSolver and solve it.
\end{lstlisting}
Finally, you can print informations concerning the resolution:
\begin{lstlisting}
\lstinline|xs.postAnalyze(instance, parser, s);
\end{lstlisting}

You can easily solve benchmarks of CSP'08 competition, or with your own problem modelize in \href{http://www.cril.univ-artois.fr/cpai08/xcsp21.pdf}{CSP'08 xml format}.

\section{How do I use the build.xml file ?}\label{faq:howdoiusethebuild.xmlfile}\hypertarget{faq:howdoiusethebuild.xmlfile}{}

The choco project provides an ant script \texttt{build.xml} for the most usual tasks of the project.
In this section, we show how to run these tasks with a terminal. Ant is fully integrated in most of Java IDE but we will not talk about it.

First, we are going into the root directory of choco
\begin{lstlisting}
	nono@arrakis:~\$ cd /path/to/choco/
	nono@arrakis:~/workspace/Choco-2.0\$ ls
	bin  build.xml  checkstyle.xml  Choco2.0.0.iml  choco-ruleset.xml  dev  lib  pom.xml
\end{lstlisting}
Then, try a simple 
\begin{lstlisting}
	nono@arrakis:~/workspace/Choco-2.0\$ ant
	Buildfile: build.xml
	
	init:
	     [echo] Ant  version                  : Apache Ant version 1.7.0 compiled on August 29 2007
	     [echo] Java version                  : 1.6.0_06
	     [echo] build of project JChoco : June 30 2008
	
	help:
	     [echo] be careful, there could have a bug in eclipse with this help message.
	     [echo] In this case, type "ant -p popart/build.xml" in a terminal.     
	     [exec] Buildfile: build.xml
	     [exec] 
	     [exec] Main targets:
	     [exec] 
	     [exec]  clean            --> deletes everything that seems useless
	     [exec]  compile          --> compiles everything
	     [exec]  dist             --> makes the distribution package (jar, src.zip, doc.zip)
	     [exec]  doc              --> generates the javadoc
	     [exec]  exec-junit-test  --> executes all junit tests.
	     [exec]  exec-pmd         --> analyzes code with PMD and CPD
	     [exec]  help             --> print this help
	     [exec] Default target: help
	
	BUILD SUCCESSFUL
	Total time: 1 second
\end{lstlisting}

Most of these tasks do not have some special requirements. However, you could need specific settings to run \texttt{exec-junit-test} and \texttt{exec-pmd}.
You need to have \texttt{junit} and \texttt{ant-junit} jars in your classpath to run \texttt{exec\_junit-test}. But, it works for my first attempt without any changes.
You need to set \texttt{pmd} jar in your classpath and probably to update the property \texttt{pmd.xslt} to run \texttt{exec\_junit-test}.
Supposes that you have installed \texttt{pmd} in \texttt{/path/to/pmd}. You have to reset the location of the following property:
\begin{lstlisting}
<property name="pmd.xslt" location="/path/to/pmd/etc/xslt/wz-pmd-report.xslt" />
\end{lstlisting}
Finally you can run the task with :
\begin{lstlisting}
ant -lib /path/to/pmd/lib/pmd-4.1.jar exec-pmd
\end{lstlisting}
It seems that using PMD with your IDE is more effective. The integration offers many filtering options and allows to correct your code on the fly.

\emph{A \href{http://www.emn.fr/x-info/choco-solver/forum/viewtopic.phpf=5&t=4&start=0&st=0&sk=t&sd=a}{post} is opened for feedbacks, for new feature requests, and for any comments about the \texttt{build.xml} file}

\section{Why do I have a error when I add my constraint ?}\label{faq:whydoihaveaerrorwheniaddmyconstraint}\hypertarget{faq:whydoihaveaerrorwheniaddmyconstraint}{}
If you have a error message like this:\\
\centerline{\emph{Component class could not be found: my.package.and.my.Constraint.ConstraintManager}}\\
and if the ConstraintManager is an inner class of your constraint, you must define the name in the component name like this:
\begin{lstlisting}
my.package.and.my.Constraint\$ConstraintManager
\end{lstlisting}

For more details, see \hyperlink{advanced:defineyourownconstraint}{define\ your\ own\ constraint}.

\section{How do I upgrade my program to Choco2.0 ?}\label{faq:howdoiupgrademyprogramtochoco2.0}\hypertarget{faq:howdoiupgrademyprogramtochoco2.0}{}
Without being very precise (see the \hyperlink{ch:doc}{documentation} if you want more details), it is really easy to transpose a program implemented on an old version of Choco to Choco2.0.

\textbf{No Problem!!} The \texttt{Problem} class does not exist anymore. It has been replaced by two new classes: \texttt{CPModel} and \texttt{CPSolver} that implement the interfaces \texttt{Model} and \texttt{Solver.} The model allows you to declare your variables and constraints and the solver allows you to define some search strategies and solve your model. As different kinds of Model and Solver will be available, everything concerning Variables and Constraints is included in the new class \texttt{Choco}. 

Now, let us see in a few steps how to transpose your program. Consider that you created the following program:
\begin{lstlisting}
	// Creation of the problem
	Problem pb = new Problem();
	
	// Declaration of variables
	IntDomainVar v1 = pb.makeEnumIntVar("v1", 1, 10);
	IntDomainVar v2 = pb.makeEnumIntVar("v1", 1, 10);
	
	// Declaration of constraints
	Constraint c1 = pb.neq(v1,v2);
	pb.post(c1);
	// Declaration of a user constraint
	Constraint prime-number = MyConstraint(v1, v2);
	pb.post(prime-number);
	
	// Definition of a search strategy
	pb.getSolver().setVarSelector(new StaticVarOrder(v1, v2));
	pb.getSolver().setValIterator(new IncreasingDomain());
	
	// Resolution of the problem
	pb.solve();
	
	// Print the solution
	System.out.println("v1"+v1.getVal());
	System.out.println("v2"+v2.getVal());
\end{lstlisting}

\begin{itemize}
	\item \textbf{A Problem becomes a Model and a Solver}
\end{itemize}

\begin{lstlisting}
	// Creation of the problem
	Problem pb = new Problem();
\end{lstlisting}
becomes
\begin{lstlisting}
	// Creation of the Model
	Model m = new CPModel();
	//Creation of the Solver
	Solver s = new CPSolver();
\end{lstlisting}

\begin{itemize}
	\item \textbf{Variables are independent of a Problem or a Model}
\end{itemize}

\begin{lstlisting}
	// Declaration of variables
	IntDomainVar v1 = pb.makeEnumIntVar("v1", 1, 10);
	IntDomainVar v2 = pb.makeEnumIntVar("v1", 1, 10);
\end{lstlisting}
becomes
\begin{lstlisting}
	// add import:
	import static choco.Choco.*;
	//...
	// Declaration of variables
	IntegerVariable v1 = makeIntVar("v1", 1, 10);
	IntegerVariable v2 = makeIntVar("v1", 1, 10);
	m.addVariable(CPOptions.V_ENUM, v1, v2);
\end{lstlisting}

\begin{itemize}
	\item \textbf{Easy declaration of constraints}
\end{itemize}

\begin{lstlisting}
	// Declaration of constraints
	Constraint c1 = pb.neq(v1,v2);
	pb.post(c1);
\end{lstlisting}
becomes
\begin{lstlisting}
	// add import (same than Variables):
	import static choco.Choco.*;
	//...
	// Declaration of constraints
	Constraint c1 = neq(v1,v2);
	m.addConstraint(c1);
\end{lstlisting}

\begin{itemize}
	\item \textbf{A specific way to define user constraints}
\end{itemize}

\begin{lstlisting}
	// Declaration of a user constraint
	Constraint prime-number = myConstraint(v1, v2);
	pb.post(prime-number);
\end{lstlisting}
becomes
\begin{lstlisting}
	// Declaration of a user constraint
	m.addConstraint(new ComponentConstraint(MyManager.class, null, v1, v2));
\end{lstlisting}

\begin{itemize}
	\item \textbf{Do not forget to read the model}
\end{itemize}

It is a \emph{\textbf{new step}}, it has to be done! 
\begin{lstlisting}
	// Read the model
	s.read(m);
\end{lstlisting}

\begin{itemize}
	\item \textbf{Clear definition of the search strategy}
\end{itemize}

\begin{lstlisting}
	// Definition of a search strategy
	pb.getSolver().setVarSelector(new StaticVarOrder(v1, v2));
	pb.getSolver().setValIterator(new IncreasingDomain());
\end{lstlisting}
becomes
\begin{lstlisting}
	// Definition of a search strategy
	s.setVarIntSelector(new StaticVarOrder(s.getVar(v1, v2)));
	s.setValIntIterator(new IncreasingDomain());
\end{lstlisting}

\begin{itemize}
	\item \textbf{And the resolution}
\end{itemize}

\begin{lstlisting}
	// Resolution of the problem
	pb.solve();
\end{lstlisting}
becomes
\begin{lstlisting}
	// Resolution of the model
	s.solve();
\end{lstlisting}

\begin{itemize}
	\item \textbf{Printing the solution}
\end{itemize}

\begin{lstlisting}
	// Print the solution
	System.out.println("v1"+v1.getVal());
	System.out.println("v2"+v2.getVal());
\end{lstlisting}
becomes
\begin{lstlisting}
	// Print the solution
	System.out.println("v1"+s.getVar(v1).getVal());
	System.out.println("v2"+s.getVar(v2).getVal());
\end{lstlisting}

And it's done!
We obtain the following code:
\begin{lstlisting}
	import static choco.Choco.*;
	...
	// Creation of the Model
	Model m = new CPModel();
	//Creation of the Solver
	Solver s = new CPSolver();
	
	// Declaration of variables
	IntegerVariable v1 = makeIntVar("v1", 1, 10);
	IntegerVariable v2 = makeIntVar("v2", 1, 10);
	m.addVariable(CPOptions.V_ENUM,v1, v2);
	
	// Declaration of constraints
	Constraint c1 = neq(v1,v2);
	m.addConstraint(c1);
	// Declaration of a user constraint
	m.addConstraint(new ComponentConstraint(MyManager.class, null, v1, v2));
	
	// Read the model
	s.read(m);
	
	// Definition of a search strategy
	s.setVarIntSelector(new StaticVarOrder(s.getVar(v1, v2)));
	s.setValIntIterator(new IncreasingDomain());
	
	// Resolution of the model
	s.solve();
	
	// Print the solution
	System.out.println("v1"+s.getVar(v1).getVal());
	System.out.println("v2"+s.getVar(v2).getVal());
\end{lstlisting}

\section{Are bounds with positive and negative infinity supported within Choco?}\label{faq:areboundswithpositiveandnegativeinfinitysupportedwithinchoco}\hypertarget{faq:areboundswithpositiveandnegativeinfinitysupportedwithinchoco}{}

Integer or Double infinity bounds are not really appreciate by CHOCO :) 
Because, during propagation, a basic test is done on bounds and the following operation can be applied: 
\emph{upper bound +1}.
As \texttt{Integer.MAX\_VALUE+1} is equal to \texttt{Integer.MIN\_VALUE}, it can corrupt the propagation. 

If you really want to have a large domain, a division with 10 should be sufficient: 
\begin{lstlisting}
IntegerVariable v1 = makeIntVar("v1", Integer.MIN_VALUE/10, Integer.MIN_VALUE/10);
RealVariable a1 = makeRealVar("A1", Double.NEGATIVE_INFINITY/10, Double.POSITIVE_INFINITY/10);
\end{lstlisting}


%\chapter{Variables}\label{ch:vars}\hypertarget{ch:vars}{}
%\section{Integer variables}\label{integervariable}\hypertarget{integervariable}{}
\texttt{IntegerVariable} is a variable whose associated domain is made of integer values. 

\subsubsection{constructors:}
      \noindent\begin{tabular}{p{.8\linewidth}p{.15\linewidth}}
        Choco method & return type \\
        \hline
        \mylst{makeIntVar(String name, int lowB, int uppB, String... options)} &\texttt{IntegerVariable}\\
		\mylst{makeIntVar(String name, List<Integer> values, String... options)} &\texttt{IntegerVariable}\\
		\mylst{makeIntVar(String name, int[] values, String... options)} &\texttt{IntegerVariable}\\
        \mylst{makeBooleanVar(String name, String... options)}  &\texttt{IntegerVariable}\\
        \mylst{makeIntVarArray(String name, int dim, int lowB, int uppB, String... options)} &\texttt{IntegerVariable[]}\\
        \mylst{makeIntVarArray(String name, int dim, int[] values, String... options)} &\texttt{IntegerVariable[]}\\
        \mylst{makeBooleanVarArray(String name, int dim, String... options)}  &\texttt{IntegerVariable[]}\\
        \mylst{makeIntVarArray(String name, int dim1, int dim2, int lowB, int uppB, String... options)}  &\texttt{IntegerVariable[][]}\\
        \mylst{makeIntVarArray(String name, int dim1, int dim2, int[] values, String... options)}  &\texttt{IntegerVariable[][]}\\
      \end{tabular}
% 	\begin{itemize}
% 		\item to create an \textbf{IntegerVariable} object:
% 		\begin{itemize}
% 			\item \mylst{makeIntVar(String name, int lowB, int uppB, String... options)}
% 			\item \mylst{makeIntVar(String name, List<Integer> values, String... options)}
% 			\item \mylst{makeIntVar(String name, int[] values, String... options)}
% 		\end{itemize}
% 		\item to create an \textbf{array of IntegerVariable} object:
% 		\begin{itemize}
% 			\item \mylst{makeIntVarArray(String name, int dim, int lowB, int uppB, String... options)}
% 			\item \mylst{makeIntVarArray(String name, int dim, int[] values, String... options)}
% 		\end{itemize}
% 		\item to create a \textbf{matrix of IntegerVariable} object:
% 		\begin{itemize}
% 			\item \mylst{makeIntVarArray(String name, int dim1, int dim2, int lowB, int uppB, String... options)}
% 			\item \mylst{makeIntVarArray(String name, int dim1, int dim2, int[] values, String... options)}
% 		\end{itemize}
% 		\item to create an \textbf{IntegerVariable} object with pre defined domain [0,1]:
% 		\begin{itemize}
% 			\item \mylst{makeBooleanVar(String name, String... options)}
% 		\end{itemize}
% 		\item to create an \textbf{array of IntegerVariable} object with pre defined domain [0,1]:
% 		\begin{itemize}
% 			\item \mylst{makeBooleanVarArray(String name, int dim, String... options)}
% 		\end{itemize}
% 	\end{itemize}
% 	\item \textbf{return type} : \texttt{IntegerVariable} \emph{or} \texttt{IntegerVariable[]} \emph{or} \texttt{IntegerVariable[][]}
\subsubsection{options:}
	\begin{itemize}
		\item \emph{no option} : equivalent to option \texttt{cp:enum}
		\item \texttt{cp:enum} : to force Solver to create enumerated domain for the variable. It is a domain in which holes can be created by the solver. It should be used when discrete and quite small domains are needed and when constraints performing Arc Consistency are added on the corresponding variables. Implemented by a \texttt{BitSet} object.
		\item \texttt{cp:bound} : to force Solver to create bounded domain for the variable. It is a domain where only bound propagation can be done (no holes). It is very well suited when constraints performing only Bound Consistency are added on the corresponding variables. It must be used when large domains are needed. Implemented by two integers.
		\item \texttt{cp:link} : to force Solver to create linked list domain for the variable. It is an enumerated domain where holes can be done and every values has a link to the previous value and to the next value. It is built by giving its name and its bounds: lower bound and upper bound. It must be used when the very small domains are needed, because although linked list domain consumes more memory than the \texttt{BitSet} implementation, it can provide good performance as iteration over the domain is made in constant time. Implemented by a \texttt{LinkedList} object.
		\item \texttt{cp:btree} : to force Solver to create binary tree domain for the variable. \emph{Under development}.
		\item \texttt{cp:blist} : to force Solver to create bipartite list domain for the variable. It is a domain where unavailable values are placed in the left part of the list, the other one on the right one.
		\item \texttt{cp:decision} : to force variable to be a decisional one
		\item \texttt{cp:no\_decision} : to force variable to be removed from the pool of decisional variables
		\item \texttt{cp:objective} : to define the variable to be the one to optimize
	\end{itemize}
\subsubsection{methods:}
      \begin{itemize}
      \item \mylst{removeVal(int val)}: remove value \emph{val} from the domain of the current variable
      \end{itemize}

A variable with $\{0,1\}$ domain is automatically considered as boolean domain.

\subsubsection{Example:}
\begin{lstlisting}
  IntegerVariable ivar1 = makeIntVar("ivar1", -10, 10);
  IntegerVariable ivar2 = makeIntVar("ivar2", 0, 10000, "cp:bound", "cp:decision");
  IntegerVariable bool = makeBooleanVar("bool");
\end{lstlisting} 

%\section{Real variables}\label{realvariable}\hypertarget{realvariable}{}
\texttt{RealVariable} is a variable whose associated domain is made of real values. Only enumerated domain is available for real variables. 

Such domain are memory consuming. In order to minimize the memory use and to have the precision you need, the model offers a way to set a precision (default value is 1.0e-6):
\begin{lstlisting}
	Model m = new CPModel();
	m.setPrecision(0.01);
\end{lstlisting}

\subsubsection{constructor:}
      \noindent\begin{tabular}{p{.8\linewidth}p{.15\linewidth}}
        Choco method & return type \\
        \hline
        \mylst{makeRealVar(String name, double lowB, double uppB, String... options)} &\texttt{RealVariable}\\
      \end{tabular}
%	\begin{itemize}
%		\item to create a \textbf{RealVariable} object:
%		\begin{itemize}
%			\item \mylst{makeRealVar(String name, double lowB, double uppB, String... options)}
%		\end{itemize}
%	\end{itemize}
%	\item \textbf{return type} : \texttt{RealVariable}
\subsubsection{options:}
	\begin{itemize}
		\item \emph{no option} : no particular choice on decision or objective.
		\item \texttt{CPOptions.V_DECISION} : to force variable to be a decisional one
		\item \texttt{cp:no\_decision} : to force variable to be removed from the pool of decisionnal variables
		\item \texttt{CPOptions.V_OBJECTIVE} : to define the variable to be the one to optimize
	\end{itemize}

\subsubsection{Example:}
\begin{lstlisting}
	RealVariable rvar1 = makeRealVar("rvar0", -10.0, 10.0);
	RealVariable rvar2 = makeRealVar("rvar2", 0.0, 100.0,CPOptions.V_DECISION, CPOptions.V_OBJECTIVE);
\end{lstlisting} 

%\section{Set variables}\label{setvariable}\hypertarget{setvariable}{}
\texttt{SetVariable} is high level modeling tool. It allows to represent variable whose values are sets. A SetVariable on integer values between $[1,n]$ has $2*n$ values (every possible subsets of $\{1..n\}$). This makes an exponential number of values and the domain is represented with two bounds corresponding to the intersection of all possible sets (called the kernel) and the union of all possible sets (called the envelope) which are the possible candidate values for the variable. The consistency achieved on SetVariables is therefore a kind of bound consistency.

\subsubsection{constructors:}
      \noindent\begin{tabular}{p{.8\linewidth}p{.15\linewidth}}
        Choco method & return type \\
        \hline
        \mylst{makeSetVar(String name, int lowB, int uppB, String... options)} &\texttt{SetVariable}\\
        \mylst{makeSetVarArray(String name, int dim, int lowB, int uppB, String... options)} &\texttt{SetVariable[]}
      \end{tabular}
%	\begin{itemize}
%		\item to create an \textbf{SetVariable} object:
%		\begin{itemize}
%			\item \mylst{makeSetVar(String name, int lowB, int uppB, String... options)}
%		\end{itemize}
%		\item to create an \textbf{array of SetVariable} object:
%		\begin{itemize}
%			\item \mylst{makeSetVarArray(String name, int dim, int lowB, int uppB, String... options)}
%		\end{itemize}
%	\end{itemize}
%	\item \textbf{return type} : \texttt{SetVariable} \emph{or} \texttt{SetVariable[]}
\subsubsection{options:}
	\begin{itemize}
		\item \emph{no option} : equivalent to option \texttt{CPOptions.V_ENUM}
		\item \texttt{CPOptions.V_ENUM} : to force Solver to create \texttt{SetVariable} with enumerated domain for the caridinality variable. It is a domain in which holes can be created by the solver. It should be used when set variable cardinality domain is discrete, quite small and constraints performing reasonings on holes in the cardinality are present in the model. Implemeted by two \texttt{BitSets} for upper and lower bounds and an enumerated \texttt{IntegerVariable} for the cardinality.
		\item \texttt{CPOptions.V_BOUND} : to force Solver to create \texttt{SetVariable} with bounded cardinality. It is a domain where only bound propagation can be done (no holes). It is very well suited when constraints performing only Bound Consistency are added on the corresponding variables. It must be used when large domains are needed. Implemented by two integers.
		\item \texttt{CPOptions.V_DECISION} : to force variable to be a decisional one
		\item \texttt{cp:no\_decision} : to force variable to be removed from the pool of decisionnal variables
		\item \texttt{CPOptions.V_OBJECTIVE} : to define the variable to be the one to optimize
	\end{itemize}

The variable representing the cardinality can be accessed and constrained using method \texttt{getCard()} that returns an \hyperlink{integervariable}{\tt IntegerVariable} object.

\subsubsection{Example:}
\begin{lstlisting}
  SetVariable svar1 = makeSetVar("svar1", -10, 10);
  setVariable svar2 = makeSetVar("svar2", 0, 10000, CPOptions.V_BOUND, CPOptions.V_NO_DECISION);
\end{lstlisting} 

Set variables are illustrated on the \hyperlink{model:example2:ternarysteinerchoco}{ternary Steiner problem}. 




\part{Elements of Choco}\label{part:elements}\hypertarget{part:elements}{}
\chapter{Variables (Model)}\label{ch:vars}\hypertarget{ch:vars}{}
This section describes the three kinds of \hyperlink{model:variables}{variables} that can be used within a Choco Model.
\section{Integer variables}\label{integervariable}\hypertarget{integervariable}{}
\texttt{IntegerVariable} is a variable whose associated domain is made of integer values. 

\subsubsection{constructors:}
      \noindent\begin{tabular}{p{.8\linewidth}p{.15\linewidth}}
        Choco method & return type \\
        \hline
        \mylst{makeIntVar(String name, int lowB, int uppB, String... options)} &\texttt{IntegerVariable}\\
		\mylst{makeIntVar(String name, List<Integer> values, String... options)} &\texttt{IntegerVariable}\\
		\mylst{makeIntVar(String name, int[] values, String... options)} &\texttt{IntegerVariable}\\
        \mylst{makeBooleanVar(String name, String... options)}  &\texttt{IntegerVariable}\\
        \mylst{makeIntVarArray(String name, int dim, int lowB, int uppB, String... options)} &\texttt{IntegerVariable[]}\\
        \mylst{makeIntVarArray(String name, int dim, int[] values, String... options)} &\texttt{IntegerVariable[]}\\
        \mylst{makeBooleanVarArray(String name, int dim, String... options)}  &\texttt{IntegerVariable[]}\\
        \mylst{makeIntVarArray(String name, int dim1, int dim2, int lowB, int uppB, String... options)}  &\texttt{IntegerVariable[][]}\\
        \mylst{makeIntVarArray(String name, int dim1, int dim2, int[] values, String... options)}  &\texttt{IntegerVariable[][]}\\
      \end{tabular}
% 	\begin{itemize}
% 		\item to create an \textbf{IntegerVariable} object:
% 		\begin{itemize}
% 			\item \mylst{makeIntVar(String name, int lowB, int uppB, String... options)}
% 			\item \mylst{makeIntVar(String name, List<Integer> values, String... options)}
% 			\item \mylst{makeIntVar(String name, int[] values, String... options)}
% 		\end{itemize}
% 		\item to create an \textbf{array of IntegerVariable} object:
% 		\begin{itemize}
% 			\item \mylst{makeIntVarArray(String name, int dim, int lowB, int uppB, String... options)}
% 			\item \mylst{makeIntVarArray(String name, int dim, int[] values, String... options)}
% 		\end{itemize}
% 		\item to create a \textbf{matrix of IntegerVariable} object:
% 		\begin{itemize}
% 			\item \mylst{makeIntVarArray(String name, int dim1, int dim2, int lowB, int uppB, String... options)}
% 			\item \mylst{makeIntVarArray(String name, int dim1, int dim2, int[] values, String... options)}
% 		\end{itemize}
% 		\item to create an \textbf{IntegerVariable} object with pre defined domain [0,1]:
% 		\begin{itemize}
% 			\item \mylst{makeBooleanVar(String name, String... options)}
% 		\end{itemize}
% 		\item to create an \textbf{array of IntegerVariable} object with pre defined domain [0,1]:
% 		\begin{itemize}
% 			\item \mylst{makeBooleanVarArray(String name, int dim, String... options)}
% 		\end{itemize}
% 	\end{itemize}
% 	\item \textbf{return type} : \texttt{IntegerVariable} \emph{or} \texttt{IntegerVariable[]} \emph{or} \texttt{IntegerVariable[][]}
\subsubsection{options:}
	\begin{itemize}
		\item \emph{no option} : equivalent to option \texttt{cp:enum}
		\item \texttt{cp:enum} : to force Solver to create enumerated domain for the variable. It is a domain in which holes can be created by the solver. It should be used when discrete and quite small domains are needed and when constraints performing Arc Consistency are added on the corresponding variables. Implemented by a \texttt{BitSet} object.
		\item \texttt{cp:bound} : to force Solver to create bounded domain for the variable. It is a domain where only bound propagation can be done (no holes). It is very well suited when constraints performing only Bound Consistency are added on the corresponding variables. It must be used when large domains are needed. Implemented by two integers.
		\item \texttt{cp:link} : to force Solver to create linked list domain for the variable. It is an enumerated domain where holes can be done and every values has a link to the previous value and to the next value. It is built by giving its name and its bounds: lower bound and upper bound. It must be used when the very small domains are needed, because although linked list domain consumes more memory than the \texttt{BitSet} implementation, it can provide good performance as iteration over the domain is made in constant time. Implemented by a \texttt{LinkedList} object.
		\item \texttt{cp:btree} : to force Solver to create binary tree domain for the variable. \emph{Under development}.
		\item \texttt{cp:blist} : to force Solver to create bipartite list domain for the variable. It is a domain where unavailable values are placed in the left part of the list, the other one on the right one.
		\item \texttt{cp:decision} : to force variable to be a decisional one
		\item \texttt{cp:no\_decision} : to force variable to be removed from the pool of decisional variables
		\item \texttt{cp:objective} : to define the variable to be the one to optimize
	\end{itemize}
\subsubsection{methods:}
      \begin{itemize}
      \item \mylst{removeVal(int val)}: remove value \emph{val} from the domain of the current variable
      \end{itemize}

A variable with $\{0,1\}$ domain is automatically considered as boolean domain.

\subsubsection{Example:}
\begin{lstlisting}
  IntegerVariable ivar1 = makeIntVar("ivar1", -10, 10);
  IntegerVariable ivar2 = makeIntVar("ivar2", 0, 10000, "cp:bound", "cp:decision");
  IntegerVariable bool = makeBooleanVar("bool");
\end{lstlisting} 

\section{Real variables}\label{realvariable}\hypertarget{realvariable}{}
\texttt{RealVariable} is a variable whose associated domain is made of real values. Only enumerated domain is available for real variables. 

Such domain are memory consuming. In order to minimize the memory use and to have the precision you need, the model offers a way to set a precision (default value is 1.0e-6):
\begin{lstlisting}
	Model m = new CPModel();
	m.setPrecision(0.01);
\end{lstlisting}

\subsubsection{constructor:}
      \noindent\begin{tabular}{p{.8\linewidth}p{.15\linewidth}}
        Choco method & return type \\
        \hline
        \mylst{makeRealVar(String name, double lowB, double uppB, String... options)} &\texttt{RealVariable}\\
      \end{tabular}
%	\begin{itemize}
%		\item to create a \textbf{RealVariable} object:
%		\begin{itemize}
%			\item \mylst{makeRealVar(String name, double lowB, double uppB, String... options)}
%		\end{itemize}
%	\end{itemize}
%	\item \textbf{return type} : \texttt{RealVariable}
\subsubsection{options:}
	\begin{itemize}
		\item \emph{no option} : no particular choice on decision or objective.
		\item \texttt{CPOptions.V_DECISION} : to force variable to be a decisional one
		\item \texttt{cp:no\_decision} : to force variable to be removed from the pool of decisionnal variables
		\item \texttt{CPOptions.V_OBJECTIVE} : to define the variable to be the one to optimize
	\end{itemize}

\subsubsection{Example:}
\begin{lstlisting}
	RealVariable rvar1 = makeRealVar("rvar0", -10.0, 10.0);
	RealVariable rvar2 = makeRealVar("rvar2", 0.0, 100.0,CPOptions.V_DECISION, CPOptions.V_OBJECTIVE);
\end{lstlisting} 

\section{Set variables}\label{setvariable}\hypertarget{setvariable}{}
\texttt{SetVariable} is high level modeling tool. It allows to represent variable whose values are sets. A SetVariable on integer values between $[1,n]$ has $2*n$ values (every possible subsets of $\{1..n\}$). This makes an exponential number of values and the domain is represented with two bounds corresponding to the intersection of all possible sets (called the kernel) and the union of all possible sets (called the envelope) which are the possible candidate values for the variable. The consistency achieved on SetVariables is therefore a kind of bound consistency.

\subsubsection{constructors:}
      \noindent\begin{tabular}{p{.8\linewidth}p{.15\linewidth}}
        Choco method & return type \\
        \hline
        \mylst{makeSetVar(String name, int lowB, int uppB, String... options)} &\texttt{SetVariable}\\
        \mylst{makeSetVarArray(String name, int dim, int lowB, int uppB, String... options)} &\texttt{SetVariable[]}
      \end{tabular}
%	\begin{itemize}
%		\item to create an \textbf{SetVariable} object:
%		\begin{itemize}
%			\item \mylst{makeSetVar(String name, int lowB, int uppB, String... options)}
%		\end{itemize}
%		\item to create an \textbf{array of SetVariable} object:
%		\begin{itemize}
%			\item \mylst{makeSetVarArray(String name, int dim, int lowB, int uppB, String... options)}
%		\end{itemize}
%	\end{itemize}
%	\item \textbf{return type} : \texttt{SetVariable} \emph{or} \texttt{SetVariable[]}
\subsubsection{options:}
	\begin{itemize}
		\item \emph{no option} : equivalent to option \texttt{CPOptions.V_ENUM}
		\item \texttt{CPOptions.V_ENUM} : to force Solver to create \texttt{SetVariable} with enumerated domain for the caridinality variable. It is a domain in which holes can be created by the solver. It should be used when set variable cardinality domain is discrete, quite small and constraints performing reasonings on holes in the cardinality are present in the model. Implemeted by two \texttt{BitSets} for upper and lower bounds and an enumerated \texttt{IntegerVariable} for the cardinality.
		\item \texttt{CPOptions.V_BOUND} : to force Solver to create \texttt{SetVariable} with bounded cardinality. It is a domain where only bound propagation can be done (no holes). It is very well suited when constraints performing only Bound Consistency are added on the corresponding variables. It must be used when large domains are needed. Implemented by two integers.
		\item \texttt{CPOptions.V_DECISION} : to force variable to be a decisional one
		\item \texttt{cp:no\_decision} : to force variable to be removed from the pool of decisionnal variables
		\item \texttt{CPOptions.V_OBJECTIVE} : to define the variable to be the one to optimize
	\end{itemize}

The variable representing the cardinality can be accessed and constrained using method \texttt{getCard()} that returns an \hyperlink{integervariable}{\tt IntegerVariable} object.

\subsubsection{Example:}
\begin{lstlisting}
  SetVariable svar1 = makeSetVar("svar1", -10, 10);
  setVariable svar2 = makeSetVar("svar2", 0, 10000, CPOptions.V_BOUND, CPOptions.V_NO_DECISION);
\end{lstlisting} 

Set variables are illustrated on the \hyperlink{model:example2:ternarysteinerchoco}{ternary Steiner problem}. 



\chapter{Operators}\label{ch:operators}\hypertarget{ch:operators}{}
This section lists and details the \hyperlink{model:expressionvariables}{operators} that can be used within a Choco Model to combine variables in expressions.
\section{abs (operator)}\label{abs:absoperator}\hypertarget{abs:absoperator}{}
Returns an expression variable that represents the absolute value of the argument (\(|n|\)).

\begin{itemize}
	\item \textbf{API} : abs(IntegerExpressionVariable n)
	\item \textbf{return type} : IntegerExpressionVariable
	\item \textbf{options} : \emph{n/a}
	\item \textbf{favorite domain} : unknown
\end{itemize}

\textbf{Example}:
\begin{lstlisting}
	CPModel m = new CPModel();
	IntegerVariable x = makeIntVar("x", 1, 5, "cp:enum");
	IntegerVariable y = makeIntVar("y", -5, 5, "cp:enum");
	m.addConstraint(eq(abs(x), y));
	CPSolver s = new CPSolver();
	s.read(m);
	s.solve();
	Assert.assertEquals(s.getVar(x).getVal(),Math.abs(s.getVar(y).getVal()));
\end{lstlisting}

%%% Local Variables: 
%%% mode: latex
%%% TeX-master: t
%%% End: 

%\part{cos}
\label{cos}
\hypertarget{cos}{}

\section{cos (operator)}\label{cos:cosoperator}\hypertarget{cos:cosoperator}{}
Returns an expression variable corresponding to the cosinus value of the argument (\(cos(x)\)).

\begin{itemize}
	\item \textbf{API} : cos(RealExpressionVariable exp)
	\item \textbf{return type} : RealExpressionVariable
	\item \textbf{options} : \emph{n/a}
	\item \textbf{favorite domain} : real
\end{itemize}

\textbf{Example}:

\emph{No valid example for the moment}

\input{chapters/Odiv.tex}
\input{chapters/Ofalse.tex}

\section{ifThenElse (operator)}\label{ifthenelse:ifthenelseoperator}\hypertarget{ifthenelse:ifthenelseoperator}{}

\emph{To complete}

% \todo{verify...}
% Returns the second argument if the first argument is satisfied, and returns the third argument otherwise.

% \begin{itemize}
% \item \textbf{API}: \mylst{ifThenElse(Constraint c1, IntegerExpressionVariable c2, IntegerExpressionVariable c3)}
% \item \textbf{return type} : IntegerExpressionVariable
% \item \textbf{options} : \emph{n/a}
% \item \textbf{favorite domain} : \emph{n/a}
% \end{itemize}

% \textbf{Example}:
% \begin{lstlisting}
% 	CPModel m = new CPModel();
% 	CPSolver s = new CPSolver();
	
% 	IntegerVariable x = makeIntVar("x", 1, 3);
% 	IntegerVariable y = makeIntVar("y", 1, 3);
% 	IntegerVariable z = makeIntVar("z", 1, 3);
	
	
% 	s.read(m);
% 	s.solveAll();
% \end{lstlisting}

%%% Local Variables: 
%%% mode: latex
%%% TeX-master: t
%%% End: 


\section{max (operator)}\label{max:maxoperator}\hypertarget{max:maxoperator}{}
Returns an expression variable equals to the greater value of the argument (\(max(x_1, x_2, ..., x_n)\)).

\begin{itemize}
	\item \textbf{API} :
	\begin{itemize}
		\item max(IntegerExpressionVariable x1, IntegerExpressionVariable x2)
		\item max(int x1, IntegerExpressionVariable x2)
		\item max(IntegerExpressionVariable x1, int x2)
		\item max(IntegerExpressionVariable[] x)
	\end{itemize}
	\item \textbf{return type}: IntegerExpressionVariable
	\item \textbf{options} : \emph{n/a}
	\item \textbf{favorite domain} : \emph{to complete}
\end{itemize}

\textbf{Example}:
\begin{lstlisting}
	Model m = new CPModel();
	m.setDefaultExpressionDecomposition(true);
	IntegerVariable[] v = makeIntVarArray("v", 3, -3, 3);
	IntegerVariable minv = makeIntVar("min", -3, 3);
	Constraint c = eq(minv, max(v));
	m.addConstraint(c);
	Solver s = new CPSolver();
	s.read(m);
	s.solveAll();
\end{lstlisting}

%%% Local Variables: 
%%% mode: latex
%%% TeX-master: t
%%% End: 


\section{min (operator)}\label{min:minoperator}\hypertarget{min:minoperator}{}
Returns an expression variable equals to the smaller value of the argument (\(min(x_1, x_2, ..., x_n)\)).

\begin{itemize}
	\item \textbf{API} :
	\begin{itemize}
		\item min(IntegerExpressionVariable x1, IntegerExpressionVariable x2)
		\item min(int x1, IntegerExpressionVariable x2)
		\item min(IntegerExpressionVariable x1, int x2)
		\item min(IntegerExpressionVariable[] x)
	\end{itemize}
	\item \textbf{return type}: IntegerExpressionVariable
	\item \textbf{options} : \emph{n/a}
	\item \textbf{favorite domain} : \emph{to complete}
\end{itemize}

\textbf{Example}:
\begin{lstlisting}
	Model m = new CPModel();
	m.setDefaultExpressionDecomposition(true);
	IntegerVariable[] v = makeIntVarArray("v", 3, -3, 3);
	IntegerVariable maxv = makeIntVar("max", -3, 3);
	Constraint c = eq(maxv, min(v));
	m.addConstraint(c);
	Solver s = new CPSolver();
	s.read(m);
	s.solveAll();
\end{lstlisting}

%%% Local Variables: 
%%% mode: latex
%%% TeX-master: t
%%% End: 

\input{chapters/Ominus.tex}

\section{mod (operator)}\label{mod:modoperator}\hypertarget{mod:modoperator}{}
Returns an expression variable that represents the integer remainder of the division of the first argument variable by the second one (\(x_1\%x_2\)).

\begin{itemize}
	\item \textbf{API}:
	\begin{itemize}
		\item mod(IntegerExpressionVariable x1, IntegerExpressionVariable x2)
		\item mod(int x1, IntegerExpressionVariable x2)
		\item mod(IntegerExpressionVariable x1, int x2)
	\end{itemize}
	\item \textbf{return type} : \texttt{IntegerExpressionVariable}
	\item \textbf{options} : \emph{n/a}
	\item \textbf{favorite domain} : \emph{n/a}
\end{itemize}

\textbf{Example}:
\begin{lstlisting}
	Model m = new CPModel();
	Solver s = new CPSolver();
	
	IntegerVariable x = makeIntVar("x", 1, 10);
	IntegerVariable w = makeIntVar("w", 22, 44);
	
	m.addConstraint(eq(1, mod(w,x)));
	
	s.read(m);
	s.solve();
\end{lstlisting}

%%% Local Variables: 
%%% mode: latex
%%% TeX-master: t
%%% End: 

%\part{mult}
\label{mult}
\hypertarget{mult}{}

\section{mult (operator)}\label{mult:multoperator}\hypertarget{mult:multoperator}{}
Returns an expression variable that corresponding to the product of variables in argument (\(x*y\)).

\begin{itemize}
	\item \textbf{API} :
	\begin{itemize}
		\item mult(IntegerExpressionVariable x, IntegerExpressionVariable y)
		\item mult(IntegerExpressionVariable x, int y)
		\item mult(int x, IntegerExpressionVariable y)
		\item mult(RealExpressionVariable x, RealExpressionVariable y)
		\item mult(RealExpressionVariable x, double y)
		\item mult(double x, RealExpressionVariable y)
	\end{itemize}
	\item \textbf{return type} :
	\begin{itemize}
		\item \texttt{IntegerExpressionVariable}, if parameters are \texttt{IntegerExpressionVariable}
		\item \texttt{RealExpressionVariable}, if parameters are \texttt{RealExpressionVariable}
	\end{itemize}
	\item \textbf{options} : \emph{n/a}
	\item \textbf{favorite domain} : \emph{to complete}
\end{itemize}

\textbf{Example}
\begin{lstlisting}
	CPModel m = new CPModel();
	IntegerVariable x = makeIntVar("x", -10, 10);
	IntegerVariable z = makeIntVar("z", -10, 10);
	IntegerVariable w = makeIntVar("w", -10, 10);
	m.addVariable(x, z, w);
	
	CPSolver s = new CPSolver();
	// x >= z * w
	Constraint exp = geq(x, mult(z,w));
	
	m.setDefaultExpressionDecomposition(true);
	m.addConstraint(exp);
	
	s.read(m);
	s.solveAll();
\end{lstlisting}

%\part{neg}
\label{neg}
\hypertarget{neg}{}

\section{neg (operator)}\label{neg:negoperator}\hypertarget{neg:negoperator}{}

Returns an expression variable that is the opposite of the expression integer variable in argument (\(-x\)).

\begin{itemize}
	\item \textbf{API} : neg(IntegerExpressionVariable x)
	\item \textbf{return type} : \texttt{IntegerExpressionVariable}
	\item \textbf{options} : \emph{n/a}
	\item \textbf{favorite domain} : \emph{n/a}
\end{itemize}

\textbf{Example}:
\begin{lstlisting}
	Model m = new CPModel();
	Solver s = new CPSolver();
	
	IntegerVariable x = makeIntVar("x", -10, 10);
	IntegerVariable w = makeIntVar("w", -10, 10);
	// -x = w - 20
	m.addConstraint(eq(neg(x), minus(w, 20)));
	
	s.read(m);
	s.solve();
\end{lstlisting}

%\part{plus}
\label{plus}
\hypertarget{plus}{}



\section{plus (operator)}\label{plus:plusoperator}\hypertarget{plus:plusoperator}{}
Returns an expression variable that corresponding to the sum of the two arguments (\(x+y\)).

\begin{itemize}
	\item \textbf{API} :
	\begin{itemize}
		\item \mylst{plus(IntegerExpressionVariable x, IntegerExpressionVariable y)}
		\item \mylst{plus(IntegerExpressionVariable x, int y)}
		\item \mylst{plus(int x, IntegerExpressionVariable y)}
		\item \mylst{plus(RealExpressionVariable x, RealExpressionVariable y)}
		\item \mylst{plus(RealExpressionVariable x, double y)}
		\item \mylst{plus(double x, RealExpressionVariable y)}
	\end{itemize}
	\item \textbf{return type} :
	\begin{itemize}
		\item \texttt{IntegerExpressionVariable}, if parameters are \texttt{IntegerExpressionVariable}
		\item \texttt{RealExpressionVariable}, if parameters are \texttt{RealExpressionVariable}
	\end{itemize}
	\item \textbf{options} : \emph{n/a}
	\item \textbf{favorite domain} : \emph{to complete}
\end{itemize}

\textbf{Example}
% \begin{itemize}
% 	\item example1:
% \end{itemize}

\lstinputlisting{java/oplus.j2t}

% \begin{itemize}
% 	\item example2:
% \end{itemize}

% \begin{lstlisting}
% 	// Build model
% 	Model model = new CPModel();
% 	// Declare every letter as a variable
% 	IntegerVariable d = makeIntVar("d", 0, 9);
% 	IntegerVariable o = makeIntVar("o", 0, 9);
% 	IntegerVariable n = makeIntVar("n", 0, 9);
% 	IntegerVariable a = makeIntVar("a", 0, 9);
% 	IntegerVariable l = makeIntVar("l", 0, 9);
% 	IntegerVariable g = makeIntVar("g", 0, 9);
% 	IntegerVariable e = makeIntVar("e", 0, 9);
% 	IntegerVariable r = makeIntVar("r", 0, 9);
% 	IntegerVariable b = makeIntVar("b", 0, 9);
% 	IntegerVariable t = makeIntVar("t", 0, 9);
	
% 	// Declare every name as a variable
% 	IntegerVariable donald = makeIntVar("donald", 0, 1000000);
% 	IntegerVariable gerald = makeIntVar("gerald", 0, 1000000);
% 	IntegerVariable robert = makeIntVar("robert", 0, 1000000);
% 	model.addVariable("cp:bound", donald, gerald, robert);
	
% 	// Array of coefficients
% 	int[] coeff = new int[]{100000, 10000, 1000, 100, 10, 1};
	
% 	// Declare every combination of letter as an integer expression
% 	IntegerExpressionVariable donaldLetters = scalar(new IntegerVariable[]{d, o, n, a, l, d}, coeff);
% 	IntegerExpressionVariable geraldLetters = scalar(new IntegerVariable[]{g, e, r, a, l, d}, coeff);
% 	IntegerExpressionVariable robertLetters = scalar(new IntegerVariable[]{r, o, b, e, r, t}, coeff);
	
% 	// Add equality between name and letters combination
% 	model.addConstraint(eq(donaldLetters, donald));
% 	model.addConstraint(eq(geraldLetters, gerald));
% 	model.addConstraint(eq(robertLetters, robert));
% 	// Add constraint name sum
% 	model.addConstraint(eq(plus(donald, gerald), robert));
% 	// Add constraint of all different letters.
% 	model.addConstraint(allDifferent(new IntegerVariable[]{d, o, n, a, l, g, e, r, b, t}));
	
% 	// Build a solver
% 	Solver s = new CPSolver();
% 	// Read the model
% 	s.read(model);
% 	// Then solve it
% 	s.solve();
	
% 	// Print name value
% 	System.out.println("donald = " + s.getVar(donald).getVal());
% 	System.out.println("gerald = " + s.getVar(gerald).getVal());
% 	System.out.println("robert = " + s.getVar(robert).getVal());
% \end{lstlisting}

\input{chapters/Opower.tex}
%\part{scalar}
\label{scalar}
\hypertarget{scalar}{}

\section{scalar (operator)}\label{scalar:scalaroperator}\hypertarget{scalar:scalaroperator}{}
Return an integer expression that corresponds to the scalar product of coefficients array and variables array (\(c_1*x_1+c_2*x_2+...+c_n*x_n\)).

\begin{itemize}
	\item \textbf{API} :
	\begin{itemize}
		\item scalar(int[] c, IntegerVariable[] x)
		\item scalar(IntegerVariable[] x, int[] c)
	\end{itemize}
	\item \textbf{return type} : IntegerExpressionVariable
	\item \textbf{options} : \emph{n/a}
	\item \textbf{favorite domain} : \emph{to complete}
\end{itemize}

\textbf{Example}:

\begin{lstlisting}
	Model m = new CPModel();
	Solver s = new CPSolver();
	
	IntegerVariable[] vars = new IntegerVariable[n * n];
	for (int i = 0; i < n; i++)
	      for (int j = 0; j < n; j++) {
	           vars[i * n + j] = makeIntVar("C" + i + "_" + j, 1, n * n);
	      }
	IntegerVariable sum = makeIntVar("S", 1, n * n * (n * n + 1) / 2);
	
	m.addConstraint(eq(sum, n * (n * n + 1) / 2));
	for (int i = 0; i < n * n; i++)
	    for (int j = 0; j < i; j++)
	        m.addConstraint(neq(vars[i], vars[j]));
	int[] coeffs = new int[n];
	for (int i = 0; i < n; i++) {
	    coeffs[i] = 1;
	}
	
	for (int i = 0; i < n; i++) {
	    IntegerVariable[] col = new IntegerVariable[n];    
	    IntegerVariable[] row = new IntegerVariable[n];
	    for (int j = 0; j < n; j++) {
	        col[j] = vars[i * n + j];
	        row[j] = vars[j * n + i];
	    } 
	    m.addConstraint(eq(scalar(coeffs, row), sum));
	    m.addConstraint(eq(scalar(coeffs, col), sum));
	}
	s.read(m);
	s.solve();
	//System.out.println("" + pretty());
	for (int i = 0; i < n; i++) {
	    for (int j = 0; j < n; j++) {   
	        System.out.print("" + s.getVar(vars[i * n + j]).getVal());
	        if (s.getVar(vars[i * n + j]).getVal() > 9) System.out.print(" ");
	           else System.out.print("  ");
	        }
	    System.out.println("");
	}
\end{lstlisting}

\input{chapters/Osin.tex}
%\part{sum}
\label{sum}
\hypertarget{sum}{}

\section{sum (operator)}\label{sum:sumoperator}\hypertarget{sum:sumoperator}{}
Return an integer expression that corresponds to the sum of the variables given in argument (\(x_1+x_2+...+x_n\)).

\begin{itemize}
	\item \textbf{API}: sum(IntegerVariable... lv)
	\item \textbf{return type} : IntegerExpressionVariable
	\item \textbf{options} : \emph{n/a}
	\item \textbf{favorite domain} : \emph{to complete}
\end{itemize}

\textbf{Example} :
\begin{lstlisting}
	CPModel pb = new CPModel();
	IntegerVariable[] vs = new IntegerVariable[n];
	for (int i = 0; i < n; i++) {
	     vs[i] = makeIntVar("" + i, 0, n - 1);
	}
	for (int i = 0; i < n; i++) {
	    pb.addConstraint(occurrence(i, vs[i], vs));
	}
	pb.addConstraint(eq(sum(vs), n));     // contrainte redondante 1
	int[] coeff2 = new int[n - 1];
	IntegerVariable[] vs2 = new IntegerVariable[n - 1];
	for (int i = 1; i < n; i++) {
	    coeff2[i - 1] = i;
	    vs2[i - 1] = vs[i];
	}
	pb.addConstraint(eq(scalar(coeff2, vs2), n)); // contrainte redondante 2
	s.read(pb);
	s.solve();
	do {
	    for (int i = 0; i < vs.length; i++) {
	        System.out.print(s.getVar(vs[i]).getVal() + " ");
	    }
	    System.out.println("");
	} while (s.nextSolution() == Boolean.TRUE);
\end{lstlisting}

\input{chapters/Otrue.tex}
\chapter{Constraints}\label{ch:constraints}\hypertarget{ch:constraints}{}
This section lists and details the \hyperlink{model:constraints}{constraints} currently available in Choco.
%\part{abs}
%\label{abs}\hypertarget{abs}{}
\section{abs (constraint)}\label{abs:absconstraint}\hypertarget{abs:absconstraint}{}
\begin{notedef}
  \texttt{abs}$(x,y)$ states that $x$ is the absolute value of $y$:
$$x = |y|$$
\end{notedef}

\begin{itemize}
	\item \textbf{API} : \mylst{abs(IntegerVariable x, IntegerVariable y)}
	\item \textbf{return type} : \texttt{Constraint}
	\item \textbf{options} : \emph{n/a}
	\item \textbf{favorite domain} : enumerated
\end{itemize}

\textbf{Example}:
\begin{lstlisting}
	CPModel m = new CPModel();
	IntegerVariable x = makeIntVar("x", 1, 5, "cp:enum");
	IntegerVariable y = makeIntVar("y", -5, 5, "cp:enum");
	m.addConstraint(abs(x,y));
	CPSolver s = new CPSolver();
	s.read(m);
	s.solve();
	Assert.assertEquals(s.getVar(x).getVal(),Math.abs(s.getVar(y).getVal()));
\end{lstlisting}


%\part{alldifferent}
\label{alldifferent}
\hypertarget{alldifferent}{}

\section{allDifferent (constraint)}\label{alldifferent:alldifferentconstraint}\hypertarget{alldifferent:alldifferentconstraint}{}
\begin{notedef}
  \texttt{allDifferent}$(x_1,\ldots,x_n)$ states that the arguments have pairwise distinct values:
 $$x_i \neq x_j,\quad \forall\ i\neq j$$  
\end{notedef}
This constraint is useful for some matching problems.
Notice that the filtering algorithm used will depend on the nature (enumerated or bounded) of the variables: 
when \emph{enumerated}, the constraint refers to the alldifferent of \cite{ReginAAAI94};
when \emph{bounded}, a dedicated algorithm for bound propagation is used \cite{LopezIJCAI03}.

\begin{itemize}
	\item \textbf{API} :
	\begin{itemize}
		\item \mylst{allDifferent(IntegerVariable... x)}
		\item \mylst{allDifferent(String options, IntegerVariable... x)}
	\end{itemize}
	\item \textbf{return type} : \texttt{Constraint}
	\item \textbf{options} :
	\begin{itemize}
		\item \emph{no option} clever choice made on domains of given variables
		\item \texttt{CPOptions.C_ALLDIFFERENT_AC} for \cite{ReginAAAI94} implementation of arc consistency
		\item \texttt{CPOptions.C_ALLDIFFERENT_BC} for \cite{LopezIJCAI03} implementation of bound consistency
		\item \texttt{CPOptions.C_ALLDIFFERENT_CLIQUE} for propagating the clique of differences
	\end{itemize}
	\item \textbf{favorite domain} : depending of options.
	\item \textbf{references} :
      \begin{itemize}
      \item  \cite{ReginAAAI94}: \emph{A filtering algorithm for constraints of difference in CSPs}
      \item  \cite{LopezIJCAI03}: \emph{A fast and simple algorithm for bounds consistency of the alldifferent constraint}
      \item global constraint catalog: \href{http://www.emn.fr/x-info/sdemasse/gccat/Calldifferent.html}{\tt alldifferent}
      \end{itemize}
\end{itemize}



\textbf{Example}:
\lstinputlisting{java/calldifferent.j2t}

%\part{and}
\label{and}
\hypertarget{and}{}

\section{and (constraint)}\label{and:andconstraint}\hypertarget{and:andconstraint}{}
\begin{notedef}
  \texttt{and}$(c_1,\ldots,c_n)$ states that every constraints in arguments are satisfied:
$$ c_1 \land c_2 \land\ldots\land c_n$$
\end{notedef}

\begin{itemize}
\item \textbf{API} : \mylst{and(Constraint... c)}
\item \textbf{return type} : \texttt{Constraint}
\item \textbf{options} : \emph{n/a}
\item \textbf{favorite domain} : \emph{n/a}
\item \textbf{references} :\\
  global constraint catalog: \href{http://www.emn.fr/x-info/sdemasse/gccat/Cand.html}{\tt and}
\end{itemize}

\textbf{Example}:
\lstinputlisting{java/cand.j2t}

%\part{atmostnvalue}
\label{atmostnvalue}
\hypertarget{atmostnvalue}{}

\section{atMostNValue (constraint)}\label{atmostnvalue:atmostnvalueconstraint}\hypertarget{atmostnvalue:atmostnvalueconstraint}{}
\begin{notedef}
\texttt{atMostNValue}$(x,z)$ states that the number of different values occurring in the array of variables $x$ is at most \emph{z}:
$$z\ge|\{x_1,\ldots,x_n\}|$$  
\end{notedef}

\begin{itemize}
	\item \textbf{API} : \mylst{atMostNValue(IntegerVariable[] x, IntegerVariable z)}
	\item \textbf{return type} : \texttt{Constraint}
	\item \textbf{options} : \emph{n/a}
	\item \textbf{favorite domain} : \emph{n/a}
	\item \textbf{references} :
      \begin{itemize}
      \item  \cite{BessiereCPAIOR05} \emph{Filtering algorithms for the NValue constraint}
      \item global constraint catalog: \href{http://www.emn.fr/x-info/sdemasse/gccat/Catmost_nvalue.html}{\tt atmost\_nvalue}
      \end{itemize}
    \end{itemize}

\textbf{Example}:
\begin{lstlisting}
	Model m = new CPModel();
	CPSolver s = new CPSolver();
	
	IntegerVariable v1 = makeIntVar("v1", 1, 1);
	IntegerVariable v2 = makeIntVar("v2", 2, 2);
	IntegerVariable v3 = makeIntVar("v3", 3, 3);
	IntegerVariable v4 = makeIntVar("v4", 3, 4);
	IntegerVariable n = makeIntVar("n", 3, 3);
	
	Constraint c2 = atMostNValue(new IntegerVariable[]{v1, v2, v3, v4}, n);
	
	m.addConstraint(c1, c2);
	        
	s.read(m);
	s.solve();
\end{lstlisting}

%\part{boolchanneling}
\label{boolchanneling}
\hypertarget{boolchanneling}{}

\section{boolChanneling (constraint)}\label{boolchanneling:boolchannelingconstraint}\hypertarget{boolchanneling:boolchannelingconstraint}{}
\begin{notedef}  
\texttt{boolChanneling}$(b,x,v)$ states that $b$ is true if and only if $x$ is equal to $v$:
$$b\quad\iff\quad (x=v)$$ 
\end{notedef}

It acts as an observer of value $v$. Imagine a bin packing problem where variable $x$ tells you on which a given bin object is placed. By stating the boolean channeling, $b$ is true if and only if the object is placed on bin $v$, the knapsack constraint for bin $v$ can then be easily stated as a scalar of the boolean variables.
\begin{itemize}
	\item \textbf{API} : \mylst{boolChanneling(IntegerVariable b, IntegerVariable x, int v)}
	\item \textbf{return type} : \texttt{Constraint}
	\item \textbf{options} : \emph{n/a}
	\item \textbf{favorite domain} : enumerated for $x$
\end{itemize}

\textbf{Example}:
\lstinputlisting{java/cboolchanneling.j2t}

%\input{chapters/Ccos.tex}
%\part{cumulative}
\label{cumulative}
\hypertarget{cumulative}{}

\section{cumulative (constraint)}\label{cumulative:cumulativeconstraint}\hypertarget{cumulative:cumulativeconstraint}{}
\begin{notedef}
  \texttt{cumulative(start,duration,height,capacity)} states that a set of tasks (defined by their starting times, finishing dates, durations and heights (or consumptions)) are executed on a cumulative resource of limited capacity. That is, the total height of the tasks which are executed at any time $t$ does not exceed the capacity of the resource:
$$\sum_{\{i\ |\ \mathtt{start}[i]\le t < \mathtt{start}[i]+\mathtt{duration}[i]\}} \mathtt{height}[i] \le \mathtt{capacity},\quad (\forall \text{ time } t)$$
\end{notedef}

The notion of task does not exist yet in Choco. The \texttt{cumulative} takes therefore as input, several arrays of integer variables (of same size $n$) denoting the starting, duration, and height of each task. When the array of finishing times is also specified, the constraint ensures that \texttt{start[i] + duration[i] = end[i]} for all task $i$.
As usual, a task is executed in the interval \texttt{[start,end-1]}.

A tutorial on the use of this constraint is available \hyperlink{schedulinganduseofthecumulative:schedulinganduseofthecumulativeconstraint}{here}

\begin{itemize}
	\item \textbf{API} :
	\begin{itemize}
		\item \mylst{cumulative(IntegerVariable[] start, IntegerVariable[] end, IntegerVariable[] duration, IntegerVariable[] height, IntegerVariable capa, String... options)}
		\item \mylst{cumulative(IntegerVariable[] start, IntegerVariable[] end, IntegerVariable[] duration, int[] height, int capa, String... options)}
		\item \mylst{cumulative(IntegerVariable[] start, IntegerVariable[] duration, IntegerVariable[] height, IntegerVariable capa, String... options)}
	\end{itemize}
	\item \textbf{return type} : \texttt{Constraint}
	\item \textbf{options} :\emph{n/a}
	\item \textbf{favorite domain} : \emph{to complete}
	\item \textbf{references} :
      \begin{itemize}
      \item  \cite{BeldiceanuCP02} \emph{A new multi-resource cumulatives constraint with negative heights}
      \item global constraint catalog: \href{http://www.emn.fr/x-info/sdemasse/gccat/Ccumulative.html}{\tt cumulative}
      \end{itemize}
\end{itemize}

\begin{lstlisting}
  CPModel m = new CPModel();
	
  // data
  int n = 11 + 3; //number of tasks (include the three fake tasks)
  int[] heights_data = new int[]{2, 1, 4, 2, 3, 1, 5, 6, 2, 1, 3, 1, 1, 2};
  int[] durations_data = new int[]{1, 1, 1, 2, 1, 3, 1, 1, 3, 4, 2, 3, 1, 1};

  // variables
  IntegerVariable capa = constant(7);
  IntegerVariable[] starts = makeIntVarArray("start", n, 0, 5, "cp:bound");
  IntegerVariable[] ends = makeIntVarArray("end", n, 0, 6, "cp:bound");
  IntegerVariable[] duration = new IntegerVariable[n];
  IntegerVariable[] height = new IntegerVariable[n];
  for (int i = 0; i < height.length; i++) {
      duration[i] = constant(durations_data[i]);
      height[i] = makeIntVar("height " + i, new int[]{0, heights_data[i]});
  }
  IntegerVariable[] bool = makeIntVarArray("taskIn?", n, 0, 1);
  IntegerVariable obj = makeIntVar("obj", 0, n, "cp:bound", "cp:objective");
	
  //post the cumulative
  m.addConstraint(cumulative(starts, ends, duration, height, capa, ""));
	
  //post the channeling to know if the task is scheduled or not
  for (int i = 0; i < n; i++) {
      m.addConstraint(boolChanneling(bool[i], height[i], heights_data[i]));
  }

  //state the objective function
  m.addConstraint(eq(sum(bool), obj));
	
  CPSolver s = new CPSolver();
  s.read(m);
	
  //set the fake tasks to establish the profile capacity of the ressource
  try {
      s.getVar(starts[0]).setVal(1); s.getVar(ends[0]).setVal(2); s.getVar(height[0]).setVal(2);
      s.getVar(starts[1]).setVal(2); s.getVar(ends[1]).setVal(3); s.getVar(height[1]).setVal(1); 
      s.getVar(starts[2]).setVal(3); s.getVar(ends[2]).setVal(4); s.getVar(height[2]).setVal(4);
  } catch (ContradictionException e) {
      System.out.println("error, no contradiction expected at this stage");
  }
  // maximize the number of tasks placed in this profile	
  s.maximize(s.getVar(obj),false);
  System.out.println("Objective : " + (s.getVar(obj).getVal() - 3));
  for (int i = 3; i < starts.length; i++) {
      if (s.getVar(height[i]).getVal() != 0)
      System.out.println("[" + s.getVar(starts[i]).getVal() + " - " 
                             + (s.getVar(ends[i]).getVal() - 1) + "]:"
                             + s.getVar(height[i]).getVal());
  }
\end{lstlisting} 

%\part{disjunctive}
\label{disjunctive}
\hypertarget{disjunctive}{}

\section{disjunctive (constraint)}\label{disjunctive:disjunctiveconstraint}\hypertarget{disjunctive:disjunctiveconstraint}{}

\begin{notedef}
  \texttt{disjunctive(start,duration)} states that a set of tasks (defined by their starting times and durations) are executed on a ddisjunctive resource, i.e. they do not overlap in time:
$$|\{i\ |\ \mathtt{start}[i]\le t < \mathtt{start}[i]+\mathtt{duration}[i]\}| \le 1,\quad (\forall \text{ time } t)$$
\end{notedef}

The notion of task does not exist yet in Choco. The \texttt{disjunctive} takes therefore as input arrays of integer variables (of same size $n$) denoting the starting and duration of each task. When the array of finishing times is also specified, the constraint ensures that \texttt{start[i] + duration[i] = end[i]} for all task $i$.
As usual, a task is executed in the interval \texttt{[start,end-1]}.

\begin{itemize}
	\item \textbf{API} :
	\begin{itemize}
		\item \mylst{disjunctive(IntegerVariable[] start, int[] duration, String...options)}
		\item \mylst{disjunctive(IntegerVariable[] start, IntegerVariable[] duration, String... options)}
		\item \mylst{disjunctive(IntegerVariable[] start, IntegerVariable[] end, IntegerVariable[] duration, String... options)}
		\item \mylst{disjunctive(IntegerVariable[] start, IntegerVariable[] end, IntegerVariable[] duration, IntegerVariable uppBound, String... options)}
	\end{itemize}
	\item \textbf{return type} : \texttt{Constraint}
	\item \textbf{options} :\emph{n/a}
	\item \textbf{favorite domain} : \emph{to complete}
	\item \textbf{references} :\\
      global constraint catalog: \href{http://www.emn.fr/x-info/sdemasse/gccat/Cdisjunctive.html}{\tt disjunctive}
\end{itemize}

\textbf{Example}:

\mylst{//TODO: complete} 

\input{chapters/Cdistanceeq.tex}
\input{chapters/Cdistancegt.tex}
%\part{distancelt}
\label{distancelt}
\hypertarget{distancelt}{}

\section{distanceLT (constraint)}\label{distancelt:distanceltconstraint}\hypertarget{distancelt:distanceltconstraint}{}
\begin{notedef}
  \texttt{distanceLT}$(x_1,x_2,x_3,c)$ states that $x_3$ plus an offset $c$ (by default $c=0$) is strictly smaller than the distance between $x_1$ and $x_2$:
$$ x_3 + c < | x_1 - x_2 |$$
\end{notedef}

\begin{itemize}
	\item \textbf{API} :
	\begin{itemize}
		\item \mylst{distanceLT(IntegerVariable x1, IntegerVariable x2, int x3)}
		\item \mylst{distanceLT(IntegerVariable x1, IntegerVariable x2, IntegerVariable x3)}
		\item \mylst{distanceLT(IntegerVariable x1, IntegerVariable x2, IntegerVariable x3, int c)}
	\end{itemize}
	\item \textbf{return type}: \texttt{Constraint}
	\item \textbf{options} : \emph{n/a}
	\item \textbf{favorite domain} : \emph{to complete}
\end{itemize}

\textbf{Example}:
\begin{lstlisting}
	Model m = new CPModel();
	Solver s = new CPSolver();
	IntegerVariable v0 = makeIntVar("v0", 0, 5);
	IntegerVariable v1 = makeIntVar("v1", 0, 5);
	IntegerVariable v2 = makeIntVar("v2", 0, 5);
	m.addConstraint(distanceLT(v0, v1, v2, 0));
	s.read(m);
	s.solveAll();
\end{lstlisting}

%\part{distanceneq}
\label{distanceneq}
\hypertarget{distanceneq}{}

\section{distanceNEQ (constraint)}\label{distanceneq:distanceneqconstraint}\hypertarget{distanceneq:distanceneqconstraint}{}
\begin{notedef}
  \texttt{distanceNEQ}$(x_1,x_2,x_3,c)$ states that $x_3$ plus an offset $c$ (by default $c=0$) is not equal to the distance between $x_1$ and $x_2$:
$$ x_3 + c \neq | x_1 - x_2 |$$
\end{notedef}

\begin{itemize}
	\item \textbf{API} :
	\begin{itemize}
		\item \mylst{distanceNEQ(IntegerVariable x1, IntegerVariable x2, int x3)}
		\item \mylst{distanceNEQ(IntegerVariable x1, IntegerVariable x2, IntegerVariable x3)}
		\item \mylst{distanceNEQ(IntegerVariable x1, IntegerVariable x2, IntegerVariable x3, int c)}
	\end{itemize}
	\item \textbf{return type}: \texttt{Constraint}
	\item \textbf{options} : \emph{n/a}
	\item \textbf{favorite domain} : \emph{to complete}
	\item \textbf{references} :\\
      global constraint catalog: \href{http://www.emn.fr/x-info/sdemasse/gccat/Call_min_dist.html}{\tt all\_min\_dist} (variant)
\end{itemize}

\textbf{Example}:
\begin{lstlisting}
	Model m = new CPModel();
	Solver s = new CPSolver();
	IntegerVariable v0 = makeIntVar("v0", 0, 5);
	IntegerVariable v1 = makeIntVar("v1", 0, 5);
	IntegerVariable v2 = makeIntVar("v2", 0, 5);
	m.addConstraint(distanceNEQ(v0, v1, v2, 0));
	s.read(m);
	s.solveAll();
\end{lstlisting}

%\input{chapters/Cdiv.tex}
%\part{domainconstraint}
\label{domainconstraint}
\hypertarget{domainconstraint}{}

\section{domainConstraint (constraint)}\label{domainconstraint:domainconstraintconstraint}\hypertarget{domainconstraint:domainconstraintconstraint}{}
\begin{notedef}  
\texttt{domainConstraint}$(bVar,values)$ states that $values[i]$ is equal to 1 if and only if $bVar$ is equal to $i$ (0 otherwise):
$$values[i]=1\quad\iff\quad (bVar=i)$$ 
\end{notedef}

It makes the link between a domain variable $bVar$ and those 0-1 variables that are associated with each potential value of $bVar$: the 0-1 variable associated with the value that is taken by variable $bVar$ is equal to 1, while the remaining 0-1 variables are all equal to 0.

\begin{itemize}
	\item \textbf{API} : \mylst{domainConstraint(IntegerVariable bVar, IntegerVariable[] values)}
	\item \textbf{return type} : \texttt{Constraint}
	\item \textbf{options} : \emph{n/a}
	\item \textbf{favorite domain} : enumerated for $bVar$
	\item \textbf{references} :\\
	  global constraint catalog: \href{http://www.emn.fr/x-info/sdemasse/gccat/Cdomain_constraint.html}{\tt domainConstraint}
\end{itemize}

\textbf{Example}:
\lstinputlisting{java/cdomainconstraint.j2t}

\input{chapters/Ceq.tex}
%\part{eqcard}
\label{eqcard}
\hypertarget{eqcard}{}

\section{eqCard (constraint)}\label{eqcard:eqcardconstraint}\hypertarget{eqcard:eqcardconstraint}{}
\begin{notedef}
  \texttt{eqCard}$(s,x)$ states that the cardinality of set $s$ is equal to $x$:
$$|s| = x$$
\end{notedef}

\begin{itemize}
	\item \textbf{API} :
	\begin{itemize}
		\item \mylst{eqCard(SetVariable s, IntegerVariable x)}
		\item \mylst{eqCard(SetVariable s, int x)}
	\end{itemize}
	\item \textbf{return type} : \texttt{Constraint}
	\item \textbf{options} : \emph{n/a}
	\item \textbf{favorite domain} : \emph{to complete}
\end{itemize}

\textbf{Example}:
\begin{lstlisting}
	Model m = new CPModel();
	Solver s = new CPSolver();
	SetVariable s = makeSetVar("s", 1, 5);
	IntegerVariable i = makeIntVar("card", 2, 3);
	m.addConstraint(member(x, 3));
	m.addConstraint(eqCard(x, i));
	s.read(m);
	s.solve();
\end{lstlisting}

%\part{equation}
\label{equation}
\hypertarget{equation}{}

\section{equation (constraint)}\label{equation:equationconstraint}\hypertarget{equation:equationconstraint}{}
\begin{notedef}
  \texttt{equation}$(x,c,z)$ states a linear equation:
$$c_1x_1+c_2x_2+...+c_nx_n = z$$
\end{notedef}
It enforces GAC using \hyperlink{regular:regularconstraint}{regular} to state a \emph{knapsack} constraint.

\begin{itemize}
	\item \textbf{API} :
	\begin{itemize}
		\item equation(IntegerVariable[] x, int[] c, int z)
	\end{itemize}
	\item \textbf{return type} : \texttt{Constraint}
	\item \textbf{options} : \emph{n/a}
	\item \textbf{favorite domain} : \emph{to complete}
\end{itemize}

\textbf{Example}:
\lstinputlisting{java/cequation.j2t}

\input{chapters/Cfalse.tex}
\input{chapters/Cfeaspairac.tex}
%\part{feastupleac}
\label{feastupleac}
\hypertarget{feastupleac}{}

\section{feasTupleAC (constraint)}\label{feastupleac:feastupleacconstraint}\hypertarget{feastupleac:feastupleacconstraint}{}
\begin{notedef}
  \texttt{feasTupleAC}$(x,feasTuples)$ states an extensional constraint on $(x_1,\ldots,x_n)$ defined by the table $feasTuples$ of compatible tuples of values, and then enforces arc consistency:
      $$\exists \text{ tuple } i\ |\quad (x_1,\ldots,x_n)=feasTuples[i]$$
\end{notedef}

The API is duplicated to define options.
\begin{itemize}
	\item \textbf{API} :
	\begin{itemize}
		\item \mylst{feasTupleAC(List<int[]> feasTuples, IntegerVariable... x)}
		\item \mylst{feasTupleAC(String options, List<int[]> feasTuples, IntegerVariable... x)}
	\end{itemize}
	\item \textbf{return type}: \texttt{Constraint}
	\item \textbf{options} :
	\begin{itemize}
		\item \emph{no option}: use AC32 (default arc consistency)
		\item \texttt{cp:ac32}: to get AC3rm algorithm (maintaining the current support of each value in a non backtrackable way)
		\item \texttt{cp:ac2001}: to get AC2001 algorithm (maintaining the current support of each value)
		\item \texttt{cp:ac2008}: to get AC2008 algorithm (maintained by STR)
	\end{itemize}
	\item \textbf{favorite domain} : \emph{to complete}
	\item \textbf{references} :\\
      global constraint catalog: \href{http://www.emn.fr/x-info/sdemasse/gccat/Cin_relation.html}{in\_relation}
\end{itemize}

\textbf{Example}:
\lstinputlisting{java/cfeastupleac.j2t}

%\part{feastuplefc}
\label{feastuplefc}
\hypertarget{feastuplefc}{}

\section{feasTupleFC (constraint)}\label{feastuplefc:feastuplefcconstraint}\hypertarget{feastuplefc:feastuplefcconstraint}{}
\begin{notedef}
  \texttt{feasTupleFC}$(x,feasTuples)$ states an extensional constraint on $(x_1,\ldots,x_n)$ defined by the table $feasTuples$ of compatible tuples of values, and then performs Forward Checking:
      $$\exists \text{ tuple } i\ |\quad (x_1,\ldots,x_n)=feasTuples[i]$$
\end{notedef}


\begin{itemize}
	\item \textbf{API} : \mylst{feasTupleFC(List<int[]> tuples, IntegerVariable... x)}
	\item \textbf{return type}: \texttt{Constraint}
	\item \textbf{options} : \emph{n/a}
	\item \textbf{favorite domain}: \emph{to complete}
	\item \textbf{references} :\\
      global constraint catalog: \href{http://www.emn.fr/x-info/sdemasse/gccat/Cin_relation.html}{in\_relation}
\end{itemize}

\textbf{Example}:
\begin{lstlisting}
	Model m = new CPModel();
	Solver s = new CPSolver();
	IntegerVariable v1 = makeIntVar("v1", 0, 2);
	IntegerVariable v2 = makeIntVar("v2", 0, 4);
	
	ArrayList feasTuple = new ArrayList();
	feasTuple.add(new int[]{1, 1}); // x*y = 1
	feasTuple.add(new int[]{2, 4}); // x*y = 1
	
	m.addConstraint(feasTupleFC(feasTuple, new IntegerVariable[]{v1, v2}));
	
	s.read(m);
	s.solve();
\end{lstlisting}

\input{chapters/Cgeost.tex}
%\part{geq}
\label{geq}
\hypertarget{geq}{}

\section{geq (constraint)}\label{geq:geqconstraint}\hypertarget{geq:geqconstraint}{}
\begin{notedef}
  \texttt{geq}$(x,y)$ states that $x$ is greater than or equal to $y$:
$$x\ge y$$
\end{notedef}

\begin{itemize}
	\item \textbf{API} :
	\begin{itemize}
		\item \mylst{geq(IntegerExpressionVariable x, IntegerExpressionVariable y)}
		\item \mylst{geq(IntegerExpressionVariable x, int y)}
		\item \mylst{geq(int x, IntegerExpressionVariable y)}
		\item \mylst{geq(RealExpressionVariable x, RealExpressionVariable y)}
		\item \mylst{geq(RealExpressionVariable x, double y)}
		\item \mylst{geq(double x, RealExpressionVariable y)}
	\end{itemize}
	\item \textbf{return type} : \texttt{Constraint}
	\item \textbf{options} : \emph{n/a}
	\item \textbf{favorite domain} : \emph{to complete}.
	\item \textbf{references} :\\
      global constraint catalog: \href{http://www.emn.fr/x-info/sdemasse/gccat/Cgeq.html}{geq}
\end{itemize}

\textbf{Examples:}
\begin{itemize}
	\item example1:
\end{itemize}

\begin{lstlisting}
	Model m = new CPModel();
	Solver s = new CPSolver();
	int c = 1;
	IntegerVariable v = makeIntVar("v", 0, 2);
	m.addConstraint(geq(v, c));
	s.read(m);
	s.solve();
\end{lstlisting}
\begin{itemize}
	\item example2
\end{itemize}

\begin{lstlisting}
	Model m = new CPModel();
	Solver s = new CPSolver();
	IntegerVariable v1 = makeIntVar("v1", 0, 2);
	IntegerVariable v2 = makeIntVar("v2", 0, 2);
	IntegerExpressionVariable w1 = plus(v1, 1);
	IntegerExpressionVariable w2 = minus(v2, 1);
	m.addConstraint(geq(w1, w2));
	s.read(m);
	s.solve();
\end{lstlisting}

%\part{geqcard}
\label{geqcard}
\hypertarget{geqcard}{}

\section{geqCard (constraint)}\label{geqcard:geqcardconstraint}\hypertarget{geqcard:geqcardconstraint}{}
\begin{notedef}
  \texttt{geqCard}$(s,x)$ states that the cardinality of set $s$ is greater than or equal to $x$:
$$|s| \ge x$$
\end{notedef}

\begin{itemize}
	\item \textbf{API} :
	\begin{itemize}
		\item \mylst{geqCard(SetVariable s, IntegerVariable x)}
		\item \mylst{geqCard(SetVariable s, int x)}
	\end{itemize}
	\item \textbf{return type} : \texttt{Constraint}
	\item \textbf{options} : \emph{n/a}
	\item \textbf{favorite domain} : \emph{to complete}
\end{itemize}

\textbf{Example}:
\begin{lstlisting}
	Model m = new CPModel();
	Solver s = new CPSolver();
	SetVariable s = makeSetVar("s", 1, 5);
	IntegerVariable i = makeIntVar("card", 2, 3);
	m.addConstraint(member(x, 3));
	m.addConstraint(geqCard(x, i));
	s.read(m);
	s.solve();
\end{lstlisting}

%\part{globalcardinality}
\label{globalcardinality}
\hypertarget{globalcardinality}{}

\section{globalCardinality (constraint)}\label{globalcardinality:globalcardinalityconstraint}\hypertarget{globalcardinality:globalcardinalityconstraint}{}
\begin{notedef}
  \texttt{globalCardinality}$(\collec{x_1}{x_n},\collec{l_0}{l_{m-1}}, \collec{u_0}{u_{m-1}}, o)$ states lower bounds $l$ and upper bounds $u$ on the occurrence numbers of the values in collection $x$ according to offset $o$: 
$$l_{j-o}\ \le\ |\{i=1..n\ |\ x_i=j\}|\ \le\ u_{j-o},\quad\forall j=0..m-1$$   
  \texttt{globalCardinality}$(\collec{x_1}{x_n},\collec{z_0}{z_{m-1}}, o)$ states that $z$ are the occurrence numbers of the values in collection $x$ according to offset $o$: 
$$z_{j-o} = \{i=1..n\ |\ x_i=j\}|,\quad\forall j=0..m-1$$   
\end{notedef}
Note that offset $o$ should be set to the minimum possible value over all variables $x$ and that the bound table length $m$ should be set to the maximum possible value plus $o+1$.

%offset is the minimum value over all variables in $x$
Several APIs exist:
\begin{itemize}
	\item \emph{constant bounds on cardinalities} \collec{l_0}{l_{m-1}} and \collec{u_0}{u_{m-1}}: use the propagator of \cite{ReginAAAI96} or of \cite{QuimperCP03} depending on the set options and the nature of the domain variables.
	\item \emph{variable cardinalities} \collec{z_0}{z_{m-1}}: use the propagator of \cite{QuimperCP03} that:      
      \begin{itemize}
      \item enforces Bound Consistency over $x$ regarding the lower and upper bounds of $z$, 
      \item maintains the upper bound of $z_j$ by counting the variables that may be instantiated to $j$, 
      \item maintains the lower bound of $z_j$ by counting the variables instantiated to $j$, 
      \item enforces $z_0 + \cdots + z_{m-1} = n$
      \end{itemize}
\end{itemize}

The APIs are duplicated to define options. 

\begin{itemize}
	\item \textbf{API} :
      \begin{itemize}
	\item \mylst{globalCardinality(IntegerVariable[] x, int[] low, int[] up, int offset)}
	\item \mylst{globalCardinality(String options, IntegerVariable[] x, int[] low, int[] up, int offset)}
	\item \mylst{globalCardinality(IntegerVariable[] x, IntegerVariable[] card, int offset)}
      \end{itemize}
	\item \textbf{return type} : \texttt{Constraint}
	\item \textbf{options}:
	\begin{itemize}
		\item \emph{no option}: 
          if $x$ have \emph{bounded} domains or if the cardinalities are variable $z$, use the propagator of~\cite{QuimperCP03} for BC, otherwise use the propagator of \cite{ReginAAAI96};
		\item \hyperlink{cgccac:cgccacoptions}{\tt Options.C\_GCC\_AC} : for \cite{ReginAAAI96} implementation of arc consistency
		\item \hyperlink{cgccbc:cgccbcoptions}{\tt Options.C\_GCC\_BC} : for  \cite{QuimperCP03} implementation of bound consistency
	\end{itemize}
	\item \textbf{favorite domain} : \emph{enumerated} for arc consistency, \emph{bounded} for bound consistency.
	\item \textbf{references} :
      \begin{itemize}
      \item \cite{ReginAAAI96}: \emph{Generalized arc consistency for global cardinality constraint},
      \item \cite{QuimperCP03}: \emph{An efficient bounds consistency algorithm for the global cardinality constraint}
      \item global constraint catalog: \href{http://www.emn.fr/x-info/sdemasse/gccat/Cglobal_cardinality.html}{global\_cardinality}
      \end{itemize}
\end{itemize}

\textbf{Examples:}
\begin{itemize}
	\item example1:
\end{itemize}

\lstinputlisting{java/cglobalcardinality1.j2t}

\begin{itemize}
	\item example2:
\end{itemize}

\lstinputlisting{java/cglobalcardinality2.j2t}

\input{chapters/Cgt.tex}
%\part{ifonlyif}
\label{ifonlyif}
\hypertarget{ifonlyif}{}

\section{ifOnlyIf (constraint)}\label{ifonlyif:ifonlyifconstraint}\hypertarget{ifonlyif:ifonlyifconstraint}{}
\begin{notedef}
  \texttt{ifOnlyIf}$(c_1,c_2)$ states that $c_1$ holds if and only if $c_2$ holds:
$$c_1\iff c_2$$
\end{notedef}

\begin{itemize}
	\item \textbf{API} : \mylst{ifOnlyIf(Constraint c1, Constraint c2)}
	\item \textbf{return type} : \texttt{Constraint}
	\item \textbf{options} : \emph{n/a}
	\item \textbf{favorite domain} : \emph{n/a}
\end{itemize}

\textbf{Example}:
\begin{lstlisting}
	Model m = new CPModel();
	Solver s = new CPSolver();
	
	IntegerVariable x = makeIntVar("x", 1, 3);
	IntegerVariable y = makeIntVar("y", 1, 3);
	IntegerVariable z = makeIntVar("z", 1, 3);
	m.addVariable("cp:bound",x ,y, z);
	
	m.addConstraint(ifOnlyIf(lt(x, y), lt(y, z)));
	
	s.read(m);
	s.solveAll();
\end{lstlisting}

\input{chapters/Cifthenelse.tex}
%\part{implies}
\label{implies}
\hypertarget{implies}{}

\section{implies (constraint)}\label{implies:impliesconstraint}\hypertarget{implies:impliesconstraint}{}
\begin{notedef}
  \texttt{implies}$(c_1,c_2)$ states that if $c_1$ holds then $c_2$ holds:
$$c_1\implies c_2$$
\end{notedef}

\begin{itemize}
	\item \textbf{API} : \mylst{implies(Constraint c1, Constraint c2)}
	\item \textbf{return type} : \texttt{Constraint}
	\item \textbf{options} : \emph{n/a}
	\item \textbf{favorite domain} : \emph{n/a}
\end{itemize}

\textbf{Example}:
\begin{lstlisting}
	Model m = new CPModel();
	Solver s = new CPSolver();
	
	IntegerVariable x = makeIntVar("x", 1, 2);
	IntegerVariable y = makeIntVar("y", 1, 2);
	IntegerVariable z = makeIntVar("z", 1, 2);
	
	m.addVariable("cp:bound",x ,y, z);
	
	Constraint e1 = implies(leq(x, y), leq(x, z));
	m.addConstraint(e1);
	
	s.read(m);
	s.solveAll();
\end{lstlisting}

%\part{infeaspairac}
\label{infeaspairac}
\hypertarget{infeaspairac}{}

\section{infeasPairAC (constraint)}\label{infeaspairac:infeaspairacconstraint}\hypertarget{infeaspairac:infeaspairacconstraint}{}
\begin{notedef}
  \texttt{infeasPairAC}$(x,y,infeasTuples)$ states an extensional binary constraint on $(x,y)$ defined by the table $infeasTuples$ of forbidden pairs of values, and then enforces arc consistency. Two APIs are available to define the forbidden pairs:
\begin{itemize}
	\item if $infeasTuples$ is encoded as a list of pairs \texttt{List<int[2]>}, then:
      $$\forall \text{ tuple } i\ |\quad (x,y)\neq infeasTuples[i]$$
	\item if $infeasTuples$ is encoded as a boolean matrix \texttt{boolean[][]}, let $\underline{x}$ and  $\underline{y}$ be the initial minimum values of $x$ and $y$, then:
      $$\forall (u,v)\ |\quad (x,y)=(u+\underline{x},v+\underline{y})\ \lor\ \neg infeasTuples[u][v]$$
\end{itemize}
\end{notedef}

The two APIs are duplicated to allow definition of options.
\begin{itemize}
	\item \textbf{API} :
	\begin{itemize}
		\item \mylst{infeasPairAC(IntegerVariable x, IntegerVariable y, List<int[]> infeasTuples)}
		\item \mylst{infeasPairAC(String options, IntegerVariable x, IntegerVariable y, List<int[]> infeasTuples)}
		\item \mylst{infeasPairAC(IntegerVariable x, IntegerVariable y, boolean[][] infeasTuples)}
		\item \mylst{infeasPairAC(String options, IntegerVariable x, IntegerVariable y, boolean[][] infeasTuples)}
	\end{itemize}
	\item \textbf{return type} : \texttt{Constraint}
	\item \textbf{options} :
	\begin{itemize}
		\item \emph{no option}: use AC3 (default arc consistency)
		\item \texttt{cp:ac3}: to get AC3 algorithm (searching from scratch for supports on all values)
		\item \texttt{cp:ac2001}: to get AC2001 algorithm (maintaining the current support of each value)
		\item \texttt{cp:ac32}: to get AC3rm algorithm (maintaining the current support of each value in a non backtrackable way)
		\item \texttt{cp:ac322}: to get AC3 with the used of \texttt{BitSet} to know if a support still exists
	\end{itemize}
	\item \textbf{favorite domain} : \emph{to complete}
\end{itemize}

\textbf{Example}:
\lstinputlisting{java/cinfeaspairac.j2t}

%\part{infeastupleac}
\label{infeastupleac}
\hypertarget{infeastupleac}{}

\section{infeasTupleAC (constraint)}\label{infeastupleac:infeastupleacconstraint}\hypertarget{infeastupleac:infeastupleacconstraint}{}
\begin{notedef}
  \texttt{infeasTupleAC}$(x,feasTuples)$ states an extensional constraint on $(x_1,\ldots,x_n)$ defined by the table $infeasTuples$ of compatible tuples of values, and then enforces arc consistency:
      $$\forall \text{ tuple } i\ |\quad (x_1,\ldots,x_n)\neq infeasTuples[i]$$
\end{notedef}

The API is duplicated to define options.
\begin{itemize}
	\item \textbf{API} :
	\begin{itemize}
		\item \mylst{infeasTupleAC(List<int[]> infeasTuples, IntegerVariable... x)}
		\item \mylst{infeasTupleAC(String options, List<int[]> infeasTuples, IntegerVariable... x)}
	\end{itemize}
	\item \textbf{return type}: \texttt{Constraint}
	\item \textbf{options} :
	\begin{itemize}
		\item \emph{no option}: use AC32 (default arc consistency)
		\item \texttt{cp:ac32}: to get AC3rm algorithm (maintaining the current support of each value in a non backtrackable way)
		\item \texttt{cp:ac2001}: to get AC2001 algorithm (maintaining the current support of each value)
		\item \texttt{cp:ac2008}: to get AC2008 algorithm (maintained by STR)
	\end{itemize}
	\item \textbf{favorite domain} : \emph{to complete}
\end{itemize}

\textbf{Example}:
\begin{lstlisting}
	Model m = new CPModel();
	Solver s = new CPSolver();
	IntegerVariable x = makeIntVar("x", 1, 5);
	IntegerVariable y = makeIntVar("y", 1, 5);
	IntegerVariable z = makeIntVar("z", 1, 5);
	
	ArrayList forbiddenTuples = new ArrayList();
	forbiddenTuples.add(new int[]{1, 1, 1});
	forbiddenTuples.add(new int[]{2, 2, 2});
	forbiddenTuples.add(new int[]{2, 5, 3});
	
	m.addConstraint(infeasTupleAC(forbiddenTuples, new IntegerVariable[]{x, y, z}));
	
	s.read(m);
	s.solveAll();
\end{lstlisting}

%\part{infeastuplefc}
\label{infeastuplefc}
\hypertarget{infeastuplefc}{}

\section{infeasTupleFC (constraint)}\label{infeastuplefc:infeastuplefcconstraint}\hypertarget{infeastuplefc:infeastuplefcconstraint}{}
\begin{notedef}
  \texttt{infeasTupleFC}$(x,feasTuples)$ states an extensional constraint on $(x_1,\ldots,x_n)$ defined by the table $infeasTuples$ of compatible tuples of values, and then performs Forward Checking:
      $$\forall \text{ tuple } i\ |\quad (x_1,\ldots,x_n)\neq infeasTuples[i]$$
\end{notedef}

\begin{itemize}
	\item \textbf{API} : \mylst{infeasTupleFC(List<int[]> infeasTuples, IntegerVariable... x)}
	\item \textbf{return type}: \texttt{Constraint}
	\item \textbf{options} : \emph{n/a}
	\item \textbf{favorite domain}: \emph{to complete}
\end{itemize}

\textbf{Example}:
\begin{lstlisting}
	Model m = new CPModel();
	Solver s = new CPSolver();
	IntegerVariable x = makeIntVar("x", 1, 5);
	IntegerVariable y = makeIntVar("y", 1, 5);
	IntegerVariable z = makeIntVar("z", 1, 5);
	
	ArrayList forbiddenTuples = new ArrayList();
	forbiddenTuples.add(new int[]{1, 1, 1});
	forbiddenTuples.add(new int[]{2, 2, 2});
	forbiddenTuples.add(new int[]{2, 5, 3});
	
	m.addConstraint(infeasTupleFC(forbiddenTuples, new IntegerVariable[]{x, y, z}));
	
	s.read(m);
	s.solveAll();
\end{lstlisting}

%\part{intdiv}
\label{intdiv}
\hypertarget{intdiv}{}

\section{intDiv (constraint)}\label{intdiv:intdivconstraint}\hypertarget{intdiv:intdivconstraint}{}
\begin{notedef}
  \texttt{intDiv}$(x,y,z)$ states that the $z$ is equal to the integer quotient of $x$ by $y$:
$$z = \lfloor x / y \rfloor$$
\end{notedef}

\begin{itemize}
	\item \textbf{API}: \mylst{intDiv(IntegerVariable x, IntegerVariable y, IntegerVariable z)}
	\item \textbf{return type} : \texttt{Constraint}
	\item \textbf{option} : \emph{n/a}
	\item \textbf{favorite domain}: bound
\end{itemize}

\textbf{Example}:
\begin{lstlisting}
	Model m = new CPModel();
	IntegerVariable x = makeIntVar("x", 3, 5);
	IntegerVariable y = makeIntVar("y", 1, 2);
	IntegerVariable z = makeIntVar("z", 0, 5);
	m.addConstraint(intDiv(x, y, z));
	s.setVarIntSelector(new RandomIntVarSelector(s, i));
	s.setValIntSelector(new RandomIntValSelector(i + 1));
	s.read(m);
	s.solve();
\end{lstlisting}

%\part{inversechanneling}
\label{inversechanneling}
\hypertarget{inversechanneling}{}

\section{inverseChanneling (constraint)}\label{inversechanneling:inversechannelingconstraint}\hypertarget{inversechanneling:inversechannelingconstraint}{}
\begin{notedef}
  \texttt{inverseChanneling}$(x,y)$ states a channeling between two arrays  $x$ and $y$ of integer variables with the same domain.It enforces that if the $i$-th element of $x$ is equal to $j$ then the $j$-th element of $y$ is equal to $i$ and conversely:
$$x_i = j\quad\iff\quad y_j = i$$
\end{notedef}
\begin{itemize}
	\item \textbf{API} : \mylst{inverseChanneling(IntegerVariable[] x, IntegerVariable[] y)}
	\item \textbf{return type} : \texttt{Constraint}
	\item \textbf{options} : \emph{no options}
	\item \textbf{favorite domain} : enumerated for x
	\item \textbf{references} :\\
      global constraint catalog: \href{http://www.emn.fr/x-info/sdemasse/gccat/Cinverse.html}{inverse}
\end{itemize}

\textbf{Example}:
\begin{lstlisting}
	int n = 8;
	Model m = new CPModel();
	IntegerVariable[] queens = new IntegerVariable[n];
	IntegerVariable[] queensdual = new IntegerVariable[n];
	for (int i = 0; i < n; i++) {
	    queens[i] = makeIntVar("Q" + i, 1, n);
	    queensdual[i] = makeIntVar("QD" + i, 1, n);
	}
	
	for (int i = 0; i < n; i++) {
	    for (int j = i + 1; j < n; j++) {
	       int k = j - i;
	       m.addConstraint(neq(queens[i], queens[j]));
	       m.addConstraint(neq(queens[i], plus(queens[j], k)));  // diagonal constraints
	       m.addConstraint(neq(queens[i], minus(queens[j], k))); // diagonal constraints
	    }
	}
	for (int i = 0; i < n; i++) {
	    for (int j = i + 1; j < n; j++) {
	        int k = j - i;
	        m.addConstraint(neq(queensdual[i], queensdual[j]));
	        m.addConstraint(neq(queensdual[i], plus(queensdual[j], k)));  // diagonal constraints
	        m.addConstraint(neq(queensdual[i], minus(queensdual[j], k))); // diagonal constraints
	    }
	}
	m.addConstraint(inverseChanneling(queens, queensdual));
	m.addVariable("cp:decision", queens);
	Solver s = new CPSolver();
	s.read(m);
	s.solveAll();
\end{lstlisting}

%\part{isincluded}
\label{isincluded}
\hypertarget{isincluded}{}

\section{isIncluded (constraint)}\label{isincluded:isincludedconstraint}\hypertarget{isincluded:isincludedconstraint}{}
\begin{notedef}
  \texttt{isIncluded}$(x,y)$ states that the second set $y$ contains the first set $x$:
 $$x\subseteq y$$
\end{notedef}

\begin{itemize}
	\item \textbf{API} : \mylst{isIncluded(SetVariable x, SetVariable y)}
	\item \textbf{return type} : \texttt{Constraint}
	\item \textbf{options} :\emph{n/a}
	\item \textbf{favorite domain} : \emph{to complete}
\end{itemize}

\textbf{Example}:
\lstinputlisting{java/cisincluded.j2t}

%\part{isnotincluded}
\label{isnotincluded}
\hypertarget{isnotincluded}{}

\section{isNotIncluded (constraint)}\label{isnotincluded:isnotincludedconstraint}\hypertarget{isnotincluded:isnotincludedconstraint}{}
\begin{notedef}
  \texttt{isNotIncluded}$(x,y)$ states that the second set $y$ does not contain the first set $x$:
 $$x\not\subseteq y$$
\end{notedef}

\begin{itemize}
	\item \textbf{API} : \mylst{isNotIncluded(SetVariable x, SetVariable y)}
	\item \textbf{return type} : \texttt{Constraint}
	\item \textbf{options} :\emph{n/a}
	\item \textbf{favorite domain} : \emph{to complete}
\end{itemize}

\textbf{Example}:
\begin{lstlisting}
	Model m = new CPModel();
	Solver s = new CPSolver();
	SetVariable v1 = makeSetVar("v1", 3, 4);
	SetVariable v2 = makeSetVar("v2", 3, 8);
	m.addConstraint(isNotIncluded(v1, v2));
	s.read(m);
	s.solveAll();
\end{lstlisting} 

%\part{leq}
\label{leq}
\hypertarget{leq}{}

\section{leq (constraint)}\label{leq:leqconstraint}\hypertarget{leq:leqconstraint}{}
\begin{notedef}
  \texttt{leq}$(x,y)$ states that $x$ is less than or equal to $y$:
$$x \le y$$
\end{notedef}

\begin{itemize}
	\item \textbf{API} :
	\begin{itemize}
		\item \mylst{leq(IntegerExpressionVariable x, IntegerExpressionVariable y)}
		\item \mylst{leq(IntegerExpressionVariable x, int y)}
		\item \mylst{leq(int x, IntegerExpressionVariable y)}
		\item \mylst{leq(RealExpressionVariable x, RealExpressionVariable y)}
		\item \mylst{leq(RealExpressionVariable x, double y)}
		\item \mylst{leq(double x, RealExpressionVariable y)}
	\end{itemize}
	\item \textbf{return type} : \texttt{Constraint}
	\item \textbf{options} : \emph{n/a}
	\item \textbf{favorite domain} : \emph{to complete}.
	\item \textbf{references} :\\
      global constraint catalog: \href{http://www.emn.fr/x-info/sdemasse/gccat/Cleq.html}{leq}
\end{itemize}

\textbf{Example:}
\begin{lstlisting}
	Model m = new CPModel();
	Solver s = new CPSolver();
	int c = 1;
	IntegerVariable v = makeIntVar("v", 0, 2);
	m.addConstraint(leq(v, c));

	IntegerVariable v1 = makeIntVar("v1", 0, 2);
	IntegerVariable v2 = makeIntVar("v2", 0, 2);
	IntegerExpressionVariable w1 = plus(v1, 1);
	IntegerExpressionVariable w2 = minus(v2, 1);
	m.addConstraint(leq(w1, w2));

	s.read(m);
	s.solve();
\end{lstlisting}

%\part{leqcard}
\label{leqcard}
\hypertarget{leqcard}{}

\section{leqCard (constraint)}\label{leqcard:leqcardconstraint}\hypertarget{leqcard:leqcardconstraint}{}
\begin{notedef}
  \texttt{leqCard}$(s,x)$ states that the cardinality of set $s$ is less than or equal to $x$:
$$|s| \le x$$
\end{notedef}

\begin{itemize}
	\item \textbf{API} :
	\begin{itemize}
		\item \mylst{leqCard(SetVariable s, IntegerVariable x)}
		\item \mylst{leqCard(SetVariable s, int x)}
	\end{itemize}
	\item \textbf{return type} : \texttt{Constraint}
	\item \textbf{options} : \emph{n/a}
	\item \textbf{favorite domain} : \emph{to complete}
\end{itemize}

\textbf{Example}:
\begin{lstlisting}
	Model m = new CPModel();
	Solver s = new CPSolver();
	SetVariable s = makeSetVar("s", 1, 5);
	IntegerVariable i = makeIntVar("card", 2, 3);
	m.addConstraint(member(x, 3));
	m.addConstraint(leqCard(x, i));
	s.read(m);
	s.solve();
\end{lstlisting}

%\part{lex}
\label{lex}
\hypertarget{lex}{}

\section{lex (constraint)}\label{lex:lexconstraint}\hypertarget{lex:lexconstraint}{}
\begin{notedef}
  \texttt{lex}$(x,y)$ enforces a strict lexicographic ordering  $x <_{lex} y$ between two arrays of same size $n$:
$$\exists\ j\in\{1,\ldots,n\}\ |\qquad x_j<y_j\quad \land\quad x_i=y_i\ (\forall\  i<j)$$
\end{notedef}

\begin{itemize}
	\item \textbf{API} : \mylst{lex(IntegerVariable[] x, IntegerVariable[] y)}
	\item \textbf{return type} : \texttt{Constraint}
	\item \textbf{options} :\emph{n/a}
	\item \textbf{favorite domain} : \emph{to complete}
	\item \textbf{references} :
      \begin{itemize}
      \item \cite{FrischCP02}: \emph{Global Constraints for Lexicographic Orderings}
      \item global constraint catalog: \href{http://www.emn.fr/x-info/sdemasse/gccat/Clex_less.html}{lex\_less}
      \end{itemize}
\end{itemize}

\textbf{Example}:
\begin{lstlisting}
	Model m = new CPModel();
	Solver s = new CPSolver();
	
	int n1 = 8;
	int k = 2;
	IntegerVariable[] vs1 = new IntegerVariable[n1 / 2];
	IntegerVariable[] vs2 = new IntegerVariable[n1 / 2];
	
	for (int i = 0; i < n1 / 2; i++) {
	   vs1[i] = makeIntVar("" + i, 0, k);
	   vs2[i] = makeIntVar("" + i, 0, k);
	}
	m.addConstraint(lex(vs1, vs2));
	
	s.read(m);
	s.solve();
\end{lstlisting} 

%\part{lexchain}
\label{lexchain}
\hypertarget{lexchain}{}

\section{lexChain (constraint)}\label{lexchain:lexchainconstraint}\hypertarget{lexchain:lexchainconstraint}{}
\begin{notedef}
\texttt{lexChain}$(x^1 ,x^2 ,x^3,\ldots)$ enforces a strict lexicographic ordering on a chain of integer vectors:
$$x^1 <_{lex} x^2 <_{lex} x^3 <_{lex}\cdots$$
%where $X^1$ contains up to $n$ variables. 
\end{notedef}

\begin{itemize}
	\item \textbf{API} : \mylst{lexChain(IntegerVariable[]... arrayOfVectors)}
	\item \textbf{return type} : \texttt{Constraint}
	\item \textbf{options} : \emph{n/a}
	\item \textbf{favorite domain} : \emph{to complete}
	\item \textbf{references} :
      \begin{itemize}
      \item \cite{BeldiceanuSICS02} \emph{Arc-Consistency for a chain of Lexicographic Ordering Constraints} 
      \item global constraint catalog: \href{http://www.emn.fr/x-info/sdemasse/gccat/Clex_chain_less.html}{lex\_chain\_less}
      \end{itemize}
\end{itemize}

%\part{lexchaineq}
\label{lexchaineq}
\hypertarget{lexchaineq}{}

\section{lexChainEq (constraint)}\label{lexchaineq:lexchaineqconstraint}\hypertarget{lexchaineq:lexchaineqconstraint}{}
\begin{notedef}
\texttt{lexChainEq}$(x^1 ,x^2 ,x^3,\ldots)$ enforces a lexicographic ordering on a chain of integer vectors:
$$x^1 \le_{lex} x^2 \le_{lex} x^3 \le_{lex}\cdots$$
%where $X^1$ contains up to $n$ variables. 
\end{notedef}

\begin{itemize}
	\item \textbf{API} : \mylst{lexChainEq(IntegerVariable[]... arrayOfVectors)}
	\item \textbf{return type} : \texttt{Constraint}
	\item \textbf{options} : \emph{n/a}
	\item \textbf{favorite domain} : \emph{to complete}
	\item \textbf{references} :
      \begin{itemize}
      \item \cite{BeldiceanuSICS02} \emph{Arc-Consistency for a chain of Lexicographic Ordering Constraints} 
      \item global constraint catalog: \href{http://www.emn.fr/x-info/sdemasse/gccat/Clex_chain_lesseq.html}{lex\_chain\_lesseq}
      \end{itemize}
\end{itemize}

\textbf{Example}:
\begin{lstlisting}
	CPModel m = new CPModel();                                 
	CPSolver s = new CPSolver();
\end{lstlisting}

%\part{lexeq}
\label{lexeq}
\hypertarget{lexeq}{}

\section{lexeq (constraint)}\label{lexeq:lexeqconstraint}\hypertarget{lexeq:lexeqconstraint}{}
\begin{notedef}
  \texttt{lexeq}$(x,y)$ enforces a lexicographic ordering  $x \le_{lex} y$ between two arrays of same size $n$:
$$\exists\ j\in\{1,\ldots,n\}\ |\qquad x_j\le y_j\quad \land\quad x_i=y_i\ (\forall\  i<j)$$
\end{notedef}

\begin{itemize}
	\item \textbf{API} : \mylst{lexeq(IntegerVariable[] x, IntegerVariable[] y)}
	\item \textbf{return type} : \texttt{Constraint}
	\item \textbf{options} :\emph{n/a}
	\item \textbf{favorite domain} : \emph{to complete}
	\item \textbf{references} :
      \begin{itemize}
      \item \cite{FrischCP02}: \emph{Global Constraints for Lexicographic Orderings}
      \item global constraint catalog: \href{http://www.emn.fr/x-info/sdemasse/gccat/Clex_lesseq.html}{lex\_lesseq}
      \end{itemize}
\end{itemize}

\textbf{Example}:
\begin{lstlisting}
	Model pb = new CPModel();
	Solver s = new CPSolver();
	
	int n1 = 8;
	int k = 2;
	IntegerVariable[] vs1 = new IntegerVariable[n1 / 2];
	IntegerVariable[] vs2 = new IntegerVariable[n1 / 2];
	for (int i = 0; i < n1 / 2; i++) {
	   vs1[i] = makeIntVar("" + i, 0, k);
	   vs2[i] = makeIntVar("" + i, 0, k);
	}
	
	m.addConstraint(lexeq(vs1, vs2));
	s.read(m);
	s.solve();
\end{lstlisting} 

%\part{leximin}
\label{leximin}
\hypertarget{leximin}{}

\section{leximin (constraint)}\label{leximin:leximinconstraint}\hypertarget{leximin:leximinconstraint}{}

\emph{TODO: verify the specifications of the implemented version.}

\begin{notedef}
Let $x = (x_1,\ldots, x_n)$ and $y = (y_1,\ldots, y_n)$ be two vectors of $n$ integers, and let $x'$ and $y'$ be respectively permutations of vectors $x$ and $y$ sorted by increasing order of the components.
Constraint \texttt{leximin(x, y)} holds if and only if $x'<_{lex} y'$:
$$\exists\ j\in\{1,\ldots,n\}\ |\qquad x'_j<y'_j\quad \land\quad x'_i=y'_i\ (\forall\  i<j)$$
  \end{notedef}

\begin{itemize}
	\item \textbf{API} :
	\begin{itemize}
		\item \mylst{leximin(IntegerVariable[] x, IntegerVariable[] y)}
		\item \mylst{leximin(int[] x, IntegerVariable[] y)}
	\end{itemize}
	\item \textbf{return type} : \texttt{Constraint}
	\item \textbf{options} :\emph{n/a}
	\item \textbf{favorite domain} : \emph{to complete}
	\item \textbf{references} :
      \begin{itemize}
      \item \cite{FrischIJCAI03}: \emph{Multiset ordering constraints} 
      \item global constraint catalog: \href{http://www.emn.fr/x-info/sdemasse/gccat/Clex_lesseq_allperm.html}{lex\_lesseq\_allperm} (variant)
      \end{itemize}
\end{itemize}

\textbf{Example}:
\begin{lstlisting}
	Model m = new CPModel();
	Solver s = new CPSolver();
	
	IntegerVariable[] u = makeIntVarArray("u", 3, 2, 5);
	IntegerVariable[] v = makeIntVarArray("v", 3, 2, 4);
	m.addConstraint(leximin(u, v));
	m.addConstraint(allDifferent(v));
	
	s.read(m);
	s.solve();
\end{lstlisting} 

%\part{lt}
\label{lt}
\hypertarget{lt}{}

\section{lt (constraint)}\label{lt:ltconstraint}\hypertarget{lt:ltconstraint}{}
\begin{notedef}
  \texttt{lt}$(x,y)$ states that $x$ is strictly smaller than $y$:
$$x<y$$
\end{notedef}

\begin{itemize}
	\item \textbf{API} :
	\begin{itemize}
		\item \mylst{lt(IntegerExpressionVariable x, IntegerExpressionVariable y)}
		\item \mylst{lt(IntegerExpressionVariable x, int y)}
		\item \mylst{lt(int x, IntegerExpressionVariable y)}
	\end{itemize}
	\item \textbf{return type} : \texttt{Constraint}
	\item \textbf{options} : \emph{n/a}
	\item \textbf{favorite domain} : \emph{to complete}.
	\item \textbf{references} :\\
      global constraint catalog: \href{http://www.emn.fr/x-info/sdemasse/gccat/Clt.html}{lt}
\end{itemize}

\textbf{Example:}
\begin{lstlisting}
	Model m = new CPModel();
	Solver s = new CPSolver();
	int c = 1;
	IntegerVariable v = makeIntVar("v", 0, 2);
	m.addConstraint(lt(v, c));
	s.read(m);
	s.solve();
\end{lstlisting}

%\part{max}
\label{max}
\hypertarget{max}{}

\section{max (constraint)}\label{max:maxconstraint}\hypertarget{max:maxconstraint}{}

\subsection{max of a list}\label{max:maxofalist}\hypertarget{max:maxofalist}{}

\begin{notedef}
\texttt{max}$(x,z)$ states that $z$ is equal to the greater element of vector $x$:
$$z = \max(x_1, x_2, ..., x_n)$$  
\end{notedef}

\begin{itemize}
	\item \textbf{API}:
	\begin{itemize}
		\item \mylst{max(IntegerVariable[] x, IntegerVariable z)}
		\item \mylst{max(IntegerVariable x1, IntegerVariable x2, IntegerVariable z)}
		\item \mylst{max(int x1, IntegerVariable x2, IntegerVariable z)}
		\item \mylst{max(IntegerVariable x1, int x2, IntegerVariable z)}
	\end{itemize}
	\item \textbf{return type}: \texttt{Constraint}
	\item \textbf{options} : \emph{n/a}
	\item \textbf{favorite domain} : \emph{to complete}
	\item \textbf{references} :\\
      global constraint catalog: \href{http://www.emn.fr/x-info/sdemasse/gccat/Cmaximum.html}{maximum}
\end{itemize}

\textbf{Example}:
\begin{lstlisting}
	Model m = new CPModel();
	Solver s= new CPSolver();
	IntegerVariable x = makeIntVar("x", 1, 5);
	IntegerVariable y = makeIntVar("y", 1, 5);
	IntegerVariable z = makeIntVar("z", 1, 5);
	m.addVariable("cp:bound", x, y, z);
	m.addConstraint(max(y, z, x));
	s.read(m);
	s.solve();
\end{lstlisting}

\subsection{max of a set}\label{max:maxofaset}\hypertarget{max:maxofaset}{}

\begin{notedef}
\texttt{max}$(s,x,z)$ states that $z$ is equal to the greater element of vector $x$ whose index is in set $s$:
$$z = \max_{i\in s}( x_i )$$
  \end{notedef}

\begin{itemize}
	\item \textbf{API}:
	\begin{itemize}
		\item \mylst{max(SetVariable s, IntegerVariable[] x, IntegerVariable z)}
	\end{itemize}
	\item \textbf{return type}: \texttt{Constraint}
	\item \textbf{options} : \emph{n/a}
	\item \textbf{favorite domain} : \emph{to complete}
\end{itemize}

\begin{lstlisting}
	Model m = new CPModel();
	Solver s= new CPSolver();
	IntegerVariable[] x = constantArray(new int[]{5,7,9,10,12,3,2});
	IntegerVariable max = makeIntVar("max", 1, 100);
	SetVariable set = makeSetVar("set", 0, x.length-1);
	m.addConstraints(max(set, x, max), leqCard(set, constant(5)));
	s.read(m);
	s.solve();
\end{lstlisting}

%\part{member}
\label{member}
\hypertarget{member}{}

\section{member (constraint)}\label{member:memberconstraint}\hypertarget{member:memberconstraint}{}

\begin{notedef}
  \texttt{member}$(x,s)$ states that integer $x$ is contained in set
  $s$:
$$x\in s.$$
\end{notedef}

\begin{itemize}
	\item \textbf{API} :
	\begin{itemize}
		\item \mylst{member(int x, SetVariable s)}
		\item \mylst{member(SetVariable s, int x)}
		\item \mylst{member(SetVariable s, IntegerVariable x)}
		\item \mylst{member(IntegerVariable x, SetVariable s)}
	\end{itemize}
	\item \textbf{return type} : \texttt{Constraint}
	\item \textbf{options} :\emph{n/a}
	\item \textbf{favorite domain} : \emph{to complete}
	\item \textbf{references} :\\
      global constraint catalog: \href{http://www.emn.fr/x-info/sdemasse/gccat/Cin_set.html}{in\_set}
\end{itemize}

\textbf{Example}:
\lstinputlisting{java/cmember.j2t}

%\part{min}
\label{min}
\hypertarget{min}{}

\section{min (constraint)}\label{min:minconstraint}\hypertarget{min:minconstraint}{}

\subsection{min of a list}\label{min:minofalist}\hypertarget{min:minofalist}{}

\begin{notedef}
  \texttt{mix}$(x,z)$ states that $z$ is equal to the smaller element
  of vector $x$:
$$z = \min(x_1, x_2, ..., x_n).$$
\end{notedef}
\begin{itemize}
	\item \textbf{API}:
	\begin{itemize}
		\item \mylst{min(IntegerVariable[] x, IntegerVariable z)}
		\item \mylst{min(IntegerVariable x1, IntegerVariable x2, IntegerVariable z)}
		\item \mylst{min(int x1, IntegerVariable x2, IntegerVariable z)}
		\item \mylst{min(IntegerVariable x1, int x2, IntegerVariable z)}
	\end{itemize}
	\item \textbf{return type}: \texttt{Constraint}
	\item \textbf{options} : \emph{n/a}
	\item \textbf{favorite domain} : \emph{to complete}
	\item \textbf{references} :\\
      global constraint catalog: \href{http://www.emn.fr/x-info/sdemasse/gccat/Cminimum.html}{minimum}
\end{itemize}

\textbf{Example}:
\begin{lstlisting}
	Model m = new CPModel();
	Solver s= new CPSolver();
	IntegerVariable x = makeIntVar("x", 1, 5);
	IntegerVariable y = makeIntVar("y", 1, 5);
	IntegerVariable z = makeIntVar("z", 1, 5);
	m.addVariable("cp:bound", x, y, z);
	m.addConstraint(min(y, z, x));
	s.read(m);
	s.solve();
\end{lstlisting}

\subsection{min of a set}\label{min:minofaset}\hypertarget{min:minofaset}{}

\begin{notedef}
  \texttt{min}$(s,x,z)$ states that $z$ is equal to the smaller
  element of vector $x$ whose index is in set $s$:
$$z = \min_{i\in s}( x_i ).$$
\end{notedef}
\begin{itemize}
	\item \textbf{API}:
	\begin{itemize}
		\item \mylst{min(SetVariable s,IntegerVariable[] x, IntegerVariable z)}
	\end{itemize}
	\item \textbf{return type}: \texttt{Constraint}
	\item \textbf{options} : \emph{n/a}
	\item \textbf{favorite domain} : \emph{to complete}
\end{itemize}

\begin{lstlisting}
	Model m = new CPModel();
	Solver s= new CPSolver();
	IntegerVariable[] x = constantArray(new int[]{5,7,9,10,12,3,2});
	IntegerVariable min = makeIntVar("min", 1, 100);
	SetVariable set = makeSetVar("set", 0, x.length-1);
	m.addConstraints(min(set, x, max), leqCard(set, constant(5)));
	s.read(m);
	s.solve();
\end{lstlisting}

%\input{chapters/Cminus.tex}
%\part{mod}
\label{mod}
\hypertarget{mod}{}

\section{mod (constraint)}\label{mod:modconstraint}\hypertarget{mod:modconstraint}{}
\begin{notedef}
  \texttt{mod}$(x_1,x_2,x_3)$ states that $x_1$ is congruent to $x_2$
  modulo $x_3$:
$$x_1 \equiv x_2 \mod x_3$$
\end{notedef}
\begin{itemize}
	\item \textbf{API} : \mylst{mod(IntegerVariable x1, IntegerVariable x2, int x3)}
	\item \textbf{return type} : \texttt{Constraint}
	\item \textbf{options} : \emph{n/a}
	\item \textbf{favorite domain} : \emph{n/a}
\end{itemize}

\textbf{Example}:
\begin{lstlisting}
	Model m = new CPModel();
	Solver s = new CPSolver();
	
	IntegerVariable x = makeIntVar("x", 0, 10);
	IntegerVariable w = makeIntVar("w", 0, 10);
	
	m.addConstraint(mod(w,x, 1));
	
	s.read(m);
	s.solve();
\end{lstlisting}

%\input{chapters/Cmult.tex}
%\part{multicostregular}
\label{multicostregular}
\hypertarget{multicostregular}{}

\section{multiCostRegular (constraint)}\label{multicostregular:multicostregularconstraint}\hypertarget{multicostregular:multicostregularconstraint}{}
\begin{notedef}
  \texttt{multiCostRegular}$(x,z,\mathcal{L}(\Pi),c)$ states that sequence $x$ is a word belonging to the regular language $\mathcal{L}(\Pi)$,
% recognized by a deterministic finite automaton (DFA) or a multicostregular expression $\Pi$:
$$(x_1,\ldots,x_n)\in\mathcal{L}(\Pi)$$
and that the bounded vector $z$ is equal to the costs of $x$ according to the assigment cost matrix $c$:
$$\sum_{i=1}^{n} c[r][i][x_i]=z[r],\quad \forall r\in\{0,\ldots,R\}$$
\end{notedef}
\texttt{multiCostRegular} is a conjunction of a \hyperlink{regular}{\texttt{regular}} constraint with $R+1$ cost functions.
It may be used in the context of personnel scheduling problems, handling complex work regulations by the mean of regular expressions, together with cardinality or financial constraints by the mean of cost functions.
The filtering algorithm associated with \texttt{multiCostRegular} is based on lagrangian relaxation and computations of shortest/longest pathes in a layered digraph~\cite{MenanaCPAIOR09}. It typically performs more filtering than the conjunction of \texttt{costRegular} and \texttt{globalCardinality} or than multiple \texttt{costRegular}.

The accepting language is specified by a deterministic finite automaton (DFA):
Automaton $\Pi$ is defined on a given \emph{alphabet} $\Sigma\subseteq\Z$ by a set $Q=\{0,\ldots,m\}$ of \emph{states}, a subset $A\subseteq Q$ of \emph{final} or \emph{accepting states} and a table $\Delta\subseteq Q\!\times\!\Sigma\!\times Q$ of \emph{transitions} between states. $\Pi$ is encoded as an object of class \texttt{Automaton}, whose API contains:
\begin{lstlisting}
  Automaton();
  int addState();
  void setStartingState(int state); 
  void setAcceptingState(int state); 
  void addTransition(int state1, int state2, int label);
  int getNbStates();
\end{lstlisting}
The cost functions are encoded as one matrix \texttt{int cost[nTime][nAct][auto.getNbStates()][nRes]} such that
\texttt{cost[i][j][s][r]} is the cost of assigning variable $x_i$ to activity $j$ at state $s$ on dimension $r+1$.

\begin{itemize}
	\item \textbf{API} :
	\begin{itemize}
		\item \mylst{multiCostRegular(IntegerVariable[] x, IntegerVariable[] z, Automaton P, int[][][][] c)}
	\end{itemize}
	\item \textbf{return type} : \texttt{Constraint}
	\item \textbf{options} :
      \begin{itemize}
      \item \texttt{MultiCostRegular.DATA\_STRUCT} is  \texttt{MultiCostRegular.BITSET} or \texttt{MultiCostRegular.LIST}: a parameter stating which backtrable data structure to use for storing the outgoing arcs of the layered digraph. The observed behaviour is until $1000$ arcs the bipartite list is much more efficient, afterwards the memory efficiency of the bitset representation allow faster operations. 
      \item \texttt{MultiCostRegular.U0}, \texttt{MultiCostRegular.R0}, \texttt{MultiCostRegular.MAXNONIMPROVEITER}, and \texttt{MultiCostRegular.MAXBOUNDITER} are value parameters of the subgradient algorithm used for solving the lagrangean relaxation.
      \item \texttt{MultiCostRegular.D\_PREC} is a double parameter stating the precision of float computation. It is set by default to $10^{-5}$.
      \end{itemize}
	\item \textbf{favorite domain} : \emph{to complete}
	\item \textbf{references} :\\
       \cite{MenanaCPAIOR09}: \emph{Sequencing and Counting with the {\tt multicost-regular} Constraint}
\end{itemize}
%\begin{notedef}
%  For further informations, see the multicost-regular description.
%\end{notedef}
\textbf{Example}:
\begin{lstlisting}
  //1- create the model
  Model m = new CPModel();

  int nTime = 14; // 2 weeks: 14 days
  int nAct = 3;   // 3 activities: DAY, NIGHT, REST
  int nRes = 4;   // 4 resources: cost (0), #DAY (1), #NIGHT (2), #WORK (3)

  //2- Create the schedule variables: the activity processed at each time slot
  IntegerVariable[] sequence = makeIntVarArray("x",nTime,0,nAct-1,"cp:enum");
  // - create the cost variables (one for each resource)
  IntegerVariable[] bounds =  new IntegerVariable[4];
  bounds[0] = makeIntVar("z_0",30,80,"cp:bound"); // 30 <= cost <= 80
  bounds[1] = makeIntVar("day",0,7,"cp:bound");   // 0 <= #DAY <= 7
  bounds[2] = makeIntVar("night",3,7,"cp:bound"); // 3 <= #NIGHT <= 7
  bounds[3] = makeIntVar("work",7,9,"cp:bound");  // 7 <= #WORK <= 9

  //3- Create the automaton
  Automaton auto = new Automaton();
  // state 0: starting and accepting state
  int start = auto.addState();
  auto.setStartingState(start); 
  auto.setAcceptingState(start);
  // state 1 and a transition (0,DAY,1)
  int first = auto.addState();
  auto.addTransition(start,first,DAY);
  // state 2 and transitions (1,DAY,2), (1,NIGHT,2), (2,REST,0), (0,NIGHT,2)
  int second = auto.addState();
  auto.addTransition(first,second,new int[]{DAY,NIGHT});
  auto.addTransition(second,start,REST);
  auto.addTransition(start,second,NIGHT);

  //4- Declare the assignment/transition costs:
  // csts[i][j][s][r]: cost on resource r of assigning Xi to activity j at state s
  int[][][][] csts = new int[nTime][nAct][auto.getNbStates()][nRes];
  for (int i = 0 ; i < csts.length ; i++) {
      csts[i][DAY][0] = new int[]{3,1,0,1}; // costs of transition (0,DAY,1)
      csts[i][NIGHT][0] = new int[]{8,0,1,1}; // costs of transition (0,NIGHT,2)
      csts[i][DAY][1] = new int[]{5,1,0,1}; // costs of transition (1,DAY,2)
      csts[i][NIGHT][1] = new int[]{9,0,1,1}; // costs of transition (1,NIGHT,2)
      csts[i][REST][2] = new int[]{2,0,0,0}; // costs of transition (2,REST,0)
  }

  //5- Set a constraint parameter
  MultiCostRegular.DATA_STRUCT = MultiCostRegular.LIST;

  //6- add the constraint
  m.addConstraint(multiCostRegular(sequence,bounds,auto,csts));	
  
  //7- create the solver, read the model and solve it
  Solver s = new CPSolver();
  s.read(m);
  s.solve();
\end{lstlisting}

%\input{chapters/Cneg.tex}
\input{chapters/Cneq.tex}
%\part{neqcard}
\label{neqcard}
\hypertarget{neqcard}{}

\section{neqCard (constraint)}\label{neqcard:neqcardconstraint}\hypertarget{neqcard:neqcardconstraint}{}

%\part{not}
\label{not}
\hypertarget{not}{}

\section{not (constraint)}\label{not:notconstraint}\hypertarget{not:notconstraint}{}
\begin{notedef}
  \texttt{not}$(c)$ holds if and only if constraint $c$ does not hold:
$$\neg c$$
\end{notedef}
\begin{itemize}
	\item \textbf{API} : \mylst{not(Constraint c)}
	\item \textbf{return type} : \texttt{Constraint}
	\item \textbf{options} : \emph{n/a}
	\item \textbf{favorite domain} : \emph{n/a}
\end{itemize}

\textbf{Example} : 
\begin{lstlisting}
	Model m = new CPModel();
	Solver s = new CPSolver();
	IntegerVariable x = makeIntVar("x", 1, 10);
	// x < 3
	m.addConstraint(not(geq(x, 3)));
	
	s.read(m);
	s.solve();
\end{lstlisting}

%\part{notmember}
\label{notmember}
\hypertarget{notmember}{}

\section{notMember (constraint)}\label{notmember:notmemberconstraint}\hypertarget{notmember:notmemberconstraint}{}
\begin{notedef}
\texttt{notMember}$(x,s)$ states that integer $x$ is not contained in set $s$:
$$x\not\in s$$  
\end{notedef}

\begin{itemize}
	\item \textbf{API} :
	\begin{itemize}
		\item \mylst{notMember(int x, SetVariable s)}
		\item \mylst{notMember(SetVariable s, int x)}
		\item \mylst{notMember(SetVariable s, IntegerVariable x)}
		\item \mylst{notMember(IntegerVariable x, SetVariable s)}
	\end{itemize}
	\item \textbf{return type} : \texttt{Constraint}
	\item \textbf{options} :\emph{n/a}
	\item \textbf{favorite domain} : \emph{to complete}
\end{itemize}

\textbf{Example}:
\lstinputlisting{java/cnotmember.j2t}

%\part{nth}
\label{nth}
\hypertarget{nth}{}

\section{nth (constraint)}\label{nth:nthconstraint}\hypertarget{nth:nthconstraint}{}
\texttt{nth} is the well known \emph{element} constraint.
Several APIs are available: 
\begin{notedef}
\begin{itemize}
\item \texttt{nth}$(i,x,y)$ ensures that $x[i]=y$
\item \texttt{nth}$(i,x,y,o)$ ensures that $x[i+o]=y$ ($o$ is an \emph{offset} for shifting values)
\item \texttt{nth}$(i,j,x,y)$ ensures that $x[i][j]=y$
\end{itemize}
\end{notedef}

\begin{itemize}
	\item \textbf{API} :
	\begin{itemize}
		\item \mylst{nth(IntegerVariable i, int[] x, IntegerVariable y)}
		\item \mylst{nth(String option, IntegerVariable i, int[] x, IntegerVariable y)}
		\item \mylst{nth(IntegerVariable i, IntegerVariable[] x, IntegerVariable y)}
		\item \mylst{nth(IntegerVariable i, int[] x, IntegerVariable y, int offset)}
		\item \mylst{nth(String option, IntegerVariable i, int[] x, IntegerVariable y, int offset)}		
		\item \mylst{nth(IntegerVariable i, IntegerVariable[] x, IntegerVariable y, int offset)}
		\item \mylst{nth(String option, IntegerVariable i, IntegerVariable[] x, IntegerVariable y, int offset)}
		\item \mylst{nth(IntegerVariable i, IntegerVariable j, int[][] x, IntegerVariable y)}
	\end{itemize}
	\item \textbf{return type} : \texttt{Constraint}
	\item \textbf{options} :
	\begin{itemize}
		\item \emph{no option} 
		\item \hyperlink{cnthg:cnthgoptions}{\tt CPOptions.C\_NTH\_G} for global consistency
	\end{itemize}
	\item \textbf{favorite domain} : \emph{to complete}
	\item \textbf{references} :\\
      global constraint catalog: \href{http://www.emn.fr/x-info/sdemasse/gccat/Celement.html}{element}
\end{itemize}

\textbf{Example}:
\lstinputlisting{java/cnth.j2t} 

%\part{occurrence}
\label{occurrence}
\hypertarget{occurrence}{}

\section{occurrence (constraint)}\label{occurrence:occurrenceconstraint}\hypertarget{occurrence:occurrenceconstraint}{}
\begin{notedef}
  \texttt{occurrence}$(v,z,x)$ states that $z$ is equal to the number of elements in $x$ with value $v$:
$$z=|\{i\ |\ x_i=v\}|$$   
\end{notedef}
  This is a specialization of the \texttt{globalCardinality} constraint.

\begin{itemize}
	\item \textbf{API}: \mylst{occurrence(int v, IntegerVariable z, IntegerVariable... x)}
	\item \textbf{return type} : \texttt{Constraint}
	\item \textbf{options} :\emph{n/a}
	\item \textbf{favorite domain} : \emph{to complete}
	\item \textbf{references} :\\
      global constraint catalog: \href{http://www.emn.fr/x-info/sdemasse/gccat/Ccount.html}{count}
\end{itemize}

\textbf{Example}:
\begin{lstlisting}
	Model m = new CPModel();
	Solver s = new CPSolver();
	
	IntegerVariable x1 = makeIntVar("X1", 0, 10);
	IntegerVariable x2 = makeIntVar("X2", 0, 10);
	IntegerVariable x3 = makeIntVar("X3", 0, 10);
	IntegerVariable x4 = makeIntVar("X4", 0, 10);
	IntegerVariable x5 = makeIntVar("X5", 0, 10);
	IntegerVariable x6 = makeIntVar("X6", 0, 10);
	IntegerVariable x7 = makeIntVar("X7", 0, 10);
	IntegerVariable y1 = makeIntVar("Y1", 0, 10);
	
	m.addConstraint(occurrence(3, y1, new IntegerVariable[]{x1, x2, x3, x4, x5, x6, x7}));
	
	s.read(m);
	s.solve();
\end{lstlisting} 

%\part{occurrencemax}
\label{occurrencemax}
\hypertarget{occurrencemax}{}

\section{occurrenceMax (constraint)}\label{occurrencemax:occurrencemaxconstraint}\hypertarget{occurrencemax:occurrencemaxconstraint}{}
\begin{notedef}
  \texttt{occurrenceMax}$(v,z,x)$ states that $z$ is at most equal to the number of elements in $x$ with value $v$:
$$z\le|\{i\ |\ x_i=v\}|$$   
\end{notedef}
  This is a specialization of the \texttt{globalCardinality} constraint.

\begin{itemize}
	\item \textbf{API}: \mylst{occurrenceMax(int v, IntegerVariable z, IntegerVariable... x)}
	\item \textbf{return type} : \texttt{Constraint}
	\item \textbf{options} :\emph{n/a}
	\item \textbf{favorite domain} : \emph{to complete}
	\item \textbf{references} :\\
      global constraint catalog: \href{http://www.emn.fr/x-info/sdemasse/gccat/Ccount.html}{count}
\end{itemize}

\textbf{Example}:
\lstinputlisting{java/coccurrencemax.j2t}
%\part{occurrencemin}
\label{occurrencemin}
\hypertarget{occurrencemin}{}

\section{occurrenceMin (constraint)}\label{occurrencemin:occurrenceminconstraint}\hypertarget{occurrencemin:occurrenceminconstraint}{}
\begin{notedef}
  \texttt{occurrenceMin}$(v,z,x)$ states that $z$ is at most equal to the number of elements in $x$ with value $v$:
$$z\le|\{i\ |\ x_i=v\}|$$   
\end{notedef}
  This is a specialization of the \texttt{globalCardinality} constraint.

\begin{itemize}
	\item \textbf{API}: \mylst{occurrenceMin(int v, IntegerVariable z, IntegerVariable... x)}
	\item \textbf{return type} : \texttt{Constraint}
	\item \textbf{options} :\emph{n/a}
	\item \textbf{favorite domain} : \emph{to complete}
	\item \textbf{references} :\\
      global constraint catalog: \href{http://www.emn.fr/x-info/sdemasse/gccat/Ccount.html}{count}
\end{itemize}

\textbf{Example}:
\lstinputlisting{java/coccurrencemin.j2t}
%\part{oppositesign}
\label{oppositesign}
\hypertarget{oppositesign}{}

\section{oppositeSign (constraint)}\label{oppositesign:oppositesignconstraint}\hypertarget{oppositesign:oppositesignconstraint}{}


\begin{notedef}
  \texttt{oppositeSign}$(x,y)$ states that the two arguments have opposite signs:
$$xy\le 0$$
\end{notedef}

\begin{notedef}
  0 is considered as both sign, if one argument is equal to 0, the constraint is not satisfied.
\end{notedef}


\begin{itemize}
	\item \textbf{API} : \mylst{oppositeSign(IntegerExpressionVariable x, IntegerExpressionVariable y)}
	\item \textbf{return type} : \texttt{Constraint}
	\item \textbf{options} :\emph{n/a}
	\item \textbf{favorite domain} : \emph{to complete}
\end{itemize}

\textbf{Example}:
\lstinputlisting{java/coppositesign.j2t} 

%\part{or}
\label{or}
\hypertarget{or}{}

\section{or (constraint)}\label{or:orconstraint}\hypertarget{or:orconstraint}{}
\begin{notedef}
  \texttt{or}$(c_1,\ldots,c_n)$ states that one or more of the constraints in arguments are satisfied:
$$ c_1 \lor c_2 \lor\ldots\lor c_n$$

  \texttt{or}$(b_1,\ldots,b_n)$ states that one or more of the 0-1  variables in arguments are true (equal to 1):
$$ b_1=1 \lor b_2=1 \lor\ldots\lor b_n=1$$
\end{notedef}

\begin{itemize}
\item \textbf{API} : 
\begin{itemize}
\item \mylst{or(Constraint... c)}
\item \mylst{or(IntegerVariable... b)}
\end{itemize}
	\item \textbf{return type} : \texttt{Constraint}
	\item \textbf{options} : \emph{n/a}
	\item \textbf{favorite domain} : \emph{n/a}
\end{itemize}

\textbf{Examples:}
\begin{itemize}
	\item example1:
\end{itemize}
\begin{lstlisting}
	Model m = new CPModel();
	Solver s = new CPSolver();
	
	IntegerVariable v1 = makeIntVar("v1", 0, 1);
	IntegerVariable v2 = makeIntVar("v2", 0, 1);
	
	m.addConstraint(or(eq(v1, 1), eq(v2, 1)));
	
	s.read(m);
	s.solve();
\end{lstlisting}
\begin{itemize}
	\item example2
\end{itemize}
\begin{lstlisting}
public void cand2() {
        //totex cor2
        Model m = new CPModel();
        Solver s = new CPSolver();
        IntegerVariable[] vars = makeBooleanVarArray("b", 10);
        m.addConstraint(or(vars));
        s.read(m);
        s.solve();
        //totex
    }
\end{lstlisting}

%\part{pack}
\label{pack}
\hypertarget{pack}{}

\section{pack (constraint)}\label{pack:packconstraint}\hypertarget{pack:packconstraint}{}

\begin{notedef}
  \texttt{pack(items, load, bin, size)} states that a collection of items is packed into different bins, such that the total size of the items in each bin does not exceed the bin capacity:
$$ \mathtt{load}[b] = \sum_{i\in\mathtt{items}[b]} \mathtt{size}[i],\quad\forall \text{ bin } b $$
%and
$$ i\in\mathtt{items}[b]\ \iff\ \mathtt{bin}[i]=b,\quad\forall \text{ bin } b,\ \forall \text{ item } i $$
\end{notedef}
%\texttt{pack}$(sizes, n, )$ states a collection of items (each of them having a specific size) is packed into different bins of given capacity such that the total weight of the items in each bin does not exceed the bin capacity.
\texttt{pack} is a \href{http://www.emn.fr/x-info/sdemasse/gccat/Cbin_packing.html}{bin packing constraint} based on \cite{ShawCP04}. 

\begin{itemize}
	\item \textbf{API} :
	\begin{itemize}
		\item \mylst{pack(SetVariable[] items, IntegerVariable[] load, IntegerVariable[] bin, IntegerConstantVariable[] size, String... options)}
		\item \mylst{pack(PackModeler modeler,String... options)}: PackModeler is a high-level modeling object.
		\item \mylst{pack(int[] sizes, int nbBins, int capacity, String... options)}: build instance with PackModeler.
	\end{itemize}
	\item \textbf{Variables}:
	\begin{itemize}
		\item \texttt{SetVariable[] items: items}$[b]$ is the set of items packed into bin $b$.
		\item \texttt{IntegerVariable[] load: load}$[b]$ is the total size of the items packed into bin $b$.
		\item \texttt{IntegerVariable[] bin: bin}$[i]$ is the bin where item $i$ is packed into.
		\item \texttt{IntegerConstantVariable[] size: size}$[i]$ is the size of item $i$.
	\end{itemize}
	\item \textbf{return type} : \texttt{Constraint}
	\item \textbf{options} : 	
      \begin{itemize}
      \item \texttt{SettingType.ADDITIONAL\_RULES}: additional filtering rules \emph{recommended}
      \item \texttt{SettingType.DYNAMIC\_LB}: feasibility tests based on dynamic lower bounds for 1D-bin packing
      \end{itemize}
	\item \textbf{favorite domain} : \emph{to complete}
	\item \textbf{references} :
      \begin{itemize}
      \item \cite{ShawCP04}: \emph{A constraint for bin packing}
      \item global constraint catalog: \href{http://www.emn.fr/x-info/sdemasse/gccat/Cbin_packing.html}{bin\_packing} (variant)
      \end{itemize}
\end{itemize}

\textbf{Example}:

Take a look at \emph{choco.samples.pack} to see advanced use of the constraint.
\begin{lstlisting}
	Model m = new CPModel();
	m.addConstraint(Choco.pack(new int[]{5,3,2,6,8,5},5,10, SettingType.ADDITIONAL_RULES.getOptionName()));
	Solver s = new CPSolver();
	s.read(m);
	s.solve();
\end{lstlisting} 

%\input{chapters/Cplus.tex}
%\input{chapters/Cpower.tex}
\input{chapters/Cprecedencereified.tex}
%\part{regular}
\label{regular}
\hypertarget{regular}{}

\section{regular (constraint)}\label{regular:regularconstraint}\hypertarget{regular:regularconstraint}{}
\begin{notedef}
  \texttt{regular}$(x,\mathcal{L}(\Pi))$ states that sequence $x$ is a word belonging to the regular language $\mathcal{L}(\Pi)$:
% recognized by a deterministic finite automaton (DFA) or a regular expression $\Pi$:
$$(x_1,\ldots,x_n)\in\mathcal{L}(\Pi)$$
\end{notedef}

The accepting language can be specified either by a deterministic finite automaton (DFA), a list of feasible or infeasible tuples, or a regular expression:
\begin{description}
\item[DFA:] Automaton $\Pi$ is defined on a given \emph{alphabet} $\Sigma\subseteq\Z$ by a set $Q=\{0,\ldots,m\}$ of \emph{states}, a subset $A\subseteq Q$ of \emph{final} or \emph{accepting states} and a table $\Delta\subseteq Q\!\times\!\Sigma\!\times Q$ of \emph{transitions} between states. $\Delta$ is encoded as \texttt{List<Transition>} where a Transition object $\delta=\texttt{new Transition}(q_i,\sigma,q_j)$ is made of three integers expressing the ingoing state $q_i$, the label $\sigma$, and the outgoing state $q_j$.
Automaton $\Pi$ is a DFA if $\Delta$ is finite and if it has only one initial state (here, state $0$ is considered as the unique initial state) and no two transitions sharing the same ingoing state and the same label.
\item[feasible tuples:] \emph{regular} can be used as an extensional constraint. Given the list of \emph{feasible} tuples for sequence $x$, this API builds a DFA from the list, and then enforces GAC on the constraint. Using \texttt{regular} can be more efficient than a standard GAC algorithm on tables of tuples if the tuples are structured so that the resulting DFA is compact. The DFA is built from the list of tuples by computing incrementally the minimal DFA after each addition of tuple. 
\item[infeasible tuples:] An another API allows to specify the list of \emph{infeasible} tuples and then builds the corresponding feasible DFA. This operation requires to know the entire alphabet, hence this API has two mandatory table fields \emph{min} and \emph{max} defining the minimum and maximum values of each variable $x_i$.
\item[regular expression:] Finally, the \texttt{regular} constraint can be based on a \href{http://en.wikipedia.org/wiki/regularexpression}{regular expression}, such as \mylst{String regexp = "(1}2)(3*)";| This expression recognizes any (possibly empty) sequences of 3 preceded by at least one 1 or one 2.
\end{description}

\begin{itemize}
	\item \textbf{API} :
	\begin{itemize}
		\item \mylst{regular(DFA pi, IntegerVariable[] x)}
		\item \mylst{regular(IntegerVariable[] x, List<int[]> feasTuples)}
		\item \mylst{regular(IntegerVariable[] x, List<int[]> infeasTuples, int[] min, int[] max)}
		\item \mylst{regular(String regexp, IntegerVariable[] x)}
	\end{itemize}
	\item \textbf{return type} : \texttt{Constraint}
	\item \textbf{options} :\emph{n/a}
	\item \textbf{favorite domain} : \emph{to complete}
	\item \textbf{references} :\\
       \cite{PesantCP04}: \emph{A regular language membership constraint}
\end{itemize}

\textbf{Examples}:
\begin{itemize}
	\item example 1 with DFA:
\end{itemize}

\begin{lstlisting}
  //1- Create the model
  Model m = new CPModel();
  int n = 6;
  IntegerVariable[] vars = new IntegerVariable[n];
  for (int i = 0; i < vars.length; i++) {
      vars[i] = makeIntVar("v" + i, 0, 5);
  }
  //2- Build the list of transitions of the DFA
  List<Transition> t = new LinkedList<Transition>();
  t.add(new Transition(0, 1, 1));
  t.add(new Transition(1, 1, 2));
  // transition with label 1 from state 2 to state 3 
  t.add(new Transition(2, 1, 3));
  t.add(new Transition(3, 3, 0));
  t.add(new Transition(0, 3, 0));
  
  //3- Two final states: 0, 3
  List<Integer> fs = new LinkedList<Integer>();
  fs.add(0); fs.add(3);
            
  //4- Build the DFA
  DFA auto = new DFA(t, fs, n);
  
  //5- add the constraint
  m.addConstraint(regular(auto, vars));
  
  //6- create the solver, read the model and solve it
  Solver s = new CPSolver();
  s.read(m);
  s.solve();
  do {
      for (int i = 0; i < n; i++)
      System.out.print(s.getVar(vars[i]).getVal());
      System.out.println("");
  } while (s.nextSolution());
  
  //7- Print the number of solution found
  System.out.println("Nb_sol : " + s.getNbSolutions());
\end{lstlisting}

\begin{itemize}
	\item example 2 with feasible tuples:
\end{itemize}

\begin{lstlisting}
  //1- Create the model
  Model m = new CPModel();
  IntegerVariable v1 = makeIntVar("v1", 1, 4);
  IntegerVariable v2 = makeIntVar("v2", 1, 4);
  IntegerVariable v3 = makeIntVar("v3", 1, 4);
	
  //2- add some allowed tuples (here, the tuples define a all_equal constraint)
  List<int[]> tuples = new LinkedList<int[]>();
  tuples.add(new int[]{1, 1, 1});
  tuples.add(new int[]{2, 2, 2});
  tuples.add(new int[]{3, 3, 3});
  tuples.add(new int[]{4, 4, 4});
	
  //3-  add the constraint
  m.addConstraint(regular(new IntegerVariable[]{v1, v2, v3}, tuples));
	
  //4- Create the solver, read the model and solve it
  Solver s = new CPSolver();	
  s.read(m);
  s.solve();
  do {
      System.out.println("("+s.getVar(v1)+","+s.getVar(v2)+","+s.getVar(v3)+")");
  } while (s.nextSolution());
	
  //5- Print the number of solution found
  System.out.println("Nb_sol : " + s.getNbSolutions());
\end{lstlisting}

\begin{itemize}
	\item example 3 with regular expression:
\end{itemize}

\begin{lstlisting}
  //1- Create the model
  Model m = new CPModel();
  int n = 6;
  IntegerVariable[] vars = makeIntVarArray("v", n, 0, 5);

  //2- add the constraint
  String regexp = "(1|2)(3*)(4|5)";
  m.addConstraint(regular(regexp, vars));
	
  //3- Create the solver, read the model and solve it
  Solver s = new CPSolver();
  s.read(m);
  s.solve();
  do {
      for (int i = 0; i < n; i++)
          System.out.print(s.getVar(vars[i]).getVal());
      System.out.println("");
  } while (s.nextSolution());
	
  //4- Print the number of solution found
  System.out.println("Nb_sol : " + s.getNbSolutions());
\end{lstlisting}

\input{chapters/Creifiedintconstraint.tex}
%\part{relationpairac}
\label{relationpairac}
\hypertarget{relationpairac}{}

\section{relationPairAC (constraint)}\label{relationpairac:relationpairacconstraint}\hypertarget{relationpairac:relationpairacconstraint}{}
\begin{notedef}
  \texttt{relationPairAC}$(x,y,rel)$ states an extensional binary constraint on $(x,y)$ defined by the binary relation $rel$:
$$(x,y)\in rel$$
\end{notedef}
Many constraints of the same kind often appear in a model. Relations can therefore often be shared among many constraints to spare memory.

The API is duplicated to allow definition of options.

\begin{itemize}
	\item \textbf{API} :
	\begin{itemize}
		\item \mylst{relationPairAC(IntegerVariable x, IntegerVariable y, BinRelation rel)}
		\item \mylst{relationPairAC(String options, IntegerVariable x, IntegerVariable y, BinRelation rel)}
	\end{itemize}
	\item \textbf{return type} : \texttt{Constraint}
	\item \textbf{options} :
	\begin{itemize}
		\item \emph{no option} : use AC3 (default arc consistency)
		\item \texttt{cp:ac3}: to get AC3 algorithm (searching from scratch for supports on all values)
		\item \texttt{cp:ac2001}: to get AC2001 algorithm (maintaining the current support of each value)
		\item \texttt{cp:ac32}: to get AC3rm algorithm (maintaining the current support of each value in a non backtrackable way)
		\item \texttt{cp:ac322}: to get AC3 with the used of \texttt{BitSet} to know if a support still exists
	\end{itemize}
	\item \textbf{favorite domain} : \emph{to complete}
\end{itemize}

\textbf{Example}:
\begin{lstlisting}
	private class MyEquality extends CouplesTest {
	
	    public boolean checkCouple(int x, int y) {
	    return x == y;
	    }
	
	    public void example(){
	        Model m = new CPModel();
	        Solver s = new CPSolver();
	        IntegerVariable v1 = makeIntVar("v1", 1, 4);
	        IntegerVariable v2 = makeIntVar("v2", 1, 4);    
	        IntegerVariable v3 = makeIntVar("v3", 3, 6);
	        m.addConstraint(relationPairAC("cp:ac32",v1, v2, new MyEquality()));
	        m.addConstraint(relationPairAC("cp:ac32",v2, v3, new MyEquality()));
	        s.read(m);
	        s.solveAll();
	   }
	}
\end{lstlisting} 

%\part{relationtupleac}
\label{relationtupleac}
\hypertarget{relationtupleac}{}

\section{relationTupleAC (constraint)}\label{relationtupleac:relationtupleacconstraint}\hypertarget{relationtupleac:relationtupleacconstraint}{}
\begin{notedef}
  \texttt{relationTupleAC}$(x,rel)$ states an extensional constraint on $(x_1,\ldots,x_n)$ defined by the $n$-ary relation $rel$, and then enforces arc consistency:
$$(x_1,\ldots,x_n)\in rel$$
\end{notedef}
Many constraints of the same kind often appear in a model. Relations can therefore often be shared among many constraints to spare memory.
The API is duplicated to define options.

\begin{itemize}
	\item \textbf{API}:
	\begin{itemize}
		\item \mylst{relationTupleAC(IntegerVariable[] x, LargeRelation rel)}
		\item \mylst{relationTupleAC(String options, IntegerVariable[] x, LargeRelation rel)}
	\end{itemize}
	\item \textbf{return type}: \texttt{Constraint}
	\item \textbf{options} :
	\begin{itemize}
		\item \emph{no option}: use AC32 (default arc consistency)
		\item \texttt{cp:ac32}: to get AC3rm algorithm (maintaining the current support of each value in a non backtrackable way)
		\item \texttt{cp:ac2001}: to get AC2001 algorithm (maintaining the current support of each value)
		\item \texttt{cp:ac2008}: to get AC2008 algorithm (maintained by STR)
	\end{itemize}
	\item \textbf{favorite domain} : \emph{to complete}
\end{itemize}

\textbf{Example} :
\lstinputlisting{java/cnotallequal.j2t}
\lstinputlisting{java/crelationtupleac.j2t}
\input{chapters/Crelationtuplefc.tex}
%\part{samesign}
\label{samesign}
\hypertarget{samesign}{}

\section{sameSign (constraint)}\label{samesign:samesignconstraint}\hypertarget{samesign:samesignconstraint}{}
\todo{verify case 0}

\begin{notedef}
  \texttt{sameSign}$(x,y)$ states that the two arguments have the same sign:
$$xy\ge 0$$
\end{notedef}

\begin{itemize}
	\item \textbf{API} : \mylst{sameSign(IntegerExpressionVariable x, IntegerExpressionVariable y)}
	\item \textbf{return type} : \texttt{Constraint}
	\item \textbf{options} :\emph{n/a}
	\item \textbf{favorite domain} : \emph{to complete}
\end{itemize}

\textbf{Example}:
\begin{lstlisting}
	Model m = new CPModel();
	Solver s = new CPSolver();
	IntegerVariable x = makeIntVar("x", -1, 1);
	IntegerVariable y = makeIntVar("y", -1, 1);
	IntegerVariable z = makeIntVar("z", 0, 1000);
	m.addConstraint(oppositeSign(x,y));
	m.addConstraint(eq(z, plus(mult(x, -425), mult(y, 391))));
	s.read(m);
	s.solve();
	System.out.println(s.getVar(z).getVal());
\end{lstlisting} 

%\input{chapters/Cscalar.tex}
%\part{setdisjoint}
\label{setdisjoint}
\hypertarget{setdisjoint}{}

\section{setDisjoint (constraint)}\label{setdisjoint:setdisjointconstraint}\hypertarget{setdisjoint:setdisjointconstraint}{}
\begin{notedef}
  \texttt{setDisjoint}$(s_1,s_2)$ states that the two set arguments are disjoint:
$$s_1\cap s_2=\emptyset$$
\end{notedef}

\begin{itemize}
	\item \textbf{API} : \mylst{setDisjoint(SetVariable s1, SetVariable s2)}
	\item \textbf{return type} : \texttt{Constraint}
	\item \textbf{options} :\emph{n/a}
	\item \textbf{favorite domain} : \emph{to complete}
\end{itemize}

\textbf{Example}:
\begin{lstlisting}
	Model m = new CPModel();
	Solver s = new CPSolver();
	SetVar x = makeSetVar("X", 1, 3);
	SetVar y = makeSetVar("Y", 1, 3);
	Constraint c1 = setDisjoint(x, y);
	m.addConstraint(c1);
	s.read(m);
	s.solveAll();
\end{lstlisting} 

%\part{setinter}
\label{setinter}
\hypertarget{setinter}{}

\section{setInter (constraint)}\label{setinter:setinterconstraint}\hypertarget{setinter:setinterconstraint}{}
\begin{notedef}
  \texttt{setInter}$(s_1,s_2,s_3)$ states that the third set $s_3$ is exactly the intersection of the two first sets:
$$s_1\cap s_2=s_3$$
\end{notedef}

\begin{itemize}
	\item \textbf{API} : \mylst{setInter(SetVariable s1, SetVariable s2, SetVariable s3)}
	\item \textbf{return type} : \texttt{Constraint}
	\item \textbf{options} :\emph{n/a}
	\item \textbf{favorite domain} : \emph{to complete}
\end{itemize}

\textbf{Example}:
\begin{lstlisting}
	Model m = new CPModel();
	Solver s = new CPSolver();
	setVar x = makeSetVar("X", 1, 3);
	SetVar y = makeSetVar("Y", 1, 3);
	SetVar z = makeSetVar("Z", 2, 3);
	Constraint c1 = setInter(x, y, z);
	m.addConstraint(c1);
	s.read(m);
	s.solveAll();
\end{lstlisting} 

%\part{setunion}
\label{setunion}
\hypertarget{setunion}{}

\section{setUnion (constraint)}\label{setunion:setunionconstraint}\hypertarget{setunion:setunionconstraint}{}
\begin{notedef}
  \texttt{setUnion}$(s_1,s_2,s_3)$ states that the third set $s_3$ is exactly the union of the two first sets:
$$s_1\cup s_2=s_3$$
\end{notedef}

\begin{itemize}
	\item \textbf{API} : \mylst{setUnion(SetVariable s1, SetVariable s2, SetVariable s3)}
	\item \textbf{return type} : \texttt{Constraint}
	\item \textbf{options} :\emph{n/a}
	\item \textbf{favorite domain} : \emph{to complete}
\end{itemize}

\textbf{Example}:
\lstinputlisting{java/csetunion.j2t}
%\input{chapters/Csin.tex}
%\part{sorting}
\label{sorting}
\hypertarget{sorting}{}

\section{sorting (constraint)}\label{sorting:sortingconstraint}\hypertarget{sorting:sortingconstraint}{}

%\input{chapters/Cstretchcyclic.tex}
\input{chapters/Cstretchpath.tex}
%\input{chapters/Csum.tex}
\input{chapters/Ctimes.tex}
%\part{tree}
\label{tree}
\hypertarget{tree}{}

\section{tree (constraint)}\label{tree:treeconstraint}\hypertarget{tree:treeconstraint}{}

Let $G=(V,A)$ be a digraph on $V=\{1,\ldots,n\}$. $G$ can be modeled by a sequence of domain variables $x=(x_1,\dots,x_n)\in V^n$ -- the \emph{successors} variables -- whose respective domains are given by $D_i=\{j\in V\ |\ (i,j)\in A\}$. Conversely, when instantiated, $x$ defines a subgraph $G_x=(V,A_x)$ of $G$ with $A_x=\{(i,x_i)\ |\ i\in V\}\subseteq A$. Such a subgraph has one particularity: any connected component of $G_x$ contains either no loop -- and then it contains a cycle -- or exactly one loop $x_i=i$ and then it is a \emph{tree} of root $i$ (literally, it is an anti-arborescence as there exists a path from each node to $i$ and $i$ has a loop).

\begin{notedef}
  \texttt{tree}$(x,restrictions)$ is a vertex-disjoint graph partitioning constraint. It states that $G_x$ is a forest (its connected components are trees) that satisfies some conditions specified by $restrictions$.
\texttt{tree} deals with several kinds of graph restrictions on:
\begin{itemize}
	\item the number of trees
	\item the number of proper trees (a tree is proper if it contains more than 2 nodes)
    \item the weigth of the partition: the sum of the weights of the edges
	\item incomparability: some nodes in pairs have to belong to distinct trees
	\item precedence: some nodes in pairs have to belong to the same tree in a given order
	\item conditional precedence: some nodes in pairs have to respect a given order if they belong to the same tree
	\item the in-degree of the nodes
	\item the time windows on nodes (given travelling times on arcs)
\end{itemize}
\end{notedef}

Many applications require to partition a graph such that each component contains exactly one \emph{resource} node and several \emph{task} nodes. A typical example is a routing problem where vehicle routes are paths (a path is a special case of tree) starting from a depot and delivering goods to several clients. Another example is a local network where each computer has to be connected to one shared printer. Last, one can cite the problem of reconstructing plylogeny trees.
The constraint \texttt{tree} can handle these kinds of problems with many additional constraints on the structure of the partition.

\begin{itemize}
	\item \textbf{API} : \mylst{tree(TreeParametersObject param)}
	\item \textbf{return type} : \texttt{Constraint}
	\item \textbf{options} :\emph{n/a}
	\item \textbf{favorite domain} : \emph{to complete}
	\item \textbf{references} :
      \begin{itemize}
      \item \cite{beldiceanuCONSTRAINTS08}: \emph{Combining tree partitioning, precedence, and incomparability constraints}
      \item global constraint catalog: \href{http://www.emn.fr/x-info/sdemasse/gccat/Cproper_forest.html}{proper\_forest} (variant)
      \end{itemize}

\end{itemize}

The tree constraint API requires a particular Model object, named \textbf{\tt TreeParametersObject}.
It can be created with the following parameters:

\begin{tabular}{p{3cm}p{3cm}p{7cm}}
parameter &type &description\\
\hline
$n$ &\texttt{int} &number of nodes in the initial graph $G$\\
$nTree$ &\texttt{IntegerVariable} &number of trees in the resulting forest $G_x$\\
$nProper$ &\texttt{IntegerVariable} &number of proper trees in $G_x$\\
$objective$ &\texttt{IntegerVariable} &(bounded) total \todo{cost} of $G_x$\\
%$objective$ &\texttt{IntegerVariable} &(bounded) total weight of $G_x$\\
$graphs$ &\texttt{List<BitSet[]>} &
\begin{minipage}[t]{7cm}
graphs encoded as successor lists,\\
  \texttt{graphs[0]} the initial graph $G$,\\
  \texttt{graphs[1]} a precedence graph,\\
  \texttt{graphs[2]} a conditional precedence graph,\\
  \texttt{graphs[3]} an incomparability graph
\end{minipage}\\
$matrix$ &\texttt{List<int[][]>} &\texttt{matrix[0]} the indegree of each node, and \texttt{matrix[1]} the starting time from each node\\
$travel$ &\texttt{int[][]} &the travel time of each arc
\end{tabular}

\textbf{Example}:
\begin{lstlisting}
	Model m = new CPModel();
	int nbNodes = 7;

	//1- create the variables involved in the partitioning problem
	IntegerVariable ntree = makeIntVar("ntree",1,5);
	IntegerVariable nproper = makeIntVar("nproper",1,1);
	IntegerVariable objective = makeIntVar("objective",1,100);

	//2- create the different graphs modeling restrictions
	List<BitSet[]> graphs = new ArrayList<BitSet[]>();
	BitSet[] succ = new BitSet[nbNodes];
	BitSet[] prec = new BitSet[nbNodes];
	BitSet[] condPrecs = new BitSet[nbNodes];
	BitSet[] inc = new BitSet[nbNodes];
	for (int i = 0; i < nbNodes; i++) {
	    succ[i] = new BitSet(nbNodes);
	    prec[i] = new BitSet(nbNodes);
	    condPrecs[i] = new BitSet(nbNodes);
	    inc[i] = new BitSet(nbNodes);
	}

    // initial graph (encoded as successors variables)
	succ[0].set(0,true); succ[0].set(2,true); succ[0].set(4,true);
	succ[1].set(0,true); succ[1].set(1,true); succ[1].set(3,true);
	succ[2].set(0,true); succ[2].set(1,true); succ[2].set(3,true); succ[2].set(4,true);
	succ[3].set(2,true); succ[3].set(4,true); // successor of 3 is either 2 or 4
	succ[4].set(2,true); succ[4].set(3,true);
	succ[5].set(4,true); succ[5].set(5,true); succ[5].set(6,true);
	succ[6].set(3,true); succ[6].set(4,true); succ[6].set(5,true);

    // restriction on precedences
	prec[0].set(4,true); // 0 has to precede 4 
	prec[4].set(3,true); prec[4].set(2,true);
	prec[6].set(4,true);

    // restriction on conditional precedences
	condPrecs[5].set(1,true); // 5 has to precede 1 if they belong to the same tree 

    // restriction on incomparability:
	inc[0].set(6,true);	inc[6].set(0,true); // 0 and 6 have to belong to distinct trees

	graphs.add(succ);
	graphs.add(prec);
	graphs.add(condPrecs);
	graphs.add(inc);

	//3- create the different matrix modeling restrictions
	List<int[][]> matrix = new ArrayList<int[][]>();

    // restriction on bounds on the indegree of each node 
	int[][] degree = new int[nbNodes][2];
	for (int i = 0; i < nbNodes; i++) {
	    degree[i][0] = 0; degree[i][1] = 2;  // 0 <= indegree[i] <= 2
	}
	matrix.add(degree);

    // restriction on bounds on the starting time at each node 
	int[][] tw = new int[nbNodes][2];
	for (int i = 0; i < nbNodes; i++) {
	    tw[i][0] = 0; tw[i][1] = 100;   // 0 <= start[i] <= 100
	}
	tw[0][1] = 15;        				// 0 <= start[0] <= 15
	tw[2][0] = 35; tw[2][1] = 40;		// 35 <= start[2] <= 45
	tw[6][1] = 5;        				// 0 <= start[6] <= 5
	matrix.add(tw);

	//4- matrix for the travel time between each pair of nodes
	int[][] travel = new int[nbNodes][nbNodes];
	for (int i = 0; i < nbNodes; i++) {
	    for (int j = 0; j < nbNodes; j++) travel[i][j] = 100000;
	}
	travel[0][0] = 0; travel[0][2] = 10; travel[0][4] = 20;
	travel[1][0] = 20; travel[1][1] = 0; travel[1][3] = 20;
	travel[2][0] = 10; travel[2][1] = 10; travel[2][3] = 5; travel[2][4] = 5;
	travel[3][2] = 5; travel[3][4] = 2;
	travel[4][2] = 5; travel[4][3] = 2;
	travel[5][4] = 15; travel[5][5] = 0; travel[5][6] = 10;
	travel[6][3] = 5; travel[6][4] = 20; travel[6][5] = 10;

	//5- create the input structure and the tree constraint
	TreeParametersObject parameters = new TreeParametersObject(nbNodes, ntree, nproper, objective, graphs, matrix, travel);
	Constraint c = Choco.tree(parameters);

	m.addConstraint(c);
	Solver s = new CPSolver();
	s.read(m);
	
	//6- heuristic: choose successor variables as the only decision variables
	s.setVarIntSelector(new StaticVarOrder(s.getVar(parameters.getSuccVars())));
	CPSolver.setVerbosity(CPSolver.SOLUTION);
	s.solveAll();
\end{lstlisting} 

\input{chapters/Ctrue.tex}
%\part{xnor}
\label{xnor}
\hypertarget{xnor}{}

\section{xnor (constraint)}\label{xnor:xnorconstraint}\hypertarget{xnor:xnorconstraint}{}
\begin{notedef}
    \texttt{xnor}$(b_1,b_2)$ states that the 0-1 variables in arguments take same value:
$$ (b_1=1 \land b_2=1) \lor (b_1=0 \land b_2=0)$$
\end{notedef}

\begin{itemize}
    \item \textbf{API} : \mylst{xnor(IntegerVariable b1, IntegerVariable b2)}
	\item \textbf{return type} : \texttt{Constraint}
	\item \textbf{options} : \emph{n/a}
	\item \textbf{favorite domain} : \emph{n/a}
\end{itemize}

\textbf{Examples:}
\lstinputlisting{java/cxnor.j2t}
%\part{xor}
\label{xor}
\hypertarget{xor}{}

\section{xor (constraint)}\label{xor:xorconstraint}\hypertarget{xor:xorconstraint}{}
\begin{notedef}
    \texttt{xor}$(b_1,b_2)$ states that the 0-1 variables in arguments take distinct value:
$$ (b_1=1 \land b_2=0) \lor (b_1=0 \land b_2=1)$$
\end{notedef}

\begin{itemize}
    \item \textbf{API} : \mylst{xor(IntegerVariable... b)}
	\item \textbf{return type} : \texttt{Constraint}
	\item \textbf{options} : \emph{n/a}
	\item \textbf{favorite domain} : \emph{n/a}
\end{itemize}

\textbf{Examples:}
\lstinputlisting{java/cxor.j2t}


\part{Tutorials}\label{ch:tut}\hypertarget{ch:tut}{}
If you look for an easy step-by-step program in CHOCO, \hyperlink{gettingstarted}{getting\ started} is for you! It introduces to basic concepts of a CHOCO program (Model, Solver, variables and constraints...).

This part also presents a collection of \hyperlink{exercises}{exercises} with their \hyperlink{solutions}{solutions}. It covers simple to advanced uses of CHOCO.

\emph{See also old pages:} \url{http://choco.sourceforge.net/tut\_expl.html}

\input{chapters/getting_started.tex}
\input{chapters/exercises.tex}
%!TEX root = ../content-tut.tex
%\part{solutions}
\label{solutions}
\hypertarget{solutions}{}

\chapter{Solutions}\label{solutions:solutions}\hypertarget{solutions:solutions}{}

\section{I'm new to CP}\label{solutions:i'mnewtocp}\hypertarget{solutions:i'mnewtocp}{}

\subsection{Solution of Exercise 1.1 (A soft start)}\label{solutions:solutionofexercise1.1}\hypertarget{solutions:solutionofexercise1.1}{}
(\hyperlink{exercises:exercise1.1}{Problem})

\noindent\emph{\textbf{Question 1}: describe the constraint network related to code}

The model is defined as :
\begin{itemize}
	\item $V = \{x_1, x_2, x_3\}$: the set of variables,
	\item $D = \{[0,5], [0,5], [0,5]\}$: the set of domain
	\item $C = \{x_1>x_2, x_1\neq x_3, x_2>x_3\}$: the set of constraints.
\end{itemize}

\noindent\emph{\textbf{Question 2}: give the variable domains after constraint propagation.}

\begin{itemize}
	\item From $x_1 = [0,5]$ and $x_2 = [0,5]$ and $x_1>x_2$, we can deduce tha : the domain of $x_1$ can be reduce to $[1,5]$ and the domain of $x_2$ can be reduce to $[0,4]$.
	\item Then, from $x_2 = [0,4]$ and $x_3 = [0,5]$ and $x_2>x_3$, we can deduce that : the domain of $x_2$ can be reduce to $[1,4]$ and the domain of $x_3$ can be reduce to $[0,3]$.
	\item Then, from $x_1 = [1,5]$ and $x_2 = [1,4]$ and $x_1>x_2$, we can deduce that : the domain of $x_1$ can be reduce to $[2,5]$.
\end{itemize}

We cannot deduce anything else, so we have reached a \textbf{fix point}, and here is the domain of each variables:
$$x_{1} : [2,5],\quad x_{2} : [1,4],\quad x_{3} : [0,3].$$


\subsection{Solution of Exercise 1.2 (DONALD + GERALD = ROBERT)}\label{solutions:solutionofexercise1.2}\hypertarget{solutions:solutionofexercise1.2}{}

(\hyperlink{exercises:exercise1.2}{Problem})

\begin{lstlisting}
  // Build model
  Model model = new CPModel();
  
  // Declare every letter as a variable
  IntegerVariable d = makeIntVar("d", 0, 9, Options.V_ENUM);
  IntegerVariable o = makeIntVar("o", 0, 9, Options.V_ENUM);
  IntegerVariable n = makeIntVar("n", 0, 9, Options.V_ENUM);
  IntegerVariable a = makeIntVar("a", 0, 9, Options.V_ENUM);
  IntegerVariable l = makeIntVar("l", 0, 9, Options.V_ENUM);
  IntegerVariable g = makeIntVar("g", 0, 9, Options.V_ENUM);
  IntegerVariable e = makeIntVar("e", 0, 9, Options.V_ENUM);
  IntegerVariable r = makeIntVar("r", 0, 9, Options.V_ENUM);
  IntegerVariable b = makeIntVar("b", 0, 9, Options.V_ENUM);
  IntegerVariable t = makeIntVar("t", 0, 9, Options.V_ENUM);
  
  // Declare every name as a variable  
  IntegerVariable donald = makeIntVar("donald", 0, 1000000,Options.V_BOUND);
  IntegerVariable gerald = makeIntVar("gerald", 0, 1000000,Options.V_BOUND);
  IntegerVariable robert = makeIntVar("robert", 0, 1000000,Options.V_BOUND);
  
  // Array of coefficients
  int[] c = new int[]{100000, 10000, 1000, 100, 10, 1}; 
  
  // Declare every combination of letter as an integer expression
  IntegerExpressionVariable donaldLetters = scalar(new IntegerVariable[]{d,o,n,a,l,d}, c);
  IntegerExpressionVariable geraldLetters = scalar(new IntegerVariable[]{g,e,r,a,l,d}, c);
  IntegerExpressionVariable robertLetters = scalar(new IntegerVariable[]{r,o,b,e,r,t}, c);
  
  // Add equality between name and letters combination
  model.addConstraint(eq(donaldLetters, donald));
  model.addConstraint(eq(geraldLetters, gerald));
  model.addConstraint(eq(robertLetters, robert));
  // Add constraint name sum
  model.addConstraint(eq(plus(donald, gerald), robert));
  // Add constraint of all different letters.
  model.addConstraint(allDifferent(new IntegerVariable[]{d,o,n,a,l,g,e,r,b,t}));
  
  // Build a solver, read the model and solve it
  Solver s = new CPSolver();
  s.read(model);
  s.solve();
  
  // Print name value
  System.out.println("donald = " + s.getVar(donald).getVal());
  System.out.println("gerald = " + s.getVar(gerald).getVal());
  System.out.println("robert = " + s.getVar(robert).getVal());
\end{lstlisting}

\subsection{Solution of Exercise 1.3 (A famous example. . . a sudoku grid)}\label{solutions:solutionofexercise1.3}\hypertarget{solutions:solutionofexercise1.3}{}

(\hyperlink{exercises:exercise1.3}{Problem})

\noindent\emph{\textbf{Question 1}: propose a way to model the sudoku problem with difference constraints. Implement your model with choco solver.}

\begin{lstlisting}
  int n = instance.length;
  // Build Model
  Model m = new CPModel();
  
  // Build an array of integer variables
  IntegerVariable[][] rows = makeIntVarArray("rows", n, n, 1, n,Options.V_ENUM);
	
  // Not equal constraint between each case of a row
  for (int i = 0; i < n; i++) {
      for (int j = 0; j < n; j++)
          for (int k = j; k < n; k++)
              if (k != j) m.addConstraint(neq(rows[i][j], rows[i][k]));
  }
                  
  // Not equal constraint between each case of a column
  for (int j = 0; j < n; j++) {
      for (int i = 0; i < n; i++)
          for (int k = 0; k < n; k++)
              if (k != i)  m.addConstraint(neq(rows[i][j], rows[k][j]));
  }

  // Not equal constraint between each case of a sub region
  for (int ci = 0; ci < n; ci += 3) {
      for (int cj = 0; cj < n; cj += 3)
          // Extraction of disequality of a sub region
          for (int i = ci; i < ci + 3; i++)
              for (int j = cj; j < cj + 3; j++)
                  for (int k = ci; k < ci + 3; k++)
                      for (int l = cj; l < cj + 3; l++)
                          if (k != i || l != j) m.addConstraint(neq(rows[i][j], rows[k][l]));
  }
	
  //...
	
  // Call solver
  Solver s = new CPSolver();
  s.read(m);
  CPSolver.setVerbosity(CPSolver.SOLUTION);
  s.solve();
  CPSolver.flushLogs();
  printGrid(rows, s);
\end{lstlisting}

\noindent\emph{\textbf{Question 2}: which global constraint can be used to model such a problem ? Modify your code to use this constraint.}

The \emph{allDifferent} constraint can be used to remplace every disequality constraint on the first Sudoku model. It improves the efficient of the model and make it more ``readable''.

\begin{lstlisting}
  // Build model
  Model m = new CPModel();
  // Declare variables
  IntegerVariable[][] cols = new IntegerVariable[n][n];
  IntegerVariable[][] rows = makeIntVarArray("rows", n, n, 1, n,Options.V_ENUM);
  
  // Channeling between rows and columns
  for (int i = 0; i < n; i++) {
      for (int j = 0; j < n; j++)
          cols[i][j] = rows[j][i];
  }
	
  // Add alldifferent constraint
  for (int i = 0; i < n; i++) {
      m.addConstraint(allDifferent(cols[i]));
      m.addConstraint(allDifferent(rows[i]));
  }

  // Define sub regions
  IntegerVariable[][] carres = new IntegerVariable[n][n];
  for (int i = 0; i < 3; i++) {
      for (int j = 0; j < 3; j++)
          for (int k = 0; k < 3; k++)
              carres[j + k * 3][i] = rows[0 + k * 3][i + j * 3];
              carres[j + k * 3][i + 3] = rows[1 + k * 3][i + j * 3];
              carres[j + k * 3][i + 6] = rows[2 + k * 3][i + j * 3];
  }
	
  // Add alldifferent on sub regions
  for (int i = 0; i < n; i++) {
      Constraint c = allDifferent(carres[i]);
      m.addConstraint(c);
  }
  
  //...
	
  // Call solver
  Solver s = new CPSolver();
  s.read(m);
  CPSolver.setVerbosity(CPSolver.SOLUTION);
  s.solve();
  printGrid(rows, s);
\end{lstlisting} 

\noindent\emph{\textbf{Question 3}: Test for both model the initial propagation step (use choco} \texttt{propagate()} \emph{method). What can be noticed ? What is the point in using global constraints ?}

The sudoku problem can be solved just with the propagation. \todo{FIXME explanation.}
The global constraint provides a more efficient filter algorithm, due to more complex deduction.

\subsection{Solution of Exercise 1.4 (The knapsack problem)}\label{solutions:solutionofexercise1.4}\hypertarget{solutions:solutionofexercise1.4}{}

(\hyperlink{exercises:exercise1.4}{Problem})

\noindent\emph{\textbf{Question 1} : In the first place, we will not consider the idea of maximizing the energetic value. Try to find a satisfying solution by modelling and implementing the problem within choco.}

\begin{lstlisting}
	Model m = new CPModel();
	
	obj1 = makeIntVar("obj1", 0, 5,Options.V_ENUM);
	obj2 = makeIntVar("obj2", 0, 7,Options.V_ENUM);
	obj3 = makeIntVar("obj3", 0, 10,Options.V_ENUM);
	c = makeIntVar("cost", 1, 1000000,Options.V_BOUND);
	
	int capacity = 34;
	int[] volumes = new int[]{7, 5, 3};
	int[] energy = new int[]{6, 4, 2};
	
	m.addConstraint(leq(scalar(volumes, new IntegerVariable[]{obj1, obj2, obj3}), capacity));
	m.addConstraint(eq(scalar(energy, new IntegerVariable[]{obj1, obj2, obj3}), c));
	
	Solver s = new CPSolver();
	s.read(m);

	s.solve();
	
	System.out.println("("+s.getVar(obj1).getVal()+","+s.getVar(obj2).getVal()+","
                       +s.getVar(obj3).getVal()+") cost = "+ s.getVar(c).getVal());
\end{lstlisting}

\noindent\emph{\textbf{Question 2} : Find and use the choco method to maximise the energetic value of the knapsack.}
Replace \mylst{s.solve()} by:
\begin{lstlisting}
	s.maximize(s.getVar(c), false);
\end{lstlisting}

\noindent\emph{\textbf{Question 3} : Propose a Value selector heuristic to improve the efficiency of the model.}

It can be improved using the following value selector strategy. It iterates over decreasing values of every domain variables: 
\begin{lstlisting}
  s.setValIntIterator(new DecreasingDomain());
\end{lstlisting}

\subsection{Solution of Exercise 1.5 (The n-queens problem)}\label{solutions:solutionofexercise1.5}\hypertarget{solutions:solutionofexercise1.5}{}
(\hyperlink{exercises:exercise1.5}{Problem})

\noindent\emph{\textbf{Question 1} : propose and implement a model based on one} $L_{i}$ \emph{variable for every row...}
\begin{lstlisting}
  Model m = new CPModel();
  
  IntegerVariable[] queens = new IntegerVariable[n];
  for (int i = 0; i < n; i++) {
      queens[i] = makeIntVar("Q" + i, 1, n,Options.V_ENUM);
  }
	
  for (int i = 0; i < n; i++) {
      for (int j = i + 1; j < n; j++) {
          int k = j - i;
          m.addConstraint(neq(queens[i], queens[j]));
          m.addConstraint(neq(queens[i], plus(queens[j], k)));  // diagonal
          m.addConstraint(neq(queens[i], minus(queens[j], k))); // diagonal
      }
  }
	
  Solver s = new CPSolver();
  s.read(m);
  CPSolver.setVerbosity(CPSolver.SOLUTION);
  int timeLimit = 60000;
  s.setTimeLimit(timeLimit);
  s.solve();
  CPSolver.flushLogs();
\end{lstlisting}

\noindent\emph{\textbf{Question 2} : Add a redundant model by considering variable on the columns ($C_i$). Continue to use simple difference constraints.}

\begin{lstlisting}
  Model m = new CPModel();
	
  IntegerVariable[] queens = new IntegerVariable[n];
  IntegerVariable[] queensdual = new IntegerVariable[n];
  for (int i = 0; i < n; i++) {
      queens[i] = makeIntVar("Q" + i, 1, n,Options.V_ENUM);
      queensdual[i] = makeIntVar("QD" + i, 1, n,Options.V_ENUM);
  }
	
  for (int i = 0; i < n; i++) {
      for (int j = i + 1; j < n; j++) {
          int k = j - i;
          m.addConstraint(neq(queens[i], queens[j]));
          m.addConstraint(neq(queens[i], plus(queens[j], k)));  // diagonal
          m.addConstraint(neq(queens[i], minus(queens[j], k))); // diagonal
      }
  }

  for (int i = 0; i < n; i++) {
      for (int j = i + 1; j < n; j++) {
          int k = j - i;
          m.addConstraint(neq(queensdual[i], queensdual[j]));
          m.addConstraint(neq(queensdual[i], plus(queensdual[j], k)));  // diagonal
          m.addConstraint(neq(queensdual[i], minus(queensdual[j], k))); // diagonal
      }
  }
  m.addConstraint(inverseChanneling(queens, queensdual));
  
  Solver s = new CPSolver();
  s.read(m);
  
  s.setVarIntSelector(new MinDomain(s,s.getVar(queens)));
  
  CPSolver.setVerbosity(CPSolver.SOLUTION);
  s.setLoggingMaxDepth(50);
  int timeLimit = 60000;
  s.setTimeLimit(timeLimit);
  s.solve();
  CPSolver.flushLogs();
\end{lstlisting}

\noindent\emph{\textbf{Question 3} : Compare the number of nodes created to find the solutions with both models. How can you explain such a difference ?}

The channeling permit to reduce more nodes from the tree search... \todo{FIXME}

\noindent\emph{\textbf{Question 4} : Add to the previous implemented model the following heuristics,
\begin{itemize}
	\item Select first the line variable ($L_i$) which has the smallest domain ;
	\item Select the value $j\in L_i$ so that the associated column variable $C_j$ has the smallest domain.
\end{itemize}
Again, compare both approaches in term of nodes number and solving time to find ONE solution for $n = 75, 90, 95, 105$.}

Add the following lines to your program (after the reading of the model):
\begin{lstlisting}
	s.setVarIntSelector(new MinDomain(s,s.getVar(queens)));
	s.setValIntSelector(new NQueenValueSelector(s.getVar(queensdual)));
\end{lstlisting}
The variable selector strategy (\texttt{MinDomain}) already exists in Choco. It iterates over variables given and returns the variable ordering by creasing domain size. 
The value selector strategy has to be created as follow:
\begin{lstlisting}
  public class NQueenValueSelector implements ValSelector {
	
      // Column variable
      protected IntDomainVar[] dualVar;
	
      // Constructor of the value selector, 
      public NQueenValueSelector(IntDomainVar[] cols) {
          this.dualVar = cols;
      }
	
      // Returns the "best val" that is the smallest column domain size OR -1
      // (-1 is not in the domain of the variables)
      public int getBestVal(IntDomainVar intDomainVar) {
          int minValue = 10000;
          int v0 = -1;
          IntIterator it = intDomainVar.getDomain().getIterator();
          while (it.hasNext()){
              int i = it.next();
              int val = dualVar[i - 1].getDomainSize();
              if (val < minValue)  {
                  minValue = val;
                  v0 = i;
              }
          }
          return v0;
      }
  }
\end{lstlisting}

\noindent\emph{\textbf{Question 5} : what changes are caused by the use of the global constraint \textbf{alldifferent} ?}

\begin{lstlisting}
  Model m = new CPModel();
	
  IntegerVariable[] queens = new IntegerVariable[n];
  IntegerVariable[] queensdual = new IntegerVariable[n];
  IntegerVariable[] diag1 = new IntegerVariable[n];
  IntegerVariable[] diag2 = new IntegerVariable[n];
  IntegerVariable[] diag1dual = new IntegerVariable[n];
  IntegerVariable[] diag2dual = new IntegerVariable[n];
  for (int i = 0; i < n; i++) {
      queens[i] = makeIntVar("Q" + i, 1, n,Options.V_ENUM);
      queensdual[i] = makeIntVar("QD" + i, 1, n,Options.V_ENUM);
      diag1[i] = makeIntVar("D1" + i, 1, 2 * n,Options.V_ENUM);
      diag2[i] = makeIntVar("D2" + i, -n, n,Options.V_ENUM);
      diag1dual[i] = makeIntVar("D1" + i, 1, 2 * n,Options.V_ENUM);
      diag2dual[i] = makeIntVar("D2" + i, -n, n,Options.V_ENUM);
  }
	
  m.addConstraint(allDifferent(queens));
  m.addConstraint(allDifferent(queensdual));
  for (int i = 0; i < n; i++) {
      m.addConstraint(eq(diag1[i], plus(queens[i], i)));
      m.addConstraint(eq(diag2[i], minus(queens[i], i)));
      m.addConstraint(eq(diag1dual[i], plus(queensdual[i], i)));
      m.addConstraint(eq(diag2dual[i], minus(queensdual[i], i)));
  }
  m.addConstraint(inverseChanneling(queens,queensdual));
	
  m.addConstraint(allDifferent(diag1));
  m.addConstraint(allDifferent(diag2));
  m.addConstraint(allDifferent(diag1dual));
  m.addConstraint(allDifferent(diag2dual));
	
  Solver s = new CPSolver();
  s.read(m);
	
  s.setVarIntSelector(new MinDomain(s,s.getVar(queens)));
  s.setValIntSelector(new NQueenValueSelector(s.getVar(queensdual)));
	
  CPSolver.setVerbosity(CPSolver.SOLUTION);
  int timeLimit = 60000;
  s.setTimeLimit(timeLimit);
  s.solve();
  CPSolver.flushLogs();
\end{lstlisting}

\section{I know CP}\label{solutions:iknowcp}\hypertarget{solutions:iknowcp}{}

\subsection{Solution of Exercise 2.1 (Bin packing, cumulative and search strategies)}\label{solutions:solutionofexercise2.1}\hypertarget{solutions:solutionofexercise2.1}{}
(\hyperlink{exercises:exercise2.1}{Problem})

\subsection{Solution of Exercise 2.2 (Social golfer)}\label{solutions:solutionofexercise2.2}\hypertarget{solutions:solutionofexercise2.2}{}
(\hyperlink{exercises:exercise2.2}{Problem})

\subsection{Solution of Exercise 2.3 (Golomb rule)}\label{solutions:solutionofexercise2.3}\hypertarget{solutions:solutionofexercise2.3}{}

\emph{under development}

(\hyperlink{exercises:exercise2.3}{Problem})

\section{I know CP and Choco2.0}\label{solutions:iknowcpandchoco2.0}\hypertarget{solutions:iknowcpandchoco2.0}{}

\subsection{Solution of Exercise 3.1 (Hamiltonian Cycle Problem Traveling Salesman Problem)}\label{solutions:solutionofexercise3.1}\hypertarget{solutions:solutionofexercise3.1}{}

(\hyperlink{exercises:exercise3.1}{Problem})

\subsection{Solution of Exercise 3.2 (Shop scheduling)}\label{solutions:solutionofexercise3.2}\hypertarget{solutions:solutionofexercise3.2}{}

(\hyperlink{exercises:exercise3.2}{Problem})

\emph{under development}


\part{Extras}\label{ch:extra}\hypertarget{ch:extra}{}
\input{chapters/choco_and_visu.tex}
\input{chapters/sudoku_and_cp.tex}
