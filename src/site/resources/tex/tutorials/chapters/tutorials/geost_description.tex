%!TEX root = ../../content-tut.tex
%\part{geost description}
\label{geostdescription}
\hypertarget{geostdescription}{}


\section{Placement and use of the Geost constraint}\label{geostdescription:placementanduseofthegeostconstraint}\hypertarget{geostdescription:placementanduseofthegeostconstraint}{}

The global constraint \texttt{\bf geost}($k,O,S,C$) handles in a generic way a variety of geometrical constraints $C$ in space and time between polymorphic $k\in\N$ dimensional objects $O$, each of which taking a shape among a set of shapes $S$ during a given time interval and at a given position in space. Each shape from $S$ is defined as a finite set of shifted boxes, where each shifted box is described by a box in a $k$-dimensional space at the given offset with the given sizes.

More precisely a \emph{shifted box} $s$= \emph{shape(sid,t[],l[])} is an entity defined by a shape id \emph{sid}, an shift offset \emph{s.t[d]}, $0\le d < k$, and a size \emph{s.l[d]}$>0$, $0\le d<k$. All attributes of a shifted box are integer values. Then, a \emph{shape} from $S$ is a collection of shifted boxes sharing all the same shape id. Note that the shifted boxes associated with a given shape may or may not overlap. This sometimes allows a drastic reduction in the number of shifted boxes needed to describe a shape.
Each \emph{object} \emph{o= object( id, sid,x[], start, duration,end)} from $O$ is an entity defined by a unique object id \emph{o.id} (an integer), a shape id \emph{o.sid}, an origin \emph{o.x[d]}, $0\le d<k$, a starting time \emph{o.start}, a duration \emph{o.duration}$>0$, and a finishingxs time \emph{o.end}.

All attributes \emph{sid, x[0],x[1],...,x[k-1], start, duration, end} correspond to domain variables. Typical constraints from the
list of constraints $C$ are for instance the fact that a given subset of objects from $O$ do not pairwise overlap.
Constraints of the list of constraints $C$ have always two first arguments $A_i$ and $O_i$ (followed by possibly some additional arguments) which respectively specify:
\begin{itemize}
	\item A list of dimensions (integers between 0 and k-1), or attributes of the objects of $O$ the constraint considers.
	\item A list of identifiers of the objects to which the constraint applies.
\end{itemize}

\subsection{Example and way to implement it}\label{geostdescription:exampleandwaytoimplementit}\hypertarget{geostdescription:exampleandwaytoimplementit}{}

We will explain how to use \emph{geost} although a 2D example. Consider we have 3 objects $o_0, o_1, o_2$ to place them inside a box $B$ (3x4) such that they don't overlap (see Figure bellow). The first object $o_0$ has two potential shapes while $o_1$ and $o_2$ have one shape. Given that the placement of the objects should be totally inside $B$ this means that the domain of the origins of objects are as follows (we start from 0 this means that the placement space is from 0 to 2 on $x$ and from 0 to 3 on $y$:
\begin{itemize}
	\item $o_0$: $x$ in 0..1, $y$ in 0..1,
	\item $o_1$: $x$ in 0..1, $y$ in 0..1,
	\item $o_2$: $x$ in 0..1, $y$ in 0..3.
\end{itemize}

\insertGraphique{.9\linewidth}{media/exp_geost.png}{Geost objects and shapes}

We describe now how to solve this problem by using Choco.

\subsubsection{Build a CP model.}\label{geostdescription:buildacpmodel}\hypertarget{geostdescription:buildacpmodel}{}
To begin the implementation we build a CP model:
\begin{lstlisting}
  Model m = new CPModel();
\end{lstlisting}
\subsubsection{Set the Dimension.}\label{geostdescription:setthedimension}\hypertarget{geostdescription:setthedimension}{}
Then we first need to specify the dimension k we are working in. This is done by assigning the dimension to a local variable that we will use later:
\begin{lstlisting}
  int dim = 2;
\end{lstlisting}
\subsubsection{Create the Objects.}\label{geostdescription:createtheobjects}\hypertarget{geostdescription:createtheobjects}{}
Then we start by creating the objects and store them in a list as such:
\begin{lstlisting}
  List<GeostObject> objects = new ArrayList<GeostObject>();
\end{lstlisting}
Now we create the first object $o_0$ by creating all its attributes.
\begin{lstlisting}
  int objectId = 0; // object id
  IntegerVariable shapeId = Choco.makeIntVar("sid", 0, 1); // shape id (2 possible values)
  IntegerVariable coords[] = new IntegerVariable[dim]; // coordinates of the origin 
  coords[0] = Choco.makeIntVar("x", 0, 1);  
  coords[1] =  Choco.makeIntVar("y", 0, 1);
\end{lstlisting}
We need to specify 3 more Integer Domain Variables representing the temporal attributes (start, duration and end), which for the current implementation of \texttt{geost} are not working, however we need to give them dummy values. 
\begin{lstlisting}
  IntegerVariable start = Choco.makeIntVar("start", 0, 0);
  IntegerVariable duration = Choco.makeIntVar("duration", 1, 1);
  IntegerVariable end = Choco.makeIntVar("end", 1, 1);
\end{lstlisting}
Finally we are ready to add the object 0 to our \emph{objects} list:
\begin{lstlisting}
  objects.add(new GeostObject(dim, objectId, shapeId, coords, start, duration, end));
\end{lstlisting}
Now we do the same for the other  object $o_1$  and  $o_2$ and add them to our \emph{objects} .


\subsubsection{Create the Shifted Boxes.}\label{geostdescription:createtheshiftedboxes}\hypertarget{geostdescription:createtheshiftedboxes}{}
To create the shapes and their shifted boxes we create the shifted boxes and associate them with the corresponding shapeId. This is done as follows, first we create a list called \emph{sb} for example 
\begin{lstlisting}
List<ShiftedBox> sb = new ArrayList<ShiftedBox> ();
\end{lstlisting}
To create the shifted boxes for the shape 0 (that corresponds to $o_0$), we start by the first shifted box by creating the sid and 2 arrays one to specify the offset of the box in each dimension and one for the size of the box in each dimension:
\begin{lstlisting}
  int sid = 0; 
  int[] offset = {0,0};
  int[] sizes = {1,3};
\end{lstlisting} 
Now we add our shiftedbox to the \emph{sb} list:
\begin{lstlisting}
  sb.add(new ShiftedBox(sid, offset, sizes));
\end{lstlisting} 
We do the same with second shifted box:
\begin{lstlisting}
  sb.add(new ShiftedBox(0, new int[]{0,0}, new int[]{2,1}));
\end{lstlisting} 
By the same way we create the shifted boxes corresponding to second shape $S_1$:
\begin{lstlisting}
  sb.add(new ShiftedBox(1, new int[]{0,0}, new int[]{2,1}));
  sb.add(new ShiftedBox(1, new int[]{1,0}, new int[]{1,3}));
\end{lstlisting}
and the third shape $S_2$ consisting of three shifted boxes:
\begin{lstlisting}
  sb.add(new ShiftedBox(2, new int[]{0,0}, new int[]{2,1})) ;
  sb.add(new ShiftedBox(2, new int[]{1,0}, new int[]{1,3})); 
  sb.add(new ShiftedBox(2, new int[]{0,2}, new int[]{2,1}));
\end{lstlisting}
and finally the last shape $S_3$
\begin{lstlisting}
  sb.add(new ShiftedBox(3, new int[]{0,0}, new int[]{2,1}));
\end{lstlisting}

\subsubsection{Create the constraints.}\label{geostdescription:createtheconstraints}\hypertarget{geostdescription:createtheconstraints}{}
First we create a Vector called ectr that will contain the external constraints. 
\begin{lstlisting}
  List <ExternalConstraint> ectr = new ArrayList <ExternalConstraint>();
\end{lstlisting}
In order to create the non-overlapping constraint we first create an array containing all the dimensions the constraint will be active in (in our example it is all dimensions) and lets name this array \emph{ectrDim} and a list of objects \emph{objOfEctr} that this constraint will apply to (in our example it is all objects).

\begin{note}
Note that in the current implementation of \texttt{geost} only the non-overlapping constraint is available. Moreover, 
\emph{ectrDim} should contain all dimensions and \emph{objOfEctr} should contain all the objects, i.e. the non-overlapping constraint applies to all the objects in all dimensions.
\end{note}
 
After that we add the constraint to a list \emph{ectr} that contains all the constraints we want to add.
The code for these steps is as follows: 
\begin{lstlisting}
  int[] ectrDim = new int[dim]; 
  for(i = 0; i < dim; i ++)  
      ectrDim[i] = i; 
  int[] objOfEctr = new int[3]; 
  for(i = 0; i < 3; i ++) 
      objOfEctr[i] = objects.get(i).getObjectId();
\end{lstlisting}
All we need to do now is create the non-overlapping constraint and add it to the \emph{ectr} list
that holds all the constraints. this is done as follows:
\begin{lstlisting}
  //Constants.NON_OVERLAPPING indicates the id of the non-overlapping constraint
  NonOverlapping n = new NonOverlapping(Constants.NON_OVERLAPPING, ectrDim, objOfEctr);
  ectr.add(n);
\end{lstlisting}

\subsubsection{Create the \texttt{geost} constraint and add it to the model.}\label{geostdescription:createthegeostconstraintandaddittothemodel}\hypertarget{geostdescription:createthegeostconstraintandaddittothemodel}{}
\begin{lstlisting}
  Constraint geost = Choco.geost(dim, objects, sb, ectr);
  m.addConstraint(geost);
\end{lstlisting}
	
\subsubsection{Solve the problem.}\label{geostdescription:solvetheproblem}\hypertarget{geostdescription:solvetheproblem}{}
\begin{lstlisting}
  Solver s = new CPSolver();
  s.read(m);
  s.solve();
\end{lstlisting}

%The full java code can be found here: \href{media/zip/geostexp.java}{geostexp.java}

\subsection{Support for Greedy Assignment within the geost Kernel}\label{geostdescription:supportforgreedyassignmentwithinthegeostkernel}\hypertarget{geostdescription:supportforgreedyassignmentwithinthegeostkernel}{}

\subsubsection{Motivation and functionality description.}\label{geostdescription:motivationandfunctionalitydescription}\hypertarget{geostdescription:motivationandfunctionalitydescription}{}
Since, for performance reasons, the \texttt{geost} kernel offers a mode where he tries to fix all objects during one single propagation step, we provide a way to specify a preferred order on how to fix all the objects in one single propagation step. This is achieved by:
\begin{itemize}
	\item Fixing the objects according to the order they were passed to the \texttt{geost} kernel.
	\item When considering one object, fixing its shape variable as well as its coordinates:
	\begin{itemize}
		\item According to an order on these variables that can be explicitly specified.
		\item A value to assign that can either be the smallest or the largest value, also specified by the user.
	\end{itemize}
\end{itemize}

\begin{note}
Note that the use of the greedy mode assumes that no other constraint is present in the problem.
\end{note}

	This is encoded by a term that has exactly the same structure as the term associated to an object of  \texttt{geost}. The only difference consists of the fact that a variable is replaced by an expression \_ (\emph{The character \_ denotes the fact that the corresponding attribute is irrelevant, since for instance, we know that it is always fixed}), $\min(I)$ (respectively, $\max(I)$), where $I$ is a strictly positive integer. The meaning is that the corresponding variable should be fixed to its minimum (respectively maximum value) in the order $I$.   We can in fact give a list of lists  $v_1,v_2,\ldots,v_p$ in order to specify how to fix objects $o_{(1+pa)},o_{(2+pa)},...,o_{(p+pa)}$.

This is illustrated by Figure bellow: for instance, Part(\emph{I}) specifies that we alternatively:
\begin{itemize}
	\item fix the shape variable of an object to its maximum value (i.e., by using max(1) ), fix the $x$-coordinate of an object to its its minimum value (i.e., by using min(2)), fix the $y$-coordinate of an object to its its minimum value (i.e., by using min(3)) and
	\item fix the shape variable of an object to its maximum value (i.e., by using max(1)), fix the $x$-coordinate of an object to its its maximum value (i.e., by using max(2)), fix the $y$-coordinate of an object to its its maximum value (i.e., by using max(3)).
\end{itemize}

In the example associated with Part (I) we successively fix objects $o_1, o_2, o_3, o_4, o_5, o_6$ by alternatively using strategies (1) \mylst{object(_,max(1),x[min(2),min(3)])} and (2) \mylst{object(_,max(1),x[max(2),max(3))}. 

\insertGraphique{\linewidth}{media/greedy.png}{Greedy placement}

\subsubsection{Implementation.}\label{geostdescription:implementation}\hypertarget{geostdescription:implementation}{}
The greedy algorithm for fixing an object o is controlled by a list \emph{v} of length \emph{k+1} such that:
\begin{itemize}
	\item The shape variable \emph{o.sid} should be set to its minimum possible value if \emph{v}[0]$<0$, and to its maximum possible value otherwise.
	\item abs(\emph{v}[1])-2 is the most significant dimension (the one that varies the slowest) during the sweep. The values are tried in ascending order if \emph{v}[1]$<0$, and in descending order otherwise.
	\item abs(\emph{v}[2])-2 is the next most significant dimension, and its sign indicates the value order, and so on.
\end{itemize}

For example, a term \mylst{object(_,min(1),[max(3),min(4),max(2)])} is encoded as the list $[-1,4,2,-3]$.
	                                      
\subsubsection{Second example.}\label{geostdescription:secondexample}\hypertarget{geostdescription:secondexample}{}
We will explain although a 2D example how to take into account of greedy mode. Consider we have 12 identical objects $o_0,o_1,\ldots,o_{11}$ having 4 potential shapes and we want to place them in a box $B$ (7x6) (see Figure bellow). Given that the placement of the objects should be totally inside B this means that the domain of the origins of objects are as follows $x\in [0,5]$, $y\in [0,4]$. Moreover, suppose that we want use two strategies when greedy algorithm is called: the term \mylst{object(_,min(1),[min(2),min(3)])} for objects $o_0, o_2, o_4, o_6, o_8, o_{10}$, and the term \mylst{object(_,max(1),[max(2),max(3)])} for objects $o_1, o_3, o_5, o_7, o_9, o_{11}$. These strategies are encoded respectively as $[-1,-2,-3]$ and $[1, 2, 3]$.

\insertGraphique{\linewidth}{media/exp2_geost.png}{A second Geost instance.}

We comment only the additional step w.r.t. the preceding example. In fact we just need to create the list of controlling lists before creating the geost constraint. Each controlling list is an array:
\begin{lstlisting}
  List <int[]> ctrlVs = new ArrayList<int[]>();
  int[] v0 = {-1, -2, -3};
  int[] v1 = {1, 2, 3};
  ctrlVs.add(v0);
  ctrlVs.add(v1);
\end{lstlisting}
and then create the \texttt{geost} constraint, by adding the list of controlling vectots as an another argument, as follows:
\begin{lstlisting}
  Constraint geost = Choco.geost(dim, objects, sb, ectr,ctrlVs);
  m.addConstraint(geost);
\end{lstlisting}

%The full java code can be reached \href{media/zip/greddyexp.java}{here}.
%\lstinputlisting{media/zip/greedyexp.java}
%TODO : add code


The placement obtained using the preceding strategies is displayed in the following figure (right side).

\insertGraphique{\linewidth}{media/solution_exp2.png}{A solution placement}

