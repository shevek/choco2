%!TEX root = ./choco-tut.tex
\addcontentsline{toc}{chapter}{Preface}
\chapter*{Preface}
Choco is a java library for constraint satisfaction problems (CSP) and constraint programming (CP). 
Choco is propagation-based constraint solver with a variable-centered propagation loop and backtrackable structures.
Choco is an open-source software, distributed under a \textbf{BSD licence} and hosted by \href{http://sourceforge.net/projects/choco/}{sourceforge.net}.

For any informations visit \url{http://choco.mines-nantes.fr} and read the documentation.
\bigskip

\noindent This document is organized as follows:
\begin{itemize}
\item \hyperlink{ch:tut}{Tutorials} provides a \hyperlink{gettingstarted:gettingstarted:welcometochoco}{fast how-to} write a Choco program, a detailed example of a \hyperlink{gettingstarted:firstexample:magicsquare}{simple program}, and several \hyperlink{exercises}{exercises} with their \hyperlink{solutions}{solutions},  several guided examples  and some examples of \hyperlink{doc:applications}{Applications}.
\end{itemize}

\part{Tutorials}\label{ch:tut}\hypertarget{ch:tut}{}
If you look for an easy step-by-step program in CHOCO, \hyperlink{gettingstarted}{getting\ started} is for you! It introduces to basic concepts of a CHOCO program (Model, Solver, variables and constraints...).

This part also presents a collection of \hyperlink{exercises}{exercises} with their \hyperlink{solutions}{solutions} and guided examples. It covers simple to advanced uses of CHOCO.

%\emph{See also old pages:} \url{http://choco.sourceforge.net/tut\_expl.html}

%!TEX root = ../content-tut.tex
%\part{getting started}
\label{gettingstarted}
\hypertarget{gettingstarted}{}

\chapter{Getting started: welcome to Choco}\label{gettingstarted:gettingstarted:welcometochoco}\hypertarget{gettingstarted:gettingstarted:welcometochoco}{}
%This introduction covers the basics of writing a program in Choco

%Choco is a java library for constraint satisfaction problems (CSP), constraint programming (CP) and explanation-based constraint solving (e-CP). It is built on a event-based propagation mechanism with backtrackable structures. 

\section{Before starting}\label{gettingstarted:beforestarting}\hypertarget{gettingstarted:beforestarting}{}

Before doing anything, you have to be sure that 
\begin{itemize}
	\item you have at least \href{http://java.sun.com/javase/6/}{Java6} installed on your environment.
	\item you have a IDE (like \href{http://www.jetbrains.com/idea/}{IntelliJ IDEA} or \href{http://www.eclipse.org/}{Eclipse}).
\end{itemize}

To install Java6 or your IDE, please refer to its specific documentation. We now assume that you have the previously defined environment.

You need to create a \textbf{New Project...} on your favorite IDE (\href{http://www.jetbrains.com/idea/training/demos.html}{create a new project on IntelliJ}, \href{https://eclipse-tutorial.dev.java.net/eclipse-tutorial/part1.html}{create a new project on Eclipse}). Our project name is \emph{ChocoProgram}.
Create a new class, named \emph{MyFirstChocoProgram}, with a main method.
\begin{lstlisting}
	public class MyFirstChocoProgram {
	
	    public static void main(String[] args) {
	        
	    }
	}
\end{lstlisting}

\section{Download Choco}\label{gettingstarted:downloadchoco}\hypertarget{gettingstarted:downloadchoco}{}
Now, before doing anything else, you need to download the last stable version of Choco. See the \href{http://www.emn.fr/z-info/choco-solver/choco-download.html}{download page}. 
Once you have download choco, you need to add it to the classpath of your project.

Now you are ready to create you first Choco program.

If you want a short introduction on what is constraint programming, you can find some informations in the \href{http://www.emn.fr/z-info/choco-solver/choco-documentation.html}{Documentation of choco}.\\
When you feel ready, solve your own problem! And if you need more tries, please take a look at the \hyperlink{exercises}{exercises}. %and \hyperlink{examples}{examples}. 

\chapter{First Example: Magic square}\label{gettingstarted:firstexample:magicsquare}\hypertarget{gettingstarted:firstexample:magicsquare}{}
A simple magic square of order 3 can be seen as the ``Hello world!'' program in Choco. First of all, we need to agree on the definition of a magic square of order 3.
\href{http://en.wikipedia.org/wiki/Magic_square}{Wikipedia} tells us that :
\begin{myquote}
A \textbf{magic square} of order $n$ is an arrangement of $n^2$ numbers, usually distinct integers, in a square, such that the $n$ numbers in all rows, all columns, and both diagonals sum to the same constant. A normal magic square contains the integers from 1 to $n^2$.
\end{myquote}

So we are going to solve a problem where unknows are cells value, knowing that each cell can take its value between 1 and $n^2$, is different from the others and columns, diagonals and rows are equal to the same constant M (which is equal to $n * (n^2 + 1) / 2$).

We have the definition, let see how to add some Choco in it.

\section{First, the model}\label{gettingstarted:first,themodel}\hypertarget{gettingstarted:first,themodel}{}
To define our problem, we need to create a Model object. As we want to solve our problem with constraint programming (of course, we do), we need to create a CPModel.
\begin{lstlisting}
	//constants of the problem:
	int n = 3;
	int M = n*(n*n+1)/2;
	
	// Our model
	Model m = new CPModel();
\end{lstlisting}
These objects require to import the following classes:
\begin{lstlisting}
	import choco.cp.model.CPModel;
	import choco.kernel.model.Model;
\end{lstlisting}

At the begining, our model is empty, no problem has been defined explicitly. A model is composed of variables and constraints, and constraints link variables to each others.

\begin{itemize}
\item 
\textbf{Variables}\label{gettingstarted:variables}\hypertarget{gettingstarted:variables}{}
A variable is an object defined by a name, a type and a domain. We know that our unknowns are cells of the magic square. So:
\mylst{IntegerVariable cell = Choco.makeIntVar("aCell", 1, n*n);}
which means that \emph{aCell} is an integer variable, and its domain is defined from \emph{1} to \emph{n*n}.
But we need $n^2$ variables, so the easiest way to define them is:
\begin{lstlisting}
	IntegerVariable[][] cells = new IntegerVariable[n][n];
	for(int i = 0; i < n; i++){
	   for(int j = 0; j < n; j++){
	      cells[i][j] = Choco.makeIntVar("cell"+j, 1, n*n); 
	      m.addVariables(cells[i][j]);
	   }
	}
\end{lstlisting}
This code requires to import the following classes:
\begin{lstlisting}
	import choco.kernel.model.variables.integer.IntegerVariable;
	import choco.Choco;
\end{lstlisting}
We add each variables to our model:
\mylst{m.addVariables(cells[i][j]);}\\
Now that our variables are defined, we have to define the constraints between variables.
\item
\textbf{Constraints over the rows}\label{gettingstarted:constraintsovertherows}\hypertarget{gettingstarted:constraintsovertherows}{}
The sum of ach rows is equal to a constant $M$.
So we need a sum operator and and equality constraint. The both are provides by the \texttt{Choco.java} class.
\begin{lstlisting}
	//Constraints
	// ... over rows
	Constraint[] rows = new Constraint[n];
	for(int i = 0; i < n; i++){
	   rows[i] = Choco.eq(Choco.sum(cells[i]), M);
	}
\end{lstlisting}
This part of code requires the following import:
\begin{lstlisting}
  import choco.kernel.model.constraints.Constraint;
\end{lstlisting}
After the creation of the constraints, we need to add them to the model:
\begin{lstlisting}
  m.addConstraints(rows);
\end{lstlisting}
\item
\textbf{Constraints over the columns}\label{gettingstarted:constraintsoverthecolumns}\hypertarget{gettingstarted:constraintsoverthecolumns}{}
Now, we need to declare the equality between the sum of each column and $M$.
But, the way we have declare our variables matrix does not allow us to deal easily with it in the column case. So we create the transposed matrix (a $90^o$ rotation of the matrix) of \emph{cells}.
\begin{note}
We do not introduce new variables. We just reorder the matrix to see the \emph{column point of view}.
\end{note}
\begin{lstlisting}
	//... over columns
	// first, get the columns, with a temporary array
	IntegerVariable[][] cellsDual = new IntegerVariable[n][n];
	for(int i = 0; i < n; i++){
	   for(int j = 0; j < n; j++){
	      cellsDual[i][j] = cells[j][i];
	   }
	}
\end{lstlisting}
Now, we can declare the constraints as before:
\begin{lstlisting}
	Constraint[] cols = new Constraint[n];
	for(int i = 0; i < n; i++){
	   cols[i] = Choco.eq(Choco.sum(cellsDual[i]), M);
	}
\end{lstlisting}
And we add them to the model:
\begin{lstlisting}
  m.addConstraints(cols);
\end{lstlisting}
\item
\textbf{Constraints over the diagonals}\label{gettingstarted:constraintsoverthediagonals}\hypertarget{gettingstarted:constraintsoverthediagonals}{}
Now, we get the two diagonals array \emph{diags}, reordering the required \emph{cells} variables, like in the previous step.
\begin{lstlisting}
	//... over diagonals                                  
	IntegerVariable[][] diags = new IntegerVariable[2][n];
	for(int i = 0; i < n; i++){                           
	    diags[0][i] = cells[i][i];                        
	    diags[1][i] = cells[i][(n-1)-i];                  
	}
\end{lstlisting} 
And we add the constraints to the model (in one step this time).
\begin{lstlisting}
	m.addConstraint(Choco.eq(Choco.sum(diags[0]), M));    
	m.addConstraint(Choco.eq(Choco.sum(diags[1]), M));
\end{lstlisting}
\item
\textbf{Constraints of variables AllDifferent}\label{gettingstarted:constraintsofvariablesalldifferent}\hypertarget{gettingstarted:constraintsofvariablesalldifferent}{}
Finally, we add the AllDifferent constraints, stating that each \emph{cells} variables takes a unique value. 
One more time, we have to reorder the variables, introducing temporary array.
\begin{lstlisting}
	//All cells are differents from each other           
	IntegerVariable[] allVars = new IntegerVariable[n*n];
	for(int i = 0; i < n; i++){                          
	    for(int j = 0; j < n; j++){                      
	        allVars[i*n+j] = cells[i][j];                
	    }                                                
	}                                                    
	m.addConstraint(Choco.allDifferent(allVars));
\end{lstlisting}
\end{itemize}

\section{Then, the solver}\label{gettingstarted:then,thesolver}\hypertarget{gettingstarted:then,thesolver}{}
Our model is established, it does not require any other information, we can focus on the way to solve it.
The first step is to create a Solver;
\begin{lstlisting}
	//Our solver              
	Solver s = new CPSolver();
\end{lstlisting}
This part requires the following imports:
\begin{lstlisting}
	import choco.kernel.solver.Solver;
	import choco.cp.solver.CPSolver;
\end{lstlisting}

After that, the model and the solver have to be linked, thus the solver \emph{read} the model, to extract informations:
\begin{lstlisting}
	//read the model
	s.read(m);
\end{lstlisting}

Once it is done, we just need to solve it:
\begin{lstlisting}
	//solve the problem
	s.solve();
\end{lstlisting}
And print the information
\begin{lstlisting}
	//Print the values                                           
	for(int i = 0; i < n; i++){                                  
	    for(int j = 0; j < n; j++){                              
	        System.out.print(s.getVar(cells[i][j]).getVal()+" ");
	    }                                                        
	    System.out.println();                                    
	}
\end{lstlisting}

\section{Conclusion}\label{gettingstarted:conclusion}\hypertarget{gettingstarted:conclusion}{}
We have seen, in a few steps, how to solve a basic problem using constraint programming and Choco. Now, you are ready to solve your own problem, and if you need more tries, please take a look at the \hyperlink{exercises}{exercises}. %and \hyperlink{examples}{examples}. 

%\part{exercises}
\label{exercises}
\hypertarget{exercises}{}

\chapter{Exercises}\label{exercises:exercises}\hypertarget{exercises:exercises}{}

\section{I'm new to CP}\label{exercises:i'mnewtocp}\hypertarget{exercises:i'mnewtocp}{}

\begin{note}
\textbf{The goal of this practical work is twofold :}
\begin{itemize}
	\item \textbf{problem modelling} with the help of variables and constraints ;
	\item \textbf{mastering the syntax} of Choco in order to tackle basic problems.
\end{itemize}

\textbf{Warning}, constraint modelling should not be solver specific. That is why you are strongly advised to write down your model before starting its implementation within Choco. %The constraint factories in Choco are static methods of the Choco API.

\end{note}

\subsection{Exercise 1.1 (A soft start)}\label{exercises:exercise1.1}\hypertarget{exercises:exercise1.1}{}

Algorithm 1 (below) describes a problem which fits the minimum choco syntax requirements.
\begin{description}
	\item[Question 1] describe the constraint network modelled in Algorithm 1.
	\item[Question 2] give the variable domains after constraint propagation.
\end{description}

\begin{lstlisting}[language={java}, title={\textbf{Algorithm 1} Mysterious model.}]
	// Build a model
	Model m = new CPModel() ;
	
	// Build enumerated domain variables
	IntegerVariable x1 = makeIntVar("var1", 0, 5);
	IntegerVariable x2 = makeIntVar("var2", 0, 5);
	IntegerVariable x3 = makeIntVar("var3", 0, 5);
	
	// Build the constraints
	Constraint C1 = gt(x1, x2) ;
	Constraint C2 = neq(x1, x3) ;
	Constraint C3 = gt(x2, x3) ;
	
	// Add the constraints to the Choco model
	m.addConstraint(C1) ;
	m.addConstraint(C2) ;
	m.addConstraint(C3) ;
	
	// Build a solver
	Solver s = new CPSolver();
	
	// Read the model
	s.read(m);
	
	// Solve the problem
	s.solve() ;
	
	// Print the variable domains
	System.out.println("var1 =" + s.getVar(x1) .getVal()) ;
	System.out.println("var2 =" + s.getVar(x2) .getVal()) ;
	System.out.println("var3 =" + s.getVar(x3) .getVal()) ;
\end{lstlisting}

(\hyperlink{solutions:solutionofexercise1.1}{Solution})

\subsection{Exercise 1.2 (DONALD +  GERALD = ROBERT)}\label{exercises:exercise1.2}\hypertarget{exercises:exercise1.2}{}
Associate a different digit to every letter so that the equation DONALD + GERALD = ROBERT is verified.

(\hyperlink{solutions:solutionofexercise1.2}{Solution})

\subsection{Exercise 1.3 (A famous example. . . a sudoku grid)}\label{exercises:exercise1.3}\hypertarget{exercises:exercise1.3}{}
A sudoku grid is a square composed of nine squares called \emph{blocks}. Each block is itself composed of 3x3 cells
(see figure 1). The purpose of the game is to fill the grid so that each block, column and row contains all the
numbers from 1 to 9 once and only once

\begin{description}
	\item[Question 1] propose a way to model the sudoku problem with difference constraints. Implement your model with Choco.
	\item[Question 2] which global constraint can be used to model such a problem ? Modify your code accordingly.
	\item[Question 3] Test, for both models, the initial propagation step (use Choco \texttt{propagate()} method). What can be noticed ? What is the point in using global constraints ?
\end{description}

\insertGraphique{.5\linewidth}{media/sudoku-grid.jpeg}{An exemple of a Sudoku grid}

(\hyperlink{solutions:solutionofexercise1.3}{Solution})

\subsection{Exercise 1.4 (The knapsack problem)}\label{exercises:exercise1.4}\hypertarget{exercises:exercise1.4}{}
Let us organise a trek. Each hiker carries a knapsack of capacity 34 and can store 3 kinds of food which respectively supply energetic values (6,4,2) for a consumed capacity of (7,5,3). The problem is to find which food is to be put in the knapsack so that the energetic value is maximal.

\begin{description}
	\item[Question 1] In the first place, we will not consider the idea of maximizing the energetic value. Try to find a satisfying solution by modelling and implementing the problem within choco.
	\item[Question 2] Find and use the choco method to \textbf{maximise} the energetic value of the knapsack.
	\item[Question 3] Propose a Value selector heuristic to improve the efficiency of the model.
\end{description}

(\hyperlink{solutions:solutionofexercise1.4}{Solution})

\subsection{Exercise 1.5 (The n-queens problem)}\label{exercises:exercise1.5}\hypertarget{exercises:exercise1.5}{}
The $n$-queens problem aims to place $n$ queens on a chessboard of size $n$ so that no queen can attack one another.
\begin{description}
\item[Question 1] propose and implement a model based on one $L_{i}$ variable for every row. The value of $L_{i}$ indicates the column where a queen is to be put. Use simple difference constraints and confirm that 92 solutions are obtained for $n= 8$.
\item[Question 2] Add a redundant model by considering variables on the columns ($C_{i}$). Continue to use simple difference constraints.
\item[Question 3] Compare the number of nodes created to find the solutions with both models. How can you explain such a difference ?
\item[Question 4] Add to the previously implemented model the following heuristics:
  \begin{itemize}
  \item Select first the line variable $L_I$ which has the smallest domain ;
  \item Select the value $j\in L_i$ so that the associated column variable $C_j$ has the smallest domain.
  \end{itemize}
  Again, compare both approaches in term of number of nodes and solving time to find ONE solution for $n = 75, 90, 95, 105$.
\item[Question 5] what changes are caused by the use of the global constraint \texttt{alldifferent} ?
\end{description}

\insertGraphique{.3\linewidth}{media/nqueen.png}{A solution of the n-queens problem for $n = 8$}

(\hyperlink{solutions:solutionofexercise1.5}{Solution})

\section{I know CP}\label{exercises:iknowcp}\hypertarget{exercises:iknowcp}{}

\subsection{Exercise 2.1 (Bin packing, cumulative and search strategies)}\label{exercises:exercise2.1}\hypertarget{exercises:exercise2.1}{}

Can $n$ objects of a given size fit in $m$ bins of capacity $C$ ? The problem is here stated has a satisfaction problem for the sake of simplicity. Your model and heuristics will be checked by generating random instances for given $n$ and $C$. The random generation must be reproducible.
\begin{description}
	\item[Question 1] Propose a boolean model (0/1 variables).
	\item[Question 2] Let us turn this satisfaction problem into an optimization one. Use your previously stated model but increase regularly the number of containers until a feasible solution is found.
	\item[Question 3] Implement a naive lower bound. This can be done by considering the occupied size globally.
	\item[Question 4] Propose a model with integer variables based on the cumulative constraint (see choco user guide/API for details). Define an objective function to minimize the number of used bins.
	\item[Bonus question] Compare different search strategies (variables/values selector) on this model for $n$ between 10 and 15.
\end{description}

Take a look at the following exercise in the old version of Choco and try to transpose it on new version of Choco.

Here is \textbf{the complete code in Choco1} : \href{media/zip/binpackingv1.zip}{BinPackingv1.zip}.

\begin{lstlisting}
	int[] instance = getRandomPackingPb(n, capaBin, seed); 
	QuickSort sort = new QuickSort(instance); //Sort objects in increasing order
	sort.sort(); 
	Problem pb = new Problem(); 
	IntDomainVar[] debut = new IntDomainVar[n]; 
	IntDomainVar[] duree = new IntDomainVar[n]; 
	IntDomainVar[] fin = new IntDomainVar[n]; 
	 
	int nbBinMin = computeLB(instance, capaBin); 
	for (int i = 0; i < n; i++) { 
	    debut[i] = pb.makeEnumIntVar("debut " + i, 0, n); 
	    duree[i] = pb.makeEnumIntVar("duree " + i, 1, 1); 
	    fin[i] = pb.makeEnumIntVar("fin " + i, 0, n); 
	} 
	IntDomainVar obj = pb.makeEnumIntVar("nbBin ", nbBinMin, n); 
	pb.post(pb.cumulative(debut, fin, duree, instance, capaBin)); 
	for (int i = 0; i < n; i++) { 
	    pb.post(pb.geq(obj, debut[i])); 
	} 
	 
	IntDomainVar[] branchvars = new IntDomainVar[n + 1]; 
	System.arraycopy(debut, 0, branchvars, 0, n); 
	branchvars[n] = obj; 
	 
	//long tps = System.currentTimeMillis(); 
	pb.getSolver().setVarSelector(new StaticVarOrder(branchvars)); 
	Solver.setVerbosity(Solver.SOLUTION); 
	pb.minimize(obj, false); 
	Solver.flushLogs(); 
	// print solution 
	System.out.println("------------------------ " + (obj.getVal() + 1) + " bins"); 
	if (pb.isFeasible() == Boolean.TRUE) { 
	    for (int j = 0; j <= obj.getVal(); j++) { 
	    System.out.print("Bin " + j + ": "); 
	    int load = 0; 
	    for (int i = 0; i < n; i++) { 
	        if (debut[i].isInstantiatedTo(j)) { 
	        System.out.print(i + " "); 
	        load += instance[i]; 
	        } 
	    } 
	    System.out.println(" - load " + load); 
	    } 
	    //System.out.println("tps " + tps + " node " 
        //     + ((NodeLimit) pb.getSolver().getSearchSolver().limits.get(1)).getNbTot()); 
	}
\end{lstlisting}

(\hyperlink{solutions:solutionofexercise2.1}{Solution})

\subsection{Exercise 2.2 (Social golfer)}\label{exercises:exercise2.2}\hypertarget{exercises:exercise2.2}{}

A group of golfers play once a week and are splitted into $k$ groups of size $s$ (there are therefore $ks$ golfers in the club). The objective is to build a game scheduling on $w$ weeks so that no golfer play in the same group than another one more than once (hence the name of the problem: \emph{social golfers}). However, it may happen that two golfers will never play together. The point is only that once they have played together, they cannot play together anymore. 

\begin{note}
You can test your model with the parameters $(w, s, g)$ set to: $\{(11, 6, 2), (13, 7, 2), (9, 8, 8), (9, 8, 4), (4, 7, 3), (3, 6, 4)\}.$
\end{note}

\begin{description}
	\item[Question 1] Propose a boolean model for this problem. Use an heuristic that consists in scheduling a golfer on every week before scheduling a new one. More precisely, a golfer can be put in the first available group of each week before considering the next golfer.
	\item[Question 2] Identify some symmetries of the problem by using every similar elements of the problem. Try to improve your model by breaking those symmetries.
\end{description}

\begin{table}[htbp]
\centering
 	\begin{tabular}{ c c c c c}
		  &  group 1 &  group 2 &  group 3 &  group 4 \\
          \cline{2-5}
		 week 1 &  1 2 3 &  4 5 6 &  7 8 9 &  10 11 12 \\
		 week 1 &  1 4 7 &  10 2 5 &  8 11 3 &  6 9 12 \\
		 week 1 &  1 5 9 &  10 2 6 &  7 11 3 &  4 8 12 \\
	\end{tabular}
\caption{A valid configuration with 4 groups of 3 golfers on 3 weeks.}
\end{table}

Here is \textbf{the complete code in Choco1} : \href{media/zip/socialgolferv1.zip}{SocialGolferv1.zip}

\begin{lstlisting}
  Problem pb = new Problem();
  int numplayers = g * s;

  // golfmat[i][j][k] : is golfer k playing week j in group i ?
  IntDomainVar[][][] golfmat = new IntDomainVar[g][w][numplayers];
  for (int i = 0; i < g; i++) {
      for (int j = 0; j < w; j++)
      	  for (int k = 0; k < numplayers; k++)
          	  golfmat[i][j][k] = pb.makeEnumIntVar("("+i+"_"+j+"_"+k+")", 0, 1);
  }

  //every week, every golfer plays in one group
  for (int i = 0; i < w; i++) {
      for (int j = 0; j < numplayers; j++) {
          IntDomainVar[] vars = new IntDomainVar[g];
          for (int k = 0; k < g; k++) {
              vars[k] = golfmat[k][i][j];
          }
          pb.post(pb.eq(pb.scalar(vars, getOneMatrix(g)), 1));
      }
  }
	
  //every group is of size s
  for (int i = 0; i < w; i++) {
      for (int j = 0; j < g; j++) {
          IntDomainVar[] vars = new IntDomainVar[numplayers];
          System.arraycopy(golfmat[j][i], 0, vars, 0, numplayers);
          pb.post(pb.eq(pb.scalar(vars, getOneMatrix(numplayers)), s));
      }
  }
	
  //every pair of players only meets once
  // Efficient way : use of a ScalarAtMost
  for (int i = 0; i < numplayers; i++) {
      for (int j = i + 1; j < numplayers; j++) {
          IntDomainVar[] vars = new IntDomainVar[w * g * 2];
          int cpt = 0;
          for (int k = 0; k < w; k++) {
              for (int l = 0; l < g; l++) {
                  vars[cpt] = golfmat[l][k][i];
                  vars[cpt + w * g] = golfmat[l][k][j];
                  cpt++;
              } 
          }
          pb.post(new ScalarAtMostv1(vars, w * g, 1));
      }
  }
	
  //break symetries among weeks
  //enforce a lexicographic ordering between every pairs of week
  for (int i = 0; i < w; i++) {
      for (int j = i + 1; j < w; j++) {
          IntDomainVar[] vars1 = new IntDomainVar[numplayers * g];
          IntDomainVar[] vars2 = new IntDomainVar[numplayers * g];
          int cpt = 0;
          for (int k = 0; k < numplayers; k++) {
              for (int l = 0; l < g; l++) {
                  vars1[cpt] = golfmat[l][i][k];
                  vars2[cpt] = golfmat[l][j][k];
                  cpt++;
              }
          }
          pb.post(pb.lex(vars1, vars2));
      }
  }
	
  //break symetries among groups
  for (int i = 0; i < numplayers; i++) {
      for (int j = i + 1; j < numplayers; j++) {
          IntDomainVar[] vars1 = new IntDomainVar[w * g];
          IntDomainVar[] vars2 = new IntDomainVar[w * g];
          int cpt = 0;
          for (int k = 0; k < w; k++) {
              for (int l = 0; l < g; l++) {
                  vars1[cpt] = golfmat[l][k][i];
                  vars2[cpt] = golfmat[l][k][j];
                  cpt++;
              }
          }
          pb.post(pb.lex(vars1, vars2));
      }
  }
	
  //break symetries among players
  for (int i = 0; i < w; i++) {
      for (int j = 0; j < g; j++) {
          for (int p = j + 1; p < g; p++) {
              IntDomainVar[] vars1 = new IntDomainVar[numplayers];
              IntDomainVar[] vars2 = new IntDomainVar[numplayers];
              int cpt = 0;
              for (int k = 0; k < numplayers; k++) {
                  vars1[cpt] = golfmat[j][i][k];
                  vars2[cpt] = golfmat[p][i][k];
                  cpt++;
              }
              pb.post(pb.lex(vars1, vars2));
          }
      }
  }
	
  //gather branching variables
  IntDomainVar[] staticvars = new IntDomainVar[g * w * numplayers];
  int cpt = 0;
  for (int i = 0; i < numplayers; i++) {
      for (int j = 0; j < w; j++) {
          for (int k = 0; k < g; k++) {
              staticvars[cpt] = golfmat[k][j][i];
              cpt++;
          }
      }
  }
  pb.getSolver().setVarSelector(new StaticVarOrder(staticvars));
	
  pb.getSolver().setTimeLimit(120000);
  Solver.setVerbosity(Solver.SOLUTION);
  pb.solve();
  Solver.flushLogs();
\end{lstlisting}

(\hyperlink{solutions:solutionofexercise2.2}{Solution})

\subsection{Exercise 2.3 (Golomb rule)}\label{exercises:exercise2.3}\hypertarget{exercises:exercise2.3}{}

\emph{under development}

(\hyperlink{solutions:solutionofexercise2.3}{Solution})

\section{I know CP and Choco}\label{exercises:iknowcpandchoco}\hypertarget{exercises:iknowcpandchoco}{}

\subsection{Exercise 3.1 (Hamiltonian Cycle Problem Traveling Salesman Problem)}\label{exercises:exercise3.1}\hypertarget{exercises:exercise3.1}{}

Given a graph $G = (V,E)$, an \emph{Hamiltonian cycle} is a cycle that goes through every nodes of G once and only once. This exercise first intorduces a naive model to solve the Hamiltonian Cycle Problem. A second part tackles with the well known Traveling Salesman Problem.

Let $V = \{2,...,n\}$ be a set of cities index to cover, and let $d$ be a single warehouse duplicated into two indices $1$ and $n+1$. Notice the duplication distinguishes the source from the sink while there is only one warehouse. Finally, let us denote by $V_{d} = V \cup \{1,n+1\}$ the set of nodes to cover by a tour. Thus, the two following problems are defined:
\begin{itemize}
	\item find an Hamiltonian path covering all the cities of $V$
	\item find an Hamiltonian cycle of minimum cost that covers all the cities of $V$.
\end{itemize}

\subsubsection{Question 1 [Hamiltonian Cycle Problem]:} We first consider the satisfaction problem. Formally, a directed graph $G = (V,E)$ represents the topology of the cities and the unfolded warehouse. There is an arc $(i,j)\in E$ iff there exists a directed road from $i\in V$ to $j\in V$. Furthermore, every arc $(1,i)$ and $(i,n+1)$, with $i\in V$, belongs to E. Such a problem has to respect the following constraints:
\begin{itemize}
	\item each node of $V_{d}$ is reached exactly once,
	\item there is no subcycle containing nodes of $V$. In other words, the single cycle involved in $G$ is hamiltonian and contains arc $(n+1,1)$.
\end{itemize}
\begin{description}
\item[Question 1.a] The first constraint can directly be modelled using those proposed by Choco. On the other way, the second one requires to implement a constraint. This can be done through the following steps (see the provided skeleton):
\begin{itemize}
	\item strictly specify your constraint signature,
	\item formalise the underlying subproblemand information that need to be maintained,
	\item ignore in a first time the Choco event based mechanism and implement your filtering algorithm directly within the \texttt{propagate()} method,
	\item once your algorithm has been checked, try to reformulate your constraint through an event based implementation with the following methods: \texttt{awakeOnInst()}, \texttt{awakeOnSup()}, \texttt{awakeOnInf()}, \texttt{awakeOnBounds()}, \texttt{awakeOnRem()}, \texttt{awakeOnRemovals()}.
\end{itemize}
\item[Question 1.b] Now, propose a search heuristic (both on variables and values) that incrementaly builds the searched path from the source node. For this purpose, you have to respectively implement java classes that inherit from \texttt{IntVarSelector} and \texttt{ValSelector}.
\end{description}

\subsubsection{Question 2 [Traveling Salesman Problem]:} We now consider the optimisation view of the Hamiltonian Cycle Problem. A quantitative information is now associated with each arc of $G$ given by a cost function $f: E \leftarrow\Z_+$. Then, the graph $G$ is now defined by the triplet $(V_d,E,f)$ and we have to find an Hamiltonian path of minimum cost in $G$. 

For this purpose, we provide a skeleton of a Choco global constraint that dynamically maintains a lower bound evaluation of the searched path cost. Here, an evaluation of a minimum spanning tree of $G$ is proposed. \emph{Be careful}: take into account the partial assignment of the variables associated with the cities. 

\begin{description}
	\item[Question 2.a] find an upper bound on the cost of the Hamiltonian path,
	\item[Question 2.b] back-propagate lower/upper bounds informations on the required/infeasible arcs of $G$.
\end{description}

(\hyperlink{solutions:solutionofexercise3.1}{Solution})

\subsection{Exercise 3.2 (Shop scheduling)}\label{exercises:exercise3.2}\hypertarget{exercises:exercise3.2}{}

Given a set of $n$ tasks $T$ and $m$ disjunctive resources $R$, the problem is to find a plan to assign tasks to resources so that for every instant $t$, each resource $r\in R$ executes at most one task. Each task $T_i\in T$ is defined by:
\begin{itemize}
	\item a starting date $s_i = [s^-_i, s^+_i]\in\Z_+$,
	\item an ending date  $e_i = [e^-_i, e^+_i]\in\Z_+$,
	\item a duration  $d_i = e_i-s_i$, %$[d^-_i, d^+_i]\in\Z_+$,
	\item a resource $r_i = \{res_{1},\ldots, res_{m}\}\subseteq R$,
	\item a set of tasks, $preds_i\subseteq T$ that need to be processed before the start of $T_i$.
\end{itemize}

Let us consider the following satisfaction problem : Can one find a schedule of tasks $T$ on the resources $R$ that
\begin{itemize}
\item satisfies all the precedence constraints:
  $$ e_j\le s_i,\qquad (\forall T_i\in T, \forall T_j\in preds_i)$$
\item the last processed task ends before a given date $D$ ?:
  $$ e_i\le D,\qquad (\forall T_i\in T)$$
\end{itemize}

Then consider the optimization version : We now aim at finding the scheduling that satisfies all the constraints and that minimizes the date of the last task processed.
You will be given for this :
\begin{itemize}
\item a class structure where you have to describe your model (\texttt{AssignmentProblem}),
\item a class structure describing a Task (\texttt{Task}),
\item a class structure (\texttt{BinaryNonOverlapping}) which defines a Choco constraint. This constraint takes two tasks as parameters and has to verify whether at any time $t$ those tasks will be processed by the same resource or not.
\item a class structure (\texttt{MandatoryInterval}) which describes for a given task, the time window it has to be processed in.
\end{itemize}

\begin{description}
	\item[Question 1] How would you model the job scheduling problem ? Make use of the constraint \texttt{BinaryNonOverlapping}.
	\item[Question 2] Implement your model as if the constraint \texttt{BinaryNonOverlapping} was implemented.
	\item[Question 3] Sketch the mandatory processing interval of a task.
	\item[Question 4] Implement the constraint \texttt{BinaryNonOverlapping}:
	\begin{itemize}
		\item Implement the following reasoning : if two tasks have to be processed on the same resource and their mandatory intervals intersect, throw a failure.
		\item Now, implement the condition : if two tasks have a mandatory interval intersection, they must be scheduled on different resources.
		\item Finally, implement the following reasoning : If two tasks have to be processed by the same resource, then the starting and ending dates of every task ought to be updated functions to their mandatory intervals.
	\end{itemize}
	\item[Question 5] Implement an variable selection heuristic on the decision variable of the problem.
	\item[Question 6] Propose a model which minimize the end date of the last assigned task.
	\item[Question 7] Can you find a way to improve the BinaryNonOverlapping constraint.
	\item[Bonus Question] Find a lower bound on the end date of the last processed task.
\end{description}

\hyperlink{solutions:solutionofexercise3.2}{Solution}

%\part{solutions}
\label{solutions}
\hypertarget{solutions}{}

\chapter{Solutions}\label{solutions:solutions}\hypertarget{solutions:solutions}{}

\section{I'm new to CP}\label{solutions:i'mnewtocp}\hypertarget{solutions:i'mnewtocp}{}

\subsection{Solution of Exercise 1.1 (A soft start)}\label{solutions:solutionofexercise1.1}\hypertarget{solutions:solutionofexercise1.1}{}
(\hyperlink{exercises:exercise1.1}{Problem})

\noindent\emph{\textbf{Question 1}: describe the constraint network related to code}

The model is defined as :
\begin{itemize}
	\item $V = \{x_1, x_2, x_3\}$: the set of variables,
	\item $D = \{[0,5], [0,5], [0,5]\}$: the set of domain
	\item $C = \{x_1>x_2, x_1\neq x_3, x_2>x_3\}$: the set of constraints.
\end{itemize}

\noindent\emph{\textbf{Question 2}: give the variable domains after constraint propagation.}

\begin{itemize}
	\item From $x_1 = [0,5]$ and $x_2 = [0,5]$ and $x_1>x_2$, we can deduce tha : the domain of $x_1$ can be reduce to $[1,5]$ and the domain of $x_2$ can be reduce to $[0,4]$.
	\item Then, from $x_2 = [0,4]$ and $x_3 = [0,5]$ and $x_2>x_3$, we can deduce that : the domain of $x_2$ can be reduce to $[1,4]$ and the domain of $x_3$ can be reduce to $[0,3]$.
	\item Then, from $x_1 = [1,5]$ and $x_2 = [1,4]$ and $x_1>x_2$, we can deduce that : the domain of $x_1$ can be reduce to $[2,5]$.
\end{itemize}

We cannot deduce anything else, so we have reached a \textbf{fix point}, and here is the domain of each variables:
$$x_{1} : [2,5],\quad x_{2} : [1,4],\quad x_{3} : [0,3].$$


\subsection{Solution of Exercise 1.2 (DONALD + GERALD = ROBERT)}\label{solutions:solutionofexercise1.2}\hypertarget{solutions:solutionofexercise1.2}{}

(\hyperlink{exercises:exercise1.2}{Problem})

Source code: \href{media/zip/exdonaldgeraldrobert.zip}{ExDonaldGeraldRobert.zip}

\begin{lstlisting}
  // Build model
  Model model = new CPModel();
  
  // Declare every letter as a variable
  IntegerVariable d = makeIntVar("d", 0, 9, "cp:enum");
  IntegerVariable o = makeIntVar("o", 0, 9, "cp:enum");
  IntegerVariable n = makeIntVar("n", 0, 9, "cp:enum");
  IntegerVariable a = makeIntVar("a", 0, 9, "cp:enum");
  IntegerVariable l = makeIntVar("l", 0, 9, "cp:enum");
  IntegerVariable g = makeIntVar("g", 0, 9, "cp:enum");
  IntegerVariable e = makeIntVar("e", 0, 9, "cp:enum");
  IntegerVariable r = makeIntVar("r", 0, 9, "cp:enum");
  IntegerVariable b = makeIntVar("b", 0, 9, "cp:enum");
  IntegerVariable t = makeIntVar("t", 0, 9, "cp:enum");
  
  // Declare every name as a variable  
  IntegerVariable donald = makeIntVar("donald", 0, 1000000,"cp:bound");
  IntegerVariable gerald = makeIntVar("gerald", 0, 1000000,"cp:bound");
  IntegerVariable robert = makeIntVar("robert", 0, 1000000,"cp:bound");
  
  // Array of coefficients
  int[] c = new int[]{100000, 10000, 1000, 100, 10, 1}; 
  
  // Declare every combination of letter as an integer expression
  IntegerExpressionVariable donaldLetters = scalar(new IntegerVariable[]{d,o,n,a,l,d}, c);
  IntegerExpressionVariable geraldLetters = scalar(new IntegerVariable[]{g,e,r,a,l,d}, c);
  IntegerExpressionVariable robertLetters = scalar(new IntegerVariable[]{r,o,b,e,r,t}, c);
  
  // Add equality between name and letters combination
  model.addConstraint(eq(donaldLetters, donald));
  model.addConstraint(eq(geraldLetters, gerald));
  model.addConstraint(eq(robertLetters, robert));
  // Add constraint name sum
  model.addConstraint(eq(plus(donald, gerald), robert));
  // Add constraint of all different letters.
  model.addConstraint(allDifferent(new IntegerVariable[]{d,o,n,a,l,g,e,r,b,t}));
  
  // Build a solver, read the model and solve it
  Solver s = new CPSolver();
  s.read(model);
  s.solve();
  
  // Print name value
  System.out.println("donald = " + s.getVar(donald).getVal());
  System.out.println("gerald = " + s.getVar(gerald).getVal());
  System.out.println("robert = " + s.getVar(robert).getVal());
\end{lstlisting}

\subsection{Solution of Exercise 1.3 (A famous example. . . a sudoku grid)}\label{solutions:solutionofexercise1.3}\hypertarget{solutions:solutionofexercise1.3}{}

(\hyperlink{exercises:exercise1.3}{Problem})

Source code: \href{media/zip/exsudoku.zip}{ExSudoku.zip}

\noindent\emph{\textbf{Question 1}: propose a way to model the sudoku problem with difference constraints. Implement your model with choco solver.}

\begin{lstlisting}
  int n = instance.length;
  // Build Model
  Model m = new CPModel();
  
  // Build an array of integer variables
  IntegerVariable[][] rows = makeIntVarArray("rows", n, n, 1, n,"cp:enum");
	
  // Not equal constraint between each case of a row
  for (int i = 0; i < n; i++) {
      for (int j = 0; j < n; j++)
          for (int k = j; k < n; k++)
              if (k != j) m.addConstraint(neq(rows[i][j], rows[i][k]));
  }
                  
  // Not equal constraint between each case of a column
  for (int j = 0; j < n; j++) {
      for (int i = 0; i < n; i++)
          for (int k = 0; k < n; k++)
              if (k != i)  m.addConstraint(neq(rows[i][j], rows[k][j]));
  }

  // Not equal constraint between each case of a sub region
  for (int ci = 0; ci < n; ci += 3) {
      for (int cj = 0; cj < n; cj += 3)
          // Extraction of disequality of a sub region
          for (int i = ci; i < ci + 3; i++)
              for (int j = cj; j < cj + 3; j++)
                  for (int k = ci; k < ci + 3; k++)
                      for (int l = cj; l < cj + 3; l++)
                          if (k != i || l != j) m.addConstraint(neq(rows[i][j], rows[k][l]));
  }
	
  //...
	
  // Call solver
  Solver s = new CPSolver();
  s.read(m);
  CPSolver.setVerbosity(CPSolver.SOLUTION);
  s.solve();
  CPSolver.flushLogs();
  printGrid(rows, s);
\end{lstlisting}

\noindent\emph{\textbf{Question 2}: which global constraint can be used to model such a problem ? Modify your code to use this constraint.}

The \emph{allDifferent} constraint can be used to remplace every disequality constraint on the first Sudoku model. It improves the efficient of the model and make it more ``readable''.

\begin{lstlisting}
  // Build model
  Model m = new CPModel();
  // Declare variables
  IntegerVariable[][] cols = new IntegerVariable[n][n];
  IntegerVariable[][] rows = makeIntVarArray("rows", n, n, 1, n,"cp:enum");
  
  // Channeling between rows and columns
  for (int i = 0; i < n; i++) {
      for (int j = 0; j < n; j++)
          cols[i][j] = rows[j][i];
  }
	
  // Add alldifferent constraint
  for (int i = 0; i < n; i++) {
      m.addConstraint(allDifferent(cols[i]));
      m.addConstraint(allDifferent(rows[i]));
  }

  // Define sub regions
  IntegerVariable[][] carres = new IntegerVariable[n][n];
  for (int i = 0; i < 3; i++) {
      for (int j = 0; j < 3; j++)
          for (int k = 0; k < 3; k++)
              carres[j + k * 3][i] = rows[0 + k * 3][i + j * 3];
              carres[j + k * 3][i + 3] = rows[1 + k * 3][i + j * 3];
              carres[j + k * 3][i + 6] = rows[2 + k * 3][i + j * 3];
  }
	
  // Add alldifferent on sub regions
  for (int i = 0; i < n; i++) {
      Constraint c = allDifferent(carres[i]);
      m.addConstraint(c);
  }
  
  //...
	
  // Call solver
  Solver s = new CPSolver();
  s.read(m);
  CPSolver.setVerbosity(CPSolver.SOLUTION);
  s.solve();
  printGrid(rows, s);
\end{lstlisting} 

\noindent\emph{\textbf{Question 3}: Test for both model the initial propagation step (use choco} \texttt{propagate()} \emph{method). What can be noticed ? What is the point in using global constraints ?}

The sudoku problem can be solved just with the propagation. \todo{FIXME explanation.}
The global constraint provides a more efficient filter algorithm, due to more complex deduction.

\subsection{Solution of Exercise 1.4 (The knapsack problem)}\label{solutions:solutionofexercise1.4}\hypertarget{solutions:solutionofexercise1.4}{}

(\hyperlink{exercises:exercise1.4}{Problem})

Source code: \href{media/zip/exknapsack.zip}{ExKnapSack.zip}

\noindent\emph{\textbf{Question 1} : In the first place, we will not consider the idea of maximizing the energetic value. Try to find a satisfying solution by modelling and implementing the problem within choco.}

\begin{lstlisting}
	Model m = new CPModel();
	
	obj1 = makeIntVar("obj1", 0, 5,"cp:enum");
	obj2 = makeIntVar("obj2", 0, 7,"cp:enum");
	obj3 = makeIntVar("obj3", 0, 10,"cp:enum");
	c = makeIntVar("cost", 1, 1000000,"cp:bound");
	
	int capacity = 34;
	int[] volumes = new int[]{7, 5, 3};
	int[] energy = new int[]{6, 4, 2};
	
	m.addConstraint(leq(scalar(volumes, new IntegerVariable[]{obj1, obj2, obj3}), capacity));
	m.addConstraint(eq(scalar(energy, new IntegerVariable[]{obj1, obj2, obj3}), c));
	
	Solver s = new CPSolver();
	s.read(m);

	s.solve();
	
	System.out.println("("+s.getVar(obj1).getVal()+","+s.getVar(obj2).getVal()+","
                       +s.getVar(obj3).getVal()+") cost = "+ s.getVar(c).getVal());
\end{lstlisting}

\noindent\emph{\textbf{Question 2} : Find and use the choco method to maximise the energetic value of the knapsack.}
Replace \mylst{s.solve()} by:
\begin{lstlisting}
	s.maximize(s.getVar(c), false);
\end{lstlisting}

\noindent\emph{\textbf{Question 3} : Propose a Value selector heuristic to improve the efficiency of the model.}

It can be improved using the following value selector strategy. It iterates over decreasing values of every domain variables: 
\begin{lstlisting}
  s.setValIntIterator(new DecreasingDomain());
\end{lstlisting}

\subsection{Solution of Exercise 1.5 (The n-queens problem)}\label{solutions:solutionofexercise1.5}\hypertarget{solutions:solutionofexercise1.5}{}
(\hyperlink{exercises:exercise1.5}{Problem})

Source code: \href{media/zip/exqueen.zip}{ExQueen.zip}

\noindent\emph{\textbf{Question 1} : propose and implement a model based on one} $L_{i}$ \emph{variable for every row...}
\begin{lstlisting}
  Model m = new CPModel();
  
  IntegerVariable[] queens = new IntegerVariable[n];
  for (int i = 0; i < n; i++) {
      queens[i] = makeIntVar("Q" + i, 1, n,"cp:enum");
  }
	
  for (int i = 0; i < n; i++) {
      for (int j = i + 1; j < n; j++) {
          int k = j - i;
          m.addConstraint(neq(queens[i], queens[j]));
          m.addConstraint(neq(queens[i], plus(queens[j], k)));  // diagonal
          m.addConstraint(neq(queens[i], minus(queens[j], k))); // diagonal
      }
  }
	
  Solver s = new CPSolver();
  s.read(m);
  CPSolver.setVerbosity(CPSolver.SOLUTION);
  int timeLimit = 60000;
  s.setTimeLimit(timeLimit);
  s.solve();
  CPSolver.flushLogs();
\end{lstlisting}

\noindent\emph{\textbf{Question 2} : Add a redundant model by considering variable on the columns ($C_i$). Continue to use simple difference constraints.}

\begin{lstlisting}
  Model m = new CPModel();
	
  IntegerVariable[] queens = new IntegerVariable[n];
  IntegerVariable[] queensdual = new IntegerVariable[n];
  for (int i = 0; i < n; i++) {
      queens[i] = makeIntVar("Q" + i, 1, n,"cp:enum");
      queensdual[i] = makeIntVar("QD" + i, 1, n,"cp:enum");
  }
	
  for (int i = 0; i < n; i++) {
      for (int j = i + 1; j < n; j++) {
          int k = j - i;
          m.addConstraint(neq(queens[i], queens[j]));
          m.addConstraint(neq(queens[i], plus(queens[j], k)));  // diagonal
          m.addConstraint(neq(queens[i], minus(queens[j], k))); // diagonal
      }
  }

  for (int i = 0; i < n; i++) {
      for (int j = i + 1; j < n; j++) {
          int k = j - i;
          m.addConstraint(neq(queensdual[i], queensdual[j]));
          m.addConstraint(neq(queensdual[i], plus(queensdual[j], k)));  // diagonal
          m.addConstraint(neq(queensdual[i], minus(queensdual[j], k))); // diagonal
      }
  }
  m.addConstraint(inverseChanneling(queens, queensdual));
  
  Solver s = new CPSolver();
  s.read(m);
  
  s.setVarIntSelector(new MinDomain(s,s.getVar(queens)));
  
  CPSolver.setVerbosity(CPSolver.SOLUTION);
  s.setLoggingMaxDepth(50);
  int timeLimit = 60000;
  s.setTimeLimit(timeLimit);
  s.solve();
  CPSolver.flushLogs();
\end{lstlisting}

\noindent\emph{\textbf{Question 3} : Compare the number of nodes created to find the solutions with both models. How can you explain such a difference ?}

The channeling permit to reduce more nodes from the tree search... \todo{FIXME}

\noindent\emph{\textbf{Question 4} : Add to the previous implemented model the following heuristics,
\begin{itemize}
	\item Select first the line variable ($L_i$) which has the smallest domain ;
	\item Select the value $j\in L_i$ so that the associated column variable $C_j$ has the smallest domain.
\end{itemize}
Again, compare both approaches in term of nodes number and solving time to find ONE solution for $n = 75, 90, 95, 105$.}

Add the following lines to your program (after the reading of the model):
\begin{lstlisting}
	s.setVarIntSelector(new MinDomain(s,s.getVar(queens)));
	s.setValIntSelector(new NQueenValueSelector(s.getVar(queensdual)));
\end{lstlisting}
The variable selector strategy (\texttt{MinDomain}) already exists in Choco. It iterates over variables given and returns the variable ordering by creasing domain size. 
The value selector strategy has to be created as follow:
\begin{lstlisting}
  public class NQueenValueSelector implements ValSelector {
	
      // Column variable
      protected IntDomainVar[] dualVar;
	
      // Constructor of the value selector, 
      public NQueenValueSelector(IntDomainVar[] cols) {
          this.dualVar = cols;
      }
	
      // Returns the "best val" that is the smallest column domain size OR -1
      // (-1 is not in the domain of the variables)
      public int getBestVal(IntDomainVar intDomainVar) {
          int minValue = 10000;
          int v0 = -1;
          IntIterator it = intDomainVar.getDomain().getIterator();
          while (it.hasNext()){
              int i = it.next();
              int val = dualVar[i - 1].getDomainSize();
              if (val < minValue)  {
                  minValue = val;
                  v0 = i;
              }
          }
          return v0;
      }
  }
\end{lstlisting}

\noindent\emph{\textbf{Question 5} : what changes are caused by the use of the global constraint \textbf{alldifferent} ?}

\begin{lstlisting}
  Model m = new CPModel();
	
  IntegerVariable[] queens = new IntegerVariable[n];
  IntegerVariable[] queensdual = new IntegerVariable[n];
  IntegerVariable[] diag1 = new IntegerVariable[n];
  IntegerVariable[] diag2 = new IntegerVariable[n];
  IntegerVariable[] diag1dual = new IntegerVariable[n];
  IntegerVariable[] diag2dual = new IntegerVariable[n];
  for (int i = 0; i < n; i++) {
      queens[i] = makeIntVar("Q" + i, 1, n,"cp:enum");
      queensdual[i] = makeIntVar("QD" + i, 1, n,"cp:enum");
      diag1[i] = makeIntVar("D1" + i, 1, 2 * n,"cp:enum");
      diag2[i] = makeIntVar("D2" + i, -n, n,"cp:enum");
      diag1dual[i] = makeIntVar("D1" + i, 1, 2 * n,"cp:enum");
      diag2dual[i] = makeIntVar("D2" + i, -n, n,"cp:enum");
  }
	
  m.addConstraint(allDifferent(queens));
  m.addConstraint(allDifferent(queensdual));
  for (int i = 0; i < n; i++) {
      m.addConstraint(eq(diag1[i], plus(queens[i], i)));
      m.addConstraint(eq(diag2[i], minus(queens[i], i)));
      m.addConstraint(eq(diag1dual[i], plus(queensdual[i], i)));
      m.addConstraint(eq(diag2dual[i], minus(queensdual[i], i)));
  }
  m.addConstraint(inverseChanneling(queens,queensdual));
	
  m.addConstraint(allDifferent(diag1));
  m.addConstraint(allDifferent(diag2));
  m.addConstraint(allDifferent(diag1dual));
  m.addConstraint(allDifferent(diag2dual));
	
  Solver s = new CPSolver();
  s.read(m);
	
  s.setVarIntSelector(new MinDomain(s,s.getVar(queens)));
  s.setValIntSelector(new NQueenValueSelector(s.getVar(queensdual)));
	
  CPSolver.setVerbosity(CPSolver.SOLUTION);
  int timeLimit = 60000;
  s.setTimeLimit(timeLimit);
  s.solve();
  CPSolver.flushLogs();
\end{lstlisting}

\section{I know CP}\label{solutions:iknowcp}\hypertarget{solutions:iknowcp}{}

\subsection{Solution of Exercise 2.1 (Bin packing, cumulative and search strategies)}\label{solutions:solutionofexercise2.1}\hypertarget{solutions:solutionofexercise2.1}{}
(\hyperlink{exercises:exercise2.1}{Problem})

Source code: \href{media/zip/binpackingv2.zip}{BinPackingv2.zip}

\subsection{Solution of Exercise 2.2 (Social golfer)}\label{solutions:solutionofexercise2.2}\hypertarget{solutions:solutionofexercise2.2}{}
(\hyperlink{exercises:exercise2.2}{Problem})

Source code: \href{media/zip/socialgolferv2.zip}{SocialGolferv2.zip}

\subsection{Solution of Exercise 2.3 (Golomb rule)}\label{solutions:solutionofexercise2.3}\hypertarget{solutions:solutionofexercise2.3}{}

\emph{under development}

(\hyperlink{exercises:exercise2.3}{Problem})

\section{I know CP and Choco2.0}\label{solutions:iknowcpandchoco2.0}\hypertarget{solutions:iknowcpandchoco2.0}{}

\subsection{Solution of Exercise 3.1 (Hamiltonian Cycle Problem Traveling Salesman Problem)}\label{solutions:solutionofexercise3.1}\hypertarget{solutions:solutionofexercise3.1}{}

(\hyperlink{exercises:exercise3.1}{Problem})

Source code: \href{media/zip/extsp.zip}{ExTSP.zip}

\subsection{Solution of Exercise 3.2 (Shop scheduling)}\label{solutions:solutionofexercise3.2}\hypertarget{solutions:solutionofexercise3.2}{}

(\hyperlink{exercises:exercise3.2}{Problem})

\emph{under development}


\chapter{Applications with global constraints}\label{doc:applications}\hypertarget{doc:applications}{}
%\part{geost description}
\label{geostdescription}
\hypertarget{geostdescription}{}


\section{Placement and use of the Geost constraint}\label{geostdescription:placementanduseofthegeostconstraint}\hypertarget{geostdescription:placementanduseofthegeostconstraint}{}

The global constraint \texttt{\bf geost}($k,O,S,C$) handles in a generic way a variety of geometrical constraints $C$ in space and time between polymorphic $k\in\N$ dimensional objects $O$, each of which taking a shape among a set of shapes $S$ during a given time interval and at a given position in space. Each shape from $S$ is defined as a finite set of shifted boxes, where each shifted box is described by a box in a $k$-dimensional space at the given offset with the given sizes.

More precisely a \emph{shifted box} $s$= \emph{shape(sid,t[],l[])} is an entity defined by a shape id \emph{sid}, an shift offset \emph{s.t[d]}, $0\le d < k$, and a size \emph{s.l[d]}$>0$, $0\le d<k$. All attributes of a shifted box are integer values. Then, a \emph{shape} from $S$ is a collection of shifted boxes sharing all the same shape id. Note that the shifted boxes associated with a given shape may or may not overlap. This sometimes allows a drastic reduction in the number of shifted boxes needed to describe a shape.
Each \emph{object} \emph{o= object( id, sid,x[], start, duration,end)} from $O$ is an entity defined by a unique object id \emph{o.id} (an integer), a shape id \emph{o.sid}, an origin \emph{o.x[d]}, $0\le d<k$, a starting time \emph{o.start}, a duration \emph{o.duration}$>0$, and a finishingxs time \emph{o.end}.

All attributes \emph{sid, x[0],x[1],...,x[k-1], start, duration, end} correspond to domain variables. Typical constraints from the
list of constraints $C$ are for instance the fact that a given subset of objects from $O$ do not pairwise overlap.
Constraints of the list of constraints $C$ have always two first arguments $A_i$ and $O_i$ (followed by possibly some additional arguments) which respectively specify:
\begin{itemize}
	\item A list of dimensions (integers between 0 and k-1), or attributes of the objects of $O$ the constraint considers.
	\item A list of identifiers of the objects to which the constraint applies.
\end{itemize}

\subsection{Example and way to implement it}\label{geostdescription:exampleandwaytoimplementit}\hypertarget{geostdescription:exampleandwaytoimplementit}{}

We will explain how to use \emph{geost} although a 2D example. Consider we have 3 objects $o_0, o_1, o_2$ to place them inside a box $B$ (3x4) such that they don't overlap (see Figure bellow). The first object $o_0$ has two potential shapes while $o_1$ and $o_2$ have one shape. Given that the placement of the objects should be totally inside $B$ this means that the domain of the origins of objects are as follows (we start from 0 this means that the placement space is from 0 to 2 on $x$ and from 0 to 3 on $y$:
\begin{itemize}
	\item $o_0$: $x$ in 0..1, $y$ in 0..1,
	\item $o_1$: $x$ in 0..1, $y$ in 0..1,
	\item $o_2$: $x$ in 0..1, $y$ in 0..3.
\end{itemize}

\insertGraphique{.9\linewidth}{media/exp_geost.png}{Geost objects and shapes}

We describe now how to solve this problem by using Choco.

\subsubsection{Build a CP model.}\label{geostdescription:buildacpmodel}\hypertarget{geostdescription:buildacpmodel}{}
To begin the implementation we build a CP model:
\begin{lstlisting}
  Model m = new CPModel();
\end{lstlisting}
\subsubsection{Set the Dimension.}\label{geostdescription:setthedimension}\hypertarget{geostdescription:setthedimension}{}
Then we first need to specify the dimension k we are working in. This is done by assigning the dimension to a local variable that we will use later:
\begin{lstlisting}
  int dim = 2;
\end{lstlisting}
\subsubsection{Create the Objects.}\label{geostdescription:createtheobjects}\hypertarget{geostdescription:createtheobjects}{}
Then we start by creating the objects and store them in a vector as such:
\begin{lstlisting}
  Vector<GeostObject> objects = new Vector<GeostObject>();
\end{lstlisting}
Now we create the first object $o_0$ by creating all its attributes.
\begin{lstlisting}
  int objectId = 0; // object id
  IntegerVariable shapeId = Choco.makeIntVar("sid", 0, 1); // shape id (2 possible values)
  IntegerVariable coords[] = new IntegerVariable[dim]; // coordinates of the origin 
  coords[0] = Choco.makeIntVar("x", 0, 1);  
  coords[1] =  Choco.makeIntVar("y", 0, 1);
\end{lstlisting}
We need to specify 3 more Integer Domain Variables representing the temporal attributes (start, duration and end), which for the current implementation of \texttt{geost} are not working, however we need to give them dummy values. 
\begin{lstlisting}
  IntegerVariable start = Choco.makeIntVar("start", 0, 0);
  IntegerVariable duration = Choco.makeIntVar("duration", 1, 1);
  IntegerVariable end = Choco.makeIntVar("end", 1, 1);
\end{lstlisting}
Finally we are ready to add the object 0 to our \emph{objects} Vector:
\begin{lstlisting}
  objects.add(new GeostObject(dim, objectId, shapeId, coords, start, duration, end));
\end{lstlisting}
Now we do the same for the other  object $o_1$  and  $o_2$ and add them to our \emph{objects} vector.


\subsubsection{Create the Shifted Boxes.}\label{geostdescription:createtheshiftedboxes}\hypertarget{geostdescription:createtheshiftedboxes}{}
To create the shapes and their shifted boxes we create the shifted boxes and associate them with the corresponding shapeId. This is done as follows, first we create a Vector called \emph{sb} for example 
\begin{lstlisting}
Vector<ShiftedBox> sb = new Vector<ShiftedBox> ();
\end{lstlisting}
To create the shifted boxes for the shape 0 (that corresponds to $o_0$), we start by the first shifted box by creating the sid and 2 arrays one to specify the offset of the box in each dimension and one for the size of the box in each dimension:
\begin{lstlisting}
  int sid = 0; 
  int[] offset = {0,0};
  int[] sizes = {1,3};
\end{lstlisting} 
Now we add our shiftedbox to the \emph{sb} Vector:
\begin{lstlisting}
  sb.add(new ShiftedBox(sid, offset, sizes));
\end{lstlisting} 
We do the same with second shifted box:
\begin{lstlisting}
  sb.add(new ShiftedBox(0, new int[]{0,0}, new int[]{2,1}));
\end{lstlisting} 
By the same way we create the shifted boxes corresponding to second shape $S_1$:
\begin{lstlisting}
  sb.add(new ShiftedBox(1, new int[]{0,0}, new int[]{2,1}));
  sb.add(new ShiftedBox(1, new int[]{1,0}, new int[]{1,3}));
\end{lstlisting}
and the third shape $S_2$ consisting of three shifted boxes:
\begin{lstlisting}
  sb.add(new ShiftedBox(2, new int[]{0,0}, new int[]{2,1})) ;
  sb.add(new ShiftedBox(2, new int[]{1,0}, new int[]{1,3})); 
  sb.add(new ShiftedBox(2, new int[]{0,2}, new int[]{2,1}));
\end{lstlisting}
and finally the last shape $S_3$
\begin{lstlisting}
  sb.add(new ShiftedBox(3, new int[]{0,0}, new int[]{2,1}));
\end{lstlisting}

\subsubsection{Create the constraints.}\label{geostdescription:createtheconstraints}\hypertarget{geostdescription:createtheconstraints}{}
First we create a Vector called ectr that will contain the external constraints. 
\begin{lstlisting}
  Vector <ExternalConstraint> ectr = new Vector <ExternalConstraint>();
\end{lstlisting}
In order to create the non-overlapping constraint we first create an array containing all the dimensions the constraint will be active in (in our example it is all dimensions) and lets name this array \emph{ectrDim} and a list of objects \emph{objOfEctr} that this constraint will apply to (in our example it is all objects).

\begin{note}
Note that in the current implementation of \texttt{geost} only the non-overlapping constraint is available. Moreover, 
\emph{ectrDim} should contain all dimensions and \emph{objOfEctr} should contain all the objects, i.e. the non-overlapping constraint applies to all the objects in all dimensions.
\end{note}
 
After that we add the constraint to a vector \emph{ectr} that contains all the constraints we want to add.
The code for these steps is as follows: 
\begin{lstlisting}
  int[] ectrDim = new int[dim]; 
  for(i = 0; i < dim; i ++)  
      ectrDim[i] = i; 
  int[] objOfEctr = new int[3]; 
  for(i = 0; i < 3; i ++) 
      objOfEctr[i] = objects.elementAt(i).getObjectId();
\end{lstlisting}
All we need to do now is create the non-overlapping constraint and add it to the \emph{ectr} vector
that holds all the constraints. this is done as follows:
\begin{lstlisting}
  //Constants.NON_OVERLAPPING indicates the id of the non-overlapping constraint
  NonOverlapping n = new NonOverlapping(Constants.NON_OVERLAPPING, ectrDim, objOfEctr);
  ectr.add(n);
\end{lstlisting}

\subsubsection{Create the \texttt{geost} constraint and add it to the model.}\label{geostdescription:createthegeostconstraintandaddittothemodel}\hypertarget{geostdescription:createthegeostconstraintandaddittothemodel}{}
\begin{lstlisting}
  Constraint geost = Choco.geost(dim, objects, sb, ectr);
  m.addConstraint(geost);
\end{lstlisting}
	
\subsubsection{Solve the problem.}\label{geostdescription:solvetheproblem}\hypertarget{geostdescription:solvetheproblem}{}
\begin{lstlisting}
  Solver s = new CPSolver();
  s.read(m);
  s.solve();
\end{lstlisting}

The full java code can be found here: \href{media/zip/geostexp.java}{geostexp.java}

\subsection{Support for Greedy Assignment within the geost Kernel}\label{geostdescription:supportforgreedyassignmentwithinthegeostkernel}\hypertarget{geostdescription:supportforgreedyassignmentwithinthegeostkernel}{}

\subsubsection{Motivation and functionality description.}\label{geostdescription:motivationandfunctionalitydescription}\hypertarget{geostdescription:motivationandfunctionalitydescription}{}
Since, for performance reasons, the \texttt{geost} kernel offers a mode where he tries to fix all objects during one single propagation step, we provide a way to specify a preferred order on how to fix all the objects in one single propagation step. This is achieved by:
\begin{itemize}
	\item Fixing the objects according to the order they were passed to the \texttt{geost} kernel.
	\item When considering one object, fixing its shape variable as well as its coordinates:
	\begin{itemize}
		\item According to an order on these variables that can be explicitly specified.
		\item A value to assign that can either be the smallest or the largest value, also specified by the user.
	\end{itemize}
\end{itemize}

\begin{note}
Note that the use of the greedy mode assumes that no other constraint is present in the problem.
\end{note}

	This is encoded by a term that has exactly the same structure as the term associated to an object of  \texttt{geost}. The only difference consists of the fact that a variable is replaced by an expression \_ (\emph{The character \_ denotes the fact that the corresponding attribute is irrelevant, since for instance, we know that it is always fixed}), $\min(I)$ (respectively, $\max(I)$), where $I$ is a strictly positive integer. The meaning is that the corresponding variable should be fixed to its minimum (respectively maximum value) in the order $I$.   We can in fact give a list of vectors  $v_1,v_2,\ldots,v_p$ in order to specify how to fix objects $o_{(1+pa)},o_{(2+pa)},...,o_{(p+pa)}$.

This is illustrated by Figure bellow: for instance, Part(\emph{I}) specifies that we alternatively:
\begin{itemize}
	\item fix the shape variable of an object to its maximum value (i.e., by using max(1) ), fix the $x$-coordinate of an object to its its minimum value (i.e., by using min(2)), fix the $y$-coordinate of an object to its its minimum value (i.e., by using min(3)) and
	\item fix the shape variable of an object to its maximum value (i.e., by using max(1)), fix the $x$-coordinate of an object to its its maximum value (i.e., by using max(2)), fix the $y$-coordinate of an object to its its maximum value (i.e., by using max(3)).
\end{itemize}

In the example associated with Part (I) we successively fix objects $o_1, o_2, o_3, o_4, o_5, o_6$ by alternatively using strategies (1) \mylst{object(_,max(1),x[min(2),min(3)])} and (2) \mylst{object(_,max(1),x[max(2),max(3))}. 

\insertGraphique{\linewidth}{media/greedy.png}{Greedy placement}

\subsubsection{Implementation.}\label{geostdescription:implementation}\hypertarget{geostdescription:implementation}{}
The greedy algorithm for fixing an object o is controlled by a vector \emph{v} of length \emph{k+1} such that:
\begin{itemize}
	\item The shape variable \emph{o.sid} should be set to its minimum possible value if \emph{v}[0]$<0$, and to its maximum possible value otherwise.
	\item abs(\emph{v}[1])-2 is the most significant dimension (the one that varies the slowest) during the sweep. The values are tried in ascending order if \emph{v}[1]$<0$, and in descending order otherwise.
	\item abs(\emph{v}[2])-2 is the next most significant dimension, and its sign indicates the value order, and so on.
\end{itemize}

For example, a term \mylst{object(_,min(1),[max(3),min(4),max(2)])} is encoded as the vector $[-1,4,2,-3]$.
	                                      
\subsubsection{Second example.}\label{geostdescription:secondexample}\hypertarget{geostdescription:secondexample}{}
We will explain although a 2D example how to take into account of greedy mode. Consider we have 12 identical objects $o_0,o_1,\ldots,o_{11}$ having 4 potential shapes and we want to place them in a box $B$ (7x6) (see Figure bellow). Given that the placement of the objects should be totally inside B this means that the domain of the origins of objects are as follows $x\in [0,5]$, $y\in [0,4]$. Moreover, suppose that we want use two strategies when greedy algorithm is called: the term \mylst{object(_,min(1),[min(2),min(3)])} for objects $o_0, o_2, o_4, o_6, o_8, o_{10}$, and the term \mylst{object(_,max(1),[max(2),max(3)])} for objects $o_1, o_3, o_5, o_7, o_9, o_{11}$. These strategies are encoded respectively as $[-1,-2,-3]$ and $[1, 2, 3]$.

\insertGraphique{\linewidth}{media/exp2_geost.png}{A second Geost instance.}

We comment only the additional step w.r.t. the preceding example. In fact we just need to create the list of controlling vectors before creating the geost constraint. Each controlling vector is an array:
\begin{lstlisting}
  Vector <int[]> ctrlVs = new Vector<int[]>();
  int[] v0 = {-1, -2, -3};
  int[] v1 = {1, 2, 3};
  ctrlVs.add(v0);
  ctrlVs.add(v1);
\end{lstlisting}
and then create the \texttt{geost} constraint, by adding the list of controlling vectots as an another argument, as follows:
\begin{lstlisting}
  Constraint geost = Choco.geost(dim, objects, sb, ectr,ctrlVs);
  m.addConstraint(geost);
\end{lstlisting}

The full java code can be reached \href{media/zip/greddyexp.java}{here}.
\lstinputlisting{media/zip/greedyexp.java}

The placement obtained using the preceding strategies is displayed in the following figure (right side).

\insertGraphique{\linewidth}{media/solution_exp2.png}{A solution placement}


\pagebreak
%\part{scheduling and use of the cumulative}
%\label{schedulinganduseofthecumulative}
%\hypertarget{schedulinganduseofthecumulative}{}

\section{Scheduling and use of the cumulative constraint}\label{schedulinganduseofthecumulative:schedulinganduseofthecumulativeconstraint}\hypertarget{schedulinganduseofthecumulative:schedulinganduseofthecumulativeconstraint}{}

\emph{This tutorial is the Choco2 version of \href{http://choco-solver.net/index.phptitle=schedulinganduseofthecumulative}{this one}}

We present a simple example of a scheduling problem solved using the cumulative global constraint.

%The problem is to schedule a given set of tasks on a single resource and in a given time horizon. 
The problem is to maximize the number of tasks that can be scheduled on  a single resource within a given time horizon.
% and to provide a corresponding valid schedule.

The following picture summarizes the instance that we will use as example. 
It shows the resource profile on the left and on the right, the set of tasks to be scheduled. Each task is represented here as a rectangle whose height is the resource consumption of the task and whose length is the duration of the task. Notice that the profile is not a straight line but varies in time. 

\insertGraphique{\linewidth}{media/schedulinstance.jpg}{A cumulative scheduling problem instance}

This tutorial might help you to deal with :

\begin{itemize}
	\item The profile of the resource that varies in time whereas the API of the cumulative only accepts a constant capacity
	\item The objective function that implies optional tasks which is a priori not allowed by cumulative
	\item A search heuristic that will first assign the tasks to the resource and then try to schedule them while maximizing the number of tasks
\end{itemize}

The first point is easy to solve by adding fake tasks at the right position to simulate the consumption of the resource. The second point is possible thanks to the ability of the cumulative to handle variable heights. We shall explain it in more details soon.

Let's have a look at the source code and first start with the representation of the instance. We need three fake tasks to decrease the profile accordindly to the instance capacity. There is otherwise 11 tasks. Their heights and duration are fixed and given in the two following int[] tables. The three first tasks correspond to the fake one are all of duration 1 and heights 2, 1, 4. 
\begin{lstlisting}
  CPModel m = new CPModel();
  // data
  int n = 11 + 3; //number of tasks (include the three fake tasks)
  int[] heights_data = new int[]{2, 1, 4, 2, 3, 1, 5, 6, 2, 1,3, 1, 1, 2};
  int[] durations_data = new int[]{1, 1, 1, 2, 1, 3, 1, 1, 3, 4,2, 3, 1, 1};
\end{lstlisting}

The variables of the problem consist of four variables for each task (start, end, duration, height). We recall here that the scheduling convention is that a task is active on the interval [start, end-1] so that the upper bound of the start and end variables need to be 5 and 6 respectively. Notice that start and end variables are BoudIntVar variables. Indeed, the cumulative is only performing bound reasonning so it would be a waste of efficiency to declare here EnumVariables. Duration and heights are constant in this problem. However, our plan is to simulate the allocation of a task to the resource by using variable height. In other word, we will define the height of the task i as a variable of domain $\{0, \mathtt{height[i]}\}$. The height of the task takes its normal value if the task is assigned to the resource and 0 otherwise. The duration is really constant and is therefore created as a ConstantIntVar.

Moreover, we add a boolean variable per task to specify if the task is assigned to the resource or not. The objective variable is created as a BoundIntVar. 

\begin{lstlisting}
  IntegerVariable capa = constant(7);
  IntegerVariable[] starts = makeIntVarArray("start", n, 0, 5, "cp:bound");
  IntegerVariable[] ends = makeIntVarArray("end", n, 0, 6, "cp:bound");
  
  IntegerVariable[] duration = new IntegerVariable[n];
  IntegerVariable[] height = new IntegerVariable[n];
  for (int i = 0; i < height.length; i++) {
      duration[i] = constant(durations_data[i]);
      height[i] = makeIntVar("height " + i, new int[]{0, heights_data[i]});
  }
  
  IntegerVariable[] bool = makeIntVarArray("taskIn?", n, 0, 1);
  IntegerVariable obj = makeIntVar("obj", 0, n, "cp:bound", "cp:objective");
\end{lstlisting}

We then add the constraints to the model. Three constraints are needed. First, the cumulative ensures that the resource consumption is respected at any time. Then we need to make sure that if a task is assigned to the cumulative, its height can not be null which is done by the use of boolean channeling constraints. Those constraints ensure that :

$$\mathtt{bool[i]} = 1 \quad\iff\quad \mathtt{height[i]} = \mathtt{heights\_data[i]}$$

We state the objective function to be the sum of all boolean variables. 
\begin{lstlisting}
  //post the cumulative
  m.addConstraint(cumulative(starts, ends, duration, height, capa, ""));                                                                            
  //post the channeling to know if the task is scheduled or not
  for (int i = 0; i < n; i++) {
      m.addConstraint(boolChanneling(bool[i], height[i], heights_data[i]));
  }
  
  //state the objective function
  m.addConstraint(eq(sum(bool), obj));
\end{lstlisting}

Finally we fix the fake task at their position to simulate the profil: 
\begin{lstlisting}
  CPSolver s = new CPSolver();
  s.read(m);
  
  //set the fake tasks to establish the profile capacity of the resource
  try {
      s.getVar(starts[0]).setVal(1); s.getVar(ends[0]).setVal(2); s.getVar(height[0]).setVal(2);
      s.getVar(starts[1]).setVal(2); s.getVar(ends[1]).setVal(3); s.getVar(height[1]).setVal(1);
      s.getVar(starts[2]).setVal(3); s.getVar(ends[2]).setVal(4); s.getVar(height[2]).setVal(4);
  } catch (ContradictionException e) {
      System.out.println("error, no contradiction expected at this stage");
  }
\end{lstlisting}
We are now ready to solve the problem. We could call a maximize(false) but we want to add a specific heuristic that first assigned the tasks to the cumulative and then tries to schedule them.
\begin{lstlisting}
  s.maximize(s.getVar(obj),false);
\end{lstlisting}
We now want to print the solution and will use the following code : 
\begin{lstlisting}
	System.out.println("Objective : " + (s.getVar(obj).getVal() - 3));
	for (int i = 3; i < starts.length; i++) {
	   if (s.getVar(height[i]).getVal() != 0)
	      System.out.println("[" + s.getVar(starts[i]).getVal() + " - " 
                                 + (s.getVar(ends[i]).getVal() - 1) + "]:" 
                                 + s.getVar(height[i]).getVal());
	}
\end{lstlisting}
Choco gives the following solution : 
\begin{lstlisting}
	Objective : 9
	[1 - 2]:2
	[0 - 0]:3
	[2 - 4]:1
	[4 - 4]:5
	[5 - 5]:6
	[0 - 2]:2
	[0 - 3]:1
	[3 - 5]:1
	[0 - 0]:1
\end{lstlisting}
This solution could be represented by the following picture :

\insertGraphique{\linewidth}{media/schedulsolution.jpg}{Cumulative profile of a solution.}

Notice that the cumulative gives a necesserary condition for packing (if no schedule exists then no packing exists) but this condition is not sufficient as shown on the picture because it only ensures that the capacity is respected at each time point. Specially, the tasks might be splitted to fit in the profile as in the previous solution.
The complete code can be found \hyperlink{cumulative:cumulativeconstraint}{here}.
